\chapter{Motivation}
\label{chap:background}

The main goal of this thesis is to develop a concept for an exergame for elderly. In our project, conducted the fall of 2012, \emph{Business Opportunities and Economics for an Exercise Game in the Health Sector} \cite{project}, we evaluated exergames to have great potential as a tool for exercising and rehabilitation. As future work we recommended to find a way to make an entertaining and motivating game concept, that should be easy to use and that should fit the needs for the end user group. This is our main motivation for this master thesis. 

The initial motivation for making an exergame is the problem of falls related to elderly, as this is a common problem that often has serious consequences. With age comes decline in physical health, where the main declines is in physical strength and balance. Falls have shown to be a very common event, and as much as 30 percent of people over 65 years old fall at least once a year. The consequences of falls can often be critical, with the most serious outcome being death. After experiencing a fall, it is common to be afraid of falling again. This often results in them being less active and the feeling of loss in confidence to carry out the activities they wish to do. Many people are afraid of leaving their house and suffer from total inactivity. This might also lead to loneliness and depression. There has been shown that exercising only once a week does not improve physical health, but with a supplement there has been seen improvements. The challenge is to get people to exercise as much as they should. This could be achieved by offering a more motivating training method, like with an entertaining game.

Because of the great share of elderly in the worlds population, there are a lot of focus on how to keep people healthier and postpone peoples' need for health care.  In the health sector in Norway the focus is now on; prevention, early intervention, and close collaboration between entities. This is stated in the reform, "Samhandlingsreformen", which was put in place from 01.01.2012. This reform also request more focus on welfare technology. Welfare Technology is in \cite{welfare} defined as: \emph{"Technological assistance that contributes to increased safety, social interaction, mobility, and physical and cultural activity. In addition welfare technology can help to strengthen an individual's ability to be independent despite sickness and social, mental or physical disabilities. Also it can work as technological support for relatives, as well as contribute to improving the services offered, when it comes to utilization of resources, availability and quality. In many cases welfare technology can prevent the need for health services or admission to an institution"} (Translated from Norwegian). The ultimate goal with the use of welfare technology is that people can take care of themselves longer.  As a part of this will every 75-year old be offered supervision to help them to promote health and own coping. When there is a need for it, every person should be provided with health care services that can be used in their home. With this, the demand in nursing homes can be decreased. The main focus in the new reform is based on "hverdagsrehabilitering", or "everyday rehabilitation" in English, where the goal is to postpone peoples need for help. The main actors here is the ergo-therapists and the physiotherapists. The new focus in the Norwegian health sector, suggest that an exergame can have a potential for health related services.

There has been great attention around the potential for exercise games in the health sector, and a lot of studies have been conducted. In \cite{project} we studied several research on that evaluated that exergames show great promise for exercise and rehabilitation, both for young and older people. This was because of the games' fun and entertaining nature, which could serve as a more motivating way to exercise. At the same time as possessing elements that might motivate people to exercice, playing games can also be a way to distract players from boring and painful treatments. Studies that tested different exergames on a group of elderly, got positive feedback. They were positive, and believed they would use games like these if they were more available. Even though exergames show great potential, the commercial games  are shown to bee not suitable, as they are made primarily for fun and entertainment, and not for rehabilitation. These games do not take elderly people's characteristics and limitations into considerations, and are therefore experienced as too complicated, including loads of information and rapid movements.

In \emph{Business Opportunities and Economics for an Exercise Game in the Health Sector}, the focus was to analyse the business potential of video games for use in exercise and rehabilitation for elderly.  Because of older people's inexperience with video games, and based on the new focus in the Norwegian health sector, we found the exergame to be relevant to be offered in a physiotherapy setting. We interviewed three physiotherapists in Trondheim. All of the interviewees were positive to the exergame, however they could not say whether they would use it in their service or not, without seeing the actual game. We were told that every patient is different, and whether they are eager to exercise or not really depends on whether they have a background as an active person or not. Therefore, it is important to not think of elderly as one common group of people, but people with different interests and needs. The physiotherapists mentioned different features the game have to offer. The game has to offer something better than what they are already offering, and at the same time ease their workload. Because of the differences in each patient's case, the game should have the ability to be customised, so that every patient can get an exercise program suited for their specific problems and needs. The game should be easy to use and understand, and should provide feedback both to let the player know if the exercises are being done right, and to motivate and engage the player to continue playing. 

With support from "Samhandlingsreformen", the new focus around welfare technology, and the interviews conducted, we concluded that an exergame have the best potential in a market where physiotherapists are the customers. The exergame could be used as a tool by the physiotherapists to engage their patients to exercise, and in that way prevent patients' decline in physical health. This can be in an individual training session with a physiotherapist, or as an activity in training groups. One of the clinics we interviewed offered a weekly training group for seniors. Training groups are also offered at various institutions in the different municipalities in Norway. This is an arena where the game could serve as a new, social, and entertaining form for exercise. The game could be a more motivating way for the patient to exercise at physiotherapy clinics, in group sessions, and possibly at home in the future. 

Having physiotherapists as the customers, while elderly are the end users, means that both actors need to be taken into account when developing the game. In this thesis we will look on aspects to consider when developing an exergame, seen from the end users point of view. Concentrating on the end users only will include creating a motivating story that the users will enjoy, and designing an appropriate user-interface. To figure out these aspects we have to meet the users to learn about and understand their needs and interests. This will be done by including a group of elderly in the design process. This thesis will not focus on the exergame from the physiotherapists' perspective. Neither will we look into how and where it will be used. In our previous project we discussed the exergame to be for both exercising and rehabilitation. In this thesis, we will only focus on making game for "general" exercising for elderly. 

