\chapter{Background}
\section{Video Games}
Video games are a genre of digital games that has become extremely popular all over the world. There exist several different gaming consoles, and for each console there has been developed an endless amount of different games that meets almost every need and interest. Video games can be described as “electronic, interactive games known for their vibrant colors, sound effects, and complex graphics” [se kilde [16] i prosjektoppgaven]. Hand held controllers or devices that capture movement are used to interact with the video game. The variety of video games makes it usable for many different purposes, like learning and education, exercising or just pure entertainment (veldig likt som oppgaven). \\ \\
One might associate video games with children and teenagers, and many hours of game play, which partly is a right assumption to make. Video games are played for several hours every day, and it has taken a great part in people's everyday life. Gaming was usually associated with being anti-social, because of all the time used looked up in a room playing, but video games has emerged from sedentary, lonely gaming to gaming involving social interaction and movement. In USA in 2010, one fourth of all gamers were under the age of 18, but an interesting fact is that as much as 26 percent of the gamers are over 50 years old. This shows that not only children and teenagers uses video games, a great share of elderly has started to use this type of technology. In Norway,  as much as 8 percent of the population in the age range from 45 - 79 use computer or video games every day. Elderly contributes to a great share of the world's population, and if the entertainment industry could see this group as potential customers, it would open up a new and inexperienced market for video games. [fra Digital Game Design for Elderly Users]. 
\section{Serious Video Games}
The widespread and assorted use of video games has made a great potential for the entertainment industry with regard to development of new games, but also in terms of developing games for additional purposes beside pure entertainment. A new way to use video games is for 
different types of training, education and learning, which is called serious games. Serious games can be described as “a mental contest, played with a computer in accordance with specific rules, that uses entertainment to further government or corporate training, education, health, public policy, and strategic communication objectives.” [fra From Visual Simulation to Virtual Reality to Games] The idea is to use the fun, motivating and captivating features that video games possess and combine it with pedagogy. The term serious games is not something new, it has existed for many years, but it did not break through when it first was introduced in the 1990s. The failure that time was that the focus of the game was on learning, which resulted in boring games with no market success \cite{susi2007serious}. Michael Zyda states that, in serious games, pedagogy has to be subordinate (SAMME ord som i teksten) to the story and entertainment component of the game. It is important that fun and entertainment are the main focus in the game \cite{zyda2005visual}. \\ \\
There exist several genres of serious games, where game-based learning, simulation games, games with a purpose, games for health and exergames are some of them. How and where serious games can be used, and are being used today, are many. With simulation games medical students can simulate surgery and soldiers can practise on how to use a rifle, students can learn their curriculum through quiz games, and people can be motivated to work out with exercise games. Research on use of serious games shows positive evolvement of skills and knowledge, and positive effect on health, which results in motivation for further research and development. (VET ikke om dette er rett ord å bruke). In addition, the new technology involving physical interfaces and improvement on graphics and animations initiates to new types of games. The market for serious games are inexperienced and unexplored, so there are a great potential for further research \cite{alfingewang}. Market sales the last couple of years also shows promise for serious games in the future. In 2008, the market for serious games sold for around 1,5 billion USD around the globe \cite{alfingewang}, and in 2010 it reached about 2 billion USD (1,5 billion EUR). The market is expecting a annual growth rate of 47 percent, up to about 13,5 billion USD (10 billion EUR) in 2015! \cite{idate} \\ \\
We will look deeper into one of the genres of serious games, exergames.  
\section{Exergames}
Skrive et kort sammendrag av hva exergames er og hvordan det blir brukt i dag. Ramse opp kjapt hva som går under exergames. 
\section{Exergames Used in Rehabilitation and Exercising}
Relevant arbeid der det er bevist at spill kan brukes for trening for eldre.. 
\section{Microsoft Kinect}
Beskrive Kinect nærmere og også grunnen til valget av Kinect. 
\section{How to Develop a Good Video Game?}
Related work.. Generelt om hva som skal til for å utvikler videospill. Dra noen konklusjoner.
\section{How to Develop a Good Exergame for Elderly}
Related work der de snakker om alle de viktige tingene vi må tenke på når man utvikelr for eldre..

