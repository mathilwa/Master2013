\chapter{Background}
This thesis is a projection of our project assignment \emph{Business Opportunities and Economics for an Exercise Game in the Health Sector} \cite{project}, conducted the fall of 2012. In this chapter, we will give a brief summary of what was done, and present findings from our previous project. To understand every detail of our project assignment, the reader will be referred to the report \cite{project}. 

In \emph{Business Opportunities and Economics for an Exercise Game in the Health Sector}, we analysed the business potential of video games for use in exercise and rehabilitation for elderly. A thorough background study was done to define the relevant actors and to understand where in the market the game could fit. We found that the game could fit well into physical therapy in a first introduction of the game. Because of older people's inexperience with video games, we did not see it as appropriate to have this group as a main customer segment, even though they are the end users of the product. Through interviews with physiotherapists and already conducted research, we found that video games can have a huge potential for exercise and rehabilitation, but that there is a need for customised games for elderly, as the commercial games are not aimed for these user. 

The main motivation for the development of an exergame for elderly is the problem of elderly and falls. This is a problem that often has serious consequences. With age comes decline in physical health. The main declines is in physical strength and balance. Falls have shown to be a very common event, and as much as 30 percent of people over 65 years old fall at least once a year. The consequences of falls can often be critical, with the most serious outcome being death. After experiencing a fall, it is common to be afraid of falling again. This often results in them being less active and the feeling of loss in confidence to carry out the activities they wish to do. Many people are afraid of leaving their house and suffer from total inactivity. This might also lead to loneliness and depression. 

We found, in our previous study \cite{project}, that there was a new focus in the health sector in Norway; prevention, early intervention, and close collaboration between entities. This is stated in the new reform, "Samhandlingsreformen", which was put in place from 01.01.2012. This reform also request more focus on welfare technology. Welfare Technology is in \cite{welfare} defined as: \emph{"Technological assistance that contributes to increased safety, social interaction, mobility, and physical and cultural activity. In addition welfare technology can help to strengthen an individual's ability to be independent despite sickness and social, mental or physical disabilities. Also it can work as technological support for relatives, as well as contribute to improving the services offered, when it comes to utilization of resources, availability and quality. In many cases welfare technology can prevent the need for health services or admission to an institution"} [Translated from Norwegian]. The ultimate goal with the use of welfare technology is that people can take care of themselves longer.  As a part of this will every 75-year old be offered supervision to help them to promote health and own coping. When there is a need for it, every person should be provided with health care services that can be used in their own house. With this, the demand in nursing homes can be decreased. The main focus in the new reform is based on "hverdagsrehabilitering", or "everyday rehabilitation" in English, where the goal is to postpone peoples need for help. The main actors here is the ergo-therapists and the physiotherapists.

Normally, physiotherapists get consulted with patients when they need rehabilitation after an incident, or when they need exercise help because of health problems that limits them from doing everyday activities. After an examination, the physiotherapist will set up a customised program for the patient related to their problems and needs.

After doing research about "Samhandlingsreformen", and after understanding what work physiotherapists do, we saw physiotherapists as an interesting potential customer for the exergame. Trying to find support in our beliefs, we interviewed three physiotherapists. The main goal of our previous study was to develop a business model around a customer segment. Therefore, the interviews were mostly directed towards physiotherapists use of tools and their buying routines. However, we also asked questions about their patients, and their attitudes towards exercise. The physiotherapists mentioned that they wished to get patients in earlier, so they could focus more on prevention instead of rehabilitation. We were told that every patient is different, and whether they are eager to exercise or not really depends on whether they have a background as an active person or not. Therefore, it is important to not think of elderly as one common group of people, but people with different interests and needs. The physiotherapists could confirm that one hour of physiotherapy a week usually is not enough to improve physical health, and that they had problems motivating patients to perform the home exercise program they were provided with. 

When asking the physiotherapists about if they believed in an exergame for elderly, they were all positive. However, it was hard to give an answer about whether they would use it or not, as the game was not yet developed. They stated some important requirements that need to be present in the game for them to be interested in buying and using it. The game has to offer something better than what they are already offering, and at the same time ease their workload. Because of the differences in each patient's case, the game should have the ability to be customised, so that every patient can get an exercise program suited for their specific problems and needs. The game should be easy to use and understand, and should provide feedback both to let the player know if the exercises are being done right, and to motivate and engage the player to continue playing. The feedback could also serve as a good way for the physiotherapist to follow their patients progression.

With support from "Samhandlingsreformen", the new focus around welfare technology, and the interviews conducted, we concluded that physiotherapists were a natural customer segment for the exergame. The exergame could be used as a tool by the physiotherapists to engage their patients to exercise, and in that way prevent patients' decline in physical health. This can be in an individual training session with a physiotherapist, or as an activity in training groups. One of the clinics we interviewed offered a weekly training group for seniors. Training groups are also offered at various institutions in the different municipalities in Norway. This is an arena where the game could serve as a new, social, and entertaining form for exercise. The game could be a more motivating way for the patient to exercise at physiotherapy clinics and in group sessions. 

There has been shown that exercising only once a week does not improve physical health, but with a supplement there has been seen improvements. The challenge is to get people to exercise as much as they should. This could be achieved by offering a more motivating training method, like with an entertaining game. The market is very immature and the majority of the end users are unfamiliar with technology. Therefore, we evaluated it as too early to aim towards the end user as a customer segment. Instead the physiotherapist will serve as a customer segment, as well as a channel towards the end user. We will refer the reader to our last study \cite{project} to read more about our evaluation of the game's potential on the market where physiotherapists are the customer segment.

Having physiotherapists as customer segment, while elderly are the end users, means that both actors need to be taken into account when developing the game. To be able to develop a game that physiotherapists can use as a tool for motivating their patients to exercise, it is as important as the relevant exercises, to have a story around the game that the user will enjoy. This means that there is a need to involve the end users, to find out what their needs are. In this thesis we will therefore concentrate on the end user only, and not so much about how the the game will be used by the physiotherapists in practise. Findings from this project assignments shows important aspects to consider when developing an exergame for elderly. We will take these into consideration in this thesis.