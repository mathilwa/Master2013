\chapter{Conclusion}

During our study we found commercial exergames to be not as inappropriate for elderly as previous research first indicated. There were functionality and elements that did not suited the characteristics of elderly, but mostly, we evaluate the commercial exergames to meet some needs for effectiveness, efficiency and satisfaction.  What has to be taken into consideration when developing an exergame for this user group are real-life experiences, more instructions, motivational factors, clear goals, simple menus, technical precision, and social aspects. This thesis presents an exergame concept, that are based on a real-life theme with familiar activities and challenges, that has included most the considerations mentioned above. We believe it is a good foundation for a future exergame for elderly. In addition, our thesis provides general guidelines and requirements that can help developers create user-friendly video games that meet the needs and characteristics of elderly.  

\section{Future Work}

Future work for an exergame for elderly can head in different directions. 

One direction would be to continue working on the video game concept presented in this thesis. This will imply inclusion and potential improvement of the aspects for future work listed in Section \ref{sec:summarydiscW2}. This should be taken into a new iteration of the cycle of user centered design. Next time design is to be prototyped, there should be focus on developing more interactive prototypes to increase the users understanding of the game. 

Another direction to go would be to look into, and study and validate, our summarised guidelines, findings, and specified requirements. We feel that our study has been thorough, but we acknowledge that there always will be further research. One approach would be to conduct new workshops, with a wider selection of participants. This implies recruiting a higher number of participants than we did, but also including elderly with more diversity due to technology experience, interests, education, physical health, and life-style. A more diverse group of elderly will provide the study with more reliable information.   

A place for this exergame to be used is, as concluded in \cite{project}, at physiotherapy clinics as an assistant tool in their work. Future work should therefore be to look at this exergame from the physiotherapists point of view, to understand and specify their wants and needs. This due to both wanted functionality and requirements to an user interface. Future work should also include looking into other possibilities for how and where the exergame could be used.   

Since the technical aspects related to the delay was such an issue, it should be the future work for developers to figure out how to minimise or eliminate this. This could be grounded in choice of software. 
