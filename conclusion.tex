\chapter{Conclusion}

This thesis concludes with presenting a design of an exergame for elderly, together with recommendations for future work. The game is meant to be compatible with the motion sensor technology provided by Microsoft Kinect. The work have been done by studying and analysing related theory and literature, which includes studying elderly and their attitude towards exercising, research done on elderly and exergames, and theory about video game design, usability and system design in general. One iteration of the cycle of user-centered design was conducted, all from gathering information about the users' needs and expectations, to specifying requirements and designing solutions, which were evaluated together with the users. This has been done by conducting two workshops with a group of elderly from the organisation "Seniornett". The motivation for this thesis is built upon the work presented in our project assignment \cite{project}. However, in that assignment we focused on the exergame's potential in the market, with physiotherapists as customers. This thesis the focus have been on the exergame's context of use, seen from the end users. 

Results from the first workshop clearly showed that existing commercial games are not developed with elderly as the intended user group. These games were at times too fast, they had to complex menus, and the games contained disturbing elements and features. In addition, there was lack of instructions and feedback, which made the informants feel that they did not master the challenges. This also made it difficult to know if what they did was done correctly. However, the commercial games held features that captivated the elderly. Well-known activities in a real-life environment enhanced the gaming experienced, and they seemed to enjoy playing. 

Based on theory, literature, and findings from the first workshop, requirements have been specified and an exergame concept has been designed. The exergame is based on the use of challenges, activities and environments that are familiar to the elderly users. The story for the game is a walk in the nature, with additional, well-known activities like picking apples, swimming, paddling and wood chopping. Exercises that are shown to be good for elderly are built into the story of the game, where the game will take the player through a full body workout. The inclusion of clear instructions is emphasised, and motivational factors, like the possibility to achieve goals, are included. The exergame contains challenges to also exercise cognitive skills, like quizzes to answer along the walk in the nature. To enhance the gaming experience, the user interface is simple and not too extensive.

The evaluation of the exergame concept in the second workshop showed that the elderly liked the idea of the familiar forest environment. What the elderly were concerned about was the inclusion of quizzes while walking or performing other challenges, as they did not like the idea of doing more than one task at the same time. They were also confused about how progression with goals and levels were presented. About the menu in the game, they had no negative comments. The general evaluation of the exergame was that the elderly liked the concept presented for them. The results from the evaluation have been listed as recommendations for future work. 

The results from the workshops presented in this thesis are based upon feedback from the chosen group of informants. This is a group of fit elderly, which exercise regularly. They all have technology experience, but not with video games. Knowing elderly as a group, we evaluate the informants to be a homogeneous group based on their interests, and physical and psychological challenges. The choice of representative users is important in the work of gathering reliable information, and we have acknowledged that the inclusion of a homogeneous group might not have provided us with information representative for elderly as a group. 

Use of exergames for exercise and rehabilitation is a popular topic, and a huge amount of research has been done. To make an exergame for the elderly users, it is crucial to take characteristics of elderly, as diversity in needs and interests, and the wide range of various physical and psychological challenges, into consideration. We believe the exergame concept presented in this thesis, together with the recommendations provided for future work, is a good foundation for a future exergame for elderly. The provided guidelines and requirements can help developers create user-friendly video games that meet the needs and characteristics of elderly.
 
\section{Future Work}

Future work for an exergame for elderly can head in different directions. 

One direction can be to continue working on the exergame concept presented in this thesis. This will imply inclusion of the aspects for future work listed in Section \ref{sec:summarydiscW2}. This should be taken into a new iteration of the cycle of user-centered design. Next time design is to be prototyped, there should be focus on developing more interactive prototypes to increase the users understanding of the game. Professionals, like music therapists and physiotherapists, should be included in this work to find appropriate music and exercises, respectively.

Another direction can be to look into, and study and validate, our summarised guidelines, findings, and specified requirements. We feel that our study has been thorough, but we acknowledge that there always will be further research. One approach can be to conduct new workshops, with a wider selection of informants. This implies recruiting a higher number of informants than we did, but also including elderly with more diversity in technology experience, interests, education, physical health, and life-style. A more diverse group of elderly will provide the study with more reliable information.   

A place for this exergame to be used is, as concluded in \cite{project}, at physiotherapy clinics as an assistant tool in their work. Future work should therefore be to look at this exergame from the physiotherapists' point of view, to understand and specify their wants and needs. This due to both wanted functionality and requirements to an user interface. Future work should also include looking into other possibilities for how and where the exergame could be used.   

Technical aspects, such as the delay problem should be addressed in future work. The goal should be to eliminate this issue. This can be grounded in choice of software. 
