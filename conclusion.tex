\chapter{Conclusion}

During our study we found commercial exergames to be not as inappropriate for elderly as previous research first indicated. There were functionality and elements that did not suited the characteristics of elderly, but mostly, we evaluate the commercial exergames to meet some needs for effectiveness, efficiency and satisfaction.  What has to be taken into consideration when developing an exergame for this user group are real-life experiences, more instructions, motivational factors, clear goals, simple menus, technical precision, and social aspects. This thesis presents an exergame concept, that are based on a real-life theme with familiar activities and challenges, that has included most the considerations mentioned above. We believe it is a good foundation for a future exergame for elderly. In addition, our thesis provides general guidelines and requirements that can help developers create user-friendly video games that meet the needs and characteristics of elderly.  

\section{Future Work}

Future work for an exergame for elderly can head in different directions. 

One direction will be to continue working on the 
- Ta tak i future work presentert i diskusjonen. Fungerer for vårt konsept. Ta dette med i ny interasjon. Lage og bruke mer interaktive prototyper for å øke forståelse. 
- Teste på en mer diverse gruppe. Også se på de som er frail. 
- Se på fysioterapeutenes side
- Validere og  utvide våre funn og guidelines.  
- Jobbe med å finne ut hvordan det kan brukes, og hvor det kan brukes.
- Finne ut hvordan man kan eliminere delay.
