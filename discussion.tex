\chapter{Discussion}
\label{chap:discussion}

\section{Discussion of Findings from Workshop 1}

In Chapter 6.1. we discussed the importance of usability which is about how easy a system is to use, learn and understand. Workshop 1 was performed to see how a set of relevant users interacted with existing commercial exergames, and to identify what aspects of these games work and do not work. \cite{usabilitydef} states three relevant elements that can say something about a system's usability: \emph{effectiveness}, \emph{efficiency}  and \emph{satisfaction}. These were aspects among others that we tried to measure in the workshop. We also tried to answer the concept \emph{context of use}. We wanted to discover the needs of the games' intended users, the needs for the functionality and the environment for the game. Where and in what circumstances the game can be used, was discussed in our previous project assignment \cite{project}. This is not our main focus in this thesis, but will briefly be discussed in Chapter \ref{subsec:whatwhere} 

In this section, we will discuss the findings from workshop 1 and relate these findings to the literature provided in precious chapters. In workshop 1, and in our literature study we have tried to answer these research questions: 

\emph{RQ1: Are existing commercial Xbox Kinect games suitable for exercising purpose for the senior user group?}

\emph{RQ2: What are the design challenges when developing video games aimed for exercising for the senior user group?}

First, we will provide a general discussion, and then we will summarize by more precisely answer the two questions at the end of this section. 

\subsection{Control of character, clear goals and feedback}
In the Game Flow model \cite{sweetser} discussed in Chapter \ref{sec:heur}. eight core elements that should be present to experience enjoyment in games are described. Of these eight core elements we found a lack of three of the elements in the commercial games tested in workshop 1. The informants expressed that they did not feel that they had control over their character. This was due to the significant delay that was present in the games. In addition it came clear from the observation that it was not always easy to understand when the game started and ended, as well as what was an instruction video and what was the actual game. The element clear goals did also seem to lack in the commercial games, and it was expressed by one of the informants that there was a need for more instructions on what was actually expected from them and the rules of the game. The last element the informants were not satisfied with was the feedback. Also two of the eight golden \cite{mmi} rules presented in Chapter \ref{subsec:golden} discuss the importance of getting informative feedback at appropriate time. The informants desired more feedback on their actions, and they especially wanted to know whether they did the exercises right or wrong. They did not feel that they got this feedback in the games played in the workshop, which made them both confused and frustrated at times. Clear goals, appropriate instruction, information and feedback, are aspects highly considered in the new game concept. The delay problematic will be discussed in more detail in Section .. 
 
\subsection{Immersion and concentration}
As presented in the GameFlow model \cite{sweetser} immersion and concentration are important aspects of the gaming experience. It was hard to evaluate from observing and interviewing informants if they were immersed into the game and concentrated on the tasks. However, comments like \emph{"I felt like the person [avatar] itself"}, and cheerful comments given by the informants while they played, like \emph{"I wanted that pineapple"}, suggested that the informants immersed into the game. If we are to evaluate, we would say that all the games required concentration to perform the tasks right. However, it did not always seem like the informants were that concentrated. One example of this is that sometimes we had to assist in reading the messages appearing on the screen, like "raise hand above head to play", while they read the same type of messages themselves at other times. We believe that the text should have been clear enough, and that the reason for them not reading the text, was because they were not concentrated enough on the game. 

\subsection{The possibility to customise}
As mentioned, in our previous project \cite{project} we evaluated the game to fit as a tool that physiotherapists can use as an alternative exercise method for their patients. The value proposition of this game we described as: \emph{"A tool with the ability to customize an exercise program, and to offer an alternative, fun and motivating training method, while at the same time ease the workload of the physiotherapist"} \cite{project}. In this thesis we have focused on one part, of this description: \emph{an alternative, fun and motivating training method}. For this game to meet these three criteria, the end-users had to be included in the development process. However, as learned from \cite{Billis}, \cite{gregor}, \cite{gerling1}, as well as from the informants, there is a need to customise these kind of games and acknowledge elderly's various limitations and disabilities. One example given by the informants was that the skiing game could have too fast pace for some people within this user group, and that it could cause problems for people with decline in their balance function. The possibility to customize the games can for example be a feature in the physiotherapists interface, where they can put together different exercises within the game story, that fits their patient. Another example is for the user herself to have an interface where she can put together her own program.  As can be found in Table \ref{tab:func2} in Chapter \ref{sec:req} we have made additional requirements for this, but we have limited our thesis to not prototype a user-interface for this. 

\subsection{Aspects to meet Player enjoyment}
In Chapter \ref{sec:motivators} we discussed self-efficacy as an important determinant of exercise behaviour. Elements that are relevant to sustain this exercise behaviour are the feeling of pleasure and satisfaction, and self-regulatory skills. This was also discussed in workshop 1 as important aspects, and the informants mentioned goal setting, the possibility for socialising, and the possibility to self decide what to do, as important. The feeling of mastery was seen as significant. They were clear that if they did not get the feeling of mastery, they would not play these games. This relates to the elements discussed in the Game Flow model \cite{sweetser}, where some of the criteria for player enjoyment in games are to include challenges that match the player's skill level, to have different levels of challenges, and clear goals. In addition, the Game Flow model says that games should support social interaction. Also in Chapter \ref{sec:exergames} the importance of social interaction in exergames are discussed, and  as much as 62 percent of all gamers say they play with others \cite{statistics2012}. Several previous studies \cite{Billis}, \cite{gerling2}, \cite{gerling1} discussed in Chapter \ref{sec:summaryguidelines} also stress the importance of the social factors of a game. Offering social interaction, can especially be important for elderly who experience loneliness in their everyday life, due to inactivity \cite{project}. It was interesting to learn that social interaction also was seen as important by the informants. The majority of the informants would rather play together than alone. However, none of the informants could see themselves playing together with others over the internet. We believe that one of the reasons for them stating this was because they did not understand the concept of "playing over the internet", as a result of their inexperience with this kind of technology. We believe that today's market is too immature for this. 

The informants' opinions did not differ much according to gender on what they liked and did not like in the games we tested. However, as discussed in Chapter \ref{sec:motivators} different people's needs and expectations, as well as aspects such as gender and ethnicity should be considered. We will meet this by making a concept with a series of games that covers a variety of interests, as well as getting to choose a man character or a woman character. Race might also be considered, but it is difficult to cover all races. Chao et al. \cite{chao} discuss that to meet these requirements it is important to be in contact with the relevant people. Even though we clearly have not covered the total group of elderly, we have included a small group to understand some needs and expectations. In Chapter \ref{sec:sergames} we discussed how video games can function as a pedagogical tool and it was shown that important factors to focus on includes motivation, effectiveness and intuitiveness. Another aspect discussed in the same chapter is behaviourism, which states that if someone is rewarded for something he or she is likely to repeat the action that triggered the reward KILDE. These aspects were also mentioned by the informants, who stated that the system needs to be easy to understand, and that the feeling of mastery and that you learn something are important.  One informant mentioned that if they did not manage to do something, they would stop doing it. In our game concept we will focus on emphasising the goals in the game, and the player will be rewarded with points. In addition we provide different levels. This will be done in two ways: Different initial difficulty levels the player can choose between, and different difficulty levels within the initial levels, where the next level depends on the previous.  

\subsection{Appropriate and simple feedback and information}
In Chapter \ref{sec:summaryguidelines} we listed typical characteristics of elderly based on findings from the literature. In our research we had a group of informants, who were relatively physically and mentally fit. From the list provided in Chapter \ref{sec:summaryguidelines} we only experiences three out of the nine characteristics. We experienced that one of the informants had problems reading the text in some of the menus. Because there was only one informant that seemed to have problems with this, we assume that this informant might have had impaired vision. However, it is important to acknowledge that impaired vision is a common problem for the older population as discussed in Chapter \ref{sec:summaryguidelines}, and in Chapter \ref{sec:designelderly}, and it should therefore be considered in a game designed for this group. The group of informants had interest in technology and used different types of technology, like computer, tablets, mobile phones, e-mail, e-banking, TV, etc. However, none of the informants had any experience with video games. In the beginning of the workshop we experienced that some of the informants were insecure, and had problems understanding what they were suppose to do. It took some time for most of them to understand that they had to use their body to play. Therefore, we see a need for clearer instructions both before and under game play. This includes an introduction to how the system works, like how to interact with the sensor. The last of the characteristics listed in Chapter \ref{sec:summaryguidelines}  we experienced, was that some of the informants expressed that it was hard to do more than one thing at the same time. Therefore, the information given, and the tasks to be done, should be limited, and adjustable. The possibility to add more functionality after the existing functionalities are managed was suggested by one informant. This is also listed as one of the guidelines in Chapter \ref{sec:summaryguidelines}, and suits well with the requirements of simplicity discussed in Chapter \ref{sec:simplicity}. 

In Chapter \ref{sec:summaryguidelines} we presented a set of previous studies and provideed a summary of some important aspects that can serve as guidelines when developing games for elderly. One of these guidelines suggested in \cite{Billis} and \cite{gerling1} is about giving motivating feedback. Some of the games that were played in workshop 1 had a lot of different features that were suppose to be motivating. However, by some of the informants this was rather seen as annoying.  In addition the amount of the information given, both text and audio, was experienced as too much. At the same time, some of the informants stated that they did not recognize these type of messages at all. In Chapter \ref{sec:simplicity} we discuss minimalistic design which is about bringing the most important elements into focus, without elements that will distract the user. Microsoft presents it as "Simple Can Be Powerful", which means that simplistic design not necessarily needs to mean lack of functionality. From this we conclude that we should avoid too much features in our video game concept. We should keep it simple, and focus on a few motivating aspects. One of the informants desired more time to read information and instructions, and suggested that there should be a way to tell the system when you are finished reading. This is also a requirement suggested in \cite{w3cTekst} which we have included in the guidelines for making user-friendly interfaces for elderly in Chapter \ref{sec:designelderly}, and will be taken into account in our game concept. 

\subsection{Cultural and lifestyle diversity}
Another guideline discussed in Chapter \ref{sec:summaryguidelines}, and the second of the eight golden rules presented in Chapter \ref{subsec:golden}, are about matching cultural and lifestyle diversity in the games. This was expressed by the informants as important, and they suggested different type of themes for a possible game concept, like dance, swimming, apple picking etc.. If they were to play a game like this they stated that it would need to include sports or activities related to real life. They also mentioned the importance of appropriate music. They did not like the music in the commercial games because they were not the type of music they used to listen to. Using appropriate music was also discussed in Chapter \ref{sec:motivators}, and \cite{schutzer} sees music also as a way to divert from pain coming from the exercises. The majority of the informants agreed that music was important, especially to keep the rhythm. However, they stated that they wanted it quite, and they wanted music more related to their generation. This is an important requirement we will set for the exergame, however, it is without our profession to say anything specific about what music to include. 

\subsection{Summary}
We will now provide a summary of the discussion to in a more precise way answer research questions 1 and 2. 

\emph{RQ1: Are existing commercial Xbox Kinect games suitable for exercising purpose for the senior user group?}

As mentioned initially, effectiveness, efficiency  and satisfaction are relevant measurements when evaluating the usability of systems. From findings done in workshop 1 we can conclude the following about the commercial games tested: 
\begin{itemize}
\renewcommand{\labelitemi}{$\bullet$}
\item The existing commercial games tested on a group of elderly do not meet the requirements of effectiveness. The games do not spend enough time on instructions and information, and do not give sufficient feedback on what the players are doing is right. The menus are too complicated and it is too many elements showing at the same time. The buttons presented in the games are too sensitive, and it is not intuitive how to press the buttons. 
\item The existing games meet only to a degree the requirement of efficiency. In FruitNinja it was clear that the informants did not understand what was required from them, and they just waved their hands uncontrollably. The majority of the the informants understood what was expected from them in the tennis game, and played through this game without problems. The skiing game was also well understood, until the second match where the two players that played together switched tracks. All of the informants had problems relating to their player after switching tracks. None of the menus met the requirements of efficiency. We had to assist the informants through the menus, and because of too much information and too sensitive buttons, the informants spend unnecessary time on getting through the menus.
\item Two of the four games played in workshop 1 meet the requirement of satisfaction. All of the informants liked playing the tennis and skiing game and they had fun while playing. FruitNinja, they did not like that much, which relates to their wish for playing games with a meaningful content that they could relate to everyday life, which is not the case for FruitNinja. The informants were not satisfied with the personal trainer game either, because this focused on "just" regular training, which they would rather do without the game. This strengthens up under Michael Zyda's statement on that the main focus in serious games should be fun and entertainment \cite{zyda2005visual}. 
\end{itemize}

It is important to mention that this was the first time the informants played games like these, and initially they did only play the games once \footnote{Three of the informants got to play the skiing game a second time because they specifically asked for it. We did not observe anything new in this second session, and have because of this, and because the rest of the informants did not try a second time, not included this in our analysis}, which did not enable us to say anything about the informants learning curve. However, we did see that they understood more what was expected from them, and that they got more confident after a while. If we had went on with several rounds with the same games, we might have seen improvements. This was also mentioned by the informants. 

\emph{RQ2: What are the design challenges when developing video games aimed for exercising for the senior user group?}

It is important that we acknowledge the difficulties about engaging a user group into a setting they are completely unfamiliar with. It is hard to evaluate some users needs, when they do not even know about these needs themselves. We have learned from workshop 1 and from our previous project that today's elderly are not necessarily the right user group for a game like this, but that the next generation of elderly are more suitable. However, it is hard to test a tool that is meant as an alternative form for exercising on people who are physically in good shape, and who keep one doing regular exercising. 

The game should meet all the requirements in the GameFlow model \cite{sweetser} presented in Chapter \ref{sec:heur}. We found this model to serve as good guidelines, as many of them was mentioned as important aspects by the informants, and also because there is a clear relation between these guidelines and the guidelines we can draw out from the literature, presented in the previous chapters. Specifically the informants desired a game with a story that appeals and relates to real life, as well as appropriate music to their age. It was urged for more instructions on how to interact with the game, as well as what was expected from the players, the goals, and the rewards. The menus in the commercial games were seen as challenging because of the amount of information and the sensitive buttons. In a game for this user group this needs to be improved. The majority of the informants would only play these type of games together with others. In \cite{chao}, \cite{statistics2012}, \cite{Billis}, \cite{gerling2} and \cite{gerling1} social aspects are also discussed to be important, as well as it is a recommended guideline in the Game Flow model \cite{sweetser}. Therefore, it is important to include this in our game concept. The delay in the games was seen as a big problem, and should be acknowledged.  This is a technical issue that we are not in the position to evaluate the reason for. We believe that if this game is to be used for exercising, both at home and in clinical settings, technical precision have to be present. 
 


\section{Discussion of Our Video Game Concept}
- Hva med å ha diskusjon og presentasjon i samme? Ref maja og william young
- Diskusjon av brukergruppen
- Diskusjon av brukergrensesnitt for eldre. Brukergrensenitt for fysioterapeut - ikke med (?)
- Si at vi har fokusert på eldre bruker, at det ville krevd en helt annen brukergruppe å ha workshop med om det skulle lages brukergrensesnitt for fysioterapeuten sin side.
- Vise fram tre menybilder, der vi har diskutert farge på knapp og tekst. 
- Viktigheten av inkludere valg for muskelgrupper. Alle spill har jo introklipp til hva som trenes, men da må man først gå helt inn til spillet for å vite "utbyttet" av treningen.

\section{Discussion of Findings from Workshop 2}

\section{Quality of the gathered information}
- Er deltakerene representative?
	- Allerede veldig spreke og aktive
	- I overkant teknologiinteresserte
- Påvirkning under workshop
- Valg av spill
- Valg av sted. Observasjon i naturlige omgivelser?
- Forutsetningene for worskhopen. 

- Utføring av workshop 1 dag 1 vs dag 2.
- Det å ikke ta akkurat de samme spørsmålene i gruppediskusjon de to dagene, har det noe å si?
- Randomisering av workshop 1. 

- Vår rolle som nøytrale under workshop 2. Vi har utviklet og designet et konsept, og vi vil jo gjerne ha en verdifull idemyldring rundt dette.  

- Brukerinvolvering. Tar med brukerene bare to ganger. En til observasjon, og en hvor vi presenterer et design. I utgangspunktet burde brukeren vært mer involvert, fra starten til presentasjon av konsept. De ble ikke inkludert så mye i starten, dette mye fordi det er en teknologi og en "verden" det er veldig vanskelig for dem å sette seg inn i og mene noe om. 


\section{Miscellaneous Aspects to Consider About the Exergame Concept}
\label{sec:misc}

\subsection{Discussion on where and in what circumstances the game can be used}
\label{subsec:whatwhere}

In this our previous project \cite{project} we evaluated the game to have the most potential in a clinical setting, offered as a training method at physiotherapy clinics. This was based on two main things. First, "Samhandlingsreformen" encourage the use of welfare technology where possible in health care. Second, most elderly, who are the end-users have little or no experience with video games, which will make it hard to reach out to these customers. In this thesis we have not focused on where the game should be implemented. However, we did ask the informants where they could see the game be used and two main settings were mentioned: in a group setting, for example in nursing homes, and in a setting with grandchildren. We decided to not study any further where the game can be implemented. This was both because we experienced that the majority of the informants had a hard time picturing themselves own and use a game like this, and also because from what we learned from the interviews conducted in our previous project, where physiotherapists could not say anything about whether they would use this game or not before they could test the actual game. For more discussion around where the game could fit, we will direct the reader to \cite{project}, and in particular Chapter 8.2 and 9 in that report. 

In Chapter \ref{sec:barriers} we discussed some challenges when it comes to motivating elderly to exercise. Some of these challenges were also mentioned by the informants. One informant saw it as a barrier to exercise if she lived far away from the training centres, and could see that the game could have a potential as a replacement for going to the gym. Another informant did not want to get controlled by time and appointments, which suggest that an exergame could be an alternative way to exercise on their own time. 

Delay
Fysioterapeuter
Enkelt å sette opp
Lastetid. Hente fram instruksjonsvideo først, og laste opp spillet mens denne vises?

\begin{table} [H]
\label{tab:nfunc2}
\centering
\begin{tabular}{|l|l|}
\hline
3.1 & The system shall be able to run on both PC and Xbox. \\ \hline
3.2 & The system shall be easy to set up (physically).\\ \hline
3.3 & The system shall include Kinect functionality, like pausing \\ & a game by holding one arm out from the body. \\ \hline
3.4 & The system shall load within few seconds.\\ \hline
3.5 & The system shall be small in size and do not require too \\&  much space.\\ \hline
3.6 & The system shall not require too much capacity. It shall \\ & be able to run on a regular PC. \\ \hline
3.7 & The system shall not require too much power. \\ \hline
3.8 & The system shall avoid delay between the player's \\ & movement and action on the screen.\\ \hline
3.9 & The system shall ensure secure storage and sharing of \\ & profiles. \\ \hline
\end{tabular}
\caption[Miscellaneous non-functional requirements]{Miscellaneous non-functional requirements}
\end{table} 

\section{General stuff}
- Diskusjon av gjennomføring av workshop
- Guidelines/tips til nestemann
