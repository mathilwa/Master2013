\chapter{Barriers and Motivation to Exercise in Elderly People  }
\label{chap:olderexercise}

An exergame for elderly needs to include the right exercises, as well as motivating elements, to engage the user group to use it. Meeting this, we have to understand what motivates elderly to exercise in general. There are different challenges and barriers that keep older people from exercising, and it is shown that only a small share of elderly meets official requirements of physical activity. There exist various proposed guidelines on how elderly people should exercise, and there are different services that both can be implemented at home, and performed outside the home. Introducing an exergame, which integrates already existing exercises in a story with focus on entertainment, can be a fun, motivating and effective alternative to existing services.  

In Section \ref{sec:exercisebehaviour} we will discuss older people's attitude towards exercising, and look at numbers that support the need for an alternative way to exercise. Guidelines on how elderly should exercise will briefly be discussed, but the interested reader are referred the original references for more information, as this is out of our area of competence. In Section \ref{sec:barriers} we will discuss typical barriers and challenges that keep elderly from exercising. At last, in Section \ref{sec:motivators} we will discuss motivating factors that will be taken into account in the development of the game concept.  

\section{Elderly and Exercising}
\label{sec:exercisebehaviour}
Physical activity is important for all people, including elderly. It can improve quality of life by reducing risk of some chronic diseases, alleviate depression, and help people towards a more independent life. Improving health by physical activity does not just apply for healthy elderly, but also for frail and very old people. Guidelines are proposed by different entities, like The Norwegian Directorate of Health and the U.S. Department of Health and Human Services, recommending physical activities like aerobic activity, as well as strengthening exercises \cite{aktivitetsbok} \cite{guidelines}. A set of guidelines for how to exercise may seem easy to follow, but it is shown to be challenging to motivate elderly to exercise. There is shown that there is only a small percentage of elderly who actually engage in physical activity \cite{olderamericans}. To be able to engage seniors to be more physically active, they should be provided with programs that will motivate them to actually perform the exercises. In this thesis, we will develop a exergame concept for this purpose. As discussed in our previous project \cite{project}, keeping the older population healthy can, in addition to decrease the risk of falling, reduce the need for health services, and also decrease economic challenges related to this. It is therefore important to have focus on the older populations' physical health. 

Norwegian elderly, aged  67-79 years old, have become more active the last years, with as much as 40 percent of this population exercising regularly several times a week \cite{statisticsnorway}. However, only one fifth of adults meet the minimum requirements recommended by The Norwegian Directorate of Health, which is 30 minutes of exercising per day \cite{statistikknorge12}. 23 percent of the group between 67-79 years old in Norway exercise less than once a month or never \cite{statisticsnorway}, which states that there are many adults that do not meet these requirements. It is shown that some exercise is better than none, and they should therefore be engaged in physical activity. Even though exercise once a week not is enough, it have some positive effects on elderly \cite{gruppetrening-trheim}. The numbers are worse in other parts of the world. The Federal Interagency Forum on Aging Related Statistics \cite{olderamericans} presented numbers in their latest report, revealing that only 11 percent of Americans aged 65 years and older, meet the 2008 Federal Physical Activity Guidelines \cite{guidelines}. The percentage decrease with increasing age, where 14 percent of older people aged 65-74 years old meet these guidelines, while this is the case for only 4 percent of people over the age of 85. It is quite common that physical activity decrease with age and that after the age of 65 years old, the level of activity is at its lowest \cite{schutzer}. 
 
The Norwegian Directorate of Health has made a handbook for physical activity for prevention and treatment. They recommend that regular activity that includes the major muscle groups should be performed 2-3 days a week, 20 minutes each time. Different activities suggested includes cycling, swimming, walking, jogging and cross-country skiing. Activities that involves resistance training has shown positive effect also for elderly people. It is recommended that exercises that include all main muscle groups should be performed 1-2 times a week, be done for 10-12 repetitions, and be progressive. In addition, it is recommended to perform exercises that will strengthen mobility, gait and balance function. Exercises for improving balance, should be customised for each individual's needs. Mobility is best maintained by regularly moving the body, both with everyday movements and physical activity. Exercises that have proved good for improving balance functions include stand on one foot, or walk in circles, sideways, or over obstacles. In addition, a regular walk in various terrain is a good and convenient activity to maintain balance function, gait and mobility \cite{aktivitetsbok}. The readers interested in the specific guidelines proposed by The Norwegian Directorate of Health and the U.S. Department of Health and Human Services will be referred to  \cite{aktivitetsbok} and \cite{guidelines}. 

In our previous project \cite{project} we presented different training programs that are made especially for elderly. One of these are "Øvelsesbanken", which is created by the physiotherapy service in Trondheim, Norway.  "Øvelsesbanken" is a service primarily meant for physiotherapists to in an convenient way put together exercise programs for their patients to perform at home. However, private users have the possibility to register, and put up their own programs, as well. The main purpose with the exercises is fall prevention, and includes exercises to increase balance and strength \cite{eldretrening}. We found these exercises relevant for the exergame. We have chosen to build the concept around these exercises, together with the activities proposed by The Norwegian Directorate of Health. This will give us a set of exercises shown to be good for elderly, that we could not have put together ourselves, as we do not have the competence to recommend exercises.

Age-related physically and mentally changes can be reduced with regular physical activity. There exist several guidelines on how and what to exercise, however, there are various challenges when it comes to motivating elderly people to exercise. These will be discussed in the next section. 

\section{Barriers and Challenges}
\label{sec:barriers}
It is hard to get people to be physical active, and this might especially apply for elderly. Because of the high fall statistics there is important to engage elderly in physical activity, as the main reason for falls are reduced physical strength and decreased balance. General guidelines suggest that elderly should perform exercising programs that contains exercises that strengthen muscles, balance, endurance and mobility \cite{gruppetrening-trheim}. There are some unique challenges and barriers when it comes to engaging elderly in physical activities discussed in the literature. Understanding these barriers will make it easier to identify how to motivate them to exercise. It is crucial when developing exercise programs, to understanding the motivational factors behind exercising \cite{chao}.

One common barrier is that elderly think their health is not good enough to start exercising \cite{schutzer}, and may believe exercising will do more harm than good \cite{chao}. Their poor health and the pain related to this, hinders them from exercising. In addition, elderly have a tendency of thinking they are too old to start exercising \cite{schutzer}. A significant challenge is to get people to exercise long enough to see positive results. Many associate exercising with pain, sweating, and muscle soreness, and the time before positive outcomes are noticed can be long. At the same time, the negative effects of not exercising, may not be that apparent \cite{chao}. 

The importance of having available and convenient resources is significant, as it is shown that people living far away from training centres, parks and walking paths are less active than people living near these facilities \cite{schutzer}. With this comes also time constraints. It is shown that many elderly look at physical activity as time consuming, thinking about the time doing the activity, as well as the time getting to the exercise site \cite{schutzer} \cite{chao}. It can be challenging to get people to perform unstructured, free-living, exercise programs, compared to structured programs. This is because people themselves have to decide when, where, and what to do. This is pretty much based on self-motivation \cite{chao}.  

Physicians play an important role when it comes to advising elderly to exercise, because most people have respect for and trust their physicians. This was also discussed in \cite{project}, where we found physiotherapists to be an appropriate and reliable mediator for the exercise game. However, in \cite{schutzer} and \cite{chao}, it is discussed that physicians do not always give sufficient exercise advice. Instead of giving a specific exercise program for the patient to perform, they simply just tell them to be more active. Many elderly have little knowledge about physical activity and its advantages. This can come from the fact that many elderly grew up in a time where people generally were more active in their everyday life, and where there were not that much attention around the importance of physical activity \cite{schutzer}. In addition, many think of physical activity more as a recreation activity, or something people do for competition, and are not looked as necessary for keeping a good health.  

\section{Strategies and Motivators}
\label{sec:motivators}
Understanding the barriers and challenges, can help to understand what factors motivate people to exercise. Having more time, more information, and living closer to exercise offerings, are basic goals that can serve as good motivators for exercise. A structured program offered at a set time, will probably be a good solution for this user group, as a common problem is to find time, as well as decide which exercises to do \cite{chao}. A strategy is to focus on that physicians are telling their patients. It is important to remember the common limitation of memory capacity among elderly. Therefore, information should be given in a precise and accurate way \cite{chao}.  

The elements that need to be in place to sustain the exercise behaviour are the feeling of pleasure and satisfaction, as well as self-regulatory skills, like goal-setting, monitoring of progress, and self-reinforcement or motivation \cite{schutzer}. In addition, self-efficacy seems to be an important determinant for exercise behaviour. Self-efficacy is defined in \cite{schutzer} as \emph{"an individuals belief in their ability to successfully perform a specific behaviour"}, and it is said that it is more likely for a person with strong self-efficacy expectations and outcomes to adopt to a specific behaviour. In \cite{white} they evaluate self-efficacy to play an important role in the relationship between physical activity and quality of life. In \cite{chao} there are suggested additional strategies to promote adherence to physical activity, which includes implementing decision-making models, modifying cognitive thoughts during activity, and increasing social support. White et al. did a study where they measured a sample of community dwelling adults' physical activity, self-efficacy, global quality of life, physical self worth, and disability limitations. They concluded that \emph{"being more active was associated with being more efficacious, having fewer disabilities limitations, reporting higher physical self-worth, and being more satisfied with one's life"} \cite{white}. In addition, \cite{white} stated that self-esteem is an important component of the physical activity and quality of life relationship. 

Chao et al. \cite{chao} discuss some additional strategies for motivation.  One must understand different peoples' needs and expectations, and take into account race, gender, ethnicity etc. To meet these expectations, contact with the relevant people is important. \cite{chao} suggests this to be done with for example interviews and focus groups. Self-regulatory skills, including goal setting, self-monitoring of progress, and environmental management, are important to keep peoples' exercise behaviour \cite{chao}. Clear goals should be set to let the participant understand that the activity has a purpose and is going through an end, and that skills will be developed through practice. To easily monitor the exercise, goals should be separated into small and more manageable parts. Environmental factors, like convenience of activity facilities, can enhance adherence to physical activity. Exercising should be looked at as an ongoing process, and it is important to remember that people's behaviour towards exercising can change with age. Therefore, prevention of relapse should be included in the planning process, to maintain the physical activity routine \cite{chao}. 

In \cite{schutzer} it is suggested some alternative motivating factors. They suggest prompts, like e-mailing and telephone contact. These prompts are typically used for home based training programs, and are shown to be at least as effective as supervision face-to-face. Telephone contact works well in a starting phase, when trying to get a person to adopt to a more active lifestyle, while e-mail contact works well to help maintain this lifestyle. These ways can actually be more effective than prolonged exercise session with face-to-face contact \cite{schutzer}. 

The last important motivator discussed in \cite{schutzer} is appropriate music to enhance the exercise experience, and to divert from pain coming from the exercises. 


\subsection{Summary of motivator factors for exercising}
\label{subsec:motivator}

From the literature we found that there are 12 motivational factor for exercise behaviour. These are listed below. 

\begin{enumerate}[{m}.1]
\item Convenience of activity facilities
\item Precise information about what to exercise 
\item Sufficient information about the benefits of exercising
\item Self-efficacy and self-esteem
\item Self-regulatory skills
\item Small and manageable goals
\item Self-monitoring of progress
\item Modifying cognitive thoughts 
\item Decision making
\item Social support
\item Meet different peoples' needs and expectations
\item Prompts, either face-to-face, by telephone or e-mail. 
\item Appropriate music
\end{enumerate} 

\bigskip


When designing physical training programs, understanding the factors that motivate people to exercise is very important. The motivational factors discussed in this chapter will be considered when we develop an exergame concept for elderly. In addition different activities presented by The Norwegian Directorate of Health as well as the exercises presented in "Øvelsesbanken" will be considered. There are additional factors that needs to be studied that are more related to video games and elderly. This will be looked into in Chapter \ref{chap:exforseniors} and \ref{sec:designelderly}. First we will briefly present what video games are, and how they can be used for exercising. 




