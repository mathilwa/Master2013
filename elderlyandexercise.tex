\chapter{Older Adults and Their Attitudes Towards Exercising}

Physical activity is important for all people, including elderly. It can improve quality of life by reducing risk of some chronic diseases, alleviate depression, and help people towards a more independent life. Improving health by physical activity does not just apply for healthy elderly, but also for frail and very old people. Guidelines are proposed by the Department of Health and Human Services, recommending aerobic activity, as well as muscle-strength training \cite{olderamericans}. These guidelines  may seem easy to follow, but it is shown to be challenging to motivate elderly to exercise, and there is shown that there is a small percentage of elderly who actually engage in physical activity \cite{olderamericans}. To be able to engage seniors to be more physically active, they should be provided with programs that will motivate them to actually perform the exercises. In this thesis, we will develop a game concept for this purpose. This game have to have the right elements in place, to motivate the user to engage in regular physical activity. ( It needs to relate to their life, and it needs to be offered at placed where they can actually use it. ) As discussed in our previous project \cite{project} and in \cite{schutzer}, keeping the older population healthy, can in addition to reduce the need for health services, also decrease economic challenges related to this.  It is therefore important to have focus on elder's physical health. 

As mentioned, a small percentage of elderly engage in physical activity. The Federal Interagency Forum on Aging-Related Statistics \cite{olderamericans} presented numbers in their latest report, revealing that only 11 percent of Americans aged 65 years and older, meet the 2008 Federal physical activity guidelines \cite{guidelines}, and that the percentage decrease with increasing age, where 14 percent of older people aged 65-74 years old meet the guidelines, while only 4 percent of people aged 85 years old and older, meet the requirements.  It is quite common that physical activity decrease with age and that after the age of 65 years old, the level of activity is at its smallest \cite{schutzer}. Even though not directly comparable with the american numbers, we found it interesting to see that Norwegian elderly, aged  67-79 years old have become more active,with as much as 40 percent of this population exercise regularly several times a week. The most popular activities include cross country skiing, swimming, cycling and brisk walking. It is also shown that jogging has become a more common activity also for the elderly. However, it is still 23 percent of the group between 67-79 years old in Norway who exercise less than once a month or never \cite{statisticsnorway}.

(It is shown, that regular physical activity, can decrease chronic diseases and disabilities. Being active can also help to increase the quality of life for the older person \cite{schutzer}.)
 
\section{Barriers and Challenges}
It is hard to get people to be physical active, and this might especially apply for elderly. Elderly have a tendency of thinking they are too old to start exercising. As the fall statistics are shown to be high, resulting from reduced physical strength and decreased balance, there is important to engage elderly in physical activity. However, there are some unique challenges or barriers when it comes to engaging elderly in physical activities, as discussed in Schutzer \cite{schutzer} and Chao \cite{chao}.  Understanding motivation factors for exercising is critical when developing exercise programs \cite{chao}.

One common barrier is that elderly think their health is not good enough to start exercising \cite{schutzer}, and may believe exercising will do more harm than good \cite{chao}. Their poor health and the pain related to this, hinders them from exercising \cite{schutzer}. It is also shown that older women associate physical exercise with something masculine \cite{chao}. It is also a challenge to get people to exercise long enough to see positive results. Many associate exercising with pain, sweating, and muscle soreness, and the time before positive outcomes are noted can be long. At the same time may the negative effects of not exercising, not be that apparent \cite{chao}. 

The importance of having available and convenient resources is significant, as it is shown that people living far from for example recreation centers, parks and walking paths are less active than people living near these facilities \cite{schutzer}. With this comes also time constraints. It is shown that many elderly look at physical activity as time consuming, thinking about the time doing the activity, as well as the time getting to the exercise site \cite{schutzer} \cite{chao}. Again, having the resources more available, will overcome this barrier. It can be challenging to get people to perform unstructured, free-living exercise programs, compared to structured programs. This is because people then have to decide when, and what to do, and is pretty much based on self-motivation \cite{chao}.  Physicians play an important role when it comes to advising elderly to exercise, because most people have respect for and trust their physicians. This was also discussed in \cite{project}, where we found physiotherapists to be a suitable and reliable mediator for the exercise game. However, as discussed in \cite{schutzer} and \cite{chao}, the physicians does not always give sufficient exercise advice. Instead of giving a specific exercise program for the patient to perform, they simply just tell them to be more active. Another aspect that may keep elderly from exercising, is the lack of knowledge about physical activity and its advantages. This can come from the fact that many elderly grew up in a time where there was not that much attention around the importance of physical activity. 

\section{Strategies and Motivators}
It is important to understand the barriers and challenges, and at the same time find what is motivating people to exercise. In \cite{schutzer} and \cite{chao} they also presents how the discussed barriers can serve as motivators to activity. For example can decrease in physical health actually motivate to start exercising. Also having more time, more information, and living closer to exercise offerings, could motivate for physical activity. What the physician is telling their patients is important, and should be done in a precise way. It is important to remember that a common limitation for this group is memory capacity, due to for example dementia and therefore information should be given in a precise and accurate way \cite{chao}.  The importance of self-efficacy is discussed in \cite{schutzer}, \cite{chao} and \cite{white}. Self-efficacy seems to be an important determinant of exercise behavior. It is more likely for a person with strong self-efficacy expectations and outcomes to adopt to a specific behavior. In \cite{white} they evaluate self-efficacy to play an important role in the relationship between physical activity and quality of life. Along with this, the elements that needs to be in place to sustain  the exercise behaviour are the feeling of pleasure and satisfaction, as well as self-regulatory skills \cite{schutzer}. There has been suggested different strategies to promote adherence to physical activity, presented in \cite{chao}, and includes goal setting, self-monitoring of progress, implementing decision-making models, modifying cognitive thoughts during activity, and increasing social support. (denne ble litt skrevet av.. vanskelig \cite{chao}). Chao et al. \cite{chao} discuss some strategies for motivation.  One must understand different peoples' needs and expectations, and take into account race, gender, ethnicity etc.  To meet these expectations contact with the relevant people is important. Chao suggest this to be done with for example interviews and focus groups.  Self-regulatory skills, including goal setting, self-monitoring of progress, and environmental management, are important to keep peoples' exercise behavior \cite{chao}. Clear goals should be set to make the participant understand that the activity has a purpose and is going through an end, and that skills will be developed through practice. To easily monitor the exercise, goals should be separated into small and more manageable parts. Activity facilities should be more available and it should be convenient for people to exercise. Exercising should be looked at as an ongoing process, and it is important to remember that people's behavior towards exercising can change with age. Therefore, prevention of relapse should be included in the planning process, to maintain the physical activity routine \cite{chao}. In the article \cite{schutzer} there is also suggested methods to motivate for physical activity. They suggest prompts, like e-mailing and telephone contact. These prompts are typically used for home based training program, and are shown to be as effective as supervision face-to-face can be. Telephone contact worked well in a starting phase, when trying to get a person to adopt to  a more active lifestyle, while e-mail contact worked well when the person had adopted the lifestyle, to help maintain this lifestyle. These ways can actually be more effective, than prolonged exercise session with face-to-face contact \cite{schutzer}. Another important motivating factor discussed in \cite{schutzer} is appropriate music to enhance the exercise experience, and to divert from pain coming from the exercises. As a last motivator is the demographics, where it is shown that people who already have an active lifestyle, who were already fit, who had lower body mass, had fewer chronic diseases and pain, were non-smokers, and had high self-efficacy, had the best adherence to exercise (siste neste avskrift.. vet ikke hvordan jeg skal skrive om en oppramsing) \cite{schutzer}.
At last, a structured program offered at a set time, will probably be a good solution for this user group, as a common problem is to find time, as well as decide what kind of exercise to do \cite{chao}.

White et al. states that "being more active was associated with being more efficacious, having fewer disabilities limitations, reporting higher physical self-worth, and being more satisfied with one's life" \cite{white}. In addition, it was stated that self-esteem is an important component of the physical activity and QOL relationship (siste skrevet av også ) \cite{white}. 