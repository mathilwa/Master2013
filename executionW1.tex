\chapter{Execution of Workshop 1}
In this chapter we will describe the planning and execution of  Workshop 1. We will describe the purpose of the workshop, provide general information about participants, location and equipment, and present how the workshop was set up and performed. 

\section{The goal for Workshop 1}
The primary goal for our first workshop was to introduce the participants, from now on called informants, to the Xbox Kinect technology and to three commercial games. We wanted to observe how the interacted with the technology and how well they enjoyed playing, and we wanted to have a discussion where the informants could talk about how they experienced the gaming session. We also used a short survey to get to know the informants technology experience and attitude towards exercising.    

\section{General information}
The workshop was held over a two day period, the 13th and 14th of March, with location at "Gulhuset, Voll gård", a place familiar to the informants from "Seniornett". The workshop started around 2 pm and lasted approximately three hours. We had recruited eight informants from "Seniornett" to our workshop, three males and five females. They were divided into two groups, one for each day. One of the recruited females had an accident, which made it difficult for her to participate in the workshop. We therefore ended up with seven informants. The informants average age was 70.6 years (with a standard deviation of 7,9 years). In addition to us and the informants, we had two Ph.D. students with us the first day who acted in the role as facilitators. The second day our supervisor, Lill Kristiansen, joined us and took the role as the facilitator.   

In advance we had sat up an agenda for how we wanted to carry out the two workshop days, see Table \ref{tab:agenda}.  

\begin{table} [ht!]
\centering
    \begin{tabular}{|l|l|}
       \hline
       \textbf{Introduction} & 15 minutes  \\ \hline
       \textbf{Survey} & 10 minutes  \\ \hline
       \textbf{Single play} & 15 minutes for each participant \\ \hline
       \textbf{Multi play} & 10 minutes for each participant \\ \hline
	   \textbf{Group discussion} & 65 minutes \\ \hline
    \end{tabular}
    \caption[Workshop Agenda]{Workshop agenda}
    \label{tab:agenda}
\end{table} 

\section{Location and Equipment}
We used "Gulhuset, Voll gård" as location for our workshop. This is a location well known for the informants. This location is used for various events, like theme lectures, song meetings, and story telling gatherings. This location has couches, tables and chairs, and it has a small kitchen where it is possible to make coffee and something to eat. "Gulhuset" was ideal to use for our workshop, not only because is is familiar to the informants, but because it was possible to make a living room-like atmosphere. It was also ideal because it is a place where we imagine that elderly could meet and play a future exercise game. At "Gulhuset" also possess a screen and a projector, which we used for our introduction. However, during the gaming session we chose to use a 42" flat screen. This was to make the gaming experience as natural as possible, as most people do not have screens and projectors in their own homes. We borrowed the flat screen from our department at NTNU. In addition to the the flat screen we had a Xbox, a Kinect sensor and three commercial games, which where use for the gaming session. This equipment was bought with support from a foundation for master thesis with computer science as subject [ENDRE litt på denne].   

In the workshop we used both video and audio recording to be able to interact with the informants. We rented a dictaphone, two video cameras (a Sony Handycam and a Panasonic 3MOS) and a rack for each camera. We wanted to use two video cameras to be able to make recordings from different angles. We had one video camera in the front of the room to capture movement and facial expressions, and we had one in the back to see movements from behind, in addition to recording interaction with the Kinect sensor and the games. The dictaphone was used for the group discussion.      

\section{Procedure}
In the introduction we shortly presented ourselves and the background and main goal for our master thesis. We also informed about the purpose of the workshop, and presented the agenda for the day. The main part of the introduction was a review of the consent form, see Appendix X, where we highlighted important aspects as video and audio recording and that participation is voluntary. After the introduction the informants had some time to look over the consent form before they signed two copies, one for themselves and one for us. We then handed out a short survey to the informants, where they were asked a few questions about themselves, their technology experience and how they attitude are towards exercising. 

After finishing this session we were ready to start gaming. The gaming session was divided into two parts, one part where the informants played some pre-chosen games individually, and one where they played together in pairs. The order of which part that where played first was for randomisation changed from day one to day two. In workshop day one there were only three informants, so they took turns playing during the multi player part. 

We presented the informants for three different pre-chosen commercial games, \emph{Fruit Ninja}, \emph{Your Shape Fitness Evolved 2012}, and \emph{Kinect Sports Season Two}. \emph{Fruit Ninja} is a game where you have to use your arms like a ninja to slice fruit that is thrown up into the air. The goal is to slice as much delicious fruit as possible in a short period of time, without hitting any bombs that are thrown up together with the fruit. This games features simultaneous multi play with both competition and cooperation. The choice for using \emph{Fruit Ninja} in the workshop was to present the informants for a game based on pure fun and movement, with a concept far from something one might experience in real life. \emph{Your Shape Fitness Evolved 2012} is a popular exercise game for Kinect. This game has over 90 hours of activities, which can be used to design your own workout program. \emph{Your Shape Fitness Evolved 2012} follows your shape, fitness level and goals, and uses this to schedule the difficulty for the activities. You have the possibility to choose exercises for specific muscle groups, you can join classes like jump rope, cardio boxing and yoga, or you can take a virtual run in New York or Paris \cite{yourshape}. We chose this game to let the informants experience a game that are designed with exercise as the main purpose, with the use of realistic exercises. We presented a workout program for the informants that are designed specific for elderly. This program consist of a set of aerobic-like movements. \emph{Kinect Sports Season Two} is game that consist of a bundle of six sports, which are tennis, darts, base ball, American football, skiing and golf. These sports stimulates movement and activity in a fun and motivating way, though exercise is not the main focus. \emph{Kinect Sports Season Two}, depending on the sport chosen, offers both competitive and cooperative multi play. We wanted to present this game for the elderly because of its amusing and real-life activities. \emph{Your Shape Fitness Evolved 2012} and tennis from \emph{Kinect Sports Season Two} was used for single play, while \emph{Fruit Ninja} and skiing from \emph{Kinect Sports Season Two} was used for multi play. 

Before the informants started on their multi player session, we introduced how the Kinect sensor works and how to interact with it. We also describe the games they where about to play, and the goal of the games. In the two games where the informants played together, we started the games for them. One of the goals with the workshop was to observe how well the informants understood the technology and the games presented for them, so when they where about to play individually, we let them walk through the menu and start up the games alone. We guided them when necessary. 

When the informants had played individually and together in pairs we sat down and had a group discussion. We asked them open questions about their experience of the technology and the gaming session. We wanted to know if they liked the games or not, and if so why. It was also in out interest to find out if this technology is something the informants would use, and if they could imagine using it for exercise. Our questions acted just as a starting point for discussion, most of the time the informants talked freely with us and each other.                   