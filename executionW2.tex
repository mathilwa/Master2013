\chapter{Execution of Workshop 2}
The aim of workshop 2 was to involve the potential user group, elderly, in the development process. We invited all the informants that participated on workshop 1 to a second workshop where we wanted to present our prototyped video game concept. The reason for doing this was to give the informants, representatives for the group of elderly, the possibility to provide us with feedback on our concept. Their feedback will be taken into consideration, which will be very helpful in the process of creating a user-friendly video game for elderly.  

\section{Execution}
Workshop 2 was held the 25th of April at "Gulhuset, Voll gård", the same location used for workshop 1. We met around 1 pm and started our presentation 20 minutes later. The total duration for workshop 2 was approximately two hours. Beforehand, we had sat up an agenda:

\begin{itemize}
\renewcommand{\labelitemi}{$\bullet$}
\item Welcome.
\item Practical information.
\item Findings from workshop 1.
\item Presentation of our video game concept.
\item Feedback on our video game concept.
\item Presentation of our menu proposal.
\item Feedback our menu proposal.
\item Summary and finish
\end{itemize}

To this workshop, we invited all the informants from workshop 1, included the informant that could not made it due to an accident. Five informants replayed positive to our invitation, three males and two females. This includes the one informant that could not participate in workshop 1. 

We presented our video game concept with a laptop, projector, and a screen. We alternated between presentation and discussion, so that there should not be information overload for the informants.
Because of our desire to give our full attention to the presentation and discussion, we recorded this workshop. This was done with the use of a video camera, because we wanted to capture facial expressions in addition to the oral feedback. 

To make a cosy and comfortable atmosphere, we served cake and coffee.   




