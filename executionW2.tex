\chapter{Execution of Workshop 2}
The aim of workshop 2 was to involve the potential user group, elderly, in the development process. We invited all the informants that participated on workshop 1 to a second workshop where we wanted to present our prototyped video game concept. The reason for inviting the informants to this second workshop, was because their feedback will be highly valuable when creating a user-friendly video game for elderly. These informant are representatives for the group of elderly, and they are the ones who knows the interest and needs for this user group. They are also themselves potential future users of the system. In our first workshop it was all about the informants interests, views on exercise, and opinions and experience from game play. In this workshop, we will give the informants an realistic impression of what an exercise game, based on their feedback, will look like. The way we will present the game is by prototypes made from PowerPoint and Photoshop. These are tools that are familiar to us, and it did not require any extra time to learn how to use them. Since we are in this early stage of the development process of a potential exergame, we have focused on using low-cost prototypes.  


\section{Execution}
Workshop 2 was held the 25th of April at "Gulhuset, Voll gård", the same location used for workshop 1. We met around 1 pm and started our presentation 20 minutes later. The total duration for workshop 2 was approximately two hours. Beforehand, we had sat up this agenda:

\begin{itemize}
\renewcommand{\labelitemi}{$\bullet$}
\item Welcome.
\item Practical information.
\item Findings from workshop 1.
\item Presentation of our video game concept.
\item Feedback on our video game concept.
\item Presentation of our menu proposal.
\item Feedback our menu proposal.
\item Summary and finish.
\end{itemize}

To this workshop, we invited all the informants from workshop 1, in addition to the informant that could not made it due to an accident. Five informants replayed positive to our invitation, three males and two females, and this includes the one informant that could not participate in workshop 1. (Si noe om dette at vi inkluderte en som ikke hadde vært med på spillingen? Evt si noe om dette i diskusjonen?)

For this presentation we only needed a laptop, a projector, and a screen. Because of our desire to give our full attention to the presentation and discussion, we also recorded this workshop. This was done with the use of a video camera, because we wanted to capture facial expressions in addition to the oral feedback. To make a cosy and comfortable atmosphere, we served cake and coffee.   

We started the presentation with an introduction, presenting ourselves, the goal for the workshop, and the agenda for the next hours. Some practical information about duration, the video recording, and requirements due to anonymity was also presented. The execution and a summary of the feedback from workshop 1 was presented, both to fresh up the informants' memory, and to give the "new" informant a recap of workshop 1. This was also done to give the informants an idea of what we have focused on when creating our video game concept.        

In this workshop we alternated between presentation and discussion, so that there should not be too much information to remember for the informants. First, we presented the overall idea for the concept, before we proceeded with a more detailed description of the video game. The video game series was presented, and we told the informants that we had focused on the video game "ut i naturen". We showed them prototypes of two challenges from this game. We presented the challenges one by one, with a individual group discussion for each of them. To make it easier for the informants to comment and discuss we handed out pictures of the prototyped scenes (Sette inn figur/bilde av bilder på bord). 

We also presented for the informants the menu prototype we had made. We tried to present the menu in a way to make it look as realistic to a Kinect video game as possible. One of us stood in front of the screen simulation game play, pretending to "push the buttons", while the other controlled the PowerPoint presentation.  After the menu presentation we opened for a new, and final, group discussion. 

After the final group discussion was over, we took a few minutes to greet the informants, and thank them for their feedback, time, and participation. It was important for us to tell the informants that we are very grateful for their help, and that their feedback is highly valuable for our master thesis.    

 