\chapter{Findings from Workshop 1}
\label{chap:findW1}

In workshop 1 we wanted to test commercial Xbox Kinect, seeking to identify what needs elderly have when it comes to a system such as this. This meets the first step of the activities in the cycle of user-centered design, understanding and specifying the context of use, depicted in Figure \ref{userdesign}. A description of the execution of workshop 1 can be found in Section \ref{sec:ws2}. 

In this chapter we will present the findings from this workshop. These findings are based upon feedback from the informants, and our own observations during the workshop. The seven informants are referred to as I1 to I7, and we have due to requirements to anonymity decided to not distinguish between male and female. Therefore, we refer to all the informants as females. Quotes are translated from Norwegian to English to  preserve the meaning. Based on the data from the questionnaire we will start by describing the informants as a group.

\section{The Informants}
The group of informants consisted of three males and four females, with an average age of 70,6 years. From the questionnaire we learned that they use various technical equipment, like TV, mobile phones, tablets, DVD-players, computers, and music players. All of the informants have their own computer, where five out of seven use this several times per week. E-banking, e-mail and news are the general reasons for using the computer. Most of the informants see social activities as important. Four out of seven said that they like to share information with family and friends through social media. When playing games together with others, two informants preferred collaboration, while the rest liked to both collaborate and compete. All the informants, but one, said that they are regularly physically active, both from doing everyday tasks, and from exercising. Motivating factors for this includes: being in activity, socialising, achieving a good mood, and maintaining a good health.

\section{Perceived Usefulness and Value of Entertainment}
\emph{"Mostly, it was quite amusing"}. This was the general feedback from the informants after the gaming session. The informants were divided in their opinions about whether they would buy a game like this or not. Some of them would rather exercise for themselves or go to training centres, while others saw the gaming as more amusing than using a treadmill or an exercise bike. I5 stated that \emph{"I think this was very fun, so I would like to buy one of these. I have one of those exercise bikes in my basement, but it is so boring that I can not bear it"}.  Some of the informants stated that one of the reasons why it was fun playing was because it was a completely new experience, and they found it as a fun way to exercise. I6 said \emph{"Now I have been involved in something I have never been a part of before. [...] It is a new world that has opened up, that is for sure"}.  They liked the idea of combining gaming and activity. I4 said that \emph{"elderly might think it would be fun to do this and be active at the same time"}. I6 explained that she felt different from when she was watching the other informants play, to when she was playing herself. \emph{"The participation was more insightful to me than I had first anticipated. When I watched the others play, it felt so unreal to have someone on the screen, but in the activity, when you got into it, it was not so stupid after all"}. The observations we did during the gaming session support the feedback the informants gave us. There were a lot of smiling and laughing, and it seemed like they had fun.  

The informants liked some games better than others. The two games from "Kinect Sports: Season Two", tennis and skiing, were the games they liked the most. They thought the activities were fun and they liked the challenges the game provided. \emph{"I liked it very much. I like that kind of activity"}, was I4's opinion. The informants also liked that the game required something from them. About skiing, I3 said \emph{"This was fun, because, here you have to pay attention and spend some effort"}. I3 also liked tennis for the same reason. It was mentioned that the pace in the skiing game might be to fast for some of the target users, like frail older people. This was also mentioned by one in the audience in our first presentation for "Seniornett". She suggested that the player should be able to choose pace herself. However, the informants liked the fast pace, because it made the game exciting and fun. The same lady from the audience also commented on too many elements in the skiing game, which could make people with decreased balance function dizzy. 

"Your Shape Fitness Evolved 2012", with its "Aging with Grace", and "Fruit Ninja" were the two games the informants as a total liked the least. They did not see the need for playing "Aging with Grace", as they rather would do these types of exercises in a training centre or by themselves. I3 said she did not like this game because it did not require anything from her. \emph{"The aerobic game I did not care for. Anyone can do that anywhere"}. However, when I1 chose which game she liked the most, she chose "Aging with Grace" as her favourite. She thought the other games had too much noise and loud sounds. Other aspects that were reasons for why the informants did not like a game were because the game did not present what they gained from doing the exercises, and it was hard to get points because of significant delay. 

The informants did not see the usefulness of "Fruit Ninja". They laughed a lot while playing, but they thought the game itself was stupid. I3 said \emph{"I think it was stupid. It does not put any requirements on you. I was just waving my arms"}. We did ask the informants if they thought the game was fun, even though it was an unrealistic game. I1 answered \emph{"Yes, I think it was a bit fun. You see all the fruits that smashes. That was lovely"}. Even though some of the informants thought the game was fun, they did not see the point in playing it. One reason was that they did not see a connection between their movements and the outcomes on the screen. \emph{"You did not get any sense of whether you hit something or not"}. This was perceived as confusing as they did not understand how to do things the right way. 

The informants said that if they were just going to do basic exercises, like in "Aging with Grace", they would rather do this without a game. It was mentioned several times that they would rather go to their local training centres and attend group sessions there. To want to use a game for exercising, it would need to contain a sport, or something else related to real life. \emph{"It would have been better to chop wood"}, I6 said after playing "Fruit Ninja". Other possible game themes mentioned by the informants were swimming, rowing, picking apples, biathlon, interval exercises, dance and a walk in the nature. \emph{"You mentioned apples. When playing together, you could pick apples and summarise the number. That might be a competition"}, I4 said. The same informant also suggested doing puzzle games on the map of Europe. In that way the player could learn geography at the same time which would make the game more meaningful.

\section{Motivation and Mastery}

\emph{"Motivation is extremely important"}, was I5's opinion about playing games for exercising. Motivation was one aspect the informants mentioned as very important for an exergame. They told us that for elderly to use a video game for exercising, it has to possess features that will make them wanting to play. The informants stated goals and socialisation as motivational aspects for a game. Goals could be either achieving a high score, or just moving their body to music. \emph{"Moving to a rhythm, that is always positive."}, I6 said. Social aspects of gaming were expressed as important. I4 said that she felt that meeting others for exercising was motivating. Another aspect that was considered as motivating was to get information about why they should play the game. The informants stated that if they could get to know the training benefits from doing the different activities, it would be more fun and motivating doing them. \emph{"We should get information about which body parts we are exercising, and why we should do it. [...] I think this information is very important for people as grown up as we are. We have to know why we should do this"}. 

One interesting observation we did during the gaming session was that the features in the games that were supposed to be motivating, were perceived as the opposite. Cheering, loud music, encouraging comments, fans and high scores were perceived as noisy and annoying, and not motivating by some, and were not noticed at all by others. \emph{"I became a bit irritated when it was a cute voice saying "yeah, that is great!", "hurrey!". I think it was stupid, I have to admit"}, said I1, while I3 said \emph{"I did not notice it at all"}. 

\emph{"Motivation and mastery are very important!"}. Mastery was also mentioned by the informants as a motivational factor. I6 said \emph{"It is all about the experience of mastery, which is essential. And the older people get, the more important it is"}. I6 emphasised how important the feeling of mastery is for elderly, in particular, because they would not continue doing something they do not master. However, the desire to master could lead to wanting to play the game over again. This is shown by I1's comment: \emph{"I want to try it one more time. To be able to master it"}. When the informants first started playing the games, they were insecure and did not know how to perform the different activities and exercises. They completed the activities, but without the feeling of mastering it. This was the case with "Aging with grace", where the movements where quite fast. In addition, some of the movement were for many difficult to perform correctly. The informants also experienced similar problems with the tennis game. The ball came quickly, and it was difficult to coordinate the hand fast enough. However, the informants did not see it as a problem that they did not manage it the first time, as they believed they would get better after playing several times. \emph{"Nothing of this is difficult. It is just a matter of training"}, I3 said. I5 replied to this with \emph{"Sure, but it is possible to make it less confusing"}.  

It seemed important for the informants to see a progress in the game, and to feel that they are learning something. \emph{"I believe in games that have the characteristic of play, where you also can see that you get better"}, I2 said. This will provide the players with something useful, that will make them want to play again.

While observing the informants interact with the technology, play the games, and walking through some of the menus, we saw that the informants mastered the different challenges better and better. They learned how to connect with the sensor and they responded faster to feedback and information. This was both our observations and the informants own perception. When we asked them if they felt it became easier as they played, the majority of the informants answered \emph{"Yes"}. One informant stated that when playing tennis she felt she was helped a lot by the game to hit the ball, but that this made the feeling of mastery come fast, and was therefore positive for the gaming experience. Some of the informants stated that they learned how they were suppose to play the game by looking at the other informants play, and that it therefore might have become easier for them to play. 

As a part of mastery, it was also mentioned that we need to consider all the different groups of elderly to make suitable tasks for everyone to master. One of the informants mentioned different groups to keep in mind: "the people who already have decided to keep their body in shape", "people who wants to know if they are doing the exercises right", and "the people who are already inactive". For the latter group she mentioned the importance of getting help, e.g. from grandchildren.  Another group mentioned are the people who can not stand or walk, who might be sitting in a wheelchair.

\section{Immersion and Engagement}
The informants showed a lot of engagement and different feelings during the gaming session. In most of the games the informants used body movements eagerly. The informants also burst out with various comments like \emph{"yes yes yes yes"}, \emph{"noooooo"} and \emph{"ooh, I wanted that pineapple!"}, while playing. In addition, the informants laughed a lot during game play. A few of the informants only used monotonous movements, like just waving their arms up and down with not that much engagement in "Fruit Ninja".  

Not all the games created feelings of immersion and engagement. "Aging with Grace" was a game that the informants "just did", without showing any particular emotions. There were not many comments during game play, and there were no laughter or engagement. We only observed concentrated faces. However, the informants tried their best, and most of them went through the exercises with controlled and powerful movements, in tempo with good rhythm. In "Fruit Ninja" the informants had a lot of questions, and some of the informant just "waved their hands", without understanding what they were suppose to do. This got in the way for the feeling of immersion for the informants. They seemed most immersed into the skiing game. However, in the second match of skiing the players switched tracks. The informants had troubles understanding which track that was their track and had comments like \emph{"Eh.. I do not understand"} and \emph{"Are we going in the same lane?"}. I5 was confused while playing, \emph{"First red and then blue and then blue and then.. [...] This does not work"}. The fun and enjoyment we observed the first round of skiing were the second round replaced with frustration. Tennis was one of the games the informants expressed they liked the most. However, we did not observe this enjoyment. They did not immerse into the game, as they talked a lot while playing. 

The informants expressed, especially in the tennis and skiing game, that there was a lot going on at the same time. We asked them if they felt it was easy to focus on the challenges, exercises and activities, although there were a lot of hustle. The informants agreed on that, while playing, they were totally focused. I7 said \emph{"When you are in the game, then you do not see the outside world, then you are concentrated"}. The informants felt that they lived into the character of the avatar. They did not feel that the avatar was a person separate from themselves. \emph{"I felt like the avatar itself"}, I3 said. 

\section{Instructions and feedback}
The first time playing the informants seemed confused and unknowing. One informant started pointing on the screen and asked \emph{"How do I point?"}, meaning how she could press the buttons. When playing "Aging with Grace" I1 asked \emph{"Press the button? Is there any button here?"}. It was clear that it was not obvious what was meant by pressing a non-physical button. None of the informants had ever played games like these, and did not understand how the game could be controlled by their body movements, even though we had explained this. They desired an introduction with instructions on how to interact with the game.

It was a general opinion that the different games should spend more time on instructions and explanations regarding the different exercises. \emph{"It needs to be a softer and more instructive introduction into the game's rules"}, I5 said. Especially in "Aging with Grace", this was the case. I1 compared this to the way they did it in an aerobic class she used to attend in her younger days. She explained that they practised the exercises slowly before they could do it fast. "Aging with Grace" started up quickly, with no room for practice or warm up, and it held a high tempo. The informants did not know how to do the required movements, since it was not given sufficient information. Most of the informants got the movements after a few steps, as they recognised the exercises from aerobic classes at their training centres. After playing, most of them agreed that the exercises were OK and that it was only a question of learning. It was also lack of instructions in "Fruit Ninja", which caused confusion. While playing, I7 had troubles understanding how she could see if she hit the fruit. It also seemed that I1 did not understand what actions did cut the fruit in half, she just waved her arms while she looked quite confused. \emph{"It is just to wave your arms. There is no system"}, I3 said, indicating that the game did not require anything from them. In addition, they did not see the relationship between what they did and what they achieved. \emph{"I did not get any feedback on my movements"}, I2 said. We observed that insufficient instruction and feedback in the games resulted in the players not understanding what was required from them. This lead to the informants just "doing something", like in "Fruit Ninja". 

All the informants agreed that there were some positive health effects from playing the games, however, they did not exactly know what they were exercising. They urged that the games should include information about which body parts that are being exercised when different tasks are done. Another general opinion was that they wanted to know if they were doing the exercises right or not when they were playing. They were also eager to receive feedback on their progress. \emph{"I think it is very nice to see how far I have walked, and how fast I have walked, and how much downhill and uphill and things like that"}, I1 said, referring to a mobile application she was using to give her information about her progress when cross-country skiing.   

In the tennis and skiing games there were introduction videos in the beginning of the game. However, not all informants understood that this was instructions and not the game itself. This became clear to us as most of the informants tried to do what the avatars did in the video, and seemed confused when nothing happened. \emph{"Should I try to hit it [the ball]? What am I suppose to do now?"}, I1 asked during the introduction video. It appeared that the second instruction video, that was shown between two matches in tennis, was more obvious, and that they all learned from it. However, when the game was finished, some of the informants did not understand that the game was over. Especially did this apply in the tennis game, where most of the informants ended up just looking at the screen. 

There were various perceptions of the information and feedback given during game play. Some of the informants were not affect by the information given, and at times they did not even recognise its presence. \emph{"I did not react to the text at all. If you had asked me, I would not know it had been text"}. I5, on the other hand, felt that there were too much information at the same time, which made it hard to concentrate. \emph{"I think you should limit the total amount of information"}. I2 said she had seen the text but it disappeared too fast, so she did not have the time to read it all. She suggested that it should be a way to confirm that you have read the information before it disappears. The instructions in both the tennis and skiing game, like "raise hand above head to play" or "move closer to the sensor" were well understood by the majority of the informants. However, we had to assist some of the informants a couple of times by reading the message given on the screen.  

\section{Complicated menus}
A general agreement was that the games had too complicated menus. Moreover, the avatar hand that was navigating on the screen was too sensitive, and it was hard to keep the hand still long enough to actually "press the button". In the sports game, buttons were pressed by holding the hand over them for a certain amount of time. This required time seemed to be too long. \emph{"This is worse than working with the mouse on the computer"}, I6 said. In "Aging with Grace" the avatar hand was, in addition to being too sensitive, obscure and at times almost invisible. It did not really look like a hand. This seemed to cause problems because not all informants understood that the object on the screen was their hand. Two informants suggested that an arrow-marker like the one used on most computers would be more intuitive and easy to use. 

\emph{"The menu was extremely difficult"}, was one comment about the menu in "Aging with Grace". The menu was big, complex and sensitive, which made it difficult to know where to go, as well as to push the right buttons. There was a great amount of information on the screen, which made it difficult to choose the right alternative. Three informants \footnote{The reason for why we did not test this on the rest of the informants, was because we experienced this to be very difficult and time-consuming, making it a bad experience for the informant.} were challenged to go through the whole menu, and they, especially, struggled a lot. To navigate between the pages in the menu, there were arrows on the sides that should be pressed. These arrows were hard to see, and they were very sensitive. When telling the informants to scroll through the menu by pressing the right arrow, I1 said \emph{"Which arrow?"}. In addition, this menu had a huge, sensitive "back"-button that the informants pressed several times without intention. 

Also in "Fruit Ninja" the informants had problems with the menu. The menu was crowded with elements, which made it hard to hit the correct buttons. In addition, the menu made it difficult for the informants to distinguish between the menu and the actual game. This came clear from observing informants waving their arms like if they were playing the game when they were in the menu. 

\section{Graphics and sound}

We asked the informants about their opinions about the music in the games, and I1 answered \emph{"I think everything about that was too much"}. I2 replied with \emph{"I like Mozart better"}. They did not want ordinary pop music and suggested music with swing rhythms to be more appropriate. All the informants agreed that music is important to keep the rhythm when exercising, but that they would prefer the music not being too noisy. It was also a general opinion that the music was inappropriate.  I7 said \emph{"You get sensitive [to sound]. You do not want it. You want it quiet. You want to be active, but without too much background noise"}. I4 added \emph{"I would prefer walking out in the nature. Then you can listen to the birds"}.

It was a general opinion that there were too much elements on the screen during game play. I7 stated this by saying \emph{"It is too much elements on the screen. Where should you look?"}. Some of the informants also felt it was too much going on at the same time during game play, and that it is in general hard to do several things at the same time. I5 suggested that there could be different levels of things that happens in the game.  \emph{"Eventually, when you get better and manage to keep track of things, you can add more elements to the game. A lot of what happens in these games are not relevant"}. 

On the question about what they thought about the avatars and the picture in general, I5 answered \emph{"I think it was very confusing to see myself. Especially, to see myself and the trainer together"}. In addition, most informants agreed that the avatar of themselves in "Aging with Grace" made them look fat, which they did not like. About the other avatars, I2 said it was OK because it was just like in the cartoons, but no further comments were made on this.  

We observed that at least one informant spent some time reading the text in the menus in "Aging with Grace" and in the tennis game. At one point the informant had to move closer to the TV-screen to read the text. This might be an indication of too small text, or poor readability, in these menus. 

\section{Delay between the player and the avatar}

\emph{"It bothered me that I did not see the correlation or the relationship between my own body movements and what was happening on the screen"}, I2 said about the gaming experience. All of the informants complained about the delay that was present in the games. This applied to all the games except from "Fruit Ninja", and it was mentioned as a problem several times. The delay made it hard to move correctly and in time. Several of the informants had problems passing through the gates and to perform a perfect jump in the skiing game because of the delay. \emph{"Missed the jump? I jumped so quickly. That is just nonsenses"}. The delay in the games made it difficult to get points and achieve high scores, which again gave the informants the feeling of not mastering the movements. 

The technical aspects of the games became a source of confusion for most of the informants. \emph{"There were some small technical details that were problematic. [...] I could not see in the skiing game that the avatar followed my movements. [...] And in tennis, it was something about the time aspect between when you tried to hit and when you actually hit the ball. I felt that the game helped me a lot in the beginning"}. The delay made it difficult for the informants to see when they actually hit the ball in the tennis game. Some of the informants expressed that the significant delay interfered with the gaming experience. I5 said that \emph{"I experienced the tennis and skiing game as games with lack of technical perfection. It was not technically good enough to create a real-time experience"}. 

\section{Physical Outcome}

The informants had a general opinion that all movements are good for your body, and that these games required you to move in a good way. They had comments like \emph{"If the game is good for your body? Yes it is. All kind of movements are of the good"}, \emph{"You get warm"}, \emph{"I felt it was useful, in a health-related way"}, and \emph{"It was amazing how much body you actually used on so little"}. However, it was mentioned that if the aim of the game was exercising, it should be built on the whole range of exercising, from warm-up to stretching. 

In tennis and "Fruit Ninja", several of the informants mentioned that their shoulder and arm were in pain after playing. In tennis, the player only uses one arm, which results in an asymmetrical workload.  
    
\section{Playing Together}
A general opinion was that this game could suit well in a nursing home setting, in a small group setting, or as an activity to do with grandchildren. \emph{"I think skiing was fun. But I am thinking it would be even more fun to play with a grandchild in a suitable age"}, I1 said about the social aspects of the different games. She continued with \emph{"I am thinking that in a nursing home we would probably like to play with someone"}. She also told about her mother in law who got inactive on her last years. She believed that the only way to encourage her to become more active by playing games would be if there were some social aspects within the game, like the possibility to play together with a grandchild. Some of the informants were clear in their opinion about whether they would play alone or with others, and I7 stated that \emph{"I do not want to do this alone at home"}. However, I4 liked both and said \emph{"I think both playing alone and together were OK. [...] I believe in exercising in small groups"}. It seemed that most informants liked to play together, however, two informants mentioned that they were indifferent. The informants enjoyed playing together in the games that were made for fun and entertainment, like skiing and tennis, while they felt that games meant for exercising, like "Aging with Grace", would be better to do alone.  

The majority of the informants seemed to like both competing and collaborating. However, I5 had a strong preference about how to play together. She said
\emph{"If there is going to be any point in doing something together, it needs to be that you are enhancing each other. Like that you get a better result if you are cooperating"}.

Everyone agreed that meeting people and being social are important in life, but that they would do it on their own time, and not be locked to a specific group or time. I5 felt that meeting together in a group for exercise would give her a sense of pressure that she did not like. \emph{"[...] I have had pressure all my life. I do not want it anymore. I do not want to put myself in that situation where people come and ask "are you going to join this?""}. I7 agreed that such pressure does not motivate. \emph{"It is like this: well-being is attractive, pressure is not attractive"}, I6 said, referring to the importance of doing things voluntarily. However, just having the opportunity to go out and meet people, like in an arranged training group, seemed to be important for all of the informants. 

\emph{"You are talented"}, and \emph{"You are starting to get it now"}, are comments the "audience" came with when they watched the other informants play. They encouraged each other, and provided positive and motivating messages. The informants that were playing seemed to enjoy this feedback, and it made the gaming session more social and fun.  

We asked the informants if they could imagine themselves playing at home in their own living room, with friends over the Internet. They were not very positive about this, and it seemed like they did not understand what we meant. \emph{"This is very distant for me"}, was I6's comment. \emph{"It would probably be more motivating to just ask her if she wanted to come over and play"}, I1 said. These answers show how being social over the Internet is quite unfamiliar to the informants. To make such a game social, all the informants agreed that meeting in person would be better than playing online. On the other hand, the informant using a training mobile application, seemed to like to share exercise information with friends over the Internet.

\section{Experience and Understanding of Technology}
None of the informants had used technology like Xbox Kinect before, and they all said that it was a completely new experience for them. I3 said \emph{"I have practically never seen it before"}. In fact, the informants said that they never had used any kind of video game technology. \emph{"Waste of time"}, I5 said, \emph{"I could never have time for something like that"}.  

The informants generally showed interest in the technology and our project during the workshop. They asked about the project, where the games were made, and how much the Xbox Kinect costs. The informants showed interest in what equipment needed for them to play at home, and wondered \emph{"Does it connect to a TV?"}. I7 commented that it is important that the technology is easy to install. The informants were also eager to ask questions about the games, where these were related to different aspects, like: the information given, their own performance, difficulty levels and general confusion. Overall they were very positive about the project and I3 complimented our work by saying \emph{"It is admirable what you are working with this, that I have to say"}. 

\section{Summary}
To summarise the findings, we will repeat aspects liked and not liked about the different games. In addition we will in a more concise way present the aspects the informants stated as important for an exergame for them. This is done to highlight the improvements that should be made.

\begin{itemize}
\item \textbf{Kinect Sports: Season Two: Tennis:} Tennis was one of two games they liked the most. This was due to its real-life activity, and because it involves fun and entertainment. It was a small delay, and they were helped a lot when hitting the ball. This was perceived as good by some, and bad by others. The informants managed the game well, but did not seem concentrated on the tasks. There were a lot going on in the tennis game (e.g. fans, commentators etc.). This was perceived as annoying by some and was not noticed at all by others. There were clear instructions, however, the first time playing, the instruction video was confused with the game itself by some. In addition, tennis requires only one arm in activity, which makes asymmetrical exercises. The informants would prefer playing this game together with others. 

\item \textbf{Your Shape Fitness Evolved 2012: Aging with Grace:} This was one of two games the informants liked the least. This was because of its distinct focus on exercising, which they could rather do at training centres.  The time spend on instructions about the exercises was short and it was little time for practice. In addition, the movements were fast.  We observed only concentration, but little enjoyment. The menu was in particular complex, with many buttons and obscure elements. The navigating hand was obscure and sensitive. This game was the one game they would prefer doing alone, if doing it at all. The informants did not like that the avatar was a recreation of themselves. In addition, there was a significant delay, which made the game confusing.

\item \textbf{Fruit Ninja:} This was the other of the two games they liked the least.  They did not understand what was required from them, and the word "stupid" was used. The menu was crowded with fruit elements, which made it hard both to distinguish between the menu and the game, and to actually "slice the button". This was the only game where they did not experience any delay.

\item \textbf{Kinect Sports: Season Two: Skiing:} Skiing was one of two games they liked the most. This was because the informants felt that it actually required something from them, in addition to the real life activity. This was the game where they seemed to immerse the most. However, switching tracks was a source for confusion, as insufficient instruction on which tracks belonged to whom were given. It was also experienced delay in this game. As in tennis, there were a lot going on in the game (e.g. fans, commentators etc.), which was perceived as annoying by some and was not noticed at all by others. There were clear instructions, however, the first time playing the instruction videos were confused with the game itself by some. The game was seen as most fun to play together with others. 
\end{itemize}

\subsection{Important Aspects for the New Exergame}
\begin{itemize}
\item Real life activities. Suggestions include: sports, wood chopping, swimming, rowing, picking apples, biathlon, interval exercises, dance, puzzle games, and a walk in the nature.
\item Instructions on what is expected from the player, why the player should do the task, and the training benefits from doing the tasks.
\item Feedback on whether the movements are done correctly or not.
\item Easier and more intuitive menus with clear objects.
\item A limited amount of information and elements at the same time.
\item Clear navigator.
\item Avoid avatars that renders the player.
\item Include the whole range of exercising, from warm-up to stretching.
\item Motivating factors, like clear goals that are possible to master.
\item See progress and experience a learning curve.
\item Different difficulty levels, where additional functionality can be added at higher levels.
\item The possibility to play with others in person, like with grandchildren, or in a training group.
\item Appropriate music. Examples provided include classical music, like Mozart, and music with swing rhythms.
\item Avoid too much noisy feedback, like loud music, sounds and comments.
\item The possibility to customise too meet different groups of elderly, like adjusting the pace in the game.
\item The system should be easy to set up.
\item Avoid delay between the player and the avatar.
\end{itemize}






