\chapter{Findings from Workshop 1}
In this chapter we will present our findings from workshop 1. The presented findings are based upon feedback and quotes from the informants, and our own observations during the workshop. The seven informants are referred to by using I1 to I7, and we have due requirements to anonymity decided not to distinguish between male and female. Quotes from the informants are translated from Norwegian as literally as possible. 

\section{Perceived Usefulness and Value of Entertainment}
\emph{"Mostly, it was quite amusing"}. This was the general feedback from the informants after the gaming session. When we asked the informants right after they were finished playing if they enjoyed it, the response came quickly and the answer was \emph{"Yes!"}. One of the informants stated that s/he could see that it could be fun and useful for someone, but it was not for him. S/he would not buy it for himself. The informants were divided in their opinions about whether they would buy a game like this or not. Some of them would rather exercise for themselves or go to training centres, while others saw the gaming as more amusing than using a treadmill or an exercise bike. I2 stated that \emph{"No, I do not think I am going to buy it. Then I would rather go to Elixia [training centre]"}, while I5 stated that \emph{"I think this seemed very fun, so I want to buy one like this. I have one of those exercise bikes in my basement, but it is so boring that I can not bear it"}.  Some of the informants stated that one of the reasons why it was fun playing was because it was a completely new experience. I6 said that \emph{"Now I have been involved in something I have never been a part of before. [...] It is a new world that has opened up, that is for sure"}. Other reasons for why they enjoyed playing were that they could imagine playing with their grandchildren, and that it was a fun way to exercise. They liked the idea of combining gaming and activity. I4 said that \emph{"they [elderly] might think it would be fun to do this and be active at the same time"}. I6 explained that s/he felt different from when s/he was watching the other informants play, to when s/he was playing herself. \emph{"It gave more to participate that I had thought. Because, when I sat and watched it felt so unreal to have someone on the screen, but in the activity, when you got in to it, it was not so stupid after all"}. The observations we did during the gaming session supports the feedback the informants gave us. There were a lot of smiling and laughing, and it seemed like they had fun. However, perceived value of entertainment and perceived usefulness does not have to be the same. 

The informants liked some games better than others. The two games from Kinect Sports Season Two, tennis and skiing, were the game that the informants liked the most. They thought it was fun activities and they liked the challenges the game provided. \emph{"I liked it very much. I like that kind of activity"}, I4 said. The informants also liked that the game required something from them. About skiing, I3 said \emph{"This was a bit fun. Yes, it was. Because, here you have to pay attention and spend some effort [...]"}. I3 also liked tennis for the same reasons.  

Your Shape Fitness Evolved 2012, the personal trainer game, and Fruit Ninja were the two games that the informants liked the least. They did not see the need for the personal trainer game, as they rather would do these type of exercises in a training centre or by themselves. I3 said s/he did not like this game because it did not require anything from him. \emph{"The aerobic game I did not care for. That game anyone could do anywhere"}. However, when I1 chose which game s/he liked the most, s/he chose the personal trainer game as her favourite. This was because the other games had too much noise and loud sounds. Other aspects that were reasons for why the the informants did not like game was because it did not present what you gained from doing the exercises, and it was hard to get points because of significant delay. This shows the importance of technical correct games, as well as a meaning or a goal, to motivate people to play. 

When it came to Fruit Ninja, the informants did not see the usefulness of this game. They laughed a lot while playing it, but they thought the game itself was stupid. I3 said \emph{No, I thought it was stupid. It does not put any requirements on you. It was just.. [showing, waving with his hands]"}. We did ask the informants if they thought the game was fun, even though it was a quite unrealistic game. I1 answered that \emph{"Yes, it was a bit fun, I think. You see all the fruit that smashes. That was lovely"}. I1 laughs. I6 stated that s/he did not see the fun at all with the smashed fruit. \emph{"[...] I think that it was just chaos. I think. It becomes much mess on the wall, but it was just chaos"}. I7 was more of the same perception as I1, and answered I6 with \emph{"yes, it was chaos, but at the same time fun!"}. Another reason for why the informants did not care for Fruit Ninja was that they did not see a connection between their movements and the outcomes on the screen. \emph{"And if you hit something or not, that you did not had any sense of. You did not get any feeling of that"}. This was perceived as confusing, and they did not understand how to do things the right way. 

During the gaming session the informants expressed a lot of confusion. There were some situations in all the games that were difficult to understand for all the informants. Even though we tried to explain before they started playing, during game play, and between the informants gaming sessions, they did not understand what the games wanted from them. It was interesting to see that what was difficult for one informant also was difficult for the next informant playing. Also, it was an interesting observation that the situations that were difficult to understand the first workshop day also were a problem for the informants that participated the second day. 

KANSKJE HA DETTE AVSNITTET PÅ SLUTTEN I OPPSUMMERINGEN. 
To summarise our observations, even though the informants experienced some confusion while playing, we saw that they had fun and enjoyed playing these games. However, most of them did not see the games so useful that they would have bought them and used them themselves. They have other ways to exercise that they would rather prefer. The games they liked the most were those involving real-life activities, like tennis and skiing, which combines movement and entertainment. They did not care for the personal trainer game, even though it involved realistic exercises, because it did not give them any value of entertainment. Fruit Ninja was perceived as stupid, and mostly, it gave them nothing.   

\section{Motivation and Mastery}

\emph{"Motivation is extremely important"}, was I5's opinion about playing games for exercising. Motivation was one of two things the informants mentioned as an important aspect for a game like this. They told us that for elderly to use a video game for exercise, it has to possess features that will make them wanting to play. The informants stated goals and socialisation as motivational aspects for a game. Goals were mentioned to be achieving a high score, or it could just be to move your body to music. \emph{"I think that what is important is one's own movement and motivation"}, I6 said. \emph{"And to move after a rhythm, that is always positive. If you do not get it all perfect, it does not have to be that important"}. Social aspects of gaming was mentioned as important for the informants. I4 said that s/he thinks meeting others for exercising is motivating. I5 felt that meeting together in a group for exercise gives him a sense of pressure that s/he does not like. \emph{"Yes.. I do not want to.. I mean, I have had pressure all my life. I do not want it anymore. I do not want to put me in that situation where people come and ask "are you going to join this?""}. I7 agreed that this type of pressure is not motivating. I6 said that \emph{"It is like that, that well-being is attractive, pressure is not attractive anymore"}.  Another aspect that was considered as motivating was to get information about why they should play the game. The informants stated that if they could get to know, e.g. the training benefits, from doing the different activities, it would be more fun and motivating doing them. \emph{"For me it would have been important to know that; okay, now we do the slalom, but what we actually do is training balance, and that we train strength in thighs and stuff like that. For me that would be important for motivation"}, I6 said. \emph{"That we get information about what we exercise when, and why we should do this. [...] I think that it is very important for people as grown up as us. We have to know why we should do this"}. We can see from the informants' feedback that entertainment alone is not motivation enough. They need to know why they should play the game, and how it can be good for them. 

One interesting observation we did during the gaming session was that the features in the games that were supposed to be motivating, were perceived as the opposite. Cheering, loud music, encouraging comments, fans and high scores were perceived as noisy and annoying, and not motivating at all. Some of the informants thought it was too much "going on", and it became too much nagging. The comments that were meant to be encouraging became irritating and stupid instead of motivating as it was intended. \emph{"I became a bit irritated. When it comes a cute voice saying "yeah, that is great!", "hurrey!". I think it is stupid, I have to admit"}.  I1 explained how s/he perceived the encouragement from the game by saying \emph{"I thought it was too much hustle. [..] Especially in the tennis game, you shall have the fans and you shall have the "hurray", and you shall have comments and stuff like that. I do not know how motivation it is, but for me it was not motivating, and I thought it was annoying. I do not think I would bare playing that game because I do not bare the hustle, in a way"}. Some informants did not even notice all this "hustle", that I1 described. \emph{"I did not notice it at all"}, I3 said, and I2 agrees with him and said that \emph{I did not think it was a problem"}. Since these features was either perceived as annoying or not noticed at all, we conclude that we should avoid features like this in our video game concept. We should keep it simple, and focus on a few motivating aspects. 

\emph{"Motivation and mastery. That is very significant"}. The other aspect I5 pointed out as an important part of a video game for elderly is mastery ("mestring"). Mastery plays a huge part for the players feeling of success and accomplishment, which affects their self esteem. I1 commented this after playing tennis: \emph{"I think I do not would do this.. hehe, if I should be completely honest. [...] Maybe if I had felt I had mastered it better"}. Mastery is also mentioned by the informants as a motivational factor. I6 tells us that s/he has been teaching an activity class and that their main goal and focus was to experience mastery. S/he continues with \emph{"It is all about the experience of mastery, which is essential. And the older people get, the more important it is"}. I6 emphasises how important the feeling of mastery is for elderly. It is not fun doing something one is not able to do, and elderly, especially, do not continue doing something they do not master. It is important for us to think about how to achieve experience of mastery for elderly when creating our video game concept. 

Some of the informants said that they felt that the game helped them to perform the various activities. \emph{"I got the feeling that in the beginning, they [the skiing game] helped me a lot"}, I5 said. This was perceived as positive, as it led to a feeling of mastery. I6 described it as \emph{"When we hit the ball here now [tennis], we mastered even though we not really did. Then we had a huge experience of mastery! And that gives, that is positive"}. S/he said that the feeling of mastery the game gave will make everyone wanting to play.  

Mastery, or the desire to master, could lead to wanting to play the game over again. I1 said that \emph{"I think that I want to try it one more time. To master it [...]"}. When the informants first started playing the games, they were quite insecure and did not know how to perform the different activities and exercises. They completed the activities, but without the feeling of mastering it. \emph{"And the feeling of mastery. This we did not master"}. But the informants did not see this as negative at all. They were all positive about it, and their common opinion were that it is all about training. \emph{"Nothing of it is difficult. It is just a matter of training"}, I3 said. 

While observing the informants interact with the technology, play the games, and walking through some of the menus, we saw that the informants mastered the different challenges better and better. They learned how to connect with the sensor, they quickly understood what was required from them in the various games, and they responded faster to feedback and information. This was both our observations and the informants own perception. When we asked them if they felt it was easier the second time they played, most of the informants answered \emph{"yes"}. We observed that it did not take long before the informants mastered exercises and understood how they were suppose to move. They just needed some time to observe and try it out. The informants stated that the feeling of mastery came fast, which was positive for the gaming experience. Some of the informants stated that they learned how they were suppose to play the game and how they should move to interact with the sensor by looking at the other informants play. Some of the informants did not fully understand e.g. Fruit Ninja before they had sat down and got the chance to see the next pair play. The first times information message was shown on the screen, the informations did not respond to it. We had to assist them in what they were suppose to do. After playing for a while, they responded to the information messages without guidance, and they did it faster for each time. This could be messages like "raise hand above head to play", "move closer to the screen" etc. It was fun to watch how fast the informants learned. Even though they felt that they did not master the different tasks at first, they kept on trying, and after short time they had learned a lot and enjoyed the feeling of mastery. 

\section{Immersion and Engagement}
The informants showed a lot of engagement and different feelings during the gaming session. They immersed into to game, and gave it all to complete the different challenges. To be able to interpret and map the informants feelings and engagement, we observed their body language, facial expressions and comments while playing. In most of the games the informants used body movements eagerly. In the skiing game they used their arms to gain high speed, they went down in deep squats, and they moved their body deeply to both sides. At one point, I7 almost hit I6 as s/he was so eager to use her arms to go faster. The informants also burst out with various comments. I1 showed a lot of engagement, and came with words and sounds like \emph{"yes yes yes yes"}, \emph{"noooooo"} and \emph{"wooooeee"} while playing. S/he also, together with most of the other informants, laughed a lot. The informants watched the instruction video before the tennis and skiing game with great attention. When the game told them tips to perform better, like to bend forward, use the arms to go faster, or to take a step forward for hit harder, the informants observed and used this information in the game. I7 went from almost standing still in the tennis game, to going out in a "flyer" position, with her hands straight out in front and one leg straight out back, after s/he got the instructions. I5 got so eager that s/he almost dived into the screen. In the tennis game the informants were good at using the floor. They move forward, backward and to both sides to hit the ball. They use big arm movements, and use the racket in various ways. Also in Fruit Ninja, even though they had problems to understand how to hit the fruit correctly, the informants wave their arms with great power. Most of the informants showed immersion and engagement by using various movements, laughing a lot, and by comments like \emph{"ooh, I wanted that pineapple!"}. I3 laughed while playing and said \emph{"This, this is completely wild, you know"}. A few of the informants ended up using monotonous movements, like just waving their arm up and down, with not that much engagement. 

Not all the games created feelings of immersion and engagement. The personal trainer game was a game that the informants "just did", without showing any particular emotion. There were not many comments during game play, and there were no laughter or engagement. We only observed concentrated faces. If this was because the informants thought it was difficult or boring, we do not know. However, the informants tried their best, and most of them went through the exercises with controlled and powerful movements, in tempo with good rhythm.       

The informants had expressed that they felt, in some of the games, that there was much going on at the same time. We asked them if they felt it was easy to focus on the challenge, exercises and activities, although there were a lot of hustle. The informants agreed on that while playing, they were totally focused. \emph{"Thus, I did not notice. I concentrated on playing [...]"}, I3 said. I1 responds with \emph{"No, I do not either think it was disturbing. It was more irritating when I sat and watched [the other informants play]"}. I7 said \emph{"When you are in the game, then you do not see the outside world, then you are concentrated"}. The informants also stated that they lived into the character of the avatar. They did not feel that the avatar was a person separate from themselves. \emph{"Yes, there I felt like the person [avatar] itself"}, I3 said. 

\section{Functionality and Usability}
\subsection{Introduction and instructions}
\emph{"It needs to be a softer and more instructive introduction in the game's rules"}, was I5's opinion about the information given about the game. It was a common opinion that the different games should spend more time on instructions and explanation on the different exercises, especially in a game like the personal trainer game. I1 compared this to the way they did it in an aerobic course s/he used to attend in her earlier days. S/he explained how they practised the exercise slowly before they could do it fast. \emph{"I thought that personal trainer game was confusing"}. In the personal trainer game it was confusion related to delay on the avatar and the lack of introduction to the exercises. The informants were participating in a workout with three aerobic movements. These movements started up quickly, with no room for practice, and they held a high tempo. The informants did not know how to do these movements, as no information where given. Most of the informants got the movements after a few steps, as they recognised the exercises from aerobic classes in their training centres. Most of the informants agreed on that the exercises were OK and that it was only a question of learning in the personal trainer game. 

As a part of improving the introduction, a thorough explanation of how the player interacts with the game should be given. When the informants played their first game, they seemed confused and unknowing. One informant started pointing on the screen and asked \emph{"How do I point?"}, meaning how s/he could press the buttons. When playing the personal trainer game I1 asked \emph{"Press the button? Is there any button here?"}. It was clear that it was not obvious what was meant by pressing a non-physical button. Another informant was confused when first trying to interact with the game and waved in front of the screen and asked \emph{"Is that mine?"}.  None of the informants had ever played games like these, and did not understand how the game could be controlled by their body movements, even though we had explained this. Therefore, we had to explain and show them how they could navigate their hand on the screen. After explaining this once for each informants, we did not need to explain this any further. This suggest that a thorough introduction video explaining the interaction between the player and the game is important, especially for this user group. 

\subsection{Complicated menus}
There was a common agreement that the games had too complicated menus.  The avatar of the hand that was navigating on the screen was too sensitive, and it was hard to keep the hand still long enough to actually "press the button". In the sports game, where the informants played tennis and skiing, buttons were pressed by holding the hand over the button for a certain amount of time. \emph{"This is worse than working with a mouse this [on the computer]"}, I6 said, when trying to navigate in the menu on the sports game. The time needed to hold on the different buttons seemed to be too long. This especially applies for people with declined strength in their arm(s). One of the informants had some trouble keeping the hand long enough out to the side to pick hands for the racket in the tennis game and needed several trials, as well as help from us.

In the personal trainer game the menu appeared to be very complicated. It was big, complex and sensitive, which made it difficult to know where to go and to push the right buttons. There were too much information on the screen, and even though we instructed them on what to choose, the three first informants had a lot of problems choosing the right alternative. To get to the right alternative, they had to scroll to the next page by pressing and arrow on the left or right side of the screen. These arrows were hard to see and they were very sensitive. When telling the informants to scroll through the menu by pressing the right arrow, I1 said \emph{"Which arrow?"}. Because of the great amount of information on every menu site, the informants kept pressing the wrong button several times. In addition, this menu had a huge "back"-button, which made the informants press this button without intention. After pushing the wrong button several times, I3 got really frustrated and asked \emph{"but, what is it that I shall have, exactly?"}. 

The avatar hand in the personal trainer game was, in addition to being to sensitive, very unclear and at times almost invisible. It did not really look like a hand. This seemed to cause some problems because not all informants understood that the object on the screen was their hand. Two informants suggested that an arrow-marker like the one used on most computers would be more intuitive and easy to use. I1 added \emph{"You could just walked up and just touched the screen, hehe. I just wanted to go over there and touch it"}, referring to today's technology where almost every device has a touch screen.    

All of the informants participating the first workshop day had problems with the personal trainer menu, and it was clear that this was a bad experience. Because of this, we decided to start further in the menu the second workshop day. This was to not spend too much time on the menu, and also to avoid the bad experience for the informants. Therefore, the last group managed the few steps in the menu they were allocated very well. We found it strange that this menu was so complicated, as it actually was aimed also for the older user group with their game "sprek alderdom", or "fit old aged" in English. 

When playing Fruit Ninja the informants also had trouble understanding the menu. The menu was very crowded with elements, which made it hard to hit the correct buttons. We started up this game for the informants, as we knew how difficult the menu is, but also we had problems. We hit the wrong menu elements several times  before we finally made it into multi player mode. Another problem with the menu was that the informants had difficulties distinguish between the menu and the actual game. I6 and I7, that were going to play together, stood in the background and waved their arms towards the screen while we tried to go through the menu. It was not before I7 saw the other informants play that s/he understood that difference. 

\subsection{Information and feedback}
The information and feedback given during the play, did not seem to affect the informants. Some of them did not even recognise it was there. \emph{"I do not think I read that much text"}, said I1. \emph{"I did not react to the text at all. If you had asked me, I would not know it had been text"}, I3 said, while I4 said \emph{"It [the text] did not need to be there [in the game]. It could as well be gone"}. I2 on the other hand said s/he had seen the text but it disappeared too fast, so s/he did not have the time to read it all. I2 suggested that it should be a way to confirm that you have read the information before it disappears. However, all of the informants said they had seen the instructions and the information in the menu. 

Most of the informants did not always understand when the game was over. Especially did this apply for the tennis game, where most of the informants were just standing looking into the screen after the game was over.  The instructions in both the tennis and skiing game, like "raise hand above head to play", was well understood by almost all informants. Although we had to assist some by reading the message, we believe that these messages should have been clear enough. A possible reason for why the informants did not see the message might be because they were not concentrated on their task [IKKE HELT klart hvilken task det er snakk om]. It was not obvious for all informants that they had the racket in one hand, and the tennis ball in the other, so we had to explain this a couple of times. 

It appeared that it might have been too small text in the menu in the personal trainer game, after observing that at least one informant spent some time reading the text. At one point the informant had to move closer to the TV-screen to read the text. This informant had the same problem when navigating through the menu in the tennis game. Because it was only one informant that expressed to have problem with this, we assume that this informant might have had impaired vision. However, it is important to acknowledge that impaired vision is a common problem for the older population (kilde kanskje?), and it should therefore be considered in a game designed for them. 

Two informants mentioned that it was hard to do several things at the same time, and that it therefore was difficult to concentrate about both the oral and the textual information. It also appeared that some of the informants had not seen all the messages shown on the screen during the play, and I5 thought it was too much. \emph{"No, limit the total amount of information, I think"}. 

Even though most of the informants agreed that it was too much unnecessary information, it seemed that it was desirable with feedback on what they were doing. \emph{"I am sitting here thinking about the program "Puls" on TV. There there are many exercises I wish I knew how to be done and how to be done right"}, I3 said. Some of the other informants agreed. \emph{"I think it is very nice to see how far I have walked, and how fast I have walked, and how much downhill and uphill and things like that"}, I1 said, when s/he was telling about a mobile application s/he was using to give her information about her progress when cross-country skiing. One of the other informant knew which application s/he was talking about, and seemed to agree. This implies that feedback on their progress is important for some of them.  

(The messages given by the commentator seem to entertain the informants watching the game being played more than the player him/herself. )

Not all informants understood that there were introduction videos and instructions in the beginning of the game. This became clear to us as most of the informants tried to do what the avatars did in the video, and seemed confused when nothing happened. \emph{"Should I try to hit it [the ball]? What am I suppose to do now?"}, I1 asked during the introduction video. However, it appeared that the second instruction video that was shown between two matches in tennis, was understood as an introduction video, and that they all learned from it. A situation related to this was when there was shown a replay video of the last set ball in tennis. Some of the informants did not understand at once that this was not a part of the actual game. One of the informants tried to stretch and bend to reach the ball, but s/he did not understand that it was just a replay video. 

\subsection{Graphics and sound}
\emph{"It was much hubbub about it, right. "Are you leaving now?"(imitating the voice of the lady in the personal trainer game). I got the feeling of when you are listening to the GPS in the car "take a u-turn", hehe"}, I1 said about what s/he felt about everything happening around the game. One informant meant strongly that it was too much fuzz around the game, and s/he was uncertain how motivating this is for people playing, and knew for sure it did not motivate her. However, these aspects were mostly irritating when watching other informants play, and not so apparent when playing themselves. 

On the question about what they thought about the music I1 answered: \emph{"I think everything about that was too much"}, while I2 answered \emph{"I like Mozart better"}. Other types of music that was mentioned as more suitable than ordinary pop music was swing. All informants agreed that music is important to keep the rhythm when exercising, but they would prefer the music not being too noisy. It was a general opinion that the music was inappropriate.  I5 said \emph{"Yes, the sound was a bit irritating. Tiring"}, while I7 says \emph{"You get sensitive [to sound]. You do not want it. You want it quite. You would want to be active, but without too much background noise"}. I4 added \emph{"[...] I would prefer walking out in the nature. Then you can listen to the birds [...]"}.

\emph{"I think it was very confusing to see myself. Especially to see both of us [himself and the trainer] [...]"}, I5 answered on the question about what they thought about the avatars and the picture in general. Except from that, there were not many comments about the avatars. However, most informants agreed that the avatar of themselves in the personal trainer game mad them look fat, which they did not like. 

\emph{"It is too much elements on the screen. Where should you look?"}, said I7 when watching I5 play. It was a general opinion that there were too much information, especially in the menu in the personal trainer game. Some of the informants also felt it was too much happening in the games at the same time and I5 suggested that there could be different levels of things that happens in the game. \emph{"[...] Eventually when you get better and manage to keep track of more things, you can add more things [to the game] that happens. A lot of what happens in these games are not relevant"}. From our observations and the feedback from the informants it seemed like most of them saw the objects presented relevant to the game, and that the only visual problem was the text that was not seen by all.

\subsection{The nature of the game}
\emph{"It bothered me that I did not see the correlation or the relationship between my own body movements and what was happening on the screen"}, I2 answered on our question about how they experienced the game. I1 agreed: \emph{"There are some delays, yes"}. All of the informants complained about the delay in the different games. This especially applied for the tennis, personal trainer and skiing game, and it was mentioned as a problem several times by the informants. In the skiing game, the delay made it hard to move correctly to pass through the gates, and that it was difficult to coordinate movements to perform a perfect jump. \emph{"Missed the jump? I jumped so quickly. That is just nonsenses"}, I1 said. 

I5 complained about the technological aspects of the games. \emph{"There was some small technical details that were problematic. [...] I could not see in the skiing game that anything special happened with me. [...] And in tennis, it was something about the time aspect, with when you tried to hit and when you actually hit [the ball]. I got a feeling that they [the game] helped me a lot in the beginning"}. This became a source for confusion for some of the informants when trying to serve in the tennis. The problem occurred when they tried to coordinate the hand with the racket and the hand with the ball. I6 tried to serve several times, and the first times s/he missed the ball. After some tries s/he was able to send the ball over to the opponent, but s/he did not hit the ball even near to perfect. \emph{"I had never hit that if it was for real}", I6 said. The delay made it difficult for the informants to see when they actually hit the ball. I6 was confused and asked several times \emph{"Did I hit it? Did I hit it?"}. I5 continued discussing the technical aspect with: \emph{"I experienced the tennis and skiing game as games with lack of technical perfection. It was not good enough technically to make a real time experience"}. I5 mentioned several times that when you get use to the delay, it might be more fun, and it looked like I5 accepted that there is a delay. 

Most of the informants had a hard time coordinating their moves in the personal trainer game, as the movements were quite fast. This applied especially for the men. I5 was one that had problems with the movements. S/he looked concentrated and confused during the all the movements. In addition, s/he moved more complicated than necessary. \emph{"No, this was difficult"}, I5 said. The last movement in the exercise was a so called "tap back", where the player was suppose to alternately tap on foot behind the other. This was the movement that was most difficult to do correctly. \emph{"The last movement I was bothered by. It was a bit unnatural for me [...]"}, I2 said. I6 commented that it was all about training, and that they would master it when they get familiar with the game. I5 answers to this with \emph{"Sure, but it is possible to make it less confusing"}. \emph{"[...] I think I could do this [the exercises] better without that [the game]"}, I1 said after playing the personal trainer game. 

There were also some problems related to coordination of movements in the tennis game. One situation was when the ball came toward the player's avatar on the opposite side of the body of where they held their racket. It was difficult to coordinate the hand fast enough to the other side of the body, and they missed that ball several times. One of the informants tried to hit the ball with his left hand, even though s/he had the racket in his right hand. Another problem was, as mentioned, to coordinate ball and racket when they were about to serve. 

In addition to the fast movements in the personal trainer game there was also a significant delay between the players actual movement and the avatar on the screen. This was a source to confusion. It also gave the informants the feeling of not mastering the movements, as the delay made it hard to perform the movements correctly, and in time. This made it difficult to get points and achieve high scores. \emph{"I think my movements on the screen look too slow. Are they that slow?"}, I6 asked while playing. We answered that s/he was in time with the trainer on the screen, but that her avatar was late. \emph{"That is confusing!"}, I5 said. I6 agreed, \emph{"No, that was no good to look at"}. On the question on whether the delay destroyed the playing experience, I5 answered \emph{"I think so"}.

\emph{"The game that required something from you was the skiing game. The rest of the games did not require you to follow the instructor apparently"}, I3 answered after being asked how they experienced the game. It was apparent that some of the informants did not understand what they were meant to do in the games and that they just "did something". All of the informants agreed that Fruit Ninja was a bit silly and that it did not require anything except from waving hands. In addition, they did not see the relationship between what they did and what they achieved. \emph{"[...] I did not get any feedback on my movements"}, I2 said. Fruit Ninja, unlike the other games, did not show a clear avatar, which could have made it difficult for the informants to see a connection between their own movements and the actions on the screen. The avatar was hard to see as it was shown as a diffuse shadow on the wall behind all the fruit. I7 told us that s/he has trouble understanding how s/he could see where s/he hit. When I1 played Fruit Ninja it seemed like s/he did not understand what actions that did cut the fruit in half, s/he just stood there and waved her arms while s/he looked quite confused. \emph{"But, I do not know when I hit.. [...] Is it possible to positioning my arms?"} I1 said. S/he laughed and continued with \emph{"I can not say that I found a way [to do it right].."}. I3 had the same opinion, \emph{"this is just to wave your arms. There is no system"}. It seemed that all informants found it important to get feedback on their movement, see results, and experience a learning effect. 

Some informants mentioned that the skiing game was very fast. At the same time there was a comment about the importance of having the fast pace, to make the game exciting and fun. I1 thought it would be too fast for older adults that lived in a nursing home, referring to her own mother. There was one situation that seemed to cause confusion and frustration for most informants when playing multi player in the skiing game. There were two matches and in the second match, the players switched tracks. The informants were not given any information about this from the game, and even though we had tried to explain it for them in beforehand and during the game play, it seemed like it was hard to understand this change. The informants kept asking which track was their. While the informants were skiing they stood up, and said things like \emph{"eh.. I do not understand"}, \emph{"are we going in the same lane?"} and \emph{"oh no, now I have ended up in the wrong lane"}. Most of the informants tried to lean hard to one side to try to "ski back" in what the thought was the right lane. I5 was confused while playing, \emph{"First red and then blue and then blue and then.. [...] This does not work"}. \emph{"I did not understand anything"}, said I2 after playing the ski game. Our observation was that the change of lanes was very difficult for all of the informants to understand and it led to a lot of confusion. The fun and enjoyment we observed the first round of skiing was the second round replaced with frustration. 

\section{Physical Outcome}

It was a general opinion that these games required you to move and that all movements are good for your body. I3 said \emph{"If it [the game] do good for your body? Yes it does. All kind of movements are of the good"}, while I4 said  \emph{"I liked it very much. I like this kind of activity. I felt it was useful, health wise(helsemessig)"}, while I7 said \emph{"You get hot"} after playing the personal trainer game. I5 was amazed by how much body s/he actually moved: \emph{"It was amazing how much body you actually used on so little"}. There were several positive comments about the games. All the informants could see that there were some kind of positive health effects, but they did not know exactly what they were exercising. They urged that the game should include information about which body parts you are exercising when you are doing the different exercises. Another general opinion was that they wanted to know if they were doing the exercises right or not when they were playing. 

From what we observed and from what was mentioned by the informants, it is clear that it is important to have a structured program where you equally, and symmetrically exercise similar body parts. This came clear from for example tennis and Fruit Ninja. Several of the informants mentioned that their shoulder and arm were in pain after playing tennis and Fruit Ninja. In tennis, the player is only playing with one hand, and the workload gets asymmetrical. In Fruit Ninja there is no structure, and all of the informants just waved uncontrollably with their arms, which might have been too strong movements.
    
\section{Social Interaction}
\subsection{Playing together}
\emph{"I think the slalom was fun. But I am thinking more if I had a grandchild in a suitable age, and I could say "should you and grandma play the ski-game together? Just for fun?""}. -I1

\emph{"I am thinking that in a nursing home we would probably like to play with someone"} -I1

\emph{"[...] I do not want to do this alone at home"} -I7

\emph{"I think both [playing alone and together] was ok. This was all new for me. It was a fun and nice experience in many ways. [...] I believe in  exercising in small groups"} -I4

A general opinion was that this game could suit well in a nursing home setting, in a small group setting, or as an activity to do with grandchildren. It seemed that most informants liked to play together, however some informants mentioned that they were indifferent. A common opinion was that it would be fun to play together in the games that were made for fun and entertainment, like skiing and tennis, while games meant for exercising, like the personal trainer game, would be better to do alone. One informant told about her mother in law that got inactive on her last years. S/he believed that the only way to encourage her to become more active by playing games would be if there were some social aspects within the game. S/he was sure that her mother in law never would have switched on the TV and started playing this game herself, but by for example the help from grandchildren this could be more likely.

Everyone agreed that being social and meeting people is important in life, but that they would do it on their own time, and not be locked to a specific group or time. However, just having the opportunity to go out and meet people for example in an arranged training group would be nice. 

\emph{"It is like this: welfare is appealing, pressure is not appealing anymore"}, said I6 referring to the importance of doing things voluntarily. 

\subsection{Watching others play}
\emph{"You are talented"}, \emph{"You are starting to get it now"}, and \emph{"You have already played this game"} are all examples of comments given by the "audience" while watching their co-informants play. They encouraged each other when playing and gave positive and motivating messages. The informants that were playing, seemed to enjoy the feedback they got from their co-informants. This seemed to make the play session more social and fun.  

\subsection{How to play together}
\emph{"[...] If there is going to be any point in doing something together, it needs to be that you are enhancing each other. Like that you get a better result if you are cooperating. [...]"} -I5

One informant had a strong meaning about how to play together, and meant that it was more motivating to cooperate rather than compete. The other informants seemed to be indifferent whether they were competing or cooperating.

On the question on if they could see themselves play the game in their living room, and compete with friends over the internet, they were not very positive.

\emph{"This is very distant for me"}, was I6's comment.

\emph{"It would probably be more motivating to just ask him if s/he wanted to come over and play"}, I1 said.

These answers show how being social over the internet is quite unfamiliar to the informants. On the other hand, one informants told about an application s/he is using to get information about her progress when s/he is out cross-country skiing. This information could be shared with friends, and s/he did seem to like that. However, to make a game like this social, they all agreed that meeting in person would be better.

\section{Experience and Understanding of Technology}
We asked the informants if they had used a technology like Xbox Kinect before, and they all answered that it was a completely new experience for them. I3 said \emph{"I have practically never seen it before"}. When we continues with questions about the use of other sorts of video games, the informants said that they never have used anything like that. I3 said that \emph{"No, the TV have just been used to look at, [laughs], and not do anything"}. \emph{"Waste of time"}, I5 said, \emph{"I could never have time for something like that"}.  Before playing we described the technology. We showed them the Xbox and the Kinect sensor, and we explain how the sensor captures their movements. The informants thought it was a bit funny to see themselves on the screen, especially in the personal trainer game where personal features like body shape, face and clothes were shown. I1 pointed at the screen and said \emph{"That is me!"}. 

The informants generally showed interest in the technology and our project during the workshop. They asked about the EU-project, where the games were made, and how much the Xbox Kinect costs. We told them that the Xbox, with the Kinect sensor and two games, costs around 2 000 NOK. The informants seemed surprised about the low price. I5 asked question about the sales process and development of new games, \emph{"how long does it take before you get general programs, like, without the need for a Xbox?"}. The informants also showed interest in what equipment needed for them to play at home. \emph{"Does it connect to a TV?"}. I7 commented that it is important that the technology is easy to install. S/he complimented us for being able to install the technology all by ourselves. \emph{"I admire you. Do you know how few.. I our days only men have worked with stuff like that. You have connected all this together, how do you do it?"}.  The informants were also eager to ask questions about the games when there was something they wondered about or did not understand. The questions were related to information given, their own performance, difficulty levels and confusion. When I6 was about to play the skiing game s/he asked the informants that played before her \emph{"Do you get the feeling that it fluctuates, or is it just straight forward?"}. While looking at two informants playing Fruit Ninja I7 had questions about what was allowed and how the point ranking worked. \emph{"Can you steal others fruit when you play? [...] What do you lose points for?"} 

During the group conversation there were questions about our work. They wanted to know how we would proceed from now, and what we imagined as our result. I1 asked \emph{"What do you mean by a concept? What is it that you are going to.. Are you going to create a game?"}. S/he also asked the facilitators that joined us the first day about their role in this project. When we were about to finish the group conversation, I6 asked \emph{"And I think.. What we are coming with, is it usable? Can you use if for something?"}. We answer that their feedback through the workshop is highly valuable, and that it will be used as a basis for the video game concept in our master thesis. I3 complimented our work with \emph{"it is price worthy what you are working with, that I have to say"}, which meant a lot for us. It is good to know that the informants, representatives for the group of elderly, feels that what we do is something useful.  

\section{Aspects to Consider About an Exergame}
\subsection{What the game should contain}

It was a general opinion that if they were just going to do a basic exercises, like in the personal trainer game, they would rather do this without a game, and that if they would use a game for exercising, it would need to contain a sport or something meaningful. 

"It would have been better to chop wood", said I6 after playing FruitNinja. 

It seemed like the informants wanted games with a meaningful content and realistic activities, like chopping woods, and gather apples. 

I6 suggested the following: "You mentioned apples. It might have worked to get it synchronized so that when you gathered apples and put it down, you could summarize the number. That might have been a competition" 

Other game themes mentioned were swimming, rowing, biathlon, interval exercises and a walk in the nature. One informant also suggested doing puzzle games on for example the map of Europe. In that way you could learn geography at the same time which would make it more meaningful.

"I think that games that have the characteristic of play and that you can see that you get better" -I2

It also seems to be important that the to see a progress in the game, and that you are learning something 

"[...] Motivation and mastering! These become very significant" , said I5. S/he also adds:

"It needs to be a relationship between event and result".

I5s comments also show the importance of a meaningful game that gives the players something back. A common agreement was that you need to see a result after doing an event, and that you should also have the possibility to learn something and get better. 

There was a common agreement that it is important with instructions and explanations on what the player is suppose to do in the game. This instructions should be thorough and it should be clear what is expected from the player. They also wished for a chance to try out the exercises slowly first. In addition it was mentioned that if the aim of the game i exercising, they need the whole range of exercising, from heating (oppvarming) to stretching. 

dette med å legge til flere ting etterhvert som man blir bedre. tror det var i5 som snakket om det..

\subsection{Different user groups}
One of the informant stress the importance of remembering the different target groups, and that it will be impossible to make one concept that suits everyone. S/he mentions different groups to keep in mind: "the people who already have decided to keep their body in shape", "people who wants to know if they are doing the exercises right", and "the people who are already inactive". For the latter group s/he mentions the importance of getting help, from for example grandchildren.  Another group mentioned are the people who can not stand or walk, who might be sitting in a wheelchair. 

\subsection{In what setting could this game be suitable}
Several of the informants could see a game like this as a fun activity to do with their grandchildren. One informant talked about her mother in law, who got total inactive her last years. S/he tried to imagine her mother in law using a game like this at home, and was sceptical. S/he added that the only way s/he could see her using a game like this, was with a grandchild or great-grandchild. This shows that it might be hard to reach the people that are already in bad physical condition and inactive, and that the focus should be put on preventing people ending up inactive. 

As all of the informants agreed that it could be relevant to play this game with their grandchildren, it is clear that the social parts of playing games are important. 

Other relevant arenas for the game mentioned were small groups and in the nursing home. However, they were not sure how people living in nursing home would manage the games. One of the informants told about her mother who in her last months before s/he died lived in a nursing home where they tried to activate their residents, and that maybe a game like this could fit as one of these activities. However, s/he was clear that a  game like for example the ski game would be too fast for her mother to do. 

"[...] Maybe if you have been skiing your whole life you would think it was fun to use the game when you are living in a nursing home. I don't know", said I1 showing that s/he was not sure about if the game could replace “real” activity. 

Only one of the seven informants could see himself use a game like this. Most of the informants were in such good physical shape that they were still doing other types of activity. However, they did agree that the game could be an alternative in the future, when they might not be able to do the same activities they are doing today. 

"[...] If I can't go out skiing anymore, for example, maybe", said I1 about seeing herself use a game like this. 

“This can be a way to shorten the long waiting lists in the hospitals, by keeping people healthy, instead of using the hospitals for every little thing”, said I3 showing his positive attitude about the exergame. 

The informants were positive about the game and believed that the game could work well as an alternative method for exercising. However, they agreed that this might be more relevant for our generation (20 years +). 

\subsection{Would rather do ordinary exercising}
“This [the exercises in the personal trainer game] is what we do at SATS [a physical health center in Norway] too. It is much more fun, I think, to be in a room with others, and to get the same instructions. However, of course, if you live far away and can't get there, this [the game] could be a replacement. I does not give the same, but could be a replacement”, said I6 about the personal trainer game. Another reason to use the game as an alternative training method was the time spent on exercising at the gym.

“I do not think I will buy it [the game]. I would rather go to Elexia [a physical health center in Norway]“, said I2.

It was clear that as long as they had good health and were in good physical shape, they would rather keep on doing their regular activities and that they would never replace these activities with a game. 

(Some of the informants also mentioned that if they knew what kind of exercises to do, they would rather do them in their living room without the game. 
\emph{"I would rather do a training program on TV for women, shown in the morning for half an hour"}, was I7s opinion. <-trenger kanskje ikke være med)


