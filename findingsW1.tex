\chapter{Findings from Workshop 1}
In this chapter we will present our findings from workshop 1. The presented findings are based upon feedback and quotes from the informants, and our own observations during the workshop. The seven informants are referred to by using I1 to I7, and we have due to requirements to anonymity decided not to distinguish between male and female. Therefore, refer to all the informants as females. Quotes from the informants are translated from Norwegian as literally as possible. 

\section{Perceived Usefulness and Value of Entertainment}
\emph{"Mostly, it was quite amusing"}. This was the general feedback from the informants after the gaming session. The informants were divided in their opinions about whether they would buy a game like this or not. Some of them would rather exercise for themselves or go to training centres, while others saw the gaming as more amusing than using a treadmill or an exercise bike. I5 stated that \emph{"I think this seemed very fun, so I want to buy one like this. I have one of those exercise bikes in my basement, but it is so boring that I can not bear it"}.  Some of the informants stated that one of the reasons why it was fun playing was because it was a completely new experience. I6 said that \emph{"Now I have been involved in something I have never been a part of before. [...] It is a new world that has opened up, that is for sure"}. Other reasons for why they enjoyed playing were that they could imagine playing with their grandchildren, and that it was a fun way to exercise. They liked the idea of combining gaming and activity. I4 said that \emph{"they [elderly] might think it would be fun to do this and be active at the same time"}. I6 explained that she felt different from when she was watching the other informants play, to when she was playing herself. \emph{"It gave more to participate that I had thought. Because, when I sat and watched it felt so unreal to have someone on the screen, but in the activity, when you got in to it, it was not so stupid after all"}. The observations we did during the gaming session supports the feedback the informants gave us. There were a lot of smiling and laughing, and it seemed like they had fun. However, perceived value of entertainment and perceived usefulness do not have to be the same. 

The informants liked some games better than others. The two games from Kinect Sports Season Two, tennis and skiing, were the game that the informants liked the most. They thought the activities were fun and they liked the challenges the game provided. \emph{"I liked it very much. I like that kind of activity"}, I4 said (om hva da?). The informants also liked that the game required something from them. About skiing, I3 said \emph{"This was a bit fun. Yes, it was. Because, here you have to pay attention and spend some effort [...]"}. I3 also liked tennis for the same reasons.  

Your Shape Fitness Evolved 2012, the personal trainer game, and Fruit Ninja were the two games that the informants liked the least. They did not see the need for the personal trainer game, as they rather would do these type of exercises in a training centre or by themselves. I3 said she did not like this game because it did not require anything from her. \emph{"The aerobic game I did not care for. That game anyone could do anywhere"}. However, when I1 chose which game she liked the most, she chose the personal trainer game as her favourite. This was because the other games had too much noise and loud sounds. Other aspects that were reasons for why the the informants did not like a game were because the game did not present what you gained from doing the exercises, and it was hard to get points because of significant delay. This shows the importance of technical correct games, as well as a meaning or a goal, to motivate people to play. 

When it came to Fruit Ninja, the informants did not see the usefulness of this game. They laughed a lot while playing it, but they thought the game itself was stupid. I3 said \emph{No, I thought it was stupid. It does not put any requirements on you. It was just.. [waving with his hands]"}. We did ask the informants if they thought the game was fun, even though it was a quite unrealistic game. I1 answered that \emph{"Yes, it was a bit fun, I think. You see all the fruit that smashes. That was lovely"}. Even though some of the informants thought the game was fun, they did not see the point in playing it. On reason was that they did not see a connection between their movements and the outcomes on the screen. \emph{"And if you hit something or not, that you did not had any sense of. You did not get any feeling of that"}. This was perceived as confusing, and they did not understand how to do things the right way. 

\section{Motivation and Mastery}

\emph{"Motivation is extremely important"}, was I5's opinion about playing games for exercising. Motivation was one of two things the informants mentioned as an important aspect for an exercise game. They told us that for elderly to use a video game for exercise, it has to possess features that will make them wanting to play. The informants stated goals and socialisation as motivational aspects for a game. Goals could be either achieving a high score, or just to move your body to music. \emph{"I think that what is important is one's own movement and motivation"}, I6 said. \emph{"And to move after a rhythm, that is always positive."}. Social aspects of gaming was mentioned as important for the informants. I4 said that she thinks meeting others for exercising is motivating.   Another aspect that was considered as motivating was to get information about why they should play the game. The informants stated that if they could get to know, e.g. the training benefits, from doing the different activities, it would be more fun and motivating doing them. \emph{"That we get information about what we exercise when, and why we should do this. [...] I think that it is very important for people as grown up as us. We have to know why we should do this"}. We can see from the informants' feedback that entertainment alone is not motivation enough. They need to know why they should play the game, and how it can be good for them. 

One interesting observation we did during the gaming session was that the features in the games that were supposed to be motivating, were perceived as the opposite. Cheering, loud music, encouraging comments, fans and high scores were perceived as noisy and annoying, and not motivating at all. \emph{"I became a bit irritated. When it comes a cute voice saying "yeah, that is great!", "hurrey!". I think it is stupid, I have to admit"}. Some informants did not even notice all this "hustle" described. \emph{"I did not notice it at all"}, I3 said. Since these features was either perceived as annoying or not noticed at all, we conclude that we should avoid features like this in our video game concept. We should keep it simple, and focus on a few motivating aspects. 

\emph{"Motivation and mastery. That is very significant"}. Mastery plays a huge part for the players feeling of success and accomplishment, which affects their self esteem. Mastery is also mentioned by the informants as a motivational factor. I6 told us that she had been teaching an activity class and that their main goal and focus was to experience mastery. She continues with \emph{"It is all about the experience of mastery, which is essential. And the older people get, the more important it is"}. I6 emphasised how important the feeling of mastery is for elderly, in particular, because they would not continue doing something they do not master. It is important for us to think about how to achieve experience of mastery for elderly when creating our video game concept. 
  
I6 presented how the feeling of mastering the game will make everyone want to play. In addition to this, the desire to master could lead to wanting to play the game over again. I1 said that \emph{"I think that I want to try it one more time. To master it [...]"}. When the informants first started playing the games, they were quite insecure and did not know how to perform the different activities and exercises. They completed the activities, but without the feeling of mastering it. But the informants did not see this as negative at all. They were all positive about it, and their common opinion were that it is all about training. \emph{"Nothing of it is difficult. It is just a matter of training"}, I3 said. 

While observing the informants interact with the technology, play the games, and walking through some of the menus, we saw that the informants mastered the different challenges better and better. They learned how to connect with the sensor, they quickly understood what was required from them in the various games, and they responded faster to feedback and information. This was both our observations and the informants own perception. When we asked them if they felt it was easier the second time they played, most of the informants answered \emph{"yes"}. The informants stated that the feeling of mastery came fast, which was positive for the gaming experience. Some of the informants stated that they learned how they were suppose to play the game and how they should move to interact with the sensor by looking at the other informants play. The first time information messages were shown on the screen, the majority of the informations did not respond to it. We had to assist them in what they were suppose to do. After playing for a while, they responded to the information messages without guidance, and they did it faster for each time. It was fun to watch how fast the informants learned. Even though they felt that they did not master the different tasks at first, they kept on trying, and after a short time they had learned a lot and enjoyed the feeling of mastery. 

\section{Immersion and Engagement}
The informants showed a lot of engagement and different feelings during the gaming session. They immersed into to game, and gave it all to complete the different challenges. In most of the games the informants used body movements eagerly. The informants also burst out with various comments like \emph{"yes yes yes yes"}, \emph{"noooooo"} and \emph{"ooh, I wanted that pineapple!"} while playing. In addition, the informants laughed a lot during game play. A few of the informants only used monotonous movements, like just waving their arm up and down with not that much engagement.  

Not all the games created feelings of immersion and engagement. The personal trainer game was a game that the informants "just did", without showing any particular emotions. There were not many comments during game play, and there were no laughter or engagement. We only observed concentrated faces. If this was because the informants thought it was difficult or boring, we do not know. However, the informants tried their best, and most of them went through the exercises with controlled and powerful movements, in tempo with good rhythm.       

The informants expressed that they felt, in some of the games, that there were a lot going on at the same time. We asked them if they felt it was easy to focus on the challenges, exercises and activities, although there were a lot of hustle. The informants agreed on that while playing, they were totally focused. I7 said \emph{"When you are in the game, then you do not see the outside world, then you are concentrated"}. The informants stated that they lived into the character of the avatar. They did not feel that the avatar was a person separate from themselves. \emph{"Yes, there I felt like the person [avatar] itself"}, I3 said. 

\section{Functionality and Usability}
\subsection{Introduction and instructions}
It was a common opinion that the different games should spend more time on instructions and explanation on the different exercises. \emph{"It needs to be a softer and more instructive introduction in the game's rules"}, I5 said. Especially in  the personal trainer game, this was the case. I1 compared this to the way they did it in an aerobic class she used to attend in her earlier days. She explained how they practised the exercise slowly before they could do it fast. The personal trainer game started up quickly, with no room for practice or warm up, and they held a high tempo. The informants did not know how to do the required movements, as no information where given. Most of the informants got the movements after a few steps, as they recognised the exercises from aerobic classes in their training centres. After playing, most of them agreed on that the exercises were OK and that it was only a question of learning. 

When the informants played their first game, they seemed confused and unknowing. One informant started pointing on the screen and asked \emph{"How do I point?"}, meaning how she could press the buttons. When playing the personal trainer game I1 asked \emph{"Press the button? Is there any button here?"}. It was clear that it was not obvious what was meant by pressing a non-physical button. None of the informants had ever played games like these, and did not understand how the game could be controlled by their body movements, even though we had explained this. This suggest that a thorough introduction explaining the interaction between the player and the game should be given. 

\subsection{Complicated menus}
There was a common agreement that the games had too complicated menus.  The avatar hand that was navigating on the screen was too sensitive, and it was hard to keep the hand still long enough to actually "press the button". In the sports game buttons were pressed by holding the hand over the button for a certain amount of time. \emph{"This is worse than working with a mouse this [on the computer]"}, I6 said. The time needed to hold over the different buttons seemed to be too long. This especially apply for people with declined strength in their arm(s). In the personal trainer game the avatar hand was, in addition to being too sensitive, unclear and at times almost invisible. It did not really look like a hand. This seemed to cause some problems because not all informants understood that the object on the screen was their hand. Two informants suggested that an arrow-marker like the one used on most computers would be more intuitive and easy to use. 

\emph{"The menu was extremely difficult [...]"}. In the same game, the menu appeared to be very complicated, as a result of being big, complex and sensitive. This made it difficult to know where to go, as well as to push the right buttons. There were a great amount of information on the screen, which made it difficult to choose the right alternative. Three informants was challenged to go through the whole menu, and they, especially, struggled a lot. To navigate between the pages in the menu, there where arrows on the sides that should be pressed. These arrows were hard to see, and they were very sensitive. When telling the informants to scroll through the menu by pressing the right arrow, I1 said \emph{"Which arrow?"}. In addition, this menu had a huge "back"-button that the informants pressed several times without intention. We found it strange that this menu was so complicated, as it actually was aimed for the older user group with their game "sprek alderdom", or "fit old aged" in English.  

Also with Fruit Ninja the informants had problems with the menu. The menu was crowded with elements, which made it hard to hit the correct buttons. In addition, the menu made it difficult for the informants to distinguish between the menu and the actual game. I6 and I7, who were playing together, stood in the background and waved their arms towards the screen while we tried to go through the menu and start up the game. 

\subsection{Information and feedback}
There were various perceptions of the information and feedback given during game play. Some of the informants was not affect by the information given, and at times they did not even recognise its presence. \emph{"I did not react to the text at all. If you had asked me, I would not know it had been text"}. I5, on the other hand, felt that there was too much information at the same time, which made it hard to concentrate. \emph{"No, limit the total amount of information, I think"}.  I2 said she had seen the text but it disappeared too fast, so she did not have the time to read it all. I2 suggested that it should be a way to confirm that you have read the information before it disappears. 

It seemed that it might have been too small text in the menu in the personal trainer game, after observing that at least one informant spent some time reading the text. At one point the informant had to move closer to the TV-screen to read the text. This informant had the same problem when navigating through the menu in the tennis game. Because there was only one informant that seemed to have problems with this, we assume that this informant might have had impaired vision. However, it is important to acknowledge that impaired vision is a common problem for the older population \cite{ijsselsteijn2007digital}, and it should therefore be considered in a game designed for this group. 

Most of the informants agreed that it was too much unnecessary information, however, it seemed that it was desirable with feedback on what they were doing. \emph{"I think it is very nice to see how far I have walked, and how fast I have walked, and how much downhill and uphill and things like that"}, I1 said, referring to a mobile application she was using to give her information about her progress when cross-country skiing. This implies that feedback on their progress is important for some of them.  

The instructions in both the tennis and skiing game, like "raise hand above head to play" or "move closer to the sensor" were well understood by almost all informants. Although we had to assist some by reading the message given on the screen, we believe that these messages should have been clear enough. A possible reason for why the informants did not see the messages might have been because they were not concentrated and focused enough. 

Not all informants understood that there were introduction videos and instructions in the beginning of the game. This became clear to us as most of the informants tried to do what the avatars did in the video, and seemed confused when nothing happened. \emph{"Should I try to hit it [the ball]? What am I suppose to do now?"}, I1 asked during the introduction video. However, it appeared that the second instruction video that was shown between two matches in tennis, was understood as an introduction video, and that they all learned from it. However, when the game was finished, some of the informants did not understand that the game was over. Especially did this apply for the tennis game, where most of the informants ended up just looking at the screen. Apparently, there was not given a clear enough information about the end of the game.  

\subsection{Graphics and sound}

We asked the informants about their opinion of the music in the games, and I1 answered that \emph{"I think everything about that was too much"}. I2 replied with \emph{"I like Mozart better"}. Other types of music that was mentioned as more suitable than ordinary pop music was swing. All the informants agreed that music is important to keep the rhythm when exercising, but that they would prefer the music not being too noisy. It was also a general opinion that the music was inappropriate.  I7 said \emph{"You get sensitive [to sound]. You do not want it. You want it quite. You would want to be active, but without too much background noise"}. I4 added \emph{"[...] I would prefer walking out in the nature. Then you can listen to the birds [...]"}.

It was a general opinion that there were too much elements and information on the screen during game play. I7 stated this by saying \emph{"It is too much elements on the screen. Where should you look?"}. Some of the informants also felt it was too much going on at the same time during game play, and I5 suggested that there could be different levels of things that happens in the game. \emph{"[...] Eventually when you get better and manage to keep track of more things, you can add more things [to the game] that happens. A lot of what happens in these games are not relevant"}. 

On the question about what they thought about the avatars and the picture in general, I5 answered \emph{"I think it was very confusing to see myself. Especially to see both of us [herself and the trainer] [...]"}. In addition, most informants agreed that the avatar of themselves in the personal trainer game made them look fat, which they did not like.

\subsection{The nature of the game}

\emph{"It bothered me that I did not see the correlation or the relationship between my own body movements and what was happening on the screen"}, I2 said about the gaming experienced. All of the informants complained about the delay that was present the games. This applied for all the games but Fruit Ninja, and it was mentioned as a problem several times by the informants. The delay made it hard to move correctly and in time, like in e.g the skiing game to pass through the gates, and to coordinate movements to perform a perfect jump. \emph{"Missed the jump? I jumped so quickly. That is just nonsenses"}. The delay in the games made it difficult to get points and achieve high scores, which again gave the informants the feeling of not mastering the movements. \emph{"I think my movements on the screen look too slow. Are they that slow?"}, I6 asked while playing. 

The technological aspects of the games became a source of confusion for most of the informants. \emph{"There were some small technical details that were problematic. [...] I could not see in the skiing game that anything special happened with me. [...] And in tennis, it was something about the time aspect, with when you tried to hit and when you actually hit [the ball]. I got a feeling that they [the game] helped me a lot in the beginning"}. For instance, this applied the tennis game, as there was a problem when the informants tried to coordinate the hand with the racket and the hand with the ball. The delay made it difficult for the informants to see when they actually hit the ball. Some of the informants expressed that the significant delay interfered with the gaming experience. I5 said that \emph{"I experienced the tennis and skiing game as games with lack of technical perfection. It was not good enough technically to make a real time experience"}. However, I5 said that if players were to get used to the delay, they might have more fun.

Most of the informants had a hard time coordinating their moves in the personal trainer game, as the movements were quite fast. Some of the movement was for many difficult to perform correctly. I6 commented that it was all about training, and that they would master it after getting familiar with the game. I5 answers to this with \emph{"Sure, but it is possible to make it less confusing"}. I1 replied that she rather would exercise without the game. The informants also experienced similar problems with the tennis game. The ball came quickly, and it was difficult to coordinate the hand fast enough. This applied especially when the ball came on the opposite side of the body from where the racket was held. One of the informants ended up trying to hit the ball with her left hand, even though she had the racket in her right hand. 

[KANSKJE HA MED I FEEDBACK??] Fruit Ninja, unlike the other games, did not show a clear avatar, which could have made it difficult for the informants to see a connection between their own movements and the actions on the screen. The avatar was hard to see as it was shown as a diffuse shadow on the wall behind all the fruit. I7 told us that she had trouble understanding how she could see where she hit. When I1 played Fruit Ninja it seemed like she did not understand what actions did cut the fruit in half, she just stood there and waved her arms while she looked quite confused. All of the informants agreed that Fruit Ninja was a bit silly and that it did not require anything from them. \emph{"This is just to wave your arms. There is no system"}. In addition, they did not see the relationship between what they did and what they achieved. \emph{"[...] I did not get any feedback on my movements"}, I2 said. It seemed that all informants found it important to get feedback on their movement, see results, and experience a learning effect.

\emph{"The game that required something from you was the skiing game. The rest of the games did not require you to follow the instructor apparently"}, I3 answered after being asked how they experienced the game. It was apparent that some of the informants did not understand what they were meant to do in the games and that they just "did something". 

Some informants mentioned that the skiing game was very fast. At the same time there was a comment about the importance of having the fast pace, to make the game exciting and fun. In addition, there was one situation in the skiing game that especially seemed to cause confusion and frustration. When playing multi player, there were two matches, where the players in the the second match would have to switch tracks. The informants had troubles understanding which track that was their track. \emph{"Eh.. I do not understand"}, \emph{"are we going in the same lane?"}. I5 was confused while playing, \emph{"First red and then blue and then blue and then.. [...] This does not work"}. The fun and enjoyment we observed the first round of skiing was the second round replaced with frustration. 

\section{Physical Outcome}

The informants has a general opinion that all movements are good for your body, and that these games required you to move in a good way. I3 said \emph{"If it [the game] do good for your body? Yes it does. All kind of movements are of the good"}, while I4 said  \emph{"[...] I felt it was useful, in a health-related way"}. I5 was amazed by how much body she actually moved, \emph{"It was amazing how much body you actually used on so little"}. All the informants could see that there were some positive health effects, however, they did not exactly know what they were exercising. They urged that the game should include information about which body parts you are exercising when you are doing the different exercises. Another general opinion was that they wanted to know if they were doing the exercises right or not when they were playing. [HA MED DETTE siste (fra However) UNDER INFORMASJON OG INSTRUKSJON???]

The informants said that if they were just going to do basic exercises, like in the personal trainer game, they would rather do this without a game. It they should use a game for exercising, it would need to contain a sport, or something else related to real life. \emph{"It would have been better to chop wood"}, I6 said after playing Fruit Ninja. It seemed like the informants wanted games with meaningful content and realistic activities. Game themes mentioned by the informants were swimming, rowing, picking apples, biathlon, interval exercises and a walk in the nature. \emph{"You mentioned apples. It might have worked to get it synchronised so that when you gathered apples and put it down, you could summarise the number. That might have been a competition"}. One informant also suggested doing puzzle games on the map of Europe. In that way you could learn geography at the same time which would make the game more meaningful.

It seemed important for the informants to see a progress in the game, and that they learning something. \emph{"I believe in games that have the characteristic of play and that you can see that you get better"}, I2 said. This will provide the players with something useful, that will make them want to play again.
 
From what we observed and from what was mentioned by the informants, it is clear that it is important to have a structured program where you equally, and symmetrically exercise similar body parts. This came clear from for example tennis and Fruit Ninja, where several of the informants mentioned that their shoulder and arm were in pain after playing. In tennis, the player only uses one arm, and the workload gets asymmetrical. In addition it was mentioned that if the aim of the game was exercising, they need the whole range of exercising, from warm-up to stretching. 
    
\section{Social Interaction}
\subsection{Playing together}
\emph{"I think the slalom was fun. But I am thinking more if I had a grandchild in a suitable age, and I could say "should you and grandma play the ski-game together? Just for fun?""}, I5 said about the social aspects of the different games. She continued with \emph{"I am thinking that in a nursing home we would probably like to play with someone"}. Some of the informants were clear in their opinion about whether they would play alone or with others, and I7 stated that \emph{"[...] I do not want to do this alone at home"}. I4 said she  enjoyed playing alone, \emph{"I think both [playing alone and together] was OK. [...] I believe in  exercising in small groups"}. This statement shows that playing in a group setting might be preferable.  

A general opinion was that this game could suit well in a nursing home setting, in a small group setting, or as an activity to do with grandchildren. It seemed that most informants liked to play together, however some informants mentioned that they were indifferent. The enjoyed playing together in the games that were made for fun and entertainment, like skiing and tennis, while with games meant for exercising, like the personal trainer game, they felt it would be better to do alone. One informant told about her mother in law that got inactive on her last years. She believed that the only way to encourage her to become more active by playing games would be if there were some social aspects within the game, like playing together with a grandchild. 

Everyone agreed that being social and meeting people are important parts in life, but that they would do it on their own time, and not be locked to a specific group or time. I5 felt that meeting together in a group for exercise gives her a sense of pressure that she does not like. \emph{[...] I have had pressure all my life. I do not want it anymore. I do not want to put myself in that situation where people come and ask "are you going to join this?""}. I7 agreed that this type of pressure is not motivating. \emph{"It is like that, that well-being is attractive, pressure is not attractive"}, I6 said, referring to the importance of doing things voluntarily. However, just having the opportunity to go out and meet people, like in an arranged training group, seemed to be important for all of the informants. 

\subsection{Watching others play}
\emph{"You are talented"}, and \emph{"You are starting to get it now"}, are examples of comments the "audience" told their co-informants while they watch the other informants play. They encouraged each other, and provided positive and motivating messages. The informants that were playing seemed to enjoy this feedback. This feedback made the gaming session more social and fun.  

\subsection{How to play together}
We asked the informants if the could imagine themselves playing at home in their own living room, and competing with friends over Internet. To this they were not very positive, and it seemed like they did not understand what we meant. \emph{"This is very distant for me"}, was I6's comment. \emph{"It would probably be more motivating to just ask her if she wanted to come over and play"}. These answers show how being social over Internet is quite unfamiliar to the informants. To make a game like this social, all the informants agreed that meeting in person would be better. On the other hand, the informants using this training mobile application, seemed to like to share exercise information with friends.

\emph{"[...] If there is going to be any point in doing something together, it needs to be that you are enhancing each other. Like that you get a better result if you are cooperating. [...]"}, I5 said. She had a strong meaning about how to play together, and meant that it was more motivating to cooperate rather than compete. The other informants seemed to be indifferent whether they were competing or collaborating.

\section{Experience and Understanding of Technology}
None of the informants had used a technology like Xbox Kinect before, and they all said that it was a completely new experience for them. \emph{"I have practically never seen it before"}. In fact, the informants said that they never had used any kind of video game technology. \emph{"Waste of time"}, I5 said, \emph{"I could never have time for something like that"}.  

The informants generally showed interest in the technology and our project during the workshop. They asked about the EU-project, where the games were made, and how much the Xbox Kinect costs. The informants showed interest in what equipment needed for them to play at home. \emph{"Does it connect to a TV?"}. I7 commented that it is important that the technology is easy to install. The informants were also eager to ask questions about the games, where this was related to information given, their own performance, difficulty levels and confusion. While looking at two informants playing Fruit Ninja I7 had questions as \emph{"Can you steal others fruit when you play?" and "What do you lose points for?"} 

During the group conversation there were questions about our work. They wanted to know how we would proceed from now, and what we imagined as our result. One informant also asked the facilitators that joined us the first day about their role in this project. When we were about to finish the group conversation, I6 asked \emph{"[...] What we are coming with, is it usable? Can you use if for something?"}. We answered that their feedback through the workshop is highly valuable, and that it will be used as a basis for the video game concept in our master thesis. I3 complimented our work with \emph{"it is price worthy what you are working with this, that I have to say"}, which meant a lot for us. It is good to know that the informants, representatives for the group of elderly, feels that what we do is something useful.  

\section{Aspects to Consider About an Exergame}

To summarise our observations, even though the informants experienced some confusion while playing, we saw that they had fun and enjoyed playing these games. However, most of them did not see the games so useful that they would have bought them and used them themselves. They had other ways to exercise that they would rather prefer. The games they liked the most were those involving real-life activities, like tennis and skiing, which combines movement and entertainment. They did not care for the personal trainer game, even though it involved realistic exercises, because it did not give them any value of 
entertainment. Fruit Ninja was perceived as stupid, and mostly, it gave them nothing.
 

\subsection{Different user groups}
One of the informant emphasised the importance of remembering the different target groups, and that it will be impossible to make a concept that suits everyone. She mentioned different groups to keep in mind, "the people who already have decided to keep their body in shape", "people who wants to know if they are doing the exercises right", and "the people who are already inactive". For the latter group she mentioned the importance of getting help, e.g. from grandchildren.  Another group mentioned are the people who can not stand or walk, who might be sitting in a wheelchair.

\subsection{In what setting could this game be suitable}
All of the informants could see a game like this as a fun activity to do with their grandchildren, which makes it clear that the social parts of playing games are important. Other relevant arenas for the game mentioned were small groups and at the nursing home. One informant talked about her mother in law, who got total inactive her last years. She said that the only way she could see her using a game like this, was with a grandchild or great-grandchild. This shows that it might be hard to reach the people that are already in bad physical condition and inactive, because they will already be depended on someone else. This build up under the importance on preventing people from ending up inactive.

Only one of the seven informants could see herself use a game like this. Most of the informants were in such good physical shape that they were still doing other types of activities. However, they did agree that the game could be an alternative in the future, when they might not be able to do the same activities they are doing today. \emph{"[...] If I can't go out skiing anymore, for example, maybe"}. 

\emph{"This can be a way to shorten the long waiting lists in the hospitals, by keeping people healthy, instead of using the hospitals for every little thing"}. Generally, the informants has a positive attitude towards the exergame, and believed that the game could work well as an alternative method for exercising. However, they agreed that this might be more relevant for our generation (20 years +). 

\subsection{Would rather do ordinary exercising}
\emph{"This [the exercises in the personal trainer game] is what we do at SATS [a Norwegian training centre] too. It is much more fun, I think, to be in a room with others, and to get the same instructions. However, of course, if you live far away and can't get there, this [the game] could be a replacement. It does not give the same, but could be a replacement"}. I2 said, \emph{"I do not think I will buy it [the game]. I would rather go to Elexia [a physical health center in Norway]"}. It became clear that as long as they had good health and were in good physical shape, they would rather keep on doing their regular activities and that they would never replace these activities with a game. 



