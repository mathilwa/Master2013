\chapter{Findings from Workshop 2}

In this chapter we will present our findings from workshop 2. The presented findings are based upon feedback and quotes from the informants, and our own observations during the workshop. The five informants are referred to by using I1 to I5, and is randomised so there is no relation to the references in workshop 1. Due to requirements to anonymity we have decided to not distinguish between male and female. Therefore, we refer to all the informants as females. 

\section{The Games}

\subsection{Game 1: Nature Trail with Quizzes}

When we presented our Nature Trail concept it took some time before the informants came with comments. It appeared that it was not clear to them from the pictures how this game really would work. \emph{"[...] I think it is impossible for me to say something now about what I think about this. This is because I do not have a sense of how it works"}, I4 said. She continues with, \emph{"Can I ask, this lady, or the girl [on the screen], if I do like this [making movements with her upper body], then she would do the same?"}. We confirmed. It seemed that the participants understood more about how the game would work, and more comments appeared. I3 said, \emph{"I think this it was a good idea because the environment you are in is familiar to me"}. The other informants agreed. I1 said several times during the discussion that she thought the concept was very nice. \emph{This was very nice"}, she said, while I4 followed with \emph{"fun development"}.  

About the involvement of quizzes in the nature trail, I1 was very sceptical. She was afraid that it would draw attention away from the physical tasks. I3 agreed and said \emph{"It will become a sort of test on how good you are, and that is not the way I have understood the point about these games"}. I1 was very clear in her opinion, that we should separate the quizzes from the rest of the game. She said \emph{"I think this is very nice. [...] However, I think you should think very thoroughly about the cognitive, the questions linked with the physical. You need to have a purpose about it. You know, a goal on what you want to achieve"}. We suggested that a solution could be to let the players choose whether they want to include the cognitive challenges or not. The informants agreed that this was a good idea.

It was not clear for all the informants when the quizzes would appear in the nature trail. They all seemed to believe that they would appear while they were doing something else, like while balancing over a log. To be able to focus on both the physical challenges and the quiz, I4 suggested that these questions should appear somewhere where it was natural to take a break, and that the player should be able to sit down and answer the questions. I4 also proposed that the player could get to see the questions before the game started, and then answer them after a while. Her experience was that if she was thinking about something, the answer usually came up eventually.  

From the discussion we saw that there were confusion about the link between the cognitive and physical challenges. This was not only about the goal of having a quiz, but also the quiz's purpose in the game. \emph{"The points [shown in the top part of the screen] are they just from the questions, if you manage the questions? It does not have anything to do with the [physical] skills?"}, I2 asked. Several of the informants wondered about how the question allocation would work. \emph{"When I'm doing the game time number three, will it still be the same assignments that appears?"}. I4 also asked if the player could choose category for the questions. 

\emph{"I did not quite understand. The heart that are floating there [points at the screen], is the idea that you should take it?"}, I4 asked. It was clear for the questioned asked that the red hearts created some confusion. I2 followed with \emph{"If you get the heart, does it disappear?"}. Further the informants wondered where the number of gathered hearts will be shown on the screen. We explained that to limit the information on the screen the hearts will only fill the "health bar", as an indication of good health. They understood, and agreed that it would be hard to count the number of hearts on the screen. Not all of the informants were positive about the hearts. \emph{"But hearts.. It is a little feminine"}, said I1, and suggested that maybe hunting butterflies would be better, especially for men because it could trigger the hunting instinct. I3 joked \emph{"A bottle of beer?"}, while I4 adds \emph{"Lets see.. A feather? Flying by?"}. 

The informants were not sure whether the players were suppose to choose their own difficulty level or not. I1 said that she would not be forced into something she would not do, and that she wanted to be able to choose herself which difficulty level to play in. \emph{"I am thinking that I do not want to get forced into something that is hard, that I do not master. Because then I get mad"}, she said. We explained how the game will remember the players progress and adjust difficulty level accordingly. She understood the point, but strongly meant that the players should be able to choose themselves. I4 agreed on the way we had organised the different levels. \emph{"I think it is an advantage that everyone starts at the easy level, and the more confident you get, the harder it gets. I think that is a good way to control"}. \emph{"Yes, but you have to choose yourself"}, I1 said repeated. 

About the different challenges and exercises we presented in the games, the only concern the informants had was about doing two things at the same time, like answering questions while doing an exercise. One challenge they commented was the one where they were suppose to balance over a log. I1 said \emph{"This exercises your balance, right, so it is easy to fall your self, if you get [takes her hand to her head as a sign of getting dizzy]"}, referring to the possibility of falling not only in the game, but yourself. There were also some technical questions about this challenge. \emph{"If I was going to balance on the log, and I fell down, would I feel that? If I did not manage to keep the balance, and fell in the water?"}, I4 asked. 

\subsection{Game 2: Picking Apples}

The informants all seemed to like the picking apple game. \emph{"I think this was nice"} I3 said. However, there were some concerns related to how the apples would appear, and how they could plan the apple picking. I4 said \emph{"I feel that two apples is fast to gather. If the tree was full of apples I would be more eager to play. [...] It might be weird if they should pop up all the time"}. We explained that the apples appear randomly because of both the cognitive and physical exercise. They all agreed that this was a nice solution. 

\section{Information, Instructions, and the Menu}

More information and instructions were stated as important in workshop 1, and we had therefore tried to include this in our exergame concept. The informants seemed satisfied with the way we had presented this, and they also seemed to recognise the instructions we had taken from the sports game, like "raise hand above head to play". We asked if they noticed the arrow in the upper left corner, and I5 immediately said \emph{"back"}. Everyone agreed that it was clear that the arrow indicated a back-button. About the combination of colors and how the pictures looked in general, I1 said \emph{"beautiful!"}. I3 mentioned that it would be better with more shadow around the buttons, to indicate that it actually is a button. Except from that, there was not many comments. On the questions on whether there were too many steps in the menu before you could start the game, the only comment was from I1, who said \emph{"[...] When you have used the menu, then you want to have shortcuts"}. 

When we presented to the informants the possibility to choose game play based on muscle groups, there arose many questions on what we meant about the term "muscle group". It became clear that we had been inconsistent with this categorisation. I5 said \emph{"[...] When you use the term muscle group, then I think that endurance do not fit under this term"}.   

All of the informants had some problems understanding the different difficulty levels in the games. \emph{"I had an immediate reaction when I saw this [points on the screen that shows the three different forests with different difficulty levels], that it took me some time before I could see that these different environments [the different forests] represents different difficulty levels"}, I3 said. The other informants agreed. Further I3 suggested, \emph{"Easy forest, medium forest, hard forest, or what you would call it"}. About the different difficulty levels, some of the informants wondered if the same challenges would appear in all levels, just more frequently. 

\section{Music and atmosphere}

The informants were curious about what kind of music the games would get. I4 asked \emph{"I am wondering about the atmosphere and environment. When I am balancing there [on the log], will I hear the sound of water?"}. We explained that we wanted to include sounds that are natural to the environment. I2 said \emph{"Not noisy, like last time [referring to the music in the games played in workshop 1]"}. We told them that we want to use peaceful music, like classical music. I1 suggested \emph{"You should think about rhythmic music. Maybe it is just as easy to pick apples to a rhythm, instead of picking as many as possible? [...] Then, when the apples ripens, it will be according to the rhythm"}. 

\section{The Delay Problem}

One of the informants remembered the technical aspects related to the delay from the last workshop. She said that \emph{"this is a technical question which is about the capacity in the computing system. It is not straightforward to solve technically. [...] Do you have any thoughts about this?"}. The informants were aware that this was beyond the scope of our thesis, but we informed them that we will look into the delay problem.

\section{Informants Showing General Interest in the Games}

Some of the informants were concerned about getting tired of the game after playing it several times. One informant asked if it would be possible to exchange the game if it got boring. We presented our idea for a video game series, where each game could be downloaded from the Internet for a small amount of money, in this case 99 NOK (about 18 USD). The informants stated that this was an affordable price. They showed interest and curiosity about how they could get the game. I1 asked \emph{"Do you think you could get it on prescription?"}, and I4 continues by asking if it would be possible to rent a game like this from the library. When discussing the exchange of games if they got bored, I3 expressed that she did not think this would be a problem. \emph{"[...] I would think that when I have completed the game I have exercised, and that it would be satisfactory for me. And then I could easily do it over again"}. After I3 had put it that way, the other informants seemed to agree.

All of the informants were eager to hear more about the development of this exergame, and they wanted to get information about when the game would appeared on the market. I4 said \emph{"It would have been fun to know when it comes. I have faith in this project"}. \emph{"Include us in the customer list!"}, I1 laughed.



