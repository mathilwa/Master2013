\chapter{Findings from Workshop 2}
\label{chap:findW2}

According to the fourth activity in the cycle of user centered design (depicted in Figure \ref{userdesign}), we wanted to conduct a second workshop to evaluate the designed exergame against the requirements. A description of the execution of this workshop can be found in Section \ref{sec:ws2}. 

In this chapter we will present the findings from workshop 2. The presented findings are based upon feedback from the informants, and our perception of their reactions to the presentation. The five informants are referred to by using I1 to I5, and are randomised so there is no relation to the references in workshop 1. Due to requirements to anonymity we have decided to not distinguish between male and female. Therefore, we refer to all the informants as females. Quotes are translated from Norwegian to English to  preserve the meaning. 

\section{General Perception Regarding the Game's Story and Elements}

\subsection{Game 1: Nature Trail}

When we presented the "Nature Trail" game it took some time before the informants had any comments. It appeared that it was not clear to them from the pictures how this game really would work. \emph{"It is impossible for me to say anything about what I think about this game. This is because I do not have any sense of how it works"}, I4 said. She continued by asking, \emph{"This girl on the screen, if I do certain movements, will she do the same?"}. After we confirmed this, it seemed that the informants understood more about how the game would work, and more comments appeared. I3 said, \emph{"I think this was a good idea because the environment you are in is familiar to me"}. The other informants agreed. Several times during the discussion, I1 complemented our concept. \emph{This was very nice"}, she said, while I4 followed with \emph{"fun development"}.  

About having quizzes in the "Nature Trail" game, I1 was very sceptical. She was afraid that it would draw attention away from the physical tasks. I3 agreed and said \emph{"Answering the questions in the quiz will become some kind of test on how good you are, and that is not how I have understood the point of these games"}. I1 meant strongly that we should separate the quizzes from the rest of the game. She said \emph{"I think this is very nice. [...] However, you should think very thoroughly through the cognitive aspects of linking the questions with the physical tasks. You need to have a purpose about it, a goal on what you want to achieve"}. 

It was not clear for all the informants when the quizzes would appear in the nature trail. They all seemed to believe that they would appear at the same time as they were doing something else, like while balancing over a log. To be able to focus on both the physical challenges and the quiz, I4 suggested that the questions should appear somewhere where it was natural to take a break, and that the player should be able to sit down while answering them. I4 also proposed that the player could get to see the questions before the game started, and then answer them after a while. Her experience was that if she was thinking about something, the answer usually came up eventually. 

From the discussion we understood that there was confusion about the link between the cognitive and physical challenges. This was not only about the goal of having a quiz, but also the quiz's purpose in the game. \emph{"The points shown in the top-bar, do you get them just by answering the questions? It does not have anything to do with the physical skills?"}, I2 asked. Several of the informants also wondered about how the question allocation would work. \emph{"When I'm doing the game time number three, will it still be the same assignments that appears?"}, I3 asked. I4 asked if the player could choose category for the questions.  

The informants were mostly concerned about doing two things at the same time, like answering questions at the same time as doing an exercise. Another challenge they commented on was the one where they were supposed to balance over a log. \emph{"This will exercise your balance, right, so it is easy to fall yourself, if you get dizzy"}, I1 said, referring to the possibility of falling not only in the game, but also in real life. There were also some technical questions about this challenge. \emph{"If I am balancing on the log, and I fall down, would I feel that? If I did not manage to keep the balance, and fell in the water?"}, I4 asked. 

Most of the elements in the game environment were familiar to the informants. However, the hearts were not that obvious. \emph{"I did not quite understand. The heart floating there on the screen, is the idea that you should take it?"}, I4 asked. I2 followed with \emph{"If you get the heart, does it disappear?"}. Further the informants wondered where the number of gathered hearts was shown on the screen. We explained that to limit the information on the screen the hearts will only fill the "health bar", as an indication of good health. They understood, and agreed that it would be hard to count the number of hearts on the screen, and that the "health bar" was a good idea. Not all of the informants were positive about the hearts. \emph{"But hearts.. It is a little feminine"}, said I1, and suggested that maybe hunting butterflies would be better, especially for men because it could trigger the hunting instinct. I3 joked \emph{"A bottle of beer?"}, while I4 added \emph{"Let's see.. A feather? Flying by?"}. 

At first the informants did not understand how the different difficulty levels would work. I1 said that she did not want to get forced into something she did not want to do, and that she wanted to be able to choose which difficulty level to play in. \emph{"I am thinking that I do not want to get forced into something that is hard, that I do not master. Because then I get mad"}, she said. We explained how the game will remember the players progress and adjust difficulty level accordingly, and that there also will be initial difficulty levels that the player can choose between. She agreed that this was a nice way to do it, as long as she had the possibility to choose herself. I4 also agreed on the way we had organised the different levels. \emph{"I think it is an advantage that everyone starts at the easy level, and the more confident you get, the harder it gets. I think that is a good way to be controlled"}.

\subsection{Game 2: Picking Apples}

The informants all seemed to like the "Picking Apples" game. \emph{"I think this was nice"} I3 said. However, there were some concerns related to how the apples would appear on the tree, and how they could plan the apple picking. I4 said \emph{"I feel that two apples is fast to gather. If the tree was full of apples I would be more eager to play. [...] It might be weird if they should pop up all the time"}. We explained that the apples appear randomly to cover both the cognitive and physical exercise. They all agreed that this was a nice solution. 


\section{Information, and the Menu}

More information was stated as important in workshop 1, and we had therefore tried to include this in the exergame. The informants seemed satisfied with the way we had presented this, and they also seemed to recognise the included instructions inspired by the sports game, like "raise hand above head to play". We asked if they noticed the arrow in the upper left corner, and I5 immediately said \emph{"back"}. Everyone agreed that it was clear that the arrow indicated a back-button. About the combination of colours and how the pictures looked in general, I1 said \emph{"beautiful!"}. I3 mentioned that it would be better with more shadow around the buttons, to indicate that it actually is a button. On the question on whether there were too many steps in the menu before the game started, the only comment was from I1, who said \emph{"When you have used the menu, then you want to have shortcuts"}. 

When we presented the possibility to choose game play based on muscle groups, there arose many questions on what we meant about the term "muscle group". It became clear that we had been inconsistent with this categorisation. I5 said \emph{"When you use the term muscle group, then I think that endurance do not fit under this term"}.   

All of the informants had problems understanding the different difficulty levels in the games. \emph{"My immediate reaction when I saw these three different forests with different difficulty levels, was that it took me some time to understand that the different environments represent different difficulty levels"}, I3 said. The other informants agreed. Further I3 suggested \emph{"Easy forest, medium forest, hard forest, or what you would call it"}. About the different difficulty levels, some of the informants wondered if the same challenges would appear in all levels, just more frequently. This was the way we have thought the game to be, in addition to adding new obstacles in the higher levels.

\section{Music and Atmosphere}

The informants were curious about the music and sounds in the games. I4 asked \emph{"I am wondering about the atmosphere and environment. When I am balancing on the log, will I hear the sound of water?"}. We explained that we wanted to include sounds that are natural to the environment. I2 said \emph{"Not noisy, like last time"}, referring to the music in the games played in workshop 1. We told them that we want to use peaceful music, like classical music. I1 suggested \emph{"You should think about rhythmic music. Maybe it is just as easy to pick apples to a rhythm, instead of picking as many as possible? [...] Then, when the apples ripen, it will be according to the rhythm"}. 


\section{Other Aspects Concerning the Games}

Some of the informants were concerned about getting tired of the game after playing it several times. One informant asked if it would be possible to exchange the game if it got boring. We presented our idea for a video game series, where each game could be downloaded from the Internet for a small amount of money, in this case 99 NOK (about 18 USD). They agreed that this was an affordable price. The informants showed interest and curiosity about how they could get the game. I1 asked \emph{"Do you think you could get it on prescription?"}, and I4 continued by asking if it would be possible to rent such a game from the library. When discussing the possibility to exchange games if they got bored, I3 expressed that she did not think this would be a problem. \emph{"I would think that when I have completed the game I have exercised, and that would be satisfactory for me. And then I could easily do it over again"}. After I3 had put it that way, the other informants seemed to agree.

One of the informants remembered the technical aspects related to the delay from the last workshop. She said \emph{"this is a technical question which is about the capacity in the computing system. It is not straightforward to solve technically. [...] Do you have any thoughts about this?"}. We informed them that we will recommend the delay issue for further work.

All of the informants were eager to hear more about the development of the exergame, and they wanted to get information about when the game would appear on the market. I4 said \emph{"It would have been fun to know when it comes. I have faith in this project"}. \emph{"Include us in the customer list!"}, I1 laughed.

\section{Summary}

We will summarise important findings from workshop 2 below:

\begin{itemize}
\item Informants liked the familiar environment, both in the "Nature Trail" game and in the "Picking Apples" game.
\item Informants were sceptical to including quizzes in the "Nature Trail" game, and suggested to separate the quizzes from the physical tasks.
\item Informants stated that the balance task could be challenging.
\item Technical details should be included, i.e. what happens if the avatar falls of the log.
\item The purpose of the hearts was not obvious. In addition, the hearts were considered as feminine.
\item Informants suggested that clearer instructions on the different difficulty levels should be included.
\item Informants were curious about music and sounds. 
\item Informants proposed to use music in rhythm with the rate of apples appearing on the tree. 
\item Informants suggested more shadow around buttons to highlight them even more.
\item Informants expressed lack of consistency related to what goes under the term "muscle group".

\end{itemize}



