\chapter{Findings from Workshop 2}

In this chapter we will present our findings from workshop 2. The presented findings are based upon feedback and quotes from the informants, and our own observations during the workshop. The five informants are referred to by using I1 to I5, and is randomized and has no relation to the references in workshop 1. Due to requirements to anonymity we have decided to not distinguish between male and female. Therefore, we refer to all the informants as females. Quotes from the informants are translated from Norwegian as literally as possible. 

\section{Information, Instructions, and the Menu}

More information and instructions were stated as important in workshop 1. The majority of the informants urged the need to understand what they were suppose to do in the game. This is something we have planned to improve in our game concept, and they all seemed satisfied with the way we have planned to do this. They also seemed to recognize the instructions we had taken from the sports game, and everyone nodded  when the screen shot with the message "Raise your hand to start the game" was shown in the picture, and confirmed they had seen it before. On the question about if they could see the arrow in the upper left corner, I5 said immediately \emph{"back"}. Everyone agreed that it was clear enough from the arrow that this indicated a back-button. About the combination of colors and how the pictures looked in general, they did not have that many comments, except from I1 who said \emph{"Beautiful!"}.  However, I3 mentioned that it would be better with more shadow around the buttons, to indicate that it actually is a button. On the questions on whether there were too many steps in the menu before you could start the game, they did not have any comments, except from I1 who mentioned \emph{"[...] When you have used the menu, then you wish to have shortcuts"}. 

In the menu the player can chose if she wants to play based on which muscle group she wants to exercise, or just which game she wants to play. There arose many questions on what we meant about muscle groups, and it became clear that we had been inconsistent with the categorization within muscle group. I1 said \emph{"Yes, muscle group, it does not fit with what you show here [refers to the picture of the menu where the player can choose different type of muscle groups]"}, while I5 said \emph{"[...] When it says muscle group, then I think that endurance.. It does not fit under muscle group"}. We acknowledge that we had not thought through this step in the menu thoroughly. However, we concluded that this is a job for professionals, like a physiotherapist, to do, and that this was just an example in our case.  

All of the informants had some problems understanding the different levels in the games. \emph{"I had an immediate reaction when I saw this [points on the screen that shows the three different forests with different difficulty levels], that it took me some time before I could see that these different levels [the different forests] represents different difficulty levels"}, said I3. The other informants agreed. Further I3 suggests: \emph{"Easy forest, medium forest, hard forest, or what you would call it"}. It came clear to us that it was not very intuitive that the three different forests we presented in the menu were meant as three different difficulty levels. This is something we have to work further on in our concept. 

\section{The games}

\subsection{Game 1: Nature Trail with Quizzes}

After going through the first part of the presentation where we presented the main concept "Out in the nature", and in particular the "Nature trail with quizzes", it took some time before the informants had a comment. It appeared that even though four out of five had participated in workshop 1 and played games like these, while the fifth participant had seen us playing two different games in our first meeting with Seniornett, it was not clear to them from the pictures how this really would work.  \emph{"[...] I think that for me it is impossible to say something now, to answer what I think about this, because I do not have a sense of how it works"}, said I4.  We tried to explain this as thoroughly as we could by comparing it with the skiing game they played in workshop 1. I4 asked again \emph{"Can I ask, is it like that this lady, or the girl [on the screen], if I do like this [making movements with her upper body], then she would do like that?"}. We confirmed. Now it seemed that the participants understood more about how this game would work and more comments appeared. I3 said: \emph{"Yes, and I think this was a good idea because the environment you are in is familiar to me"}. The other informants agreed. I1 said several times during the discussion that she thought the concept was very nice. \emph{No, this was very nice"}, she said, while I4 followed with \emph{"Fun development"}.  

Even though the informants mostly liked the game in general, there were some concerns and uncertainties about integrating quizzes in the trail. I1 was very sceptical to having quizzes in the game and was afraid that it would defocus away from the physical tasks. I3 agreed and said \emph{"It will in a way become a sort of test on how good you are, and that is not the way I have understood is the point about these games. [...]"}. It was neither clear for all informants when the quizzes would appear on the trail. They all seemed to believe that they would appear while they were doing something else, like for example while balancing over a log. I4 suggested that these questions should appear somewhere where it was natural to take a break, and that the player should be able to sit down to answer the question. In that way the player could focus on the physical challenges and the quizzes separately. The way I4 described this was also the way we had imagined it to be. It seemed that we did not present this clear enough, as there was never meant to be the way in which they would have to answer a question at the same time as they were balancing over the log, like I4 thought. However, from the informants comments about this, it seemed that we all agreed on when the quizzes should appear. I1 was very clear in her opinion when it came to the quizzes. She meant that we should separate them from the rest of the game. \emph{"I think this is very nice. [...] However, I think you should think very thoroughly about the cognitive, the questions linked with the physical. You need to have a purpose about it. You know, a goal on what you want to achieve, I think. Because if not it might knock each other out, to say it in that way. It becomes counter productive"}, was her concluding comment. We suggested that one solution might be to choose when starting up the game whether you want to include the cognitive challenges or not. The informants agreed that this was a good idea. I4 proposed that the player could get to see the questions before the game started, and then answer them after a while. Her experience was that if she was thinking about something, the answer usually came up eventually. In that way the player could go through the game, and focus on the physical tasks, and at the same time have the question in their head and let it mature. 

It came clear from the discussion that there were some confusion about the link between the cognitive and physical challenges, not only when it came to the goal about the questions, but also its role in the game. \emph{"The points [shown in the top part of the screen] are they just from the questions, if you manage the questions? It does not have anything to do with the [physical] skills?"}, asked I2. We confirmed, and explained that the physical skills would be shown in the red bar. Several of the informants also wondered about how the question allocation would work.  \emph{"When I'm doing the game time number three, will it still be the same assignments that appears?"}, asked I3. We explained that the question-allocator will be random and intelligent, so that they never will get the same questions, and also that the difficulty will increase after playing several times. I4 wondered if the player could choose category for the questions. Initially we had thought the questions and tasks to be randomized, but categorized questions can also be considered. 

Another aspect of the game that raised a little confusion was the red hearts. I4 asked \emph{"I did not quite understand. The heart that are floating there [points at the screen], is the idea that you should take it?"}. We confirmed. I2 asked so \emph{"If you get the heart, does it disappears?"}. We explained that it would disappear after they had captured it. Further they wonder where on the screen it would show how many hearts they had gathered. We explained that to limit the information on the screen the hearts will only fill the bar in the middle of the top screen with red color, as an indication of good health. They understood and agreed that it would be hard to count the number of hearts on the screen. Not all of the informants were positive about the hearts. \emph{"But hearts.. It is a little feminine"}, said I1, and suggested that maybe hunting butterflies would be better, especially for men, because then you could trigger the hunting instinct. I3 joked \emph{"A bottle of beer?"}, while I4 adds \emph{"Lets see.. A feather? Flying by?"}. 

There were also some technical questions about the game. I4 said \emph{"I am thinking. There were you are going to balance on the log. If I fell down, would I feel that? If I did not manage to keep the balance, and fell in the water?"}. These kind of limitations that have to be programmed into the game, was one type of aspect we had not considered and will be something we will look further into and integrate into the requirements. 

When it came to the different levels in the game, some of the informants wondered if the same challenges would appear in all levels, just more frequently. We answered that most likely this would be the case, but that it also might be possible with new and different challenges. 

It arose some further confusion on whether the player could choose their own difficulty level or not. I1 was strongly against getting forced into something she would not do, and wanted to be able to choose what kind of level she was playing at herself. \emph{"I am thinking that I do not want to get forced into something that is hard, that I do not master. Because then I get mad"}, she said. We explained the relationship between the different levels, and how the sensor would remember each player, so that the player could move on to a more difficult level the next time she was playing. I1 quickly added \emph{"Yes, if you want to?"}.  She understood the point, but strongly meant that the player should be able to choose herself. Eventuelt skrive noe her om at denne personen også snakket en del om dette i WS1.
I4 agreed on the way we had organized the different levels. \emph{"
I think it is an advantage that everyone starts at the easy level, and the more confident you get, the harder it gets. I think that is a good way to control"}, said I4. \emph{"Yes, but you have to choose yourself"}, I1 said again. We explained the way we wanted it to be, that the player could choose between the initial difficulty levels, and that it would be three different levels within each initial level that is controlled by the game. We also explained that everything was voluntarily, and that the player could choose to stay on the easy level. Everyone seemed to agree that this was a good solution, and I1 who initially was sceptical to the different levels seemed satisfied. 

Some informants were concerned about getting tired of the game after playing it several times. I3 said \emph{"I think, for me, on the questions about exchanging the games, I would think that when I have completed the game I have exercised and that that ultimately is satisfactory and then it is OK to do it over again. Because in the exercise centers I do the same, using the same apparatus and follow [the same programs]. Yes, its nice for me when I leave there [the training center]"}. After I3 had put it that way, everyone seemed to agree. 

About the different challenges and exercises we presented in the games, they did not have many comments. The only concern they had was about doing two things at the same time, like answering questions while doing an exercise. The only challenge they commented was the one where they were suppose to balance over a log. I1 said \emph{"This exercises your balance, right, so it is easy to fall your self, if you get [takes her hand to her head as a sign of getting dizzy]"}, referring to the possibility of falling not only in the game, but yourself. 

\subsection{Game 2: Picking Apples}

The picking apple game they all seemed to like. I3 said \emph{"I think this was nice"}, and the other informants agreed. However, they had some concerns about how the apples would appear and how they could plan the apple picking. I4 said \emph{"Yes, I get a little like this.. 2 apples is fast to gather, right? If the three was full of apples I would get very eager. [...] But it might be weird if they should pop up all the time"}.  We explained that the reason for having the apples appear randomly was because of exercise purposes. They all agreed that this was a nice solution. 

\section{Music and atmosphere}

The informants were curious about what kind of music the games would get. I4 asked \emph{"I am wondering, about the atmosphere and environment, when I am balancing there [on the log], will I hear the sound of water?"}. We explained that we wanted to include all the natural sounds from the forests and make calm, nice sounds to listen to, and that there in addition will be suitable music when certain exercises had to be performed. I2 said \emph{"Not noisy, like last time [referring to the music in the games we played in workshop 1]"}. We explain further that we want music more related to their generation, like for example classical music. When it came to the apple picking game, I1 suggested \emph{"You should think about rhythmic music. Maybe it is just as easy to pick apples to a rhythm, instead of picking as many as possible?"}, and referred this to the way you do it at the gym. \emph{"Then it will be like when the apples matures it will fit with the rhythm"}, she added. 

\section{The Delay Problematic}

One of the informants remembered from the last workshop that there was a significant delay between the players movement and the movements that appeared on the screen. I1 said \emph{"This is a technical question which is about the capacity in the computing system, to put it in that way, so it is not just straightforward to solve technical. [...] Do you have any thoughts about this?"}. The informants was aware that this was not a part of our project. However, I1 mentioned the great business possibilities for the gaming industry to solve this problem. We informed them that we will look into the delay problematic, and try to find out where it lies, but that we will not be able to solve the problem in this project. 

\section{Informants Show General Interest in the Games}

There were some concerns about if they would be bored about the game if they played it several times, and one informant had a question on if there was a possibility to exchange games if you got tired of the game you had. We explained that the marketing and selling plan for this game was not yet decided, and that this would be up to the company that is going to develop this game. However, we presented a possible solution for our series of games, where every type of game could be downloaded from the internet for a small amount of money, in this case 99 NOK, or about 18 USD. Everyone agreed that this was an affordable price. The informants seemed interested and curious about how they could get the game. I1 asked \emph{"Do you think you could get it on prescription?"}, while I4 asked if we thought it was possible to rent a game like this on the library.

All of the informants were eager to hear more about the project and get information about the game when it appeared on the market. I4 said \emph{"It would have been fun to know when it comes. Because I believe it will be something out of this"}. 

We wanted the informants to be critical, and to give us as much feedback as possible. We urged them to give us feedback, both positive and negative. However, the informants were mostly positive, and I1 said \emph{"No, it is nothing negative"}, and asked us some other questions that we will not discuss here, because of irrelevance for this project. 

