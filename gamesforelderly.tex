\chapter{Exergames for Seniors}
\section{Related Work}
Billis et al. discuss some important issues that need to be taken into account when developing games for elderly. Elderly often suffer from decline in visual acuity, decreased audition, mobility changes and cognitive functions’ decline. In addition, many elderly are not familiar with technology. The writers of the paper suggest that it should be able to customize the game for every players’ special needs. Font, size and color should be adjustable, and information should be provided in different multimedia alternatives, like text, voice and images. The objects should be of sufficient size and the elements should not move too fast. The overall interface should be as simple as possible, without the need to remember information given earlier in the gaming process, and it should be given sufficient information and guidance throughout the whole game. The writer of this paper also stress the importance of the social factors of the game, and suggest the ability to multi-play. At last, for the players to get interested and engaged in the game, the content of the game should match their cultural and lifestyle diversity. \\ \\
de Bruin et al. write about the potential of virtual reality environment, with use of for example games, for exercise. Virtual reality platforms can provide naturalistic movements in a safe environment that can be customized after the patients' needs. It can offer a consistent program that enables for comparison over time. In addition, the use of games, can distract the player from any pain they may have. For a game, stepping exercises can be suitable. They have found that stepping exercises can be a good predictor of falls, and it is proved that repetitive training program with stepping exercises can improve balance in elderly. Also here, they express that the problem of already existing exergames is that they are too complex for the older user group. Therefore, there is a need to develop games specifically for this group where physical and cognitive limitations, as well as typical interests of elderly, are taken into consideration. The writers also present a study where it was shown that there was a significant decrease in relative dual task costs (DTC) of walking for elderly who was training physically combined with a VR dance game that required decision making, while training traditionally did not change this walking parameters. This comes from the fact that elderly often faces problems when they have to do more than one task at the same time. \\ \\
Gregor et al. discuss the particular issues when designing for the older population, and propose a paradigm and methodology to support the process of designing software as close to the universal accessibility ideal as possible. These methods are called: “Design for Dynamic Diversity (D3)” and a modified version of this: “User Sensitive Inclusive Design (USID)”.\\ \\
The writers describe older people through three different groups: \\
- Fit older people: elderly who do not suffer from any diseases or dysfunctionalities, but who are different from when they were young.\\
- Frail older people: elderly who in  general have a reduction in many of their functionalities and who often have one or more disabilities.\\
- Disabled people who grow older: These are the people who have long-term disabilities which have affected their aging process. \\ \\
In addition, they define some important characteristics of older people (directly drawn from the paper, kilde):\\
-”The individual variability of physical, sensory, and cognitive functionality of people increases with increasing age”\\
-”The rate of decline in that functionality (that begins to occur at a surprising early age) can increase significantly as people move into the “older” category”\\
-”There are different, and more widely appearing problems with cognition, e.g. dementia, memory dysfunction, the ability to learn new techniques.\\
-”Many older users of computer systems can be affected by multiple disabilities. Such multiple minor (and something major) impairments can interact, at a human computer interface level to produce a handicap that is greater than the effects of the individual impairments. Thus research into accessibility focused on single impairments may not always appropriate solutions”\\
-”Older people may have significantly different needs and wants due to the stage of their lives they have reached”\\
-”The environment in which older people live and work can significantly change their usable functionality - e.g. the need to use a walking frame, to avoid long periods of standing, or the need to wear warm gloves”\\
-”On a more positive note, older people can have access to a much wider experience and knowledge of the world than younger people, and a more mature approach to problem solving”\\ \\
Most software design are static where there are no possibility to adapt to the different needs of the users. To develop a technology system for the older user group with different characteristics, a modified version of User Centered Design (UCD) principles must be used. The writers propose a methodology which they call “User Sensitive Inclusive Design” (USID). The issues this methodology address are (directly drawn from the paper): \\
-”Much greater variety of user characteristics and functionality” \\
-”Finding and recruiting “representative users”” \\
-”Conflicts of interest between user groups (including “temporarily able-bodied”)” \\
-”The need to specify exactly the characteristics and functionality of the user group” \\
-”Tailored, personalisable and adaptive interfaces” \\
-”Provision for accessibility using additional components (hardware and software)”\\ \\
As an example of the advantages of the proposed methodology, the writers present a case study that was developed at the Speech Project at Oxford Brookes University. In this case study they designed a web browser for people that were visually impaired. 200 visually impaired users evaluated the system.  The findings they did when it comes to older people were:
- elderly seemed to lack confidence in handling IT systems. However, the confidence increased after they had experienced a successful interaction and decreased after experiencing an unsuccessful interaction.
-many elderly have difficulties remembering too much information. This indicates that there are important memory related factors that need to be taken into account when designing for elderly. 
-following a need for less information, will most likely also mean less functionality. From an other study they found that it was a need for the possibility that more functionality could be added after the user had mastered the initial, simple functionalities. 
-after some kind of assessment is passed, the user can be moved to a higher level. This can be done for example by self-assessment. To reinforce user-confidence the user should be able to reach goals.
kanskje noe mer passende avsluttende her.. Men det var ikke noe ordentlig konklusjon i paperen.. hmm \\ \\
Gerling et al. also discuss the chances and challenges when developing a game for the elderly users. They talk about the same age-related challenges as the previous reviewed papers, and suggest four major guidelines that should be followed when designing games for elderly: \\
1. The player should have the possibility to interact with the game both when sitting and standing. \\
2. Avoid too extensive and sudden movements.\\
3. It should be possible to adjust the level when it comes to difficulty, game speed and device sensitivity and the player should be able to adjust the level themselves. \\
4. Interaction mechanisms should be simple and player frustration should be avoided, and it should provide constructive feedback.\\ \\
To verify these guidelines, the research group prototyped a exergame called SilverBalance. This game was made for the Wii Balance Board and consisted of two balance tasks and have the possibility to be played both sitting and standing. This game was tested on 9 seniors with an average age of 84. The following observations were made: \\
-All of the participants were able to play the game adequately. \\
-Participants expressed that the fact that the design was so minimalistic, made it possible for them to focus on the purpose of the game. \\
-The possibility to sit while playing the game was necessary because all participants were dependent on devices to assist them when standing and walking.\\
-The players started to compare their results and comment on each others results, (suggesting that social factors are important.. slang på den selv jeg)\\
-Impairments and diseases made it difficult for some participants after a longer period of playing, suggesting that alternative interactions should be included in such games.\\ \\
The SilverBalance game and the focus group test shows that the four criterions set serve as good guidelines for developing games for elderly, (but that further research with the focus on the impact of age on different structural elements of games should be carried out. hva nå enn dette betyr?)\\ \\
Gerling et al presents a case study executed on a group of frail elderly living in full-time nursing homes. They introduce and evaluate a video game called SilverPromenade, which is a virtual walk through the forest. SilverPromenade is stripped from complexity in functionality and design to be senior-friendly. The game can be played as a single-player game or a multiplayer game. In each mode it is possible to engage three different roles. SilverPromenade is played using Nintendo Wii Remote and Wii Balance Board. \\ \\
The main concept of the game is the combination of the walk in the forest with optional mini games to engage the players. One player can be walking and solving the mini games, or the mini games can be used for multiplayer mode, engaging the maximum number of three players in SilverPromenade. The main task is to step on the Wii Balance Board to walk through the forest, the walker role. The two mini games consist of catching a butterfly by pointing at it with the Wii Remote, the pointer role, and counting rabbits by shaking the Wii Remote every time a rabbit appears on the screen, the shaker role. This scenario is quite simplistic, which is important when including elderly in digital game play. The concept of the game, a virtual walk in the forest, appeals to elderly in nursing-homes because it offers the possibility to “visit” the forest, which generally might be inaccessible for them. \\ \\
Playing this game requires resources from the elderly. They have to focus on cognitive, mental and physical abilities. They have to understand the basics of the game, they have to pay attention to elements appearing on the screen, they have to be prepared for challenging situations, and the person in the role of the walker has to continuously step on the Balance Board, which requires physical exercise. \\ \\
SilverPromenade has an easy and understandable user interface that is intuitive also for people without any gaming experience. It consist of an menu structure which easily guides the user to playing-mode. The input devices used for playing makes it possible for elderly to both sit and stand during game-play, which takes the individuals abilities into consideration. Complex graphics and visual effects are avoided, while important elements are highlighted. It one of the three roles is not suitable, SilverPromenade offers the possibility of not including one or more roles. \\ \\
Gerling et al performed a case study with SilverPromenade where they let the elderly play the game and afterwards fill out a short questionnaire. During the case study they examined three research question related to interface design, game design, and player experience. Two groups of elderly participated in the case study, one group consisting of 9 seniors with prior experience with this type of  technology, and one group with 9 seniors without any experience. The participants suffered from age-related changes, and most of them needed help from devices to walk. During the game play they observed how the elderly used the controllers, how they understood the menu, and how easily they perceived the game behavior. The gaming results were also observed. \\ \\
Results from this case study shows that there is a clear difference in performance between experienced players and inexperienced players, both in single-player mode and multiplayer mode. Experienced players clearly had an advantage. The one role that was handled equally well was the role of the shaker in single-player mode. Some observations that were done under game play was that some roles were difficult to execute. Players in the walker roles experienced difficulties with performing correct movements on the Balance Board, and players in the pointer role had difficulties using the Wii Remote because it requires that one point it directly towards the sensor. However, the results showed that the overall experience was positive. SilverPromenade gave an impression of being outside, and the use of a real-world scenario engaged elderly to play. The concept of the game lead to communication in terms of identifying objects, and helping and encouraging each other. One important observation was that the elderly where sharing and discussing their scores. The conclusion of the case study is that elderly enjoyed playing digital games, and that SilverPromenade could be an appropriate game to use in this age group. 