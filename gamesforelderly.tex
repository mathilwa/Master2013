\chapter{Exergames for Seniors}
\label{chap:exforseniors}

\section{Related Research}
Exergames for elderly has become a popular topic in the past couple of years, and several research on how games can be developed for this particular user group have been conducted. It is quite common today to develop technology systems to a homogenous user group. This means that the characteristics of specific user groups, like the elderly, are being ignored. Elderly in particular, have some special characteristics that needs to be taken into account when developing technology systems for them. Most video games existing today have not taken these characteristics into account, and are therefore not suitable for this group. In this section, we will review some interesting literature that addresses what to consider when developing technology systems, and in particular video games, for elderly.  

Billis et al. \cite{Billis} discuss some important issues that need to be taken into account when developing games for elderly. Elderly often suffer from decline in visual acuity, decreased audition, mobility changes and cognitive functions' decline. In addition, many elderly are not familiar with technology. The writers of the paper suggest that it should be possible to customize the game for every players' special needs. Font, size and color should be adjustable, and information should be provided in different multimedia alternatives, like text, voice and images. The objects should be of sufficient size and the elements should not move too fast. The overall interface should be as simple as possible, without the need to remember information given earlier in the gaming process, and it should be given sufficient information and guidance throughout the whole game. Games should also provide motivating messages to encourage the player. The writers also stress the importance of the social factors of the game, and suggest the ability to multiplay. At last, for the players to get interested and engaged in the game, they suggest that the content of the game should match the users' cultural and lifestyle diversity \cite{Billis}.

de Bruin et al. \cite{bruin} write about the potential of virtual reality environment, with use of for example games, for exercise. Virtual reality platforms can provide naturalistic movements in a safe environment that can be customized after the patients' needs. It can offer a consistent program that enables for comparison over time. In addition, the use of games can distract the player from any pain they may have. They suggest stepping exercises to be suitable, because they have found that stepping exercises can be a good predictor of falls. It is also proved that a repetitive training program with stepping exercises can improve balance in elderly. Like in other research (sett opp noen kilder jeg finner her), they also here express that the problem of already existing exergames is that they are too complex for the older user group. Therefore, there is a need to develop games specifically for this group where physical and cognitive limitations, as well as typical interests of elderly, are taken into consideration. The writers also present a study where it was shown that there was a significant decrease in relative \ac{dtc} of walking for elderly who was training physically combined with a virtual reality dance game that required decision making, while training traditionally did not change this walking parameters. This comes from the fact that elderly often faces problems when they have to do more than one task at the same time \cite{bruin}.

Gregor et al. \cite{gregor} discuss some particular issues when designing for the older population, and propose a paradigm and methodology to support the process of designing software as close to the universal accessibility ideal as possible. These methods are called: \ac{d3} and a modified version of this: \ac{usid} \cite{gregor}.

The writers of \cite{gregor} describe older people through three different groups:
\begin{itemize}
\item Fit older people: elderly who do not suffer from any diseases or dysfunctionalities, but who are different from when they were young.
\item Frail older people: elderly who in  general have a reduction in many of their functionalities and who often have one or more disabilities.
\item Disabled people who grow older: These are the people who have long-term disabilities which have affected their ageing process.
\end{itemize}

In addition, \cite{gregor} define some important characteristics of older people:
\begin{itemize}
\item As people get older the individual variability of physical, sensory and cognitive functionality will increase. 
\item Functional decline will go faster when people gets older. 
\item Cognition problems, like memory dysfunction and the ability to learn new things, are widely appearing.
\item Based on where they are in life, elderly may have different wants and needs. 
\item How people live, like if they are needing a walking frame or needing warm glows, can change their usability function.
\item Elderly have more life experience than younger people, and can therefore have more knowledge of the world, as well as a more mature ways of solving problems. 
\end{itemize}

Most software design is static with no possibility to adapt to the different needs of the users. \ac{ucd} principles should be followed when designing technology systems. However, these principles have been developed for homogeneous user groups, and not for specific types of users. To make it easier to develop a technology system for the older user group with different characteristics, \cite{gregor} propose a modified version of the \ac{ucd} principles, which they call \ac{usid}. The issues this methodology address are (Directly drawn from \cite{gregor}): 
\begin{itemize}
\item "Much greater variety of user characteristics and functionality". 
\item "Finding and recruiting "representative users"". 
\item "Conflicts of interest between user groups (including "temporarily able-bodied")".
\item "The need to specify exactly the characteristics and functionality of the user group".
\item "Tailored, personalisable and adaptive interfaces".
\item "Provision for accessibility using additional components (hardware and software)".
\end{itemize}

As an example of the advantages of the proposed methodology, the writers present a case study were a  web browser for people that were visually impaired was developed. 200 visually impaired users evaluated the system.  The findings they did and the conclusions they took out from this study were \cite{gregor}: 
\begin{itemize}
\item Elderly seemed to lack confidence in handling IT systems. However, the confidence increased after they had experienced a successful interaction and decreased after experiencing an unsuccessful interaction.
\item Many elderly had difficulties remembering too much information. This indicates that there are important memory related factors that need to be taken into account when designing for elderly. 
\item Following a need for less information, will most likely also mean less functionality. From an other study they found that it was a need for the possibility that more functionality could be added after the user had mastered the initial, simple functionalities.
\item After some kind of assessment is passed, the user can be moved to a higher level. This can be done for example by self-assessment. To reinforce user-confidence the user should be able to reach goals. 
\end{itemize}

Gerling et al. \cite{gerling1} discuss the chances and challenges when developing a game for the elderly users. They suggest four major guidelines that should be followed when designing games for elderly:
\begin{enumerate}
\item The player should have the possibility to interact with the game both when sitting and standing. 
\item Avoid too extensive and sudden movements.
\item It should be possible to adjust the level when it comes to difficulty, game speed and device sensitivity. This changes should be possible to be adjusted by the player. 
\item Interaction mechanisms should be simple, player frustration should be avoided, and the game should provide constructive feedback.
\end{enumerate}

To verify these guidelines, the research group prototyped an exergame called SilverBalance. This game was made for the Wii Balance Board, consisted of two balance tasks and had the possibility to be played both sitting and standing. The game was tested on 9 seniors with an average age of 84. The following observations were made \cite{gerling1}
\begin{itemize}
\item All of the participants were able to play the game adequately. 
\item Participants expressed that the fact that the design was so minimalistic, made it possible for them to focus on the purpose of the game. 
\item The possibility to sit while playing the game was necessary because all participants were dependent on devices to assist them when standing and walking.
\item The players started to compare their results and comment on each others results.
\item Impairments and diseases made it difficult for some participants after a longer period of playing, suggesting that alternative interactions should be included in such games.
\end{itemize}

The tesing of SilverBalance shows that the four criteria contributes to the development of exergames for elderly, but that further studies on how age-related changes can affect games, should be carried out \cite{gerling1}.

In another study Gerling et al. presents a case study where they introduce and evaluate another video game for elderly, called SilverPromenade \cite{gerling2}.  SilverPromenade is developed for the Nintendo Wii technology and utilizes the Wii Remote and Wii Balance Board. The game is stripped from complexity in functionality and design to be senior-friendly. The theme of the game is "a walk through the forest" and it can be played as a single-player game or a multi-player game. In each mode of the game it is possible to engage three different roles, walking through the forest by stepping on the Wii Balance Board which requires physical exercise, catching a butterfly by pointing at it with the Wii Remote, and counting rabbits by shaking the Wii Remote when a rabbit appears. The scenarios are simplistic, which is important when including elderly in digital game play. The concept of the game, "a virtual walk in the forest", was chosen because it appeals to elderly living in nursing-homes because it offers the possibility to "visit" the forest, which generally might be inaccessible for them. Playing this game requires the older player to focus on cognitive, mental and physical abilities. They have to understand the basics of the game, they have to pay attention to elements appearing on the screen, and they have to be prepared for challenging situations. 

SilverPromenade has an easy and understandable user interface. Complex graphics and visual effects are avoided, while important elements are highlighted. It consist of a menu structure which easily guides the user to the playing-mode. The input devices used for playing makes it possible for elderly to both sit and stand during game-play, which takes the different individuals abilities into consideration.  If one of the three roles is not suitable, SilverPromenade offers the possibility of not including one or more roles.

A case study with SilverPromenade was executed on a group of frail elderly living in full-time nursing homes. The participants were asked to play the game and afterwards fill out a short questionnaire. During the case study they examined three research question related to interface design, game design, and player experience. Two groups of elderly participated, one group with prior experience with this type of technology, and one group without any experience. The participants suffered from age-related changes, and most of them needed assistive devices to walk. During the game play the researchers observed how the elderly used the controllers, how they understood the menu, and how easily they perceived the game behaviour. The gaming results were also observed.

Results from this case study show that there is a clear difference in performance between experienced players and inexperienced players, both in single-player mode and multi-player mode. Experienced players clearly had an advantage. Some observations done during game play was that some roles were difficult to execute. When walking on the Balance Board they experienced difficulties with performing correct movements, and when pointing at the butterflies they had difficulties using the Wii Remote because it requires that one point it directly towards the sensor. However, the results showed that the overall experience was positive. SilverPromenade gave an impression of being outside, and the use of a real-world scenario engaged elderly to play. The concept of the game lead to communication in terms of identifying objects, and helping and encouraging each other. One important observation was that the elderly where sharing and discussing their scores. The conclusion of the case study was that elderly enjoyed playing digital games, and that SilverPromenade could be an appropriate game to use in this age group \cite{gerling2}.

\section{Summary of Findings from Related Research}
\label{sec:summaryguidelines}
It is clear from the literature that elderly has some specific characteristics that need to be taken into account when developing technology systems aimed for them. Based on the reviewed research we will summarize important aspects when developing technology systems, and in particular video games, for elderly. In \cite{bruin} and \cite{gerling2} they discuss stepping as a relevant exercise for elderly. This is because this kind of exercise has proved to be a good predictor of fall and that repetitive stepping exercises can improve balance. In addition stepping, requires significant physical activity, which is important when improving physical health. Walking or stepping exercises is also recommended by the Department of Health and Human Services, discussed in Chapter \ref{chap:olderexercise}.

It is important to remember the limitations related to elderly users. We will now summarize the different aspects discussed in the literature. First we will list some of the typical characteristics of the older user. Then we will provide a list of relevant guidelines to be followed when developing user interfaces for elderly. 

\subsection{Characteristics of elderly people:}
\begin{itemize}
\renewcommand{\labelitemi}{$\bullet$}
\item Cognitive functions' decline, e.g. dementia, memory dysfunction, and the ability to learn new things \cite{Billis}, \cite{gregor}.
\item Decline in sensory functionality, like  visual acuity and audition \cite{Billis}, \cite{gregor}.
\item The problems elderly face, often increase significantly with increasing age \cite{gregor}.
\item Mobility change \cite{Billis}.
\item Needs and wants related to cultural and lifestyle diversity, as well as at what stage they are in life \cite{Billis}, \cite{gregor}.
\item Many are in need of assistive tools (for example when standing and walking) \cite{gregor}.
\item Inexperienced with technology. Many elderly seems to lack confidence when it comes to IT-systems. A better confidence can be achieved when experiencing a successful interaction with an IT-system \cite{Billis}, \cite{gregor}.
\item It is important to take into account memory related factors. Many elderly have difficulties remembering too much information \cite{Billis}, \cite{gregor}.
\item It can be hard to do two things at the same time \cite{bruin}.
\end{itemize}

\subsection{Guidelines when developing technology systems for elderly}

\begin{itemize}
\renewcommand{\labelitemi}{$\bullet$}
\item The game should have the possibility to be customized for every user's needs, condition, interests etc. This can be done by offering alternative interactions \cite{Billis}, \cite{gregor}, \cite{gerling1}.
\item Offer adjustable font, size and color \cite{Billis}.
\item The interface should have different alternatives for multimedia presentation (for example text, voice and images) \cite{Billis}.
\item The interface should be simple and not too extensive. The objects should be of sufficient size and there should be no sudden movements. Important elements should be highlighted \cite{Billis}, \cite{gerling1}, \cite{gerling2}.
\item Sufficient guidance and information should be given during the process, without the need to remembering earlier giver information \cite{Billis}, \cite{gregor}.
\item Constructive feedback should be given in a motivating form, to encourage play \cite{Billis}, \cite{gerling1}.
\item The story of the game should match cultural and lifestyle diversity. An example presented in \cite{gerling2} is a game with a real-world scenario, where the players have to walk through a forest. This seemed to appeal to elderly living in a nursing home, because they do not have the same possibilities to "just take a walk" in the forest \cite{Billis}, \cite{gregor}, \cite{gerling2}. 
\item Social factors should be included, by for example offering the possibility to multiplay \cite{Billis}, \cite{gerling2}, \cite{gerling1}.
\item It should be offered variety in user characteristics and functionality \cite{gregor}, \cite{gerling1}.
\item The game should not have too much functionality, but instead offer the possibility to add more functionality after the existing functionalities are managed \cite{gregor}, \cite{gerling2}.
\item Offer different levels with different difficulties. With this comes the possibility to reach goals. The device sensitivity and the game speed should also be adjustable \cite{gregor}, \cite{gerling1}.
\item It should be possible for the player to adjust levels themselves \cite{gregor}, \cite{gerling1}. 
\item It is important during the development to test on representative users \cite{gregor}.
\item The possibility to interact with the game both when sitting and standing is important \cite{gerling1}, \cite{gerling2}.
\end{itemize}

The guidelines listed from the literature will be used when we are developing a concept for an exergame for elderly. In Chapter \ref{sec:designelderly} we will discuss more specific guidelines when it comes to user-interface for elderly. 





