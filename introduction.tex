\chapter{Introduction}

The world's population is ageing, in addition to people living longer. This together with baby boomers has led to a great share of older people in the world. In 2007, 20 percent of the population in the developed world was over the age of 60 \cite{dickinson2007methods}, and it is expected that 25 percent of the population in Europe will be over the age of 60 by 2020 \cite{ijsselsteijn2007digital}. A problem with such a great share of elderly in the world, is that it will be difficult to provide health care services to everyone that might need help, due to physical and psychological decline. Cognitive impairments, decreased mobility functions, and visual and auditory decline are common problems related to ageing and the older population. One common problem for this group of people is increased risk of falling, due to reduced balance and physical strength. This is an important problem to take into consideration, as falls are one of the main factors when it comes to loss of function, independence and quality of life for elderly [Ganske likst som kilden (smartsenior)]. The main focus should be to make the older population healthier, so that they can experience greater quality of life, where strong health is the key to achieve this. In Norway, a goal for the older population, is to offer everyone that needs it, a place in care homes by 2015. If this is to be possible, something has to change.     

Use of technology for different purposes has become an important part of peoples' everyday life. One of these purposes, is use of video game technology to enhance people in physical activity. Exercise is an important aspect in slowing down older peoples ageing effects, as decline in mobility, balance and physical strength. The introduction of an exergame could be part of the solution of getting an healthier older population. The Norwegian "Samhandlingsreformen" supports this solution, by including welfare technology in the health sector where possible. However, many older people do not use technology as computers, mobile phones, tablets, and maybe especially video games, as it is very unfamiliar to them. In addition, most technology developed today are aimed towards younger user group, and are therefore not suitable for elderly. A first step to make the older population accept the use of video games for health related purposes, would be to develop a game aimed especially for their needs and interest, where their physical and psychological declines are taken into consideration. 

Based on this we will in this thesis focus on gathering sufficient knowledge and information to design a video game concept, with special focus on elderly and exercise.    

There are a lot of important aspects around a game developed for the older user, as we will be discussed in Chapter \ref{chap:exforseniors} and \ref{sec:designelderly}. First, in Chapter \ref{chap:olderexercise}, we will discuss some challenges related to engage the older population to exercise, to see what kind of barriers we will be confronted with. We will also discuss typical motivational factors for getting older people to exercise.  

\section{Research Questions}
\label{sec:researchq}

Previous research suggest that commercial video games are not suitable for elderly and exercise. 
\emph{RQ1: Are existing commercial Xbox Kinect games suitable for exercising purpose for the senior user group?} 

\emph{RQ2: What are the design challenges when developing video games aimed for exercising for the senior user group?} 

Based on theory and knowledge about video game development, w
\emph{RQ3: What would be suitable system requirements, and appropriate design, for a video game concept for the senior user group?}

\emph{RQ4: Is it possible in an early development phase to involve a user group that is inexperienced with the type of technology being developed?}


\section{Scope and Limitations}

\section{Contribution}

\section{Concepts and Terms}

\section{Outline}