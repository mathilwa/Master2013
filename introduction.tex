\chapter{Introduction}

Today, technology is used for all kind of purposes, and has become an important part of people's everyday life. Different kinds of technology meet a wide range of needs, all from entertainment and socialisation, to education and health. One focus that has got great attention the last years is how to use video games with motion sensor technology to engage people in exercise and physical activity. These games are called exercise games, or exergames, because they require body movements to play. There have been done a lot of research on different motion sensor technologies used for exercise and it has shown positive effect for health related purposes. 

Due to baby boomers and the fact that people are living longer, the world's population now consist of a great share of older people. As a part of growing older, elderly often meet disabilities like physical and psychological decline. Due to reduced balance function and physical strength, one serious problem for elderly people is the risk of falling. Fall is the leading cause of injuries in older people, and can often have serious consequences. These problems, together with the world's ageing population, lead to both strategic and economic challenges for the government when it comes to providing health care services to everyone that need it. This clearly shows that new initiatives have to be taken to figure out how to prevent falls, and keep the older population healthy. However, engaging the elderly population in physical activity can be challenging, and it is shown that a great share of elderly do not exercise enough. The introduction of an exergame can meet this problem. This can serve as an alternative way to exercise which is more motivating and fun, than regular exercising.  

The use of exergames for health related purposes is supported by the new reform "Samhandlingsreformen" in Norway, were one focus is about using welfare technology in the health sector where possible. However, many older people, are unfamiliar with technology, and especially with video game technology. In addition, the existing commercial exergames are aimed towards a younger user group, and are therefore not suitable for elderly. A first step to make the older population accept the use of video games for health related purposes, would be to develop a game aimed especially for their needs and interest, where their physical and psychological declines are taken into consideration. 

This thesis is build upon our project assignment \emph{Business Opportunities and Economics for an Exercise Game in the Health Sector} \cite{project}, where we developed a business model for an exergame, with the physiotherapy service as the customer. In this thesis we will focus on the end users of this game, the elderly population. Our contribution in this thesis will be to explore and identify the users' needs and wants, gather sufficient knowledge and information about system design, and from this create a concept for an exergame proper for the older user group.

\section{Objectives}
\label{sec:researchq} Exergames have shown promise in health related purposes like exercise and rehabilitation. However, previous research have found that commercial exergames are not suitable for elderly, as they are not specially aimed and developed for this user group. 

In this thesis, we will study how elderly interact with commercial video games to see what do and do not work, and to validate previous findings. With this we will answer research question 1 and 2: 

\emph{RQ1: Are existing commercial Xbox Kinect games suitable for exercising purpose for the senior user group?} 
\emph{RQ2: What are the design challenges when developing video games aimed for exercising for the senior user group?}

Based on theory and knowledge about video game development, in addition to findings from our study, we will try to answer research question 3:

\emph{RQ3: What would be suitable system requirements, and appropriate design, for a video game concept for the senior user group?}

\section{Contribution}
Our contribution in this thesis has been to develop a design for an exergame concept aimed at the elderly users. We have used aspects from our previous project assignment as motivation for our work, which are summarised in Chapter \ref{chap:background}. To be able to understand the elderly users, we have looked into which challenges we might meet when it comes to elderly and exercise. We will also discuss typical motivational factors for getting older people to exercise. This will be presented in Chapter \ref{chap:olderexercise}. We have studied important aspects related to game developed for the older user, as we will be discussed in Chapter \ref{chap:exforseniors} and \ref{sec:designelderly}. To get a thorough basis for our exergame concept, we have summarised a set of guidelines from related literature, studied video game theory and system design in general. This can be found respectively in Section \ref{sec:summaryguidelines}, Chapter \ref{chap:vg} and Chapter \ref{chap:generalsystemdesign}. In accordance with the importance of user involvement, we have conducted two workshops where we have engaged users in the development of the exergame concept. We recruited informants for our workshops by holding a presentation for "Seniornett", an organisation working with learning the older population technology. This will be described in Section \ref{sec:recruitment}. Workshop 1 was conducted to observe and understand how elderly would interact with commercial Xbox Kinect games. In this workshop methods like questionnaire, participatory observation and focus group interviews were used. These are described in \ref{chap:metode}. We let them play games, and followed with a focus group interview. A more detailed description of the execution, and findings from workshop 1, will be presented in Section \ref{sec:ws1} and \ref{chap:findW1}, respectively. The main focus for this thesis have been to specify system requirements and create design for a concept for an exergame for elderly. This work has been based upon information gathered from theory, previous studies and findings from workshop 1. The design was presented by making prototypes. The exergame concept is the most important contribution in this thesis, and will be presented in Chapter \ref{chap:concept}. The concept was brought to a second workshop, to be evaluated by the users. Execution of and findings from workshop 2 are presented in Section \ref{sec:ws2} and Chapter \ref{chap:findW2}. In our discussion we suggest a detailed description of future work for an exergame for elderly, based upon findings from workshop 2. This thesis provides general guidelines for how to design an exergame for elderly. The concept presented can serve as a foundation and starting point for developing an exergame for elderly.     

\section{Scope and Limitations}
Initially, we worked in collaboration with a company who was going to develop an exergame for elderly. Our work in this collaboration was to come up with a concept for their game. They were suppose to provide us with a library of exercises that are shown to be good for elderly, and that the game was meant to be build around. Unfortunately, this collaboration ended two months into the work period, which slowed down our progress.  One important limitation of our thesis was the sample of informants in our workshops. Elderly can not be seen as one common group of people. They are people with different interests and needs. In this thesis, we only included one small group of elderly. As the first workshop was critical for our further work, we did not have much time recruiting informants. Therefore, we recruited only informants from the organization "Seniornett", where all have one common interest for technology. We recruited informants through a presentation we held at one of their meetings in Trondheim. This gave us a small number of informants. Due to the time limit, we did not use more arenas for recruiting. The informants had never played video games before, which made it challenging to include them in the development process of our exergame concept. Putting a group of elderly in setting playing the different Kinect games, is not a natural setting. This made it difficult to fit this study into defined research methodologies. The last limitation concerns the choice of how to present our design. We used prototypes, but limited them to not involve any interaction, even though they represented scenarios for a highly interactive game.  


\section{Outline}