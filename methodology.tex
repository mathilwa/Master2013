\chapter{Methodology}
\section{Qualitative Research}
Qualitative research method is used to give an understanding of a phenomenon and is well suited when studying sensitive and personal topics. Interpretation is therefore important in qualitative method.  The focus lies on how and why things are done, and not how many who does it, like in quantitative methods. Quantitative methods include a huge sample, while qualitative research can give more information about a small sample. While in quantitative methods structuring is very important, flexibility, in the sense that the scheme can be changed during the research if needed, is important in qualitative research. Also different from quantitative research is that it is not common to use numbers because the number are too small. The process can be divided into phases, that partially overlap. The first phase consist of defining what the research will find out. The next phase is the data gathering phase. This phase can be performed by different methods. The last phase includes interpreting and analysing of the data as well as formulating theories. In the last phase, the results will be presented. Data gathering and analysis should be done in parallel so that further data gathering can be adjusted from what have been found in earlier analysis. It is common in qualitative research that close connection between the researcher and the people who is being studied. Qualitative research is well suited for topics that there has not been done a lot of research on, because on this kind of topics, openness and flexibility is required. The different data gathering methods are divided in four \cite{qualitative}: \\ \\
- observation \\
- interview \\ 
- document analysis, and \\
- analysing of video and audio recordings \\ \\
The most common methods used are interview and participatory observation. Common for  these methods are that the researcher will establish direct contact with the people who is being studied. ,This contact is important for the data the researcher will gain from the study \cite{qualitative}. \\ \\
It is common for most qualitative methods is to document the data to be analysed textual. The documentation can include what people do, their statements, their intentions or their perspectives. The text can be notes from the field, printouts of  recorded interviews. \\ \\
The close contact established between researcher and informants introduces some ethical  challenges, described in the next subsection. Ta med noe om systematikk og innlevelse? All results conducted from the research needs to be precisely and correctness in the presentation of the results and when other researchers work is being evaluated. This includes not plagiarize other people's work, which means to copy other people’s work and take credit for it. When referring to other people’s work, it is important to properly state the resource. When working in close contact with the informants, the researcher often gets personal information about the informants. With personal information we mean information that can be related to individuals. In project were personal information needs to be handled the project needs to be reported. In research projects performed at universities, where personal information is being handled, the project needs to be reported to “Norsk Samfunnsvitenskapelig Datatjeneste” (NSD), which is in care of data protection for these institutions. NSD will evaluate each project in accordance to research ethical rules \cite{qualitative}. \\ \\
\subsection{Ethical Challenges:}
As mentioned, there are some ethical guidelines that needs to be followed when working in close contact with informants. These will be described in this section. \\ \\
\textbf{Informed Consent:} \\
In a research project with people involved, these is a requirement that the researcher has the participant’s informed consent. This means that the informants have gotten all the information they need to know about the participation and have self chosen to participate and that they can withdraw at any time without any consequences. One challenge about this, is that in some projects too much information can affect the participants behaviour (If they for example know too much about what the researchers are looking for). In this thesis we does not see this as a problem, and the participants will be informed about every important aspect of the study \cite{qualitative}. \\ \\
\textbf{Confidentiality:}\\
Researchers are required to keep all the information they collect about a participant confident. This means that the information have to be anonymized. This also involved strict requirements to how personal data that makes it possible to identify individuals should be stored and annulled. There are rules about how long data can be stored, and general principles are that data should be stored for only the amount of time there is use for the data, and that data that can be directly linked to an individual should be stored separately and not electronically.  Reuse of data is not allowed without consent from the participants \cite{qualitative}. \\ \\
\textbf{Consequences of participating in research projects:}\\
The researcher has responsibility over the participants safety and should respect their wishes. It is important to have thought through what consequences the execution of the research has for the participant. The researcher is required to protect the participants’ integrity during the process \cite{qualitative}. \\ \\
\subsection{First Phase} 
The first phase in a qualitative research method is to define what wants to be found from the research. A problem description (research questions?) should be prepared and a design for how the research will be performed should be developed. The design consist of guidelines of how the project should be carried out and includes: what the study will focus on, who are the informants, where the study will be performed, and how the study will be performed. Vet ikke om vi skal ha med noe om prosjektbeskrivelse? det går på hvem som skal jobbe, hvordan finansiere osv..A problem description needs to be developed in order to know what the study should be focusing on. This will contain research questions that the researcher wants to get information about. The problem description should be clearly prepared and it should be limited realistically within the framework of the project, and at the same time be open enough so that interesting topics that will appear during the process can be studied. It is important to that the problem description is being modified and worked with during the research process, because the researcher will gain new knowledge and understanding during the process that can be important for the research. The choice of problem description should be justified be describing why the problem the researchers will study is important. 
\subsection{Observation}
The observation method is used when the researcher wants to see how a group of people behave in a specific setting. In observation studies, is one important decision how the observer will perform in the field. This varies from project to project. The observer can be a participant or just an observer, and the observation can be open or undisclosed. Participation observation is something in between being a complete observer and complete participation, where the researcher participate in the same way as the informants. This involves the researcher being present in the setting of the participants and observe how they act. The researcher participate in the session, in the sense of interacting with the participants while they are performing the tasks. This means that the researcher does not do the same as the participant. Participation observation is well suited in research of a new and immature topic. It is very common to combine observation with interviews. This is for the researcher to verify or discard the understanding he or she has acquired during the observation. In most research it is important to study the behaviour in the informants own environment. This will give a more natural behavior. However, it is important to acknowledge that the informants may not find the environment “natural” when the researchers are there observing them. How the researcher presents the project for the informants is important for the people the researcher wants to observe will be willing to participate. The researcher should in a trusting way, present him or herself and the project.\\ \\
We will use the observation method to see how elderly respond to the use of Xbox Kinect. We will look for things that supports what we have read in the literature, as well as identify new aspects that we were not aware of. This is for us to get an understanding of what works, and what does not work with existing commercial Kinect games.
\subsection{Qualitative Interviews}
With an interview, a researcher can gain comprehensive information about peoples views and opinions about a topic. Interview is one one of the most important tools used in qualitative research. There are two extremities in interview methods: unstructured and structured interviews. The former is more like a conversation between the researcher and the informants. The topic is chosen, but there are no interview guide involved. The questions can be adjusted during the interview. The latter has a structured form where the questions are chosen beforehand. The advantage with this method, is that all informants will answer the same questions, and therefore the answers can be compared. The third, and most common, type of interview is semi-structured interviews. This method is something in between the two extremities. This type of interview is called qualitative interviews. This method includes a interview guide, where some questions are decided beforehand, but the order in which the questions are asked, are chosen during the interview. In this way it is easier to follow the story of the interviewee. In addition, the researcher needs to be open to discuss other topics that might appear during the conversation. The most common interview setting is with one individual at the same time. However, group interviews are becoming more common. In a group interview several people discuss a topic, while a researcher serve as a moderator (ordstyrer?). Kan evt. skrive mer om gruppeintervju hvis det er det vi ender opp å gjøre. To get a successful interview, it is important that the researcher has an understanding of the informants’ situation so that the questions asked will be relevant. Questions should be asked in a way so that the informant can reflect over the question and not only answer “yes” or “no”. 
Evt. ta med pitfalls her (fra forrige oppgave). 

