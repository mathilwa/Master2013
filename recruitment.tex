\chapter{Recruitment of Participants for the Workshops}
To be able to perform a workshop we had to recruit participants from the appropriate user group for our exergame. This chapter is about how we held a presentation to recruit participants from "Seniornett". We  will describe the execution of and feedback from this presentation. We will also present information about what "Seniornett" is and does.  

"Seniornett" is an organization in Norway which work to include the older generation (55 years +) in the emerging information technology and to help them gain digital competence. The organization was started in 1997 and is represented in every county in Norway \cite{seniornett}. Even though this group includes seniors who are interested in learning about new technology, something not every senior is, we decided that we wanted to start with this group for our workshop. This group may still have the same physical limitations, but may have a higher understanding of technology than other seniors. Every semester the local clubs arrange 3-4 meetings with different topics related to technology.

We contacted the manager of the club in Trondheim. He invited us to present the topic of this thesis in their next meeting, which was 02.25.13. This presentation would both serve as a contribution to this club, as well as promotion of our project. The latter was mostly to gain interest in participating in the planned workshop. The presentation slides can be found in appendix B. We will now shortly present what we presented in the meeting as well as some of the comments and questions that arose during the meeting. 

The main focus of the presentation was not our master thesis, but rather about the evolution of computer and video games, and how these can be used for different purposes. We had a brief description of our master thesis in the beginning of the presentation. After this, we introduced computer and video games. With the assumption that the audience had little knowledge or experience with this kind of technology, we started out very easy, with the description of computer games, with examples like solitaire, scrabble and chess. We also presented "The Sims" as an example of a single-player game with very innocent actions, and "Counterstrike" as an examples of a more violent game, where the player is playing over the internet alone or in a team against an enemy. Later we discussed violent computer and video games. Even though most people think of violent games as something that affect the players in a negative way, we wanted to describe the positive aspects of violent games. We used World of Warcraft as an example. Violent games often requires some kind of cooperation in a team, where each player has to understand the other team members' strengths and weaknesses and coordinate these within the team. These are skills that are important also in real-life, and people playing these type of games, has shown to have good ability to obtain the same skills in real life. We decided that a short description of this topic with a positive view, was suitable in this setting, as most focus around this topic is negative. However, we did not get any comments or feedback on this topic. 

From computer games we moved over to video games. We described how video games has emerged from being sedentary games to be more active games controlled by body movements. Because of the popularity around the Super Mario series, we decided to present two different versions of this game. We showed an old version in two-dimensional space and a newer version in three-dimensional space. This was to show the audience how games with the same concept, can change throughout the years. The audience seemed to like the video clips of Super Mario and laughed during the showing. One women asked if the clip was already recorded and where the player was in the game? We described that this was a clip that someone had recorded and published on YouTube and that the player was a human being that controlled Mario with a controller. We held up the controller for her to see. She immediately understood. It struck us that these kind of things that we, who are familiar with this kind of games, take for granted, is maybe not so intuitive for the older generation. Therefore, the next time we are talking to an audience about unfamiliar technology, we should be more precise in our descriptions. Further we showed a video clip of a car race game, called "Need for Speed". The audience also seemed to like this, but did not comment it. 

We also decided to talk about the topic "Serious Games" because exergames fall under this category. We believe a description of this gave exergames more credibility among the audience. We started by describing how games could be used in education, and we used a Norwegian mobile application, called "QuizBattle", as an example. We talked about the importance of focusing on the game story and entertainment, while keeping the pedagogics and educational factors as underlying factors. Further we described how games can be used as simulators in for example surgery, military actions and flight controlling. Finally we discussed exergames. We briefly talked about how dance pads have been taken in use in physical education at different schools. We also described how so called exergames emerged from being "just" fun and entertaining games to be more "serious" with focus on physical exercise after realising that these type of games actually gave health improvements.

From this, we naturally proceeded to describing Xbox Kinect. We described how this technology has gained acceptance within the health community, because of its accuracy and easiness, and that this is why this technology will be used in our project. We also presented the advantages and the limitations with the use of a  game like this for the elderly user, and explained why there was a need for making a customized game. One man told us that his wife, who is 72 years old, already was exercising with Nintendo Wii, and that it seemed like she enjoyed it.

To make it easier for the audience to understand what we presented, we provided a demonstration where we played Kinect Sports 2, showed on a big canvas. We played tennis and skiing. With tennis, we showed them how only one player could physically play against a computer controlled player. With skiing we showed how two players could play and compete against each other at the same time. The audience seemed to enjoy the demonstration and they seemed impressed. However, they had some concerns. One women expressed a concern when it comes to the setup of the game, and she wondered if she would need a projector. We explained that the only thing needed was the Xbox, the Kinect sensor, the games and either a television or a computer. We also admitted that the setup and introduction of the game are not very intuitive today, and that this will be something we will consider when developing a game concept. A man asked how much the game costs, and we answered that a package with the Xbox, a Kinect sensor and two games (Kinect Sports 2 and Kinect Adventures), costs about 2000 NOK. A woman expresses a concern when it comes to dizziness when playing the ski game. She thought it was too many details. She did not think these games would fit a person with decreased balance. We confirmed that it is not the first time this is mentioned about these kind of games. An other women had some comments about the music and the theme of the games. She felt that the games were too "childish" to appeal to her. She mentioned dance as a possible theme for a game, and a Norwegian folk dance called "Rørospols" where specifically mentioned. In addition, she expressed a wish for music that appealed to her generation. A third women said that she thought this game would fit best in a group setting and not necessarily at home. She expressed concerns about the space in her living room. Others from the audience agreed that the game would fit well as a group activity, because it would be more fun in that way.

After the discussion, we presented the planned workshop and invited the audience to help us with this. We told them that to make a user-friendly game, it is very important to involve the user, and that their help would mean a lot for the development of a game concept. In addition, we explained what was expected from them during the workshop and we explained that we would video record the workshop and sound record the interviews. We also made them aware that participating is voluntary and that the project is legally reported to \ac{nsd}. Everyone in the audience got a document containing information about the project and informed consent, see appendix Y. Four people signed up at that time. X people signed up later by e-mail.

After the meeting was ended we had some casual conversations with the audience. The same lady who suggested that there should be more suitable music for their generation, gave us an example from SATS, a fitness center in Norway, where she is regularly exercising. She told that they arrange weekly senior exercise groups, called "Senior Pulse", where they only used music more related to this generation's interests. An other woman expressed big concern when it comes to the pace of the game. She meant that this would be one of the biggest challenges. She suggested that a game for their generation should have the possibility to choose pace, so that the player could start slowly and speed up the pace after getting familiar with the game.

The meeting was was very useful for us. We got participants for our workshop and we got many useful questions and comment, that will be taken into account in our further work. See appendix B for more detailed description of the presentation and appendix C for more detailed notes taken during the meeting (both appendix B and C are only provided in Norwegian).
    