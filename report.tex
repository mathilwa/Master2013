\documentclass[b5paper,twoside,openright,11pt]{report}
\usepackage[printonlyused]{acronym}
\usepackage{graphicx}
\usepackage{enumerate}
\usepackage{cite}
\usepackage{float}
\usepackage[utf8]{inputenc}
\usepackage[hyphens]{url}
\usepackage{rotating}
\usepackage{array}
\usepackage{amsmath}
\usepackage{fancyhdr}
\usepackage{multirow}
\usepackage[parfill]{parskip}
\setlength{\headheight}{25.23pt}
\renewcommand{\headrulewidth}{0pt}
\pagenumbering{gobble}
\begin{document}
\begin{flushleft}
\begin{figure}[htb]
\includegraphics[scale=0.6]{NTNU-logo}
\end{figure}
\bigskip
\bigskip
\bigskip
\bigskip
\begin{huge}
\textbf{A Concept for an Exergame for Elderly People}\\
\end{huge} 
\bigskip
\bigskip
\bigskip
\bigskip
\bigskip
\bigskip
\bigskip
\begin{Large}
\textbf{Kine Aa. Omholt and Mathilde Wærstad \\}
\end{Large}
\bigskip
\bigskip
\bigskip
\bigskip
\bigskip
\bigskip
\begin{large}
\textbf{Master Thesis\\}
\end{large}
Delivered: \today\\
Professor: Lill Kristiansen\\
Supervisor: Ather Nawaz\\
\bigskip
\bigskip
\bigskip
\bigskip
\bigskip
Norwegian University of Science and Technology\\ 
Faculty of Information Technology, Mathematics and Electrical Engineering\\
Department of Telematics
\end{flushleft}
\cleardoublepage
\begin{abstract}

The aim of this study is to identify important aspects for an exergame for elderly, and based on this, specify system requirements and develop a concept for an exergame. Due to "baby boomers" and the fact that people live longer, the share of older people in the world is growing. One common problem elderly people face is reduced balance function and physical strength, which increase the risk of falling for this group. Engaging elderly in physical activity can help overcome this problem. However, this can be challenging. Today, there is a great focus on the use of technology for health related purposes, including using video games with motion sensor technology to promote physical activity. These types of games are called exercise games, or exergames, because they require the player to use body movements to play. The Microsoft Kinect sensor is a console that has shown promise as a tool for use in clinical practice. In a previous project we evaluated exergames used as a tool in physical therapy. Apparently, customised games are required for the elderly population. The commercial exergames that exist are made for a younger user group, and many games are not suitable for the older population. It is important to identify aspects of exergames for elderly for making valuable games for them. In this master thesis we have developed a concept for an exergame for elderly people. Research on relevant topics has been conducted, to find important aspect for the game. We have looked into guidelines for how elderly should exercise, as well as motivating factors for exercising in general. We have studied related research, identifying different characteristics of elderly, as well as proposing guidelines on how to develop and design exergames for this group. In addition, official guidelines proposed by three different organizations on how to design user-friendly interfaces for elderly are provided. In the development of an exergame for elderly, usability and simplistic design become significant. Meeting this, the relevant users have to be involved in the development process. We have included a group of elderly in one iteration of the cycle of user-centered design. This was done by arranging two workshops. The first included an experiment, where a group of elderly people got to play three different commercial Xbox Kinect games. During the gaming session, we observed how they interacted with the games, what did and did not work, and how they seemed to like the games. A focus group interview was performed after the gaming session. From workshop 1 we learned that the group liked playing the games, but that there were important aspects that could be improved, such as more instruction and feedback, easier menus, more information about what was expected from them, which body parts that were exercised, and whether they did the exercises correctly or not. There was a desire for activities that could be related to real life, with appropriate music. Clear goals and the feeling of mastery were seen as important. Based on the findings from workshop 1 and the research on relevant topics, system requirements have been proposed and a design for an exergame concept has been developed. The exergame includes one compounded game, exercising the whole body, and four single games, exercising specific muscle groups. The games are made with relevant exercises in a familiar environment. Early stage prototypes were made to visually show what different scenarios in the games will look like. In addition, a menu proposal was provided. To include the users, and to get feedback on the concept, a second workshop was held. The exergame concept was presented by showing prototypes, simulating gameplay and explaining scenarios. Focus group discussions were held, to get feedback on the exergame. The overall perception was positive, however, some aspects of the games were unclear, and some suggestions for improving the games were made. We conclude that existing commercial Xbox Kinect games contain some elements not suitable for the elderly user, and that they lack certain functionality. However, the games that contained real life activities were experienced as fun. The proposed system requirements are appropriate for this user group, and the exergame concept was appreciated. However, some adjustments should be made in the future work on the exergame. We acknowledge that the group of users involved was physically and mentally fit, and that their opinions and experiences may differ from another group with other characteristics.

\end{abstract}
\newpage
 \pagestyle{empty}
\renewcommand{\abstractname}{Sammendrag}
\begin{abstract}
Målet med denne studien er å identifisere viktige aspekter for et treningspill for eldre, og basert på dette, spesifisere systemkrav og utvikle et konsept for et treningsspill. På grunn av "baby boomers" og det faktum at folk lever lenger, har andelen av eldre mennesker blitt større. Et vanlig problem eldre mennesker ofte møter redusert balansefunksjon og fysisk styrke. Denne reduksjonen øker risikoen for å falle for denne gruppen. Å engasjere eldre i fysisk aktivitet kan bidra til å overkomme dette problemet. Men dette kan være utfordrende. I dag er det et stort fokus på bruk av teknologi for helserelaterte formål. Dette inkluderer bruken av videospill med bevegelsessensorteknologi for å fremme fysisk aktivitet. Disse typer spill kalles treningsspill fordi de krever at spilleren må bruke kroppsbevegelser for å spille. Microsoft Kinect-sensoren har vist seg å være et lovende verktøy for bruk i klinisk praksis. I et tidligere prosjekt evaluerte vi treningsspill som et verktøy i fysioterapi. Det viser seg at et tilpasset spill trengs for eldre og at de kommersielle treningsspillene som finnes er laget for en yngre brukergruppe og at disse spillene ikke er egnet for eldre mennesker. Det er viktig å identifisere hvilke aspekter som er viktig for et trenignsspill for eldre, for å lage passende spill for dem. I denne masteroppgaven har vi utviklet et konsept for et treningsspill for eldre mennesker. For å identifisere viktige aspekter for spillet, har forskning på aktuelle temaer blitt gjennomført. Vi har sett nærmere retningslinjer for hvordan eldre bør trene, samt hva som er motiverende faktorer for å trene generelt. Vi har studert relatert forskningsarbeid, identifisere ulike karakteristikker av eldre, samt foreslått retningslinjer for hvordan man bør utvikle og designe treningsspill for denne gruppen. I tillegg presenteres offisielle retningslinjer foreslått av tre ulike organisasjoner om hvordan å designe brukervennlige grensesnitt. I utviklingen av et treningsspill for eldre, er brukervennlighet og enkelt design viktig. For å møte dette, bør aktuelle brukerne bli involvert i utviklingsprosessen. Vi har inkludert en gruppe eldere mennesker i en iterasjon av sykelen for brukersentrert design. Dette ble gjennomført gjennom to workshoper. Den første inkluderte et eksperiment, der en gruppe eldre mennesker fikk spille tre forskjellige kommersielle Xbox Kinect-spill. Under spilløkten, observerte vi hvordan de samhandlet med spillene, hva fungerte og fungerte ikke, og hvordan det så ut til at de like spillene. Etter spilløkten hadde vi et fokus-gruppe-intervju. Fra workshop 1 lærte vi at gruppen likte å spille spill, men at det var viktige aspekter som kan forbedres. Eksempler på dette er mer instruksjon og tilbakemeldinger, enklere menyer, mer informasjon om hva som er forventet av spilleren, hvilke kroppsdeler som ble trent, og informasjon om øvelsene var gjort riktig eller ikke. De ønsket øvelser som kunne relateres til det virkelige liv, samt passende musikk. Klare mål og følelsen av mestring ble sett på som viktig. Basert på funn fra workshop 1 og forskning på aktuelle temaer, har systemkrav blitt foreslått og et design for et treningsspill-konseptet blitt utviklet. Treningsspillet inneholder et sammensatt spill som trener hele kroppen, og fire enkelspill som trener spesifikke muskelgrupper. Spillene er laget med relevante øvelser i et kjent miljø. Enkle prototyper ble laget for å visuelt vise de ulike scenarioene i spillene. I tillegg er en meny foreslått. For å inkludere brukerne og for å få tilbakemelding på konseptet, arrangerte vi en til workshop. Treningsspillet ble presentert ved å vise prototyper, simulere spill, og forklare scenarier. Fokusgruppediskusjoner ble holdt, for å få tilbakemeldinger på spillet. Den generelle oppfatningen var positiv, men det var noen aspekter ved spillene som var uklare, og noen forslag til forbedringer ble gitt. Vi konkluderer med at eksisterende kommersielle Xbox Kinect-spill inneholder noen elementer som ikke egner seg for eldre brukere, samt at de mangler noen funksjoner. Imidlertid, ble spillene som inneholdt virkelige aktiviteter opplevd som morsomme. De foreslåtte systemkravene er egnet for denne brukergruppen, og treningsspillet var likt. Allikevel, er det noen justeringer og forbedringer som bør gjøres i et videre arbeid med dette treningsspillet. Vi erkjenner at gruppen av brukere som ble involvert var fysisk og psykisk oppegående, og at deres meninger og erfaringer kan variere fra en gruppe med andre egentskaper. 
\end{abstract}
\cleardoublepage
\chapter*{Preface}
This thesis is written as a part of our master's degree in Communication Technology, with focus in Tele-economics, at the Norwegian University of Science and Technology (NTNU). The focus of this thesis has been to design a video game concept for elderly, and it is a continuation of the work done in our project assignment. 

We would like to thank our professor Lill Kristiansen for valuable guidance and feedback on our work during this final semester. The result of this thesis would not have been the same without her ideas and comments. We would also like to thank Post Doc Ather Nawaz and Professor Leif Arne Rønningen for stepping in the last couple of weeks before submission. Their help has been highly appreciated.  

"Seniornett" deserve a great thank you for their contribution to our assignment. This, especially, yields Arne Sølvberg, the manager of "Seniornett" in Trondheim. He has helped us a lot with arrangement of meeting, booking of facilities, moving needed equipment, and engaging elderly to join our workshops. We would not have been able to perform these workshops without his help. We are grateful to the eight seniors from "Serniornett" who had time and willingness to participate in our workshops. Their feedback and opinions have been used as part of the basis for our master thesis. We would also thank Pål Sæther from the Department of Telematics for always helping us finding the equipment we would need. The main findings in out thesis was presented in a workshop with the FARSEEING EU-project, and we would like to thank the participants in this workshop for their appreciation of our study. 

Finally, we would thank our families for all the support we have got during all our time as students. We are forever grateful! And to all our classmates and friends, thank you for all the fun and good memories! The time here at NTNU has been unforgettable. And to each other, a thank you for a great semester. We finally made it!   

\cleardoublepage
\pagenumbering{roman}
\tableofcontents
\cleardoublepage
\chapter*{Acronyms}
\begin{acronym}
\acro{nsd}[NSD]{Norwegian Social Science Data Services}
\acro{usid}[USID]{User Sensitive Inclusive Design}
\acro{ucd}[UCD]{User Centered Design}
\acro{d3}[D3]{Design for Dynamic Diversity}
\acro{dtc}[DTC]{Dual Task Costs}
\acro{sdk}[SDK]{Software Development Kit}
\acro{cmos}[CMOS]{Complimentary Metal-Oxide Semiconductor}
\acro{sdi}[SDI]{Stepwise Deductive Inductive Method}
\acro{vr}[VR]{Virtual Reality}
\acro{hci}[HCI]{Human–computer Interaction}

\end{acronym}
\cleardoublepage
\listoffigures
\cleardoublepage
\listoftables
\cleardoublepage
\pagenumbering{arabic}
\pagestyle{fancy}
\fancyhead[LE]{\thepage}
\fancyhead[RE]{\leftmark}
\fancyhead[RO]{\thepage}
\fancyhead[LO]{\rightmark}
\fancyfoot{}
\cleardoublepage
\chapter{Introduction}

Today, technology is used for all kind of purposes, and it has become an important part of people's everyday life. Different kinds of technologies meet a wide range of needs, all from entertainment and socialisation, to education and health. One focus that has got great attention the last years is how to use video games with motion sensor technology to engage people in exercise and physical activity \cite{exergamesforelderly}, \cite{gerling1}, \cite{garcia2012exergames}. These games are called exercise games, or exergames, because they require body movements to play. There have been done a lot of research on different motion sensor technologies used for exercise and it has shown positive effect when it comes to improving peoples health. 

Due to "baby boomers" and the fact that people are living longer, the world's population now consist of a great share of older people \cite{dickinson2007methods}. As a part of growing older, elderly often meet disabilities like physical and psychological decline. Due to reduced balance function and physical strength, one serious problem for elderly people is the risk of falling. Fall is the leading cause of injuries in older people, and can often have serious consequences \cite{otago}. These problems, together with the world's ageing population, lead to both strategical and economical challenges for the government when it comes to providing health care services to everyone that need it. This clearly shows that new initiatives have to be taken to identify how to prevent falls, and keep the older population healthy. However, engaging the elderly population in physical activity can be challenging, and it is shown that a great share of elderly do not exercise enough \cite{statistikknorge12}. The introduction of an exergame can meet this problem. This can serve as a more fun and entertaining alternative than regular exercise.  

The use of exergames for health related purposes is supported by the new reform "Samhandlingsreformen" in Norway, were one focus is to use welfare technology in the health sector where possible \cite{welfare}. However, many older people are unfamiliar with technology, and especially with video games. In addition, existing commercial exergames are aimed towards a younger user group, and are therefore not suitable for elderly \cite{exergamesforelderly}. A first step to make the older population accept the use of video games for health related purposes, would be to develop a game aimed especially for their needs and interest, where their physical and psychological declines are taken into consideration. 

This thesis is built upon our project assignment \emph{Business Opportunities and Economics for an Exercise Game in the Health Sector} \cite{project}, where we developed a business model for an exergame, with the physiotherapy service as the customer. In this thesis we will focus on the end users of this game, which are elderly people. Our contribution in this thesis will be to explore and identify the users' needs and wants, gather sufficient knowledge and information about system design, and from this create a concept\footnote{We have choose use the term \emph{concept} which means \emph{an idea, or a general notion} http://dictionary.reference.com/browse/concept} for an exergame appropriate for the older user group.

\section{Objectives}
\label{sec:researchq} Exergames have shown promise in health related purposes like for exercise and rehabilitation. However, previous research suggest that commercial exergames are not suitable for elderly, as they are not specially aimed and developed for this user group \cite{exergamesforelderly} \cite{gerling2} \cite{bruin} \cite{project}. 

We will study how elderly interact with commercial Xbox Kinect games to see what do and do not work, and to validate previous findings. With this we will answer research question 1 and 2: 

\emph{RQ1: Are existing commercial Xbox Kinect games suitable for exercising purpose for elderly people?} 

\emph{RQ2: What are the design challenges when developing an exergame for elderly people, and what aspects need to be considered in this game?}

Based on the answers to research questions 1 and 2, and theory and knowledge about video game development, usability and system design, we will answer research question 3:

\emph{RQ3: What would be suitable system requirements, and appropriate design, for an exergame for elderly people?}

\section{Contribution}

Our contribution in this thesis is the development of system requirements and a concept for an exergame for elderly people. Some aspects from our previous project \cite{project} have been used as motivation for this thesis. A brief summary of this will be provided in Chapter \ref{chap:background}. To be able to understand the users, we have looked into challenges and motivating factors when it comes to engaging elderly to perform physical activity. This will be presented in Chapter \ref{chap:olderexercise}. For the reader to get an insight into what exergames are, we provide in Chapter \ref{chap:exergames} a brief overview of this. As we are making an exergame concept, we have studied important aspects related to game development for the older user, and summarised a set of guidelines that should be followed when developing for this group. This can be found in Chapter \ref{chap:exforseniors}. Understanding of the different elements included in a video game, as well as general system design, is needed to develop the exergame. We provide an introduction to these topics in Chapter \ref{chap:vg} and \ref{chap:generalsystemdesign}, respectively. In accordance with the importance of user involvement, we have conducted two workshops where we have engaged users in the development of the exergame. Informants were recruited by holding a presentation for "Seniornett", an organisation working with teaching the older population technology. This will be described in Section \ref{sec:recruitment}. Workshop 1 was conducted to observe and understand how a group of elderly people interact with commercial Xbox Kinect games. In this workshop, methods like questionnaire, participatory observation and focus group interviews were used. These methods are described in Chapter \ref{chap:metode}. We observed the informants play three games chosen by us, and followed with a focus group interview. A description of the execution, and findings from workshop 1, will be presented in Section \ref{sec:ws1} and in Chapter \ref{chap:findW1}, respectively. The main focus for this thesis has been to specify system requirements and create design for a concept for an exergame for elderly people. This work has been based upon information gathered from theory, previous studies and findings from workshop 1. The design was presented by making prototypes. The exergame concept is the most important contribution in this thesis, and will be presented in Chapter \ref{chap:concept}. The concept was presented in a second workshop, to be evaluated by the users. The execution of, and findings from, workshop 2 are presented in Section \ref{sec:ws2} and Chapter \ref{chap:findW2}. In our discussion, provided in Chapter \ref{chap:discussion}, we provide a detailed description of future work for an exergame for elderly, based upon findings from workshop 2. This thesis provides general guidelines for how to design an exergame for elderly, and can be used in the continuation of our exergame concept, or in the development of a new exergame. The main findings in this thesis, together with the design proposal, have been presented in a workshop with the EU-project FARSEEING \footnote{FARSEEING is a collaborative European Commission funded research project, which are working on promoting healthy, independent living for elderly people with the use of technology. See http://farseeingresearch.eu/ for more information.}.     

\section{Scope and Limitations}
One important limitation of this thesis is the sample of informants in our workshops. Elderly cannot be seen as one common group of people. They are people with different interests and needs. In this thesis, we only included one small group of elderly. Due to time limits before the first workshop, which was critical for our further work, we only recruited informants from the organisation "Seniornett". We did not have the time to look into other arenas for recruiting. The informants had never played video games before, which made it challenging to include them in the development process of the exergame concept. Gathering a group of elderly, and observing them while they played different Xbox Kinect games, is not observing a natural setting. This made it difficult to fit this study into defined research methodologies. The last limitation concerns the choice of how we presented our design. We used prototypes, but limited them to not involve any interaction, even though they represent scenarios for a highly interactive game.  


\section{Outline}
The report is structured in the following way:

\begin{itemize}
\item In \textbf{Chapter 2} the main background and motivation for this master thesis will be presented.
\item To make a video game aimed for exercising, there is a need to understand what actually motivated elderly to exercise in general. This will be provided in \textbf{Chapter 3}. In addition, we will briefly discuss how elderly should exercise.
\item In \textbf{Chapter 4} we will describe what video games, and in particular exergames, are. We will also present the exergame technology we have made an exergame concept for, namely Microsoft Kinect.
\item Elderly people are a group with diversity, and have some typical characteristics. A lot of research on how to make appropriate exergames for elderly people, where these characteristics are been taken into account, have been conducted. In \textbf{Chapter 5} we summarise interesting research and provide a list of typical characteristics of elderly, as well as guidelines for how exergames and user-interfaces should be made for this group.  
\item In \textbf{Chapter 6} we present video game design in general. This is to make it easier to structure our exergame proposal, and to include important elements that a video game should consist of.
\item It is also important to understand how to design systems in general. In \textbf{Chapter 7} we will discuss the four pillars of design, system requirements and the importance of usability. At last, we will present a model that can be used for usability evaluation of games or as guidelines in an early development phase. 
\item It is important to involve the user in the development process of a system. This can be done with several methods, which we will discuss in \textbf{Chapter 8}. This includes; experimental simulation, participatory observation, focus group interview, questionnaire and video and audio recording. At last in this chapter, a thorough description on how we have worked, and how we have used the different methods will be described.
\item In \textbf{Chapter 9} the findings from workshop 1 will be presented.
\item Based on the theory discussed in the previous chapters together with the findings from workshop 1, we will present the functional and interface requirements and the exergame concept. This can be found in \textbf{Chapter 10}. As a part of this, prototypes of some scenarios in the game will be shown.
\item The exergame was evaluated by a group of elderly in a second workshop. In \textbf{Chapter 11} the findings from workshop 2 will be presented.
\item In \textbf{Chapter 12} our findings and result will be discussed and we will provide recommendations for future work on our exergame, based on the feedback gotten from workshop 1. The quality of the research will also be discussed.
\item Finally, in \textbf{Chapter 13}, we provide a conclusion on the work done and we suggest future work.

\end{itemize}

\cleardoublepage
\chapter{Background}
\section{Video Games}
Video games are a genre of digital games that has become extremely popular all over the world. There exist several different gaming consoles, and for each console there has been developed an endless amount of different games that meets almost every need and interest. Video games can be described as “electronic, interactive games known for their vibrant colors, sound effects, and complex graphics” [se kilde [16] i prosjektoppgaven]. Hand held controllers or devices that capture movement are used to interact with the video game. The variety of video games makes it usable for many different purposes, like learning and education, exercising or just pure entertainment (veldig likt som oppgaven). \\ \\
One might associate video games with children and teenagers, and many hours of game play, which partly is a right assumption to make. Video games are played for several hours every day, and it has taken a great part in people's everyday life. Gaming was usually associated with being anti-social, because of all the time used looked up in a room playing, but video games has emerged from sedentary, lonely gaming to gaming involving social interaction and movement. In USA in 2010, one fourth of all gamers were under the age of 18, but an interesting fact is that as much as 26 percent of the gamers are over 50 years old. This shows that not only children and teenagers uses video games, a great share of elderly has started to use this type of technology. In Norway,  as much as 8 percent of the population in the age range from 45 - 79 use computer or video games every day. Elderly contributes to a great share of the world's population, and if the entertainment industry could see this group as potential customers, it would open up a new and inexperienced market for video games. [fra Digital Game Design for Elderly Users]. 
\section{Serious Video Games}
The widespread and assorted use of video games has made a great potential for the entertainment industry with regard to development of new games, but also in terms of developing games for additional purposes beside pure entertainment. A new way to use video games is for 
different types of training, education and learning, which is called serious games. Serious games can be described as “a mental contest, played with a computer in accordance with specific rules, that uses entertainment to further government or corporate training, education, health, public policy, and strategic communication objectives.” [fra From Visual Simulation to Virtual Reality to Games] The idea is to use the fun, motivating and captivating features that video games possess and combine it with pedagogy. The term serious games is not something new, it has existed for many years, but it did not break through when it first was introduced in the 1990s. The failure that time was that the focus of the game was on learning, which resulted in boring games with no market success \cite{susi2007serious}. Michael Zyda states that, in serious games, pedagogy has to be subordinate (SAMME ord som i teksten) to the story and entertainment component of the game. It is important that fun and entertainment are the main focus in the game \cite{zyda2005visual}. \\ \\
There exist several genres of serious games, where game-based learning, simulation games, games with a purpose, games for health and exergames are some of them. How and where serious games can be used, and are being used today, are many. With simulation games medical students can simulate surgery and soldiers can practise on how to use a rifle, students can learn their curriculum through quiz games, and people can be motivated to work out with exercise games. Research on use of serious games shows positive evolvement of skills and knowledge, and positive effect on health, which results in motivation for further research and development. (VET ikke om dette er rett ord å bruke). In addition, the new technology involving physical interfaces and improvement on graphics and animations initiates to new types of games. The market for serious games are inexperienced and unexplored, so there are a great potential for further research \cite{alfingewang}. Market sales the last couple of years also shows promise for serious games in the future. In 2008, the market for serious games sold for around 1,5 billion USD around the globe \cite{alfingewang}, and in 2010 it reached about 2 billion USD (1,5 billion EUR). The market is expecting a annual growth rate of 47 percent, up to about 13,5 billion USD (10 billion EUR) in 2015! \cite{idate} \\ \\
We will look deeper into one of the genres of serious games, exergames.  
\section{Exergames}
Skrive et kort sammendrag av hva exergames er og hvordan det blir brukt i dag. Ramse opp kjapt hva som går under exergames. 
\section{Exergames Used in Rehabilitation and Exercising}
Relevant arbeid der det er bevist at spill kan brukes for trening for eldre.. 
\section{Microsoft Kinect}
Beskrive Kinect nærmere og også grunnen til valget av Kinect. 
\section{How to Develop a Good Video Game?}
Related work.. Generelt om hva som skal til for å utvikler videospill. Dra noen konklusjoner.
\section{How to Develop a Good Exergame for Elderly}
Related work der de snakker om alle de viktige tingene vi må tenke på når man utvikelr for eldre..


\cleardoublepage
\chapter{Older Adults and Their Attitudes Towards Exercising}

Physical activity is important for all people, including elderly. It can improve quality of life by reducing risk of some chronic diseases, alleviate depression, and help people towards a more independent life. Improving health by physical activity does not just apply for healthy elderly, but also for frail and very old people. Guidelines are proposed by the Department of Health and Human Services, recommending aerobic activity, as well as muscle-strength training \cite{olderamericans}. These guidelines  may seem easy to follow, but it is shown to be challenging to motivate elderly to exercise, and there is shown that there is a small percentage of elderly who actually engage in physical activity \cite{olderamericans}. To be able to engage seniors to be more physically active, they should be provided with programs that will motivate them to actually perform the exercises. In this thesis, we will develop a game concept for this purpose. This game have to have the right elements in place, to motivate the user to engage in regular physical activity. ( It needs to relate to their life, and it needs to be offered at placed where they can actually use it. ) As discussed in our previous project \cite{project} and in \cite{schutzer}, keeping the older population healthy, can in addition to reduce the need for health services, also decrease economic challenges related to this.  It is therefore important to have focus on elder's physical health. 

As mentioned, a small percentage of elderly engage in physical activity. The Federal Interagency Forum on Aging-Related Statistics \cite{olderamericans} presented numbers in their latest report, revealing that only 11 percent of Americans aged 65 years and older, meet the 2008 Federal physical activity guidelines \cite{guidelines}, and that the percentage decrease with increasing age, where 14 percent of older people aged 65-74 years old meet the guidelines, while only 4 percent of people aged 85 years old and older, meet the requirements.  It is quite common that physical activity decrease with age and that after the age of 65 years old, the level of activity is at its smallest \cite{schutzer}. Even though not directly comparable with the american numbers, we found it interesting to see that Norwegian elderly, aged  67-79 years old have become more active,with as much as 40 percent of this population exercise regularly several times a week. The most popular activities include cross country skiing, swimming, cycling and brisk walking. It is also shown that jogging has become a more common activity also for the elderly. However, it is still 23 percent of the group between 67-79 years old in Norway who exercise less than once a month or never \cite{statisticsnorway}.

(It is shown, that regular physical activity, can decrease chronic diseases and disabilities. Being active can also help to increase the quality of life for the older person \cite{schutzer}.)
 
\section{Barriers and Challenges}
It is hard to get people to be physical active, and this might especially apply for elderly. Elderly have a tendency of thinking they are too old to start exercising. As the fall statistics are shown to be high, resulting from reduced physical strength and decreased balance, there is important to engage elderly in physical activity. However, there are some unique challenges or barriers when it comes to engaging elderly in physical activities, as discussed in Schutzer \cite{schutzer} and Chao \cite{chao}.  Understanding motivation factors for exercising is critical when developing exercise programs \cite{chao}.

One common barrier is that elderly think their health is not good enough to start exercising \cite{schutzer}, and may believe exercising will do more harm than good \cite{chao}. Their poor health and the pain related to this, hinders them from exercising \cite{schutzer}. It is also shown that older women associate physical exercise with something masculine \cite{chao}. It is also a challenge to get people to exercise long enough to see positive results. Many associate exercising with pain, sweating, and muscle soreness, and the time before positive outcomes are noted can be long. At the same time may the negative effects of not exercising, not be that apparent \cite{chao}. 

The importance of having available and convenient resources is significant, as it is shown that people living far from for example recreation centers, parks and walking paths are less active than people living near these facilities \cite{schutzer}. With this comes also time constraints. It is shown that many elderly look at physical activity as time consuming, thinking about the time doing the activity, as well as the time getting to the exercise site \cite{schutzer} \cite{chao}. Again, having the resources more available, will overcome this barrier. It can be challenging to get people to perform unstructured, free-living exercise programs, compared to structured programs. This is because people then have to decide when, and what to do, and is pretty much based on self-motivation \cite{chao}.  Physicians play an important role when it comes to advising elderly to exercise, because most people have respect for and trust their physicians. This was also discussed in \cite{project}, where we found physiotherapists to be a suitable and reliable mediator for the exercise game. However, as discussed in \cite{schutzer} and \cite{chao}, the physicians does not always give sufficient exercise advice. Instead of giving a specific exercise program for the patient to perform, they simply just tell them to be more active. Another aspect that may keep elderly from exercising, is the lack of knowledge about physical activity and its advantages. This can come from the fact that many elderly grew up in a time where there was not that much attention around the importance of physical activity. 

\section{Strategies and Motivators}
It is important to understand the barriers and challenges, and at the same time find what is motivating people to exercise. In \cite{schutzer} and \cite{chao} they also presents how the discussed barriers can serve as motivators to activity. For example can decrease in physical health actually motivate to start exercising. Also having more time, more information, and living closer to exercise offerings, could motivate for physical activity. What the physician is telling their patients is important, and should be done in a precise way. It is important to remember that a common limitation for this group is memory capacity, due to for example dementia and therefore information should be given in a precise and accurate way \cite{chao}.  The importance of self-efficacy is discussed in \cite{schutzer}, \cite{chao} and \cite{white}. Self-efficacy seems to be an important determinant of exercise behavior. It is more likely for a person with strong self-efficacy expectations and outcomes to adopt to a specific behavior. In \cite{white} they evaluate self-efficacy to play an important role in the relationship between physical activity and quality of life. Along with this, the elements that needs to be in place to sustain  the exercise behaviour are the feeling of pleasure and satisfaction, as well as self-regulatory skills \cite{schutzer}. There has been suggested different strategies to promote adherence to physical activity, presented in \cite{chao}, and includes goal setting, self-monitoring of progress, implementing decision-making models, modifying cognitive thoughts during activity, and increasing social support. (denne ble litt skrevet av.. vanskelig \cite{chao}). Chao et al. \cite{chao} discuss some strategies for motivation.  One must understand different peoples' needs and expectations, and take into account race, gender, ethnicity etc.  To meet these expectations contact with the relevant people is important. Chao suggest this to be done with for example interviews and focus groups.  Self-regulatory skills, including goal setting, self-monitoring of progress, and environmental management, are important to keep peoples' exercise behavior \cite{chao}. Clear goals should be set to make the participant understand that the activity has a purpose and is going through an end, and that skills will be developed through practice. To easily monitor the exercise, goals should be separated into small and more manageable parts. Activity facilities should be more available and it should be convenient for people to exercise. Exercising should be looked at as an ongoing process, and it is important to remember that people's behavior towards exercising can change with age. Therefore, prevention of relapse should be included in the planning process, to maintain the physical activity routine \cite{chao}. In the article \cite{schutzer} there is also suggested methods to motivate for physical activity. They suggest prompts, like e-mailing and telephone contact. These prompts are typically used for home based training program, and are shown to be as effective as supervision face-to-face can be. Telephone contact worked well in a starting phase, when trying to get a person to adopt to  a more active lifestyle, while e-mail contact worked well when the person had adopted the lifestyle, to help maintain this lifestyle. These ways can actually be more effective, than prolonged exercise session with face-to-face contact \cite{schutzer}. Another important motivating factor discussed in \cite{schutzer} is appropriate music to enhance the exercise experience, and to divert from pain coming from the exercises. As a last motivator is the demographics, where it is shown that people who already have an active lifestyle, who were already fit, who had lower body mass, had fewer chronic diseases and pain, were non-smokers, and had high self-efficacy, had the best adherence to exercise (siste neste avskrift.. vet ikke hvordan jeg skal skrive om en oppramsing) \cite{schutzer}.
At last, a structured program offered at a set time, will probably be a good solution for this user group, as a common problem is to find time, as well as decide what kind of exercise to do \cite{chao}.

White et al. states that "being more active was associated with being more efficacious, having fewer disabilities limitations, reporting higher physical self-worth, and being more satisfied with one's life" \cite{white}. In addition, it was stated that self-esteem is an important component of the physical activity and QOL relationship (siste skrevet av også ) \cite{white}. 
\cleardoublepage
\chapter{Exercise Games}
In this chapter we will present what exercise games are and how it can be used for exercising. To understand the concept of exergames and where it originates from, we will start with a brief presentation of video games and serious games. We will also present the technology used for the game we are developing a concept for, namely Microsoft Kinect. 

\section{Video Games}

Video games are a genre within digital games that has become extremely popular all over the world. There exist several different gaming consoles, and for each console there has been developed an endless amount of various games that meets almost every need and interest. Video games can be described as \emph{"electronic, interactive games known for their vibrant colors, sound effects, and complex graphics"} \cite{videogamedef}. Hand held controllers or devices that capture movement are used to interact with the video game. The variety of video games makes it usable for many different purposes, like learning, education, physical activity, or just for the sake of amusement \cite{project}. 

One might associate video games with children and teenagers, and many hours of game play. Generally, this is a right assumption to make. In USA, video games are played for several hours every week \cite{foxnews}, and one third of all gamers are under the age of 18 \cite{videogames2012}. Gaming are usually associated with being anti-social, because of all the time spent looked up alone in a room, playing. However, the gaming lifestyle has changed. Video games have emerged from sedentary, lonely gaming, to gaming involving social interaction and movement, and it has taken a great part in people's everyday life. In Norway, 6 percent of the population in the age range 45 - 79 use computer or video games every day \cite{mediebarometer2012}. This might indicate that not only children and teenagers use video games, but also the elderly population has started to use this type of technology. Statistics from USA in 2011 provided numbers that might support this assumption, as they show that as much as 29 percent of all gamers where over the age of 50 \cite{videogames2011}. Elderly contribute to a great share of the world's population, and if the entertainment industry could see this group as potential customers, it would open up a new and inexperienced market for video games \cite{ijsselsteijn2007digital}. 

\section{Serious Games}
\label{sec:sergames}
The widespread and assorted use of video games have made a great potential for the entertainment industry regarding development of new games, and also in terms of developing games for additional purposes beside pure entertainment. A new way to use video games is to include it in training, education and learning. This is called serious games. Serious games can be described as \emph{"a mental contest, played with a computer in accordance with specific rules, that uses entertainment to further government or corporate training, education, health, public policy, and strategic communication objectives"} \cite{zyda2005visual}. The idea is to use the fun, motivating and captivating features that video games possess and combine it with pedagogy.

The first serious attempt of developing video games with the main focus on learning did not become very successful \cite{understandingvg}. The failure that time was that the focus in these  video games were on pure learning, which resulted in boring games that did not create motivation and curiosity \cite{understandingvg} \cite{susi2007serious}. Michael Zyda states that, in serious games, pedagogy has to be subordinate to the story and entertainment component of the game. It is highly important that fun and amusement is the main focus in the game \cite{zyda2005visual}. For a video game to function as a pedagogical tool it has to focus on motivation, effectiveness, and intuitiveness. These games have to provide motivating factors that engage people to play the game, and they have to be effective in the matter of learning outcome. When it comes to intuitiveness, people have to be able to play and understand the video games without the need for guidance \cite{understandingvg}. 

There exist several genres of serious games, where edutainment and game-based learning, advertaintment, games for politics, simulation games, and games for health are some of them. The genre of edutainment and game-based learning combines fun and entertainment with education and learning. Advertainment is about using video games as a media for marketing, and games for politics influences players through hidden, underlying political messages in the game. Simulation games are about bringing various activities into life with the use of video games, and gaming for health is about using video games to engage activity, to become physically stronger, and to improve quality of life \cite{understandingvg} \cite{alfingewang}. Gaming for health, with training its main focus, will be the genre suitable for exergames.

An important feature with serious games is that they provide the opportunity to experience real-life situations and adventures one may otherwise not be able to enjoy. This might be due to expenses, time, distance, risk, or physical capabilities. With serious games it is possible to play golf, be the star of a boxing match, paddle down turbulent rivers, and explore nature and creatures one have never seen before. In a more serious matter, video games can be used to simulate brain surgery or a battle in a war zone. This can teach students how to perform medical procedures or it could prepare soldiers for war. Serious games can also be used for students to learn their curriculum through quiz games, and people can be motivated to work out with exercise games. Serious games can be used for several purposes and situations, and it can help people to develop a wide range of different skills. 

Research on the use of serious games show positive development of skills and knowledge, in addition to positive effect on health, which have resulted in motivation for further research and development. In addition, the new technology involving physical interfaces and improvement on graphics and animations initiates to new types of games. The market for serious games are inexperienced and unexplored, so there are a great potential for further research \cite{alfingewang}. Market sales the last couple of years also shows promise for serious video games in the future. In 2008, the market for serious games sold for around 1,5 billion USD around the globe \cite{alfingewang}, and in 2010 it reached about 2 billion USD \footnote{Converted from EUR from \cite{idate}}. The market is expecting a annual growth rate of 47 percent, up to impressive 13,5 billion USD \footnote{Converted from EUR from \cite{idate}} in 2015 \cite{idate}. 

We will look closer into the genre of gaming for health, with its exercise games.  

\section{Exergames}
\label{sec:exergames}
Exergames are a type of serious video games that combine traditional game play with physical activity. The combination of movement and amusement is used to stimulate exercise and engage people to be more physically active in a fun and motivating way. A lot of research have been done within this topic, and results show that exergames can have a positive effect on users health. In addition to being a new, and more amusing way to work out, exergames provide an opportunity for social interaction. A great amount of today's existing computer and video games possess this feature, and as much as 62 percent of all gamers say they play with others either in-person or online. One of the two top-selling video games in 2011 was Just Dance 3, which is an exergame that has the possibility to play with others \cite{statistics2012}. This shows that the social aspect of gaming is important. The social aspect these games provide may also be especially important for people who experience loneliness in their everyday life \cite{exergamesforelderly}.

Exergames use technology like remote hand held controllers and motion sensors to capture body movements. Users are therefore required to get up and move their body to be able to play the game. There exist several consoles that support this technology, where Nintendo Wii, Playstation Move, Dance Dance Revolution and Xbox Kinect are some of them. The existing consoles use different type of devices to capture body movement, like hand held controllers, different boards and pads with embedded sensors, and video cameras with motion sensors. In this assignment we will focus on the Kinect sensor technology. This is because we in our previous project evaluated Kinect to be an appropriate choice as it provides the possibility to play without holding on to any devices, and because the sensor has proved to measure different body movements accurately so that the parameters are applicable for use in clinical practice. The reader will be referred to Section 4.2 in our project assignment \cite{project} to read what this is based on. We will now provide a brief description of the chosen technology.

\section{Microsoft Kinect for Xbox 360 and Windows}
Microsoft Kinect is a motion sensor that captures movement without the use of any controllers. This gives players the opportunity to play and interact with the game only by gestures and body movement. Xbox presents the Kinect experience this way: \emph{"Kinect for Xbox 360 brings games and entertainment to life in extraordinary new ways without using a controller"} \cite{kinectxboxdef}. The word Kinect is a fusion of two words, \emph{kinetic}, which means movement, and \emph{connect}, which relates to the social aspect \cite{howstuffworksKinect}. The Kinect sensor was released by Microsoft in November 2010 as an input device to the Xbox 360 console, and it did not take long before the Microsoft Kinect got extremely popular. Only ten days after the release in 2010 it had been sold 1 million Kinect motion sensors, and in January 2012, sales numbers had climbed up to over 18 million sold units \cite{kinectsales}. These sales numbers put Kinect into history, as no other consumer electronic device has experienced such great sales numbers in such a short amount of time  \cite{kinectsales} \cite{microsoftnews}. The Kinect technology has the possibility to be experienced without the Xbox 360 console, as it exist a separate Kinect Sensor compatible with Windows computers \cite{kinectforwindows}. A more detailed information about developing for Kinect, and a presentation of the technology provided by the Kinect Sensor, will for the interested reader be presented in Appendix A.

\begin{figure} 
\centering
\includegraphics[scale=0.36]{sensorandtv}
\caption[The Kinect sensor]{The Kinect sensor (a) together with the rest of the needed equipments, and (b) alone}
\label{kinectsensor}
\end{figure} 
 
Kinect has not just experienced success as an entertainment device within the living room, researches have also started to see the possibility to use the Kinect for other non-gaming purposes, like within healthcare, education and industry \cite{microsoftnews}. The key reasons for the broad use of Kinect are its accessibility and low price, in addition to the ground breaking technology it provides. The release of the Kinect for Windows \ac{sdk} is also a reason for the wealth of non-gaming applications, as it makes the the Kinect platform available for everyone to develop on \cite{microsoftnews}. 


\cleardoublepage
\chapter{Designing Exergames for Elderly}
\label{chap:exforseniors}

Exergames for elderly have become a popular topic in the past couple of years, and several research on how games can be developed for this particular user group have been conducted. Most of the research done on \ac{hci} are performed on young people \cite{dickinson2007methods}, and it is common to develop technology systems for a homogeneous user group. This means that the characteristics of specific user groups, like the elderly, are being ignored. With the wave of elderly ahead of us system designers have to put a greater focus on older people and their needs. Elderly in particular, have some special characteristics due to ageing that needs to be taken into account when developing technology systems for them. Studies indicate that most existing video games have not considered these characteristics, and are therefore not suitable for this group. 

In this chapter we will discuss elderly as users of a technology system. We will discuss characteristics of elderly that are important to acknowledge when designing technology systems for them. Based on previous research done on exergames for elderly people, we will draw out some specific guidelines that need to be considered when developing for this group. This can be found in Section \ref{sec:relatedresearch}. We will also provide previous research on elderly and user interfaces, as this is an important part of the gaming experience.  In addition, we will present some official guidelines for developing user-friendly interfaces for the elderly user. This will be provided in Section \ref{sec:designelderly}. 

\section{Related Research}
\label{sec:relatedresearch}
There have been done a lot of research on how to make exergames for seniors. In this section, we will review research that we have found to have interesting findings.

Billis et al. \cite{Billis} discuss important issues that need to be taken into account when developing games for elderly. Elderly often suffer from decline in visual acuity, decreased audition, mobility changes and cognitive functions' decline. In addition, many elderly are not familiar with technology. The writers suggest that it should be possible to customise the game for every player's special needs. Font, size and colour should be adjustable, and information should be provided in different multimedia alternatives, like text, voice and images. The objects should be of sufficient size and the elements should not move too fast. The overall interface should be as simple as possible, without the need to remember information given earlier in the gaming process, and it should be given sufficient information and guidance throughout the whole game. Games should also provide motivating messages to encourage the player. The writers also stress the importance of the social factors of the game, and suggest the ability to multi-play. At last, for the players to get interested and engaged in the game, they suggest that the content of the game should match the users' cultural and lifestyle diversity \cite{Billis}.

de Bruin et al. \cite{bruin} write about the potential of \ac{vr} environment, like exergames, for exercise. \ac{vr} platforms can evoke naturalistic movements in a safe environment that can be customised according to the patients' needs. It can offer a consistent program that enables for comparison over time. In addition, the use of games can distract the player from any pain they may have. They suggest stepping exercises to be suitable, because they have found that stepping exercises can be a good predictor of falls. It is also proved that a repetitive training program with stepping exercises can improve balance in elderly. Like they indicate in other research, \cite{gerling1} \cite{exergamesforelderly}, they also here express that the problem with already existing exergames is that they are too complex for the older user group. Therefore, there is a need to develop games specifically for this group where physical and cognitive limitations, as well as typical interests, are taken into consideration. The writers also present a study where it was shown a clear decrease in relative \ac{dtc} of walking for elderly. These elderly were training physically, combined with a virtual reality dance game that required decision making. Training traditionally, without the additional cognitive challenge, did not change these walking parameters. This comes from the fact that elderly people often face problems when they have to do more than one task at the same time \cite{bruin}.

Brox et al. \cite{exergamesforelderly} suggest some persuasive strategies for motivating elderly to exercise with games. They suggest that too much detailed information about the player's progress should be avoided. The information should rather be shown visually, like a fish that gets larger and healthier as the player is exercising. Information from previous sessions should be provided, to help players set new goals for the next session. The player should be provided with positive feedback as they achieve goals, and should not be punished if they do not achieve goals. This information and feedback should be given at an appropriate time, and should not disturb the player. The game should have an easy, understandable and good looking interface. The writers see social interaction as very important and suggest that this should be combined with exergaming. This is because it is shown that people get more engaged in activities when doing them with others, and because this group of people often suffer from loneliness and depression \cite{exergamesforelderly}. 

John et al. \cite{john2012smartsenior} write about the development of an interactive training system for seniors, called "Interactive Trainer". The main motivation for this system is the problem of falls. They refer to studies where it is shown that strengthening exercises, and balance and functional training, are suitable in the matter of reducing the risk of falls. They did a review of studies done on therapy for people over the age of 50, to find appropriate exercises to use in their training system. They found 12 concepts suitable for the "Interactive Trainer" system, where exercises for balance and strength were integrated. Their first exercises were (directly drawn from \cite{john2012smartsenior}: "walking while sitting", "weight shifting while sitting", "weight shifting to both sides while standing", "weight shifting to the front and rear while standing", "one-leg standing", and "sit-to-stand from chair".
      
The system consists of a computer and two motion sensor cameras. The movements tracked from these sensors are portrayed on a TV-screen or PC monitor. Ordinary internet connection makes it possible to communicate with the therapists through the system, both to get assistance and to transmit training results. The idea of "Interactive Trainer" is to provide individual therapeutic help to elderly, which they can benefit from in their own living room. They will be given a customised exercise program, and they will be provided with an evaluation of quality of the performed exercises. The seniors will be introduced for this game by therapists, where they together create a user profile, set up a training program, and choose avatar and preferred scenery. The seniors will then take the game home, where they will be guided by a virtual trainer. The exercises in the program have to be completed in a specific sequence, and a completion of one sequence of exercises will be rewarded with an "unlocking" of a new game. The therapist can follow the training from their "Interactive Trainer" editor, where they can supervise exercises, progress and success, and adjust the game accordingly. The writers acknowledge that success for the system is dependent on user acceptance, and essential for this is a good user interface and integration of motivational factors. To achieve this they use \cite{john2012smartsenior}:

\begin{itemize}
\item Avatar design in 3D
\item Virtual therapist in 3D
\item Motivational feedback systems, which will be used as encouragement  to reach goals
\item Corrective feedback systems, which will be used as guidance for exercise 
\item Tutorials for how to learn the system and chosen exercises
\item Target group-specific environment, with the use of familiar and real-life activities and exercises 
\end{itemize} 

Gregor et al. \cite{gregor} discuss some particular issues when designing for the older population. They propose a paradigm and methodology to support the process of designing software as close to the universal accessibility ideal as possible \cite{gregor}.

They describe older people through three different groups: Fit older people, frail older people and disabled people who grow older. In addition, they define some important characteristics of older people \cite{gregor}:
\begin{itemize}
\item As people get older the individual variability of physical, sensory and cognitive functionality will increase. 
\item Functional decline will go faster when people gets older. 
\item Cognition problems, like memory dysfunction and the ability to learn new things, are widely appearing.
\item Based on where they are in life, elderly may have different wants and needs. 
\item How people live, like if they are needing a walking frame or needing warm glows, can change their usability function.
\item Elderly have more life experience than younger people, and can therefore have more knowledge of the world, as well as a more mature ways of solving problems. 
\end{itemize}

To make it easier to develop a technology system for the older user group with different characteristics, Gregor et al. proposed a modified version of the \ac{ucd} principles, which they called \ac{usid}. The issues this methodology addresses are (directly drawn from \cite{gregor}): 
\begin{itemize}
\item "Much greater variety of user characteristics and functionality". 
\item "Finding and recruiting "representative users"". 
\item "Conflicts of interest between user groups (including "temporarily able-bodied")".
\item "The need to specify exactly the characteristics and functionality of the user group".
\item "Tailored, personalisable and adaptive interfaces".
\item "Provision for accessibility using additional components (hardware and software)".
\end{itemize}

As an example of the advantages of the proposed methodology, the writers present a case study where a web browser for people that are visually impaired was developed. 200 visually impaired users evaluated the system. The findings they did and the conclusions they made from this study were \cite{gregor}: 
\begin{itemize}
\item Elderly seemed to lack confidence in handling IT systems. However, the confidence increased after they had experienced a successful interaction and decreased after experiencing an unsuccessful interaction.
\item Many elderly had difficulties remembering too much information. This indicates that there are important memory related factors that need to be taken into account when designing for elderly. 
\item Following a need for less information, will most likely also mean less functionality. From another study they found that it was a need for the possibility that more functionality could be added after the user had mastered the initial, simple functionalities.
\item After some kind of assessment is passed, the user should be moved to a higher level. This can be done by self-assessment. To reinforce user-confidence the user should be able to reach goals. 
\end{itemize}

Gerling et al. \cite{gerling1} discuss possibilities and challenges when developing a game for elderly users, and they suggest four major guidelines to be followed:
\begin{itemize}
\item The player should have the possibility to interact with the game both when sitting and standing. 
\item Avoid too extensive and sudden movements.
\item It should be possible to adjust the level when it comes to difficulty, game speed and device sensitivity. These changes should be possible to be adjusted by the player. 
\item Interaction mechanisms should be simple, player frustration should be avoided, and the game should provide constructive feedback.
\end{itemize}

To verify these guidelines, the research group prototyped an exergame, called SilverBalance, made for the Wii Balance Board. SilverBalance consisted of two balance tasks and had the possibility to be played both when sitting and standing. The game was tested on nine seniors with an average age of 84. The following observations were made \cite{gerling1}.
\begin{itemize}
\item All of the participants were able to play the game adequately. 
\item Participants expressed that the fact that the design was so minimalistic, made it possible for them to focus on the purpose of the game. 
\item The possibility to sit while playing the game was necessary because all participants were dependent on devices to assist them when standing and walking.
\item The players started to compare their results and comment on each other's results.
\item Impairments and diseases made it difficult for some participants after a longer period of playing, suggesting that alternative interactions should be included in such games.
\end{itemize}

The testing of SilverBalance shows that the four guidelines can contribute to the development of exergames for elderly people, but the writers state that further studies on how age-related changes can affect games, should be carried out \cite{gerling1}.

In another study, Gerling et al. present a case study where they introduce and evaluate an additional video game for elderly, called SilverPromenade \cite{gerling2}.  SilverPromenade is developed for the Nintendo Wii technology and utilises the Wii Remote and Wii Balance Board. The game is stripped from complexity in functionality and design to be senior-friendly. The theme of the game is a virtual "walk through the forest" and it can be played both as a single-player and multi-player game. In each mode of the game it is possible to engage three different roles: walking through the forest by stepping on the Wii Balance Board which requires physical exercise, catching a butterfly by pointing at it with the Wii Remote, and counting rabbits by shaking the Wii Remote when a rabbit appears. The scenarios are simplistic, which is important when including elderly in digital game play. The concept of the game was chosen because it appeals to elderly living in nursing-homes, as it offers them the possibility to "visit" the forest, which generally might be inaccessible for them. Playing this game requires the older player to focus on cognitive, mental and physical abilities. 

SilverPromenade has an easy and understandable user interface. Complex graphics and visual effects are avoided, while important elements are highlighted. It consists of a menu structure which easily guides the user to the playing-mode. The input devices used for playing makes it possible for elderly to both sit and stand during game-play, which takes the different individuals abilities into consideration.  If one of the three roles is not suitable, SilverPromenade offers the possibility of not including one or more roles.

A case study with SilverPromenade was executed on a group of frail elderly living in full-time nursing homes. The participants were asked to play the game and afterwards fill out a short questionnaire. Two groups of elderly participated, one group with experience with this type of technology, and one group without any experience. During the game play the researchers observed how the elderly used the controllers, how they understood the menu, and how easily they perceived the game behaviour. The gaming results were also observed.

Results from this case study show that experienced players clearly had an advantage over inexperienced players. It was observed during game play that some roles were difficult to execute, like when pointing at the butterflies they had difficulties using the Wii Remote because it required that one point it directly towards the sensor. However, the results showed that the overall experience was positive. SilverPromenade gave an impression of being outside, and the use of a real-world scenario engaged elderly to play. The concept of the game led to communication in terms of identifying objects, and helping and encouraging each other. One important observation was that the participants were sharing and discussing their scores. The conclusion of the case study was that elderly enjoyed playing digital games, and that SilverPromenade could be an appropriate game to use in this age group \cite{gerling2}.

Most studies done on elderly and the use of digital games, suggest that social interaction is important to include. Social interaction can be multi-play both in person in the same room and online. Online multi-play could be a good way to socialise for elderly who are less mobile and are lonely. However, not many studies on playing games together over the Internet have been conducted. We know that elderly often are unfamiliar with this type of technology \cite{Billis}, \cite{gregor}, and it is therefore important to acknowledge the challenges that can be faced by introducing multi-play over the Internet for this group. Gajadhar et al. \cite{Gajadhar} did a study, where they wanted to find out how players experiences playing together with different social presence. They tested a digital game on 40 participants in pairs with three different degrees of social presence: playing in the same room at the same time, playing against a computer controlled player, and playing together online. Results showed that while playing online elderly experienced low enjoyment, competence and challenge score, which indicates that they did not like playing online. It was also shown that they had more fun when playing alone, than when playing online with others. When playing together with others, they preferred seeing their co-players, and be able to speak with them. Seniors experienced more fun, competence, challenge and immersion when playing together in the same room than when playing with others over the internet. It was also found that seniors seemed to not care if they were winning or losing. They appeared to not be very competitive. Instead they enjoyed helping each other. The writers link this to previous studies that had found that elderly rather liked to help and teach others how to play games, than compete. The writers suggest that when including social interaction in games, the focus should be on cooperation rather than competition \cite{Gajadhar}. 


\section{Summary of Findings from Related Research}
\label{sec:summaryguidelines}
It is clear that elderly people have some specific characteristics that need to be considered when developing technology systems aimed for them. Based on the reviewed research we will summarise and list the discussed characteristics. In addition, we will draw out important aspects found in the literature, and list these as a proposal for guidelines to follow when designing an exergame for elderly.

\subsection{Characteristics of Elderly:}
\label{subsec:characteristics}
\begin{enumerate}[{c}.1]
\item Cognitive functions' decline, e.g. dementia, memory dysfunction, and the ability to learn new things \cite{Billis}, \cite{gregor}.
\item Decline in sensory functionality, like  visual acuity and audition \cite{Billis}, \cite{gregor}.
\item The problems elderly face, often increase significantly with increasing age \cite{gregor}.
\item Mobility change \cite{Billis}.
\item Needs and wants related to cultural and lifestyle diversity, as well as at what stage they are in life \cite{Billis}, \cite{gregor}.
\item Many are in need of assistant tools (for example when standing and walking) \cite{gregor}.
\item Inexperienced with technology. Many elderly people seem to lack confidence when it comes to IT-systems. A better confidence can be achieved when experiencing a successful interaction with an IT-system \cite{Billis}, \cite{gregor}.
\item It is important to take into account memory related factors. Many elderly have difficulties remembering too much information \cite{Billis}, \cite{gregor}.
\item It can be challenging to do more than one task at the same time \cite{bruin}.
\end{enumerate}

\subsection{Proposed Guidelines from Related Research}
\label{subsec:guidelines}

\begin{enumerate}[{g}.1]
\renewcommand{\labelitemi}{$\bullet$}
\item The game should have the possibility to be customised for every user's needs, conditions, interests etc. \cite{Billis}, \cite{bruin}, \cite{gregor}, \cite{gerling1}, \cite{john2012smartsenior}.
\item The interface should offer adjustable font, size and colours \cite{Billis}.
\item The interface should have different alternatives for multimedia presentation, for example, text, voice and images \cite{Billis}.
\item The interface should be simple and not too extensive. The objects should be of sufficient size and there should be no sudden movements. Important elements should be highlighted \cite{exergamesforelderly}, \cite{Billis}, \cite{gerling1}, \cite{gerling2}.
\item The game should have a menu, which easily guides the user to the gaming session \cite{gerling2}, \cite{john2012smartsenior}.
\item Sufficient guidance and information should be given before and during the process. There should not be a need to remember earlier given information \cite{Billis}, \cite{gregor}, \cite{john2012smartsenior}.
\item Motivating feedback should be given to encourage play \cite{Billis}, \cite{gerling1}, \cite{exergamesforelderly}, \cite{john2012smartsenior}.
\item Constructive feedback should be given to guide and correct exercises \cite{john2012smartsenior}. 
\item Information and feedback should be given when appropriate, to not disturb the player \cite{exergamesforelderly}.
\item The story of the game should match cultural and lifestyle diversity \cite{Billis}, \cite{gregor}, \cite{gerling2}. 
\item The game should use environments and activities that are familiar to the target group. An example presented in \cite{gerling2} is a game with a real-world scenario, where the players have to walk through a forest. \cite{john2012smartsenior}, \cite{gerling2}.
\item Social factors should be included by offering the possibility to multi-play \cite{Billis}, \cite{gerling2}, \cite{gerling1}, \cite{exergamesforelderly}.
\item Multi-play should focus more on cooperation than competition \cite{Gajadhar}.
\item It should be offered variety in user characteristics and functionality \cite{gregor}, \cite{gerling1}.
\item The game, with its environment and characters, should be in 3D to enhance the gaming experience \cite{john2012smartsenior}.
\item The game should not have too much functionality, but instead offer the possibility to add more functionality after the existing functionalities are managed \cite{gregor}, \cite{gerling2}.
\item The game should offer different levels with different difficulties. With this comes the possibility to reach goals. The device sensitivity and the game speed should also be adjustable \cite{gregor}, \cite{gerling1}, \cite{john2012smartsenior}.
\item The game should give the player information about progression and results. This information should not be too extensive \cite{exergamesforelderly}, \cite{john2012smartsenior}.
\item It should be possible for the player to adjust levels themselves \cite{gregor}, \cite{gerling1}. 
\item The possibility to interact with the game both when sitting and standing is important \cite{gerling1}, \cite{gerling2}.
\item The game should use exercises which are proven to be good for elderly \cite{john2012smartsenior}. 
\item Exergames for elderly should have the opportunity to be used in a clinical setting \cite{Billis}, \cite{gerling2}, \cite{john2012smartsenior}.
\item There should be made a profile where the user's progress and results can be saved. If the game is used in a clinical setting, it should be possible to share this profile with therapists \cite{john2012smartsenior}. 

As an additional guideline, it is worth mentioning that both \cite{bruin} and \cite{gerling2} discuss stepping as a relevant exercise for elderly. This is because this kind of exercise has proved to be a good predictor of fall and that repetitive stepping exercises can improve balance. Stepping requires significant physical activity, which is important when improving physical health. Walking or stepping exercises are also recommended by the Department of Health and Human Services, discussed in Chapter \ref{chap:olderexercise}. In addition, \cite{john2012smartsenior} mentions balance and functional training, together with strengthening exercises, as appropriate for elderly as it can lead to reducing the risk of falling.    

\section{Guidelines from Official Organisations for How to Design User-Friendly Interfaces for Elderly}
\label{sec:designelderly}

As seen from the characteristics in the previous section, ageing will often lead to both physical and psychological negative changes, like decline in cognitive skills, visual and auditory acuity, and in balance and physical strength. An important part of game development is designing the user interface, where having an intuitive and user-friendly interface is crucial for the game to experience acceptance and success.  When working on interface design, the users' needs have to be taken into consideration, and designers should therefore take into account the characteristics of elderly when developing a game for this user group \cite{mmi}. Especially, when designing an interface for elderly, the aspect of visual decline is important to focus on \cite{blindeforbundetTekst}. 

There exist various examples of guidelines for how to design good and user-friendly interfaces for elderly, or people in general, that are partially sighted or visually impaired. Visual decline implies less contrast sensitivity, decrease in peripheral vision, and more dark adoption \cite{ijsselsteijn2007digital}. Therefore, some of the main guidelines are simple design, large font size, and sharp contrast between background colour and font colour. The users might have different levels of visual decline, so it is preferable to allow users to change settings themselves \cite{blindeforbundetTekst} \cite{actionforblindpeopleTekst} \cite{w3cTekst}. Design guidelines are provided by various organisations, as e.g. "Blindeforbundet", a Norwegian federation for everyone with poor vision \cite{blindeforbundet}, "Action for Blind People", a charity in the UK which provides support to people that are blind or partially sighted \cite{actionforblindpeople}, and "World Wide Web Consortium (W3C)", a community working together to create web standards \cite{w3c}. These web standards are meant to be used as guidelines for making usable interfaces for the general user, regardless of age or disabilities \cite{w3cTekst}. These three organisations are just a few out of several others with focus on making technology accessible for elderly and people with visual disabilities. 

We will now present a summary of guidelines to consider when designing an interface for elderly, gathered from the three organisations mentioned above. The numbering of the guidelines will continue from the numbers presented in the previous section as all these will be used together as guidelines when developing the exergame.


\textbf{Make the text easy to read} \\ \\

\item Use a relatively large font size, minimum 12 pt \cite{blindeforbundetTekst}. A font size of 14-16 pt is preferable for those who are visually impaired. For large prints 16-22 pt should be used \cite{actionforblindpeopleTekst}. Use of font size larger than 20 pt will not increase readability in a text \cite{blindeforbundetTekst}.     
\item Use a clear print. This mean that a sans serif font should be used (see Table \ref{tab:fonts}). Arial, Helvetica and Futura are fonts that are easy to read \cite{actionforblindpeopleTekst}.

\begin{table}
\centering
\begin{tabular}{ l p{8cm} }

\multirow{1}{*}{\includegraphics[scale=0.6]{font1.jpg}} & This is the letter X in Times New Roman, a widely used font. The lines have various widths, and there is a broader design at the ends. \\ \\ \\ \\
\multirow{1}{*}{\includegraphics[scale=0.6]{font2.jpg}} & This is the same letter in Arial. This font is easier to read for those with visual decline. Here the lines are equally wide all over, without the expansion at the ends. \\ \\

\end{tabular}
\caption[Fonts]{Fonts with and without serifs [modified from \cite{blindeforbundetTekst}]}
\label{tab:fonts}
\end{table}
\item Avoid fonts with special styles \cite{blindeforbundetTekst} \cite{actionforblindpeopleTekst}.
\item For titles, use upper-case letters in beginning of words \cite{actionforblindpeopleTekst}. Do not use capital letters in large amounts. They give little variation, which makes the text difficult to read \cite{blindeforbundetTekst}. 
\item Do not use italics, underlining and bold typing in a larger coherent text, that will disturb the reading \cite{blindeforbundetTekst} \cite{actionforblindpeopleTekst}.  
\item If possible, allow the users to change font and text settings themselves \cite{blindeforbundetTekst} \cite{w3cTekst}. 
\item Use a language with ordinary, well-known words. This makes the text more easy to understand \cite{w3cTekst}. \\ 


\textbf{Use simple design}

\item Use simple design, as this makes it easy for important elements to stand out \cite{actionforblindpeopleTekst}.
\item Be consistent in design and layout throughout the system \cite{actionforblindpeopleTekst}.
\item Always use left-aligned text. Centered text appears unclear \cite{actionforblindpeopleTekst}.  
\item Keep spacing between words permanent. Do not stretch the text to get an even right margin, as this will change the spacing randomly. An uneven right margin is good for readability, as it help leading the eyes to the next line \cite{blindeforbundetTekst} \cite{actionforblindpeopleTekst}.
\item Do not use too much or too little spacing between lines \cite{blindeforbundetTekst} \cite{actionforblindpeopleTekst}. 
\item Avoid big blocks of text. The text will be easier to read and understand if it is divided into smaller sections \cite{blindeforbundetTekst} \cite{actionforblindpeopleTekst}. \\  


\textbf{Use of colours and contrasts}
\item It is important with sharp and clear contrast between background colour and font colour to ensure good readability \cite{blindeforbundetTekst} \cite{actionforblindpeopleTekst}. The actual choice of colours is not that essential.   
\item In general, the most preferred combination of background colour and font colour is black text on either white or yellow background \cite{actionforblindpeopleTekst}. Black on white or yellow creates a very good contrast \cite{blindeforbundetTekst}. 
\item People are different. Some might prefer the opposite, like light font colours on dark background \cite{blindeforbundetTekst}. This again suggests that it should be possible to customise settings. When using light font colours on dark background, the font should be both bigger and bolder to make the text more readable \cite{actionforblindpeopleTekst}.
\item Avoid pale colours on coloured background \cite{blindeforbundetTekst}.  
\item Choose a one-coloured background. Try not to write text on pictures. The picture will take the attention away from the text, and make the text harder to read. If necessary, put the text where there is light areas in the picture  \cite{blindeforbundetTekst}.  
\item Reading speed is also highly connected to colours and contrast  \cite{blindeforbundetTekst}. Black text on white background provides the highest reading speed in a text. For more combinations, see Figure \ref{fig:contrastreadingspeed}.

\begin{figure} [ht!]
\centering
\includegraphics[scale=0.5]{readingcolors.jpg}
\caption[Colours and contrasts - reading speed]{Relationship between choice of text and background colour, and reading speed. The best combinations are those shown on top. (Translated from Norwegian, \cite{blindeforbundetTekst})}
\label{fig:contrastreadingspeed}
\end{figure}

\item By choosing text and background colour according to the surrounding environment, readability can be increased \cite{blindeforbundetTekst}. Table \ref{tab:contrastenvironment} shows appropriate colour combinations for four environments. These combinations are meant for digital information boards out in the real world. The choice of colours will also make the board stand out from its environment.  

\begin{table} [ht!]
\centering
    \begin{tabular}{|p{3,8cm}|p{2,3cm}|p{4,9cm}|}
       \hline
       \textbf{Environment} & \textbf{Background} & \textbf{Text} \\ \hline
		Red brick, dark stone & White & Black, dark green, dark blue \\ \hline
		Light brick, light stone & Black, dark & White, yellow \\ \hline
		Whitewashed wall & Black, dark & White, yellow \\ \hline
		Green vegetation & White & Black, dark green, dark blue \\ \hline
    \end{tabular}
    \caption[Colours and contrasts - environment]{This figure shows how text and background colour should be chosen according to the environment to ensure good readability. (Translated from Norwegian, \cite{blindeforbundetTekst})}
    \label{tab:contrastenvironment}
\end{table} 


\textbf{Provide accessible information}
\item Provide alternatives for how to present information. For those who are visually impaired, an audio presentation might be preferable \cite{blindeforbundetTekst} \cite{w3cTekst}. 
\item Text is a better way to present information than use of images and icons \cite{w3cTekst}.
\item Give the users time to read the given information. Let them tell the system when they are finished reading \cite{w3cTekst}.  
\item Help users to easily navigate to the content of interest \cite{w3cTekst}.
\item Assist users on how to avoid and correct mistakes. Describe the error for the user, and provide a suggestion for how they can undo the action that caused the error \cite{w3cTekst}.      
\end{enumerate} 

The characteristics and guidelines drawn from previous studies together with the official guidelines proposed by "Blindeforbundet", "Action for Blind People" and "World Wide Web Consortium (W3C) will become very important  when we develop a concept for an exergame for elderly. To understand how to design this exergame, we also have to understand how video games in general can be designed. This will be described in the next chapter.





\cleardoublepage
\chapter{Designing Video Games} 
\label{chap:vg}
Exergames are one type of video games. Developing a concept for an exergame require us to understand what elements a video game is build up from. There are different phases of video game development. Making a design document is mainly the one video game designers are concerned about. This design document consists of several aspects of the video game, which are important for the developer to describe. These aspects will be important in our further work.

First, we will briefly present the development phases, with emphasise on the design document. Then we will describe important elements that a video game should consist of, and that should be included in the design document.  

\section{The Design Phase}
\label{sec:designphase}
In \cite{understandingvg} four phases of video game development are discussed: \emph{a conceptual phase}, \emph{a design phase}, \emph{a pre production phase} and \emph{a testing phase}. The first is a short description of the video game concept to sell the idea. This formulation often describes which platform the game will be build on and sometimes also concept arts. The next part of this phase is for the developers to make a \emph{game proposal} which includes market analysis, technical issues, budget projections, audiovisual style and how the game would feel to play \cite{understandingvg}. The following phase is the \emph{design phase}, which includes making a design document. The design document is what the game designers are most concerned about \cite{gamedesign}, and its goal is \emph{"[...] to fully describe and detail the gameplay of the game"} \cite{gamedesign}.  In \cite{gamedesign} they state that the design document should not include  technical details of the game design. However, in \cite{understandingvg} they include both the functional and technical specifications in the design document. The design document should at least include the story of the game, the different levels in the game world, what the players are allowed to do in the game and how they do it. In addition the design document will describe what characters and objects will be included in the game world \cite{gamedesign}. In \cite{understandingvg} they also include text, illustrations, mock-ups and concept drawings.  In \cite{gamedesign} they have separate documents for illustrations, mock-ups and concept drawings, which they describe as art bible, flow charts, and storyboards. Flow charts can be useful for visually showing decision making in the game, and is quite useful for scenarios where a lot of branching are involved. The art bible is described as \emph{"[...] the place where the look and feel of the game is comprehensively established in detail"} \cite{gamedesign}. Here it will be described what kind of art style will be used, in addition to sketches of the style. Making storyboards is a way to sketch or mock-up shots before they are filmed, and can for example be used as concept sketches to show how the game world will be seen from the player's view. 

The design document, should include the overall story-line. With a complex story line, the details can be described in what \cite{gamedesign} call the story bible.  The technical design document is in \cite{gamedesign} separated from the design document, while they join them in \cite{understandingvg}. This document focus on how the functionality will be implemented. This is out of the scope of this thesis, and will not be discussed any further. 

After the design document is finished the decision on the game engine must be made. The game engine is the software which provides the basic architecture of the game but not the concrete content, and includes handling the artificial intelligence, the audiovisuals and the physics. In addition, third-party software tools can be used to make for example some of the graphics. After the development of the design document, and the choices of the game engine and third-party software, it is common to start making working prototypes. These do not need to be playable \cite{understandingvg}. The next phases include production and testing. 

The reader should remember that the names of the different documents described in this section, in addition to the way they are divided or joined, is not the standard way of making game development documentation. This varies from developer to developer.

We will now move on to the next section, where we will describe the different elements that needs to be included in video games.

\section{Characteristics of Video Games}

Game designer Sid Meier came up with one of the most famous definitions of a game: "A game is a series of interesting choices" \cite{understandingvg}. This of course, does not apply for every video game. In \cite{understandingvg} the authors present the MDA-model, which was developed by Robin Hunicke, Marc LeBlanc and Robert Zubeck, after several workshops conducted at the "Game Developers Conference" in California between 2001 and 2004. This model divides games into three elements: \emph{mechanics}, \emph{dynamics}, and \emph{aesthetics}, and is a very useful tool for designers to understand games. \emph{Mechanics} are not something we can see or hear, but rather the rules and basic code of the game. This is for example the algorithms that lies in the ground for creating the reaction pattern of a computer-controlled character. \emph{Dynamics} are based on the mechanics and describe what events do and can occur during the gameplay, seen from the players point of view. \emph{Aesthetics} are about everything that can be experienced by the player. 

The MDA model has its limitations. It is very simple and it does not really represents how gameplay works. Also the aspects around playing experience fall outside this model, as well as the expressive side of the game. However, the writers of \cite{understandingvg} suggest that Sid Meier's definition and the MDA-model provide the most salient definitions of game, and works good for helping developers in their design work. In this thesis the \emph{aesthetics} is the most important entity, because this are all the aspects that the players experience, and it describes the entities that actually form a game. Gameplay is in \cite{understandingvg} an entity that falls under aesthetics. However, since this is a quite important entity, we will provide an own section for this. We will also describe the importance of a story, with settings and characters. First we present what aesthetics are and include. Some definitions and words used in this chapter might differ from how these words are used in "the real world". We will remind the reader that this chapter describes video games, which have virtual worlds, and that therefore some terms would not be used in the same way in real life.    


\subsection{Aesthetics}
The aesthetics describe everything that can be experienced by the player. These are the elements that actually makes the game. The aesthetics can be divided in three, namely \emph{rules}, \emph{geography and representation}, and \emph{number of players}. The rules say something about what the players can and cannot do, as well as what actions will make the score increase or decrease. The geography and representation element says something about how the video game is represented through graphics and sound. Here there are a lot of different design possibilities. The number of players is important, because there are huge differences between a single-player game and a multi-player game \cite{understandingvg}.

We will now look a little closer into the three different categories.

\subsubsection{Rules}
In \cite{understandingvg}, they distinguish between two types of rules; \emph{interplay rules} and \emph{evaluation rules}. The former are the physical laws in the gamespace, and determines what properties the different elements in the game should have, what can be done, as well as what will happen corresponding to the player input. An example of this is "What will happen when the player presses button A? Jump". The latter defines what rewarding and punishments will be given at certain actions \cite{understandingvg}. 

\subsubsection{Geography and Representation}
\label{sec:georep}
Geography and representation are about how the game is represented through graphics and sounds \cite{understandingvg}. Graphics and sounds are used to set the games environment and enhance the players enjoyment of the game, but has no effect on how the game is played.  

\emph{Graphics}\\
Graphics are needed in a videogame to in a best possible way imitate reality and enhance the player experience. This means that instead of telling out what is happening, it will be shown visually \cite{umlapproach}.
There are many aspects to consider when it comes to how the game should be represented. Table \ref{tab:graphic} shows an overview of basic geographical and representational characteristics. We will only discuss the choice of perspective, dimension and exploration in this thesis. 

\begin{table}
\centering
    \begin{tabular}{|l|l|l|l|l|}
        \hline
       \textbf{Perspective} & First person & Third person & &  \\ \hline
       \textbf{Dimension} & 2 & 3 & & \\ \hline
       \textbf{Space type} & Torus & Abstract & Free & \\ \hline
	   \textbf{Off-screen} & Dynamic & Static & None & \\ \hline
	   \textbf{Scroll} & Vertical & Horizontal & Free & None \\ \hline
	   \textbf{Exploration} & Forced & Free & None & \\
        \hline
    \end{tabular}
    \caption[Graphical Characteristics]{Basic graphical game characteristics \cite{understandingvg}.}
    \label{tab:graphic}
\end{table} 

It has to be decided if the game will be two-dimensional or three-dimensional and in first-person or third-person. Two-dimensional graphics are represented by only two coordinates and do not have any depth. This makes a very unrealistic scene. Three-dimensional graphics, on the other hand, has depth, and is therefore more realistic. All games will either have a first-person perspective, a third-person perspective, or a mix of both. In a first-person perspective, the game is played through the characters eyes, while in a third-person perspective the player can see the character she controls through the game. Games will also either have an isometric perspective or a top-down perspective. Isometric perspective is when three-dimensional objects are presented in a two-dimensional form, like in an architectural drawing, while a top-down perspective is, as the name suggest, when the scene is shown from above. The choice of perspective is important because it decides how the player will perceive the game world. It is important to choose the right perspective with the right dimension. A third-person game can be played in a two-dimensional and a three-dimensional space, while a first-person game should only be played in a three-dimensional space. Exploration says something about whether the player can move in the game at forced or desired pace. Another aspect experienced by the player is time. It is proposed to distinguish between play time and event time, where the former is the time a player uses playing the game, and the latter is the time in the game world. If play time and event time are the same, we can say that the game is played in real time. In many games it is possible to save the game, and return to the same state at a later time \cite{understandingvg}.

There are many different ways to present a game graphically. In \cite{understandingvg} they describe three different styles common for video games that were identified by Aki Järvinen: 

\begin{itemize}
\item \emph{Photorealism:} is defined as \emph{"a style of painting that tried to completely mimic photographs"} \cite{understandingvg}. There are two subcategories of photorealism: \emph{Televisualism} tries to give the same visual look as shown on the television. This is commonly used for sports games. \emph{Illusionism} is the other category. Here it is non-realistic content presented as photorealistic content. An example of this is when zombies or similar objects are presented in a way that seems like it is in real life.  
\item \emph{Caricaturism:} Here so called caricatures are used. Caricatures are drawings that presents an object by exaggerating the prominent features of the object, and often gives the feeling of a cartoon. 
\item \emph{Abstractionism:} In this category there are no real people or real-life objects involved. Instead  the game has a rather abstract form. An example of this is the well-known game Tetris. A problem with these kind of games is that they often face a hard time on the market. This is because humans mostly get attracted by the story, and it can therefore be hard to create attention just around the mechanics of the game. Therefore, it is within this category very important with the story \cite{understandingvg}. 
\end{itemize}

\emph{Sounds}\\
Sounds are important in video games for the same reasons as why graphics are important. Background music can set the atmosphere of the game, as well as give an indication on when actions will change and if the atmosphere is changing (for example from happy to sad). Sound can also be used as feedback to the player \cite{umlapproach}. There are different types of sound included in a  video game. \cite{understandingvg} distinguishes between four categories:
\emph{vocalization}, which is the voice of the characters in the game,
\emph{sound effects}, which are sounds made by the different objects in the game, \emph{ambient effects}, which are non-specific sounds that makes the atmosphere in the game, and \emph{music}, which is the game's soundtrack, and also a part of setting the atmosphere. The latter is a very important part of the game. Elements that can affect sound effects are \emph{the environment}, \emph{spatiality} and \emph{physics}. An example of how the environment can affect the sounds is what kind of floor the characters are walking on or what kind of weather conditions there are in the scene. The sound will also be affected by where in space it will come from. For example, a sound from far away sounds different from a sound close up. In addition, sound will be affected by movements. Imagine the sound of the sirens on a fast passing ambulance.

\subsubsection{Number of players}
\label{subsec:numbers}
The number of players is an important choice because it affects a number of game elements. In a single-player game artificial intelligence (AI) is important, because the player has to play against the computer and not real humans. The computer can never be as intelligent as the human brain, and generally has a very limited set of strategies. Therefore, it can be easy for an experienced player to win over the opponent. When designing multi-player games, artificial intelligence is not necessarily needed, as the player will play against other human players. In multi-player games it has to be possible to distinguish between characters. This can be done by giving them different unique features, but at the same avoid making any character superior. It is also important to facilitate social interaction between the players, like for example cooperation or competition \cite{understandingvg}. Often in multi-player games, it is possible to chat with the opponents or team members. 

\subsection{Gameplay}

"[...] gameplay is the most important element when designing a game" \cite{umlapproach}.  In \cite{understandingvg} gameplay is defined as "the game dynamics emerging from the interplay between rules and game geography". Gameplay is a result of the game's rules, while the feeling of the gameplay is influenced by the sounds and graphics. The dynamics can be of different types. They can be entertaining or they can be predictable, or they can be none of those. In a video game, everything really depends on the gameplay because it defines how the game is played. In short, gameplay can be describes as the interaction between the player and the game \cite{umlapproach}. Siang et al. describe two different kinds of interactions: player-object and object-object. When creating a game, a set of rules about what kind of interactions can be made have to be set. These rules are set by determining what kind of interactions are allowed between the two components: players and game objects. \cite{umlapproach} describes the player-token, or what we have called player-object, interactions with Figure \ref{fig:playerobject} (a), which is a description of gameplay, set aside the players experience.
\begin{figure}
\begin{center}
\includegraphics[scale=0.4]{player-object-merged}
\caption[The player-token interaction]{(a) Gameplay independent of player experience, (b) Gameplay depended of player experience (modiefied from \cite{umlapproach})}
\label{fig:playerobject}
\end{center}
\end{figure} 


As a part of gameplay there is a user interface. The interface is necessary to enable interaction between the player and the object. In \cite{umlapproach} Siang et al. present a figure describing the gameplay, see Figure \ref{fig:playerobject} (b). The interface is where and with what the player interacts with the game and includes the input that is sent through the input devices, like a game control, and the output the user receives on the screen and through the speakers \cite{umlapproach}.

\subsection{Story}
One important element of a video game is the story, or the narrative. The story is included in the game to make the involvement and enjoyment better for the player. The story is just a part of the game, and is not the game \cite{umlapproach}.  Different events make up the narrative and will usually contain settings and characters \cite{understandingvg}. Unlike literature and film, which centers on the story, the game centers on play. Therefore, the narrative should be looked at in a player-centric context \cite{gametheory}. Players often make their own story throughout the game \cite{umlapproach}. This means that the game designer creates the background storyline, while the player creates the experimental story while interacting with the game or others. 

In \cite{understandingvg} they introduce interesting elements of the narrative in a video game, and discuss three different categories: \emph{The fictional world}, \emph{Mechanics}, and \emph{Reception}.

\subsubsection{The fictional world}
\label{subsub:fictionalworld}
This category includes settings and actors, which can also be thought of as the \emph{what} and \emph{who}. In \cite{beram}, this is described as objects. Objects are defined as the elements that form the gameplay by adding depth, atmosphere and interactivity to the setting. The \emph{setting}, or \emph{location} as called in \cite{beram}, can also be considered as an object. The setting will be described later in this section.

\cite{beram} defines three categories of objects:
\begin{itemize}
\item \emph{Structural-objects (static):} These objects do not change visually and are in the game to add functionality. Examples are trees, table, walls.
\item \emph{Interactive-objects/entities (dynamic):} These objects will react to the input from the player and other events in the game and will add more functionality to the game. The character is an example of an interactive-object.
\item \emph{Scripting-objects (behaviour):} These objects are not physical, but they are conceptual parameters that adds functionality to the the other object-types. The parameters say something about how the objects will react to events. One example is a game manager that contains parameters like number of life and high scores \cite{umlapproach}. 
\end{itemize}

The game space, which is the \emph{game setting} or \emph{location}, is an important part of a game. The game space is a reproduction of some of the features of the real world and contains specific rules to make it possible to play the game. There are different characters in a game. There are both the characters that the game is about and characters that make things happen and that interacts with the main characters. Therefore, all the different characters need to be considered. \cite{understandingvg} proposes four different categories of characters: 

\begin{itemize}
\item \emph{Stage Characters:} These characters can not be interacted with, and serve more as just a part of a scenario. \\
\item \emph{Functional Characters:} These are also just a part of a scenario, but in addition, they have a general function, which makes it possible for the player to interact with them in some way. \\
\item \emph{Cast Characters:} These have specific functions in the game that have something to do with the story. \\
\item \emph{Player Characters:} These are the characters that the player is in control of. 
\end{itemize}

The different characters can be constructed in different ways. They can be constructed through description, which means through the way we can see them on the screen. This can be done in a symbolic, naturalistic or a "real-life" way. See Table \ref{tab:description} for examples.

\begin{table}
\centering
    \begin{tabular}{|l|l|}
        \hline
        \textbf{Type of description} & \textbf{Example} \\ \hline
       Symbolic & Red haired girl is diabolic  \\ \hline
       Naturalistic & Villains are ugly, while heroes are good-looking \\ \hline
       Real-life & David Beckham as a football player \\ \hline
    \end{tabular}
    \caption[Different ways to describe characters]{Different ways to describe characters \cite{understandingvg}.}
    \label{tab:description}
\end{table} 

The characters can also be described through their actions, through their relationship to space, through other character’s views, or just through a meaningful name. The player character is the most important type, and was by Toby Gard divided in three different categories, in accordance to how easy the player will identify with them. The different categories are \emph{avatars}, \emph{actors} and \emph{roleplaying}, and are defined as:

\emph{"Avatars are a non-intrusive representation of ourselves, actors are always part of a story (or have a story, albeit minimal sometimes), and roleplaying characters have very different abilities that we can raise according to our performance"}. \cite{understandingvg}

Usually, the use of avatars implies that the view is in first-person, which means that the avatars can not be seen, while actors usually will be seen in third-person. In addition, actors will usually have a personality and they are well integrated in the story, while an avatar usually do not have a personality at all. Roleplayers are quite different. Here the players create their own characters. The player can choose their name, and their abilities, as well as if they want to play in first-person or third-person.  

\subsubsection{Mechanics}
This category describes how to organise narrative action. There is a basic concept for how to organise the story in a game, which is called "branching". This means to have multiple paths in the story. 

\begin{figure}
\begin{center}
\includegraphics[scale=1.0]{linearFiction}
\caption[Classical linear fiction]{A model of classical linear fiction \cite{understandingvg}}
\label{fig:linearfiction}
\end{center}
\end{figure} 
\begin{figure}
\begin{center}
\includegraphics[scale=1.0]{interactiveFiction}
\caption[Interactive fiction]{A model of interactive fiction \cite{understandingvg}}
\label{fig:interactivefiction}
\end{center}
\end{figure} 

Figure \ref{fig:linearfiction} shows the standard progression of a story of linear fiction. In traditional stories, it goes through a resolution. This can not be applied in a game, because it would not let the player do anything. Therefore, a model of interactive fiction is applied for a game. As shown in Figure \ref{fig:interactivefiction}, this model has no continuous curve. In this model it is more about finishing each chapter by solving puzzles, and it relies on the emotional satisfaction the player gets from the victory of solving a puzzle. Even though an action in a game can lead to different endings, the player is actually solving a story. These games are called \emph{progression games} because the player has to finish different actions and chapters, before proceeding. The different chapters are cumulative, meaning that each chapter are building on each other. It is very common in these kind of games that it is a climax or a resolution at the end of each chapter (for example "fight the boss", which is quite common in many games). 

The other structure of narratives, is \emph{emergence games}. This structure depends on a more active artificial intelligence where all the objects in the game have behaviours. To understand the difference between progression games and emergence games we provide an example directly drawn from \cite{understandingvg}: \emph{"In a progression game, the dragon will always attack the player when she steps into the cave, but in an emergence game, this might depend on how the player behaves towards the dragon, which is a more "active" object with a few possible different responses."} 

To be able to easily explore the smaller events in a game, game designers are making so called "quests", which is the missions players have to perform in the game. When the different quests are defined, it can be put together and be told as a story. In \cite{understandingvg} they say that \emph{"ideally, quests are the glue where world, rules and themes come together in a meaningful way".}

In this chapter we have discussed different elements of a video game. These are important elements that need to be studied to understand video games, and can be used both for evaluating existing games, and for developing new games. This elements will be used as a framework for structuring the functional design of the exergame concept. This can be found in Chapter \ref{chap:concept}.


\cleardoublepage
\chapter{How to Design a System}
To design and develop a system, there are several aspects to take into consideration. What is the system going to be? What functionality shall the system hold, and what should the system look like? Who are the users, and what are their needs? What software should the system be built upon? There are many questions to be asked and figured out before starting with the development. In this chapter we will present theory about some aspects of system development. We will in Section \ref{sec:fourpillarsofdesign} look into general guidelines for how to design a system. An important part of system design is to set up system requirements, and this will be presented in Section \ref{sec:systemreq}. Since the focus of this master thesis is to design an exergame for elderly, we will also discuss various aspects of usability, both in general and specific for elderly. This will be done in Section \ref{sec:usability} and \ref{sec:designguide}. At the end of this chapter we will present a list of heuristics implemented in a model of "flow". This model will serve good to evaluate a system's usability and for developing a system in an early phase.

\section{The Four Pillars of Design}
\label{sec:fourpillarsofdesign}
It is the work of an interactive system designer to combine the sense of what attracts the user with system functionality. To help these designers develop successful systems, a theory called the four pillars of design has been developed. The four pillars of design does not guarantees brilliant systems, but it could be helpful along the way in a development process. The four pillars of design consist of \emph{user-interface requirements}, \emph{guidelines documents and processes}, \emph{user-interface software tools} and \emph{expert reviews and usability testing} \cite{mmi}.    

\subsubsection{User-interface requirements (Ethnographic observations)}   
A major key to success when developing a system is connected to specifying the user requirements, and how well these requirements are defined and understood. The way to specify requirements differs from system to system, but what the final result always should include is the same, the system's \emph{context of use}: who should use the system, where should it be used and what should it be used for. 

\subsubsection{Guidelines documents and process (Theories and models)}
It is important for the interactive system designer to generate a document that obtains a set of guidelines which specifies how the design should be. This could be design guidelines for the whole system, or it can be design guidelines for parts of the system, like functional design, implementation design, and interface design. Companies like e.g. Apple uses guidelines documents to specify design principles developers should follow. This is to create consistency in design across systems and products. Design may differ as different systems have different needs, but there are still some elements that should be considered in the guidelines document. It is important that the guidelines are flexible, so that they can adapt to changes in needs and experiences \cite{mmi}. Examples of what guidelines should describe are:

\begin{itemize}
\renewcommand{\labelitemi}{$\bullet$}
\item Words, icons, and graphics.
\item Screen-layout issues.
\item Input and output devices.
\item Action sequences.
\item Training.
\end{itemize}

\subsubsection{User-interface software tools (Algorithms and prototypes)}
In the early stages of development, it is difficult for users to picture what the final result will look like. One way to address this problem is to let the users get a realistic impression of the final result, by presenting different types of mock-ups and prototypes \cite{mmi}. What prototypes are, and how and when they should be used will be presented in Section \ref{sec:prototypes}. 

When deciding on which development environment to use, there are numbers of good products to choose from. Most of them are easy to use, and offers good features. The important part is for the developers to choose the development environment that is most suitable for the product they are going to make, due to performance, cost, and how easy it is to use and learn \cite{mmi}.
	
\subsubsection{Expert reviews and usability testing (Controlled experiments)}
To be able to launch a successful system, it is important with testing along the way in the development process. System testing could involve both experts and the intended users \cite{mmi}. 

\section{System Requirements}
\label{sec:systemreq}
To be able to make a system, designers must start by finding out what the system actually is going to be, what it should do and what functionality that should be included. Finding out and documenting all this is called a requirement analysis. The requirements in this analysis should focus on the role and the purpose of the system, viewed from the environment. They should include what is essential to the system, and avoid details that are unnecessary. \emph{How} the system is going to be realised, is not common to include in these requirements. From a requirements analysis there is natural to produce what is called a requirement specification \cite{braude2000software}. Requirements specification is defined in \cite{systemutviklingDel1} as \emph{"A specification that sets forth the requirements for a system [...]. Typically included are: functional requirements, performance requirements, interface requirements, design requirements and development standards"}. The different requirements are often separated into two categorisations, functional and non-functional requirements. 

\subsection{Functional Requirements}
Functional requirements are about system behaviour, and the provided functionality seen by the user \cite{systemutviklingDel1}. In addition it says something about input/output devices, range of users etc. \cite{mmi}. \cite{systemutviklingDel1} mentions examples of functional requirements:
\begin{itemize}
\item \emph{Requirements to ideal (functional) services and dialogues, possibly on several layers of abstraction.}
\item \emph{Requirements to ideal (conceptual) knowledge about the environment.}
\item \emph{Requirements to size: numbers of users, terminals etc.}
\end{itemize}     

\subsubsection{Functional Design}
Functional requirements will be used as a primary input to functional design. Functional design should say something about the complete functionality of the system that can be seen by the users, and is defined as \emph{"what the system shall do in a way that can be compared to the functional requirements. [...] it provides a basis for selecting the implementation. It is therefore idealised with respect to the concrete system and will hold for a range of technical solutions"}. Functional design is independent from technology and will not say anything about how the system is going to be implemented \cite{systemutviklingDel1}. 

\subsection{Non-Functional Requirements}
Non-functional requirements include performance requirements, interface requirements, reliability and availability requirements, error handling, and constraints. Performance requirements address speed, capacity and memory usage. Reliability and availability requirements specify reliability in quantitative terms, and the amount of time the system should be available to the users. Error handling is about how the system should respond to errors, and interface design says something about how the system will interact with the users. Constraints restrict how the system should be designed and implemented. This is done by describing accuracy, tool and language to be used, design constraints, which standards that should be used, and the hardware platform the system should be built upon \cite{braude2000software}. To summarise, non functional requirements are about ensuring quality of a system, and it says something about constraints on hardware, software, and the implementation of the system in general \cite{mmi}. These requirements are hidden from the users point of view. Examples of non-functional requirements are presented in \cite{systemutviklingDel1} as:
\begin{itemize}
\item \emph{Requirements to (concrete) physical interfaces.}
\item \emph{Requirements to physical conditions like temperature, humidity, power, consumption etc.}
\item \emph{Requirements to processing capacity: response times, traffic load etc.}
\item \emph{Requirements to exception handling.} \\ \\ 
\end{itemize}   

\subsubsection{Implementation Design}
Non-functional requirements form a basis for implementation design. While functional design is about what the system shall do, implementation design describes how the system should be realised. Implementation design connects the technical solution with the functional design, which makes the "manual" for how the final system will be implemented \cite{systemutviklingDel1}.  

An important part of developing a system is to decide and specify functional and non-functional requirements. It is not always easy to distinguish between the different requirements, in terms of which category they belong under. It is important to notice that categorisation of requirements is not the essential part. What is the main goal is to define and express the requirements as clear, simple and understandable as possible \cite{systemutviklingDel1}.  

\section{The Importance of Usability}
\label{sec:usability}
When developing a system it is important to keep in mind usability. Usability says something about how easy it is to use, learn and understand a human-made system. Examples of systems can be a machines, software applications, websites, tools, or anything else that involves human interaction with an object. Usability is often used in association with technology development, in terms of making digital systems understandable and intuitive for the users through user-friendly interfaces. Usability has played a huge part in the evolvement of bringing digital systems into people's homes and everyday life. The first computers and digital systems that were developed consisted of complex and not understandable applications that only professionals with special knowledge could use. There was little focus on simple and accessible systems, and complex interfaces were actually appreciated and gave the system credibility. First, when computers and digital systems were developed with the intention of being used by the normal user, developers had to think about usability. The developers had to put the user in the center of the computer system, and not only focus on functionality and system features \cite{mmi}.

We have experienced a great shift in technology from the first computer was invented and until today. Technology has been more mobile due to laptops, smart phones and other portable devices, and it is also used more often because of instant messaging, e-business and social networks \cite{mmi}. Users are no longer just "computer professionals", but normal people in all age groups, with different skills and interests, that are both experienced and inexperienced with technology. This has been possible because of designers and researchers with focus on human needs. The term human-computer interaction was created, which is about including psychology in developing human-centric design. This is not an easy task, and it includes people from  many different sciences. Ben Shneiderman lists "psychologist, instructional and graphic designers, technical writers, experts in human factors or ergonomics, information architects, and adventuresome anthropologists and sociologist" as people to be included in the process of saying something about usability and human-computer interaction \cite{mmi}.  

Usability is a wide and quite abstract term, and it is not easy to understand, to measure, or to practise right. ISO 9241-11 states usability as \emph{"The extent to which a product can be used by specified users to achieve specified goals with effectiveness, efficiency and satisfaction in a specified context of use."} \cite{usabilitydef}. From this definition we can see that there are three elements that could help us say something about a system's usability; \emph{effectiveness}, \emph{efficiency} and \emph{satisfaction}. \emph{Effectiveness} measures to which degree the systems covers all necessary functionality, and how easy they are to use and understand. \emph{Efficiency} is about how well different tasks are performed. This requires measurement on how much time that is used to accomplish a task. The last element is \emph{satisfaction}, which is about the overall user experience. This could be measures through interviews, studies, questionnaires etc. The degree of satisfaction is important for the system to be accepted \cite{mmi}. 

\subsection{Context of Use}
\emph{Context of use} is an important concept within the definition of usability, and is defined in ISO 9241-11 as \emph{"users, tasks and equipment (hardware, software and materials), and the physical and social environment in which a product is used"} \cite{maguire2001context}. The degree of usability and quality in user experience for a system is dependent on how well it is related to its specified context of use \cite{bevan1995human}. A system will be used by a specific population for specific reasons within a specific environment. It is therefore crucial that the system fits the needs of its intended users, tasks, equipment, and environment, as depicted in Figure \ref{contextofuse}. Analysing a system's context of use will help developers to specify who the users are, what are their characteristics, which functionality do they want, and where and in which circumstances do they want to use it. This understanding about the system can be used all through the development process, from system specification to the test phase \cite{maguire2001context}.

\begin{figure} [ht!]
\centering
\includegraphics[scale=0.5]{contextOfUse.jpg}
\caption[Context of use]{This figure shows how ISO 9241-11 presents a product's context of use. Modified from \cite{bevan1995human}.}
\label{contextofuse}
\end{figure} 

\subsection{Simplicity}
\label{sec:simplicity}
Making good, intuitive, easy to understand systems is essential for a system to be successful, accepted and used. "Make it or break it" is a slogan that connects well with success and acceptance. A system can possess the best functionality there is, but if the users do not understand how to use it, the system will fail. An example of this is Apple's huge breakthrough when they launched their iPhone in 2007 \cite{iphone2007}. One might associate the invention of the touch phone with Apple's iPhone, but the truth is that touch phones was invented long before the iPhone. The first touch screen was published as early as in 1968, where it was used for air traffic control, and the first smart phone with touch screen technology was released early in the 1990s \cite{touchphone}. Apple was an eager participant in the development of touch screen devices. Already in 1983 Apple had a prototype of a touch screen phone \cite{applefirst1983}. Apple's success with their iPhone is based on focus on user's needs throughout the development process, which has resulted in good, intuitive and user-friendly design and interfaces. 

"KISS" and "Less is more" are other terms related to usability. "KISS" is an acronym that stands for "Keep it simple, stupid". This principle was formulated by the American aircraft designer Kelly Johnson in the middle of the 1900s, and it states that simple systems work better than complex ones. KISS is not related to stupidity, but rather to intelligent systems that due to their simplistic design may be perceived as stupid. The KISS principle has been adopted into software engineering, and subjects as design and usability. It states that simplicity should be the main focus in design, and that every element that leads to unnecessary complexity should be avoided \cite{kiss1} \cite{kiss2}. Ludwig Mies van der Rohe was a German architect that used the term "Less is more" to describe his extreme simplistic and minimalistic design style, and his use of that term became a guiding principle in modern design. "Less is more" has also been widely used as a slogan in association with usability. \cite{rohe}. Minimalistic design can be described as \emph{"design at its most basic, stripped of superfluous elements, colors, shapes and textures"}. With minimalistic design, the most important elements are brought into focus. In this way the user will not be distracted from, or miss out on, the content that is important \cite{lessismore}. Also big companies, like Microsoft, focus on simplicity in their design. Microsoft has launched an article called "The Importance of Simplicity" in their developer network. This is about how to design user-friendly systems while still keeping good functionality. Microsoft presents a topic called "Simple Can Be Powerful". This means that simplistic design not necessarily implies lack of functionality. Simplistic design will provide ease of use for first timers. The idea is to present a design that is intuitive, understandable and easy to learn, with a possibility for the experienced user to choose to add more functionality. A possible solution could be to include customisation so the users can set up their own workspace, and include more features if wanted \cite{msdnsimple}.            
    
\section{Design Guidelines}
\label{sec:designguide}
In order for a system to become successful it has to be easy to interact with, and it has to offer functionality that are attractive to the user. There have been developed several guidelines to help designers make successful, user-friendly systems. In this section we will present a list of eight principles with focus on interface design, called "The Eight Golden Rules". We will also look into designing interfaces with focus on the senior user group. 

\subsection{The Eight Golden Rules}
\label{subsec:golden}
The "Eight Golden Rules", presented in \cite{mmi}, are a set of guidelines that have been developed over three decades with research and experiences. It does not exist a solution for how to make good and user-friendly interface design, but these "Eight Golden Rules" can serve as a starting point and a helpful design guide if they are used correctly. When using the "Eight Golden Rules", it is important that designers refines and implements the principles into the environment they are working in. 

We will now present the "Eight Golden Rules" \cite{mmi}:

\begin{enumerate} 
\item \emph{Strive for consistency:} Consistency in interfaces requires identical terminology for actions and layout. This is important for users not to wonder whether words, icons or situations means the same. 
\item \emph{Cater to universal usability:} Designers have to see the need for making a design that fits the diversity of users. There could be differences in age and technology experience, that requires transformation of content. Beginners would need guiding and explanations, while experts should have features for short cuts. This could improve quality of the system experience. 
\item \emph{Offer informative feedback:} The users should always receive feedback on their actions. Appropriate system feedback should be chosen in accordance to the importance of the actions performed. Process bars, sound as a response for clicking a button, or visual presentation for showing object in actions, is possible ways to give users feedback on actions.  
\item \emph{Design dialogues to yield closure:} It is important to create distinct work steps in dialogues. This means organising similar actions into separate groups with a clear start, middle, and an ending. To provide users with a feeling of accomplishment, feedback should be provided when a particular sequence is finished.     
\item \emph{Prevent errors:} The best solution to this problem would be not to experience any errors at all. Designers should prevent users from doing serious errors by e.g. not allowing inappropriate digits in a field or "hiding" buttons that could cause errors. However, when errors do occur users should be provided with informative instructions for how to recover from the problem.   
\item \emph{Permit easy reversal of actions:} Users should always be provided with the possibility to regret a performed action. This will make the system easy and comforting to use, as users know that every action can be undone. 
\item \emph{Support internal locus of control:} User should feel that they are controlling the interface, and not the other way around. This might be especially important for experienced users. Surprising changes in design and actions, in addition to boring, time-consuming situations, will not be well received. 
\item \emph{Reduce short-term memory load:} Designers should reduce the need for memorising information and how actions should be performed. The focus should be on designing an interface with visible information and intuitive actions.
\end{enumerate}

These presented guidelines are far from being the only guides for how to design a user interface. There have been done a huge amount of research in the area of usabilit. Jacob Nielsen is one of the participants \cite{nielsen2005ten}. He is a Ph.D from Denmark, and an expert in human-computer interaction. He has established a movement for how to easy improve user interfaces, invented several methods for how to achieve good usability, and he has also published a great amount of articles and books with usability as main subject \cite{JNielsen}. As a part of his work Nielsen has created a list of ten usability heuristics, which can be used as general principles when designing a user interface\cite{nielsen2005ten}. We will discuss heuristics in more detail in Section \ref{sec:heur}. Now, we will present difficulties and possible solutions related to developing interfaces for elderly users.

\section{Heuristics}
\label{sec:heur}
Heuristics are designed guidelines made to assess how good software design is, and it has become a widely used method for usability evaluation in software development. As learned from this chapter so far, it is important to develop software interfaces that are easy to understand, learn and conduct. Heuristic evaluation method allows for insight into users' point of view, even before there is an actual system, and is actually best suited in an early phase, before spending a lot of money on expensive prototypes \cite{desurvire}. As mentioned in \ref{subsec:golden}, Jacob Nielsen developed a set of heuristics which can be used as guidelines when developing user interfaces. These can be found in \cite{nielsen2005ten}. However, these heuristics are made for software development in general. We sought to find heuristics that were more applicable for game development. 

We found that a lot of research have been done on heuristics for games and different sets have been suggested. Some worth mentioning are Desurvire et al. \cite{desurvire}, Malone \cite{malone}, Shelley \cite{shelley}, and Federoff \cite{federoff}. Many of these overlap, and in some way, they all tell the same. 

Sweetser and Wyeth discovered that many of the heuristics proposed in the literature, did not evaluate the enjoyment in games. They argue that the many valid sets of heuristics presented in the literature should be integrated in a model where also player enjoyment can be assessed. How much someone enjoys something can be described by the concept of flow. The concept of flow was first proposed by  Mihaly Csikszentmihalyi, when he many years ago started  to study how people could be so immersed and engaged in something they did not get money for. He wanted to find out why they did these things. He found that the reason was the enjoyment they felt when doing it. He called this state "flow" because "many of the respondents described the feeling when things were going well as an almost automatic, effortless, yet highly focused state of consciousness" \cite{flow}.  Sweetser and Wyeth integrated the already existing heuristics into the model of "flow", and called this new model "GameFlow".  They argued that the nature of flow fits well as a way to structure the different heuristics found in the literature, into a model of player enjoyment. The "GameFlow" model has eight core elements which are related to Csikszentmihalyi's defined elements. The core elements are: \emph{concentration, challenge, skills, control, clear goals, feedback, immersion} and \emph{social interaction}, see Figure \ref{fig:gameflow1} and \ref{fig:gameflow2} \cite{sweetser}. 


\begin{figure} [ht!]
\centering
\includegraphics[scale=0.9]{gameflow1}
\caption[GameFlow criteria for player enjoyment in games, part 1]{GameFlow criteria for player enjoyment in games, part 1 \cite{sweetser}}
\label{fig:gameflow1}
\end{figure}  

\begin{figure} [ht!]
\centering
\includegraphics[scale=0.7]{gameflow2}
\caption[GameFlow criteria for player enjoyment in games, part 2]{GameFlow criteria for player enjoyment in games, part 2 \cite{sweetser}}
\label{fig:gameflow2}
\end{figure}  

We find most of the guidelines listed in the model also relevant for an exercise game for elderly. We will now emphasise the guidelines that based on literature discussed in previous chapters, are found to be most relevant. 
To do the exercises right and properly the person doing it should be \emph{concentrated} on what she is doing. Therefore, we evaluate guidelines 3 and 6 to be important. Not all exercises are fun to do, however, most exercises are good for your body. 4 states that the player should not have to do tasks that do not feel important. Therefore, it is crucial to give the player information about how the tasks in the game relates to exercise so that the tasks seem meaningful. \emph{Challenge} is also important in games for elderly. This is a user group with diversity, where many suffer from different types of physical and mental decline. It is not fun to play something that is too easy, and it is definitely not fun to play something that the player is not able to do. Therefore, number 7 is in particular important.  In addition, 8, 9 and 10 are important, because as with all other user groups, elderly will experience learning effects, and will get better. The possibility to add more functionality after initial tasks are managed, and offering different difficulty levels were suggested in Chapter \ref{subsec:guidelines}. In Chapter \ref{sec:motivators} self-efficacy is discussed as an important determinant of exercise behaviour and self-esteem is shown to be an element that makes people continue doing physical activities. Self-esteem will likely be reached when challenges are overcome.  The element \emph{player skills} is related to challenge. Because of memory related problems seen in many elderly as discussed in Chapter \ref{subsec:guidelines} we found especially 14 to be relevant. Players should not be bothered with long textual information beforehand, but should rather learn through tutorials as they are playing. 15 is important for the same reasons as why 8,9 and 10 are, and should be considered. In this case, the Kinect sensor have the property of being able to detect the people playing (see Appendix \ref{app:kinectsensortech} for a brief description of the Kinect technology), and can in that way detect if the movements are been done right or not. This makes it possible to individually increase the players skills at an appropriate pace for this player. As suggested as guideline G.4 in Section \ref{subsec:guidelines} 17 should be met. This is in particular important for this user group, because of their inexperience with technology, as well as the common problem of decline in visual acuity \cite{Billis}, \cite{gregor}. The feeling of \emph{control} is an element in both the game flow model and in the eight golden rules presented in \ref{subsec:golden}. Control is especially applicable for exercise games. This is because for the player to do the exercises right a realistic picture of the player and her movements should by shown on the screen.  18 should therefore be considered. 19 and 20 are crucial for the player to be able to interact with menus and to start and stop the game. Because of many elderly's lack of technology experience, this should be presented in an easy and understandable way. If the player does not understand how to interact with the game, she is likely to not use the game. As for every other technology system, errors should not affect the user. This makes also 21 relevant. The game should meet the requirement of \emph{clear goals} and especially 25. It is especially important to present intermediate goals, because of memory related problems which are quite common in elderly people, as discussed in Chapter \ref{subsec:characteristics}. Goal setting is also discussed as important to promote adherence to physical activity in Chapter \ref{sec:motivators}. Appropriate \emph{feedback} should, as suggested in \ref{subsec:guidelines}, be given as the player achieve goals, but at an appropriate time without disturbing the player. Therefore, 27 and 28 are relevant. \emph{Immersion} is about feeling involved in the game and is a determinant of player enjoyment. In Chapter \ref{sec:barriers} it is discussed that many look at exercising as time-consuming, and many see exercising as boring (Finne en kilde på at trening er kjedelig). Therefore, an altered sense of time will be positive for the exergaming experience, as described in guideline 31. At last, social interaction is, as mentioned several times in the previous chapters, a very important aspect to get elderly to exercise. 35 and 36 are not necessarily relevant in this case, because as discussed in Chapter \ref{chap:exforseniors} the elderly user group does not like to play over the internet. However, 34 should be considered, with emphasis on cooperation, as discussed as the preferable way of multi-play by \cite{Gajadhar}.

The GameFlow model will be used in our further work, both when evaluating elderly's enjoyment of existing commercial games, and when developing a early-phase game concept. To come up with a game appealing to elderly, this user group has to be involved in the development phase. We will now move on to the next chapter, where we will discuss the methods used to involve the user in the development process.

\cleardoublepage
\chapter{Methodology}
\section{Qualitative Research}
Qualitative research method is used to give an understanding of a phenomenon and is well suited when studying sensitive and personal topics. Interpretation is therefore important in qualitative method.  The focus lies on how and why things are done, and not how many who does it, like in quantitative methods. Quantitative methods include a huge sample, while qualitative research can give more information about a small sample. While in quantitative methods structuring is very important, flexibility, in the sense that the scheme can be changed during the research if needed, is important in qualitative research. Also different from quantitative research is that it is not common to use numbers because the number are too small. The process can be divided into phases, that partially overlap. The first phase consist of defining what the research will find out. The next phase is the data gathering phase. This phase can be performed by different methods. The last phase includes interpreting and analysing of the data as well as formulating theories. In the last phase, the results will be presented. Data gathering and analysis should be done in parallel so that further data gathering can be adjusted from what have been found in earlier analysis. It is common in qualitative research that close connection between the researcher and the people who is being studied. Qualitative research is well suited for topics that there has not been done a lot of research on, because on this kind of topics, openness and flexibility is required. The different data gathering methods are divided in four \cite{qualitative}: \\ \\
- observation \\
- interview \\ 
- document analysis, and \\
- analysing of video and audio recordings \\ \\
The most common methods used are interview and participatory observation. Common for  these methods are that the researcher will establish direct contact with the people who is being studied. ,This contact is important for the data the researcher will gain from the study \cite{qualitative}. \\ \\
It is common for most qualitative methods is to document the data to be analysed textual. The documentation can include what people do, their statements, their intentions or their perspectives. The text can be notes from the field, printouts of  recorded interviews. \\ \\
The close contact established between researcher and informants introduces some ethical  challenges, described in the next subsection. Ta med noe om systematikk og innlevelse? All results conducted from the research needs to be precisely and correctness in the presentation of the results and when other researchers work is being evaluated. This includes not plagiarize other people's work, which means to copy other people’s work and take credit for it. When referring to other people’s work, it is important to properly state the resource. When working in close contact with the informants, the researcher often gets personal information about the informants. With personal information we mean information that can be related to individuals. In project were personal information needs to be handled the project needs to be reported. In research projects performed at universities, where personal information is being handled, the project needs to be reported to “Norsk Samfunnsvitenskapelig Datatjeneste” (NSD), which is in care of data protection for these institutions. NSD will evaluate each project in accordance to research ethical rules \cite{qualitative}. \\ \\
\subsection{Ethical Challenges:}
As mentioned, there are some ethical guidelines that needs to be followed when working in close contact with informants. These will be described in this section. \\ \\
\textbf{Informed Consent:} \\
In a research project with people involved, these is a requirement that the researcher has the participant’s informed consent. This means that the informants have gotten all the information they need to know about the participation and have self chosen to participate and that they can withdraw at any time without any consequences. One challenge about this, is that in some projects too much information can affect the participants behaviour (If they for example know too much about what the researchers are looking for). In this thesis we does not see this as a problem, and the participants will be informed about every important aspect of the study \cite{qualitative}. \\ \\
\textbf{Confidentiality:}\\
Researchers are required to keep all the information they collect about a participant confident. This means that the information have to be anonymized. This also involved strict requirements to how personal data that makes it possible to identify individuals should be stored and annulled. There are rules about how long data can be stored, and general principles are that data should be stored for only the amount of time there is use for the data, and that data that can be directly linked to an individual should be stored separately and not electronically.  Reuse of data is not allowed without consent from the participants \cite{qualitative}. \\ \\
\textbf{Consequences of participating in research projects:}\\
The researcher has responsibility over the participants safety and should respect their wishes. It is important to have thought through what consequences the execution of the research has for the participant. The researcher is required to protect the participants’ integrity during the process \cite{qualitative}. \\ \\
\subsection{First Phase} 
The first phase in a qualitative research method is to define what wants to be found from the research. A problem description (research questions?) should be prepared and a design for how the research will be performed should be developed. The design consist of guidelines of how the project should be carried out and includes: what the study will focus on, who are the informants, where the study will be performed, and how the study will be performed. Vet ikke om vi skal ha med noe om prosjektbeskrivelse? det går på hvem som skal jobbe, hvordan finansiere osv..A problem description needs to be developed in order to know what the study should be focusing on. This will contain research questions that the researcher wants to get information about. The problem description should be clearly prepared and it should be limited realistically within the framework of the project, and at the same time be open enough so that interesting topics that will appear during the process can be studied. It is important to that the problem description is being modified and worked with during the research process, because the researcher will gain new knowledge and understanding during the process that can be important for the research. The choice of problem description should be justified be describing why the problem the researchers will study is important. 
\subsection{Observation}
The observation method is used when the researcher wants to see how a group of people behave in a specific setting. In observation studies, is one important decision how the observer will perform in the field. This varies from project to project. The observer can be a participant or just an observer, and the observation can be open or undisclosed. Participation observation is something in between being a complete observer and complete participation, where the researcher participate in the same way as the informants. This involves the researcher being present in the setting of the participants and observe how they act. The researcher participate in the session, in the sense of interacting with the participants while they are performing the tasks. This means that the researcher does not do the same as the participant. Participation observation is well suited in research of a new and immature topic. It is very common to combine observation with interviews. This is for the researcher to verify or discard the understanding he or she has acquired during the observation. In most research it is important to study the behaviour in the informants own environment. This will give a more natural behavior. However, it is important to acknowledge that the informants may not find the environment “natural” when the researchers are there observing them. How the researcher presents the project for the informants is important for the people the researcher wants to observe will be willing to participate. The researcher should in a trusting way, present him or herself and the project.\\ \\
We will use the observation method to see how elderly respond to the use of Xbox Kinect. We will look for things that supports what we have read in the literature, as well as identify new aspects that we were not aware of. This is for us to get an understanding of what works, and what does not work with existing commercial Kinect games.
\subsection{Qualitative Interviews}
With an interview, a researcher can gain comprehensive information about peoples views and opinions about a topic. Interview is one one of the most important tools used in qualitative research. There are two extremities in interview methods: unstructured and structured interviews. The former is more like a conversation between the researcher and the informants. The topic is chosen, but there are no interview guide involved. The questions can be adjusted during the interview. The latter has a structured form where the questions are chosen beforehand. The advantage with this method, is that all informants will answer the same questions, and therefore the answers can be compared. The third, and most common, type of interview is semi-structured interviews. This method is something in between the two extremities. This type of interview is called qualitative interviews. This method includes a interview guide, where some questions are decided beforehand, but the order in which the questions are asked, are chosen during the interview. In this way it is easier to follow the story of the interviewee. In addition, the researcher needs to be open to discuss other topics that might appear during the conversation. The most common interview setting is with one individual at the same time. However, group interviews are becoming more common. In a group interview several people discuss a topic, while a researcher serve as a moderator (ordstyrer?). Kan evt. skrive mer om gruppeintervju hvis det er det vi ender opp å gjøre. To get a successful interview, it is important that the researcher has an understanding of the informants’ situation so that the questions asked will be relevant. Questions should be asked in a way so that the informant can reflect over the question and not only answer “yes” or “no”. 
Evt. ta med pitfalls her (fra forrige oppgave). 


\cleardoublepage
\chapter{Findings from Workshop 1}
In this chapter we will present our findings from workshop 1. The presented findings are based upon feedback and quotes from the informants, and our own observations during the workshop. The seven informants are referred to by using I1 to I7, and we have due to requirements to anonymity decided not to distinguish between male and female. Therefore, refer to all the informants as females. Quotes from the informants are translated from Norwegian as literally as possible. 

\section{Perceived Usefulness and Value of Entertainment}
\emph{"Mostly, it was quite amusing"}. This was the general feedback from the informants after the gaming session. The informants were divided in their opinions about whether they would buy a game like this or not. Some of them would rather exercise for themselves or go to training centres, while others saw the gaming as more amusing than using a treadmill or an exercise bike. I5 stated that \emph{"I think this seemed very fun, so I want to buy one like this. I have one of those exercise bikes in my basement, but it is so boring that I can not bear it"}.  Some of the informants stated that one of the reasons why it was fun playing was because it was a completely new experience. I6 said that \emph{"Now I have been involved in something I have never been a part of before. [...] It is a new world that has opened up, that is for sure"}. Other reasons for why they enjoyed playing were that they could imagine playing with their grandchildren, and that it was a fun way to exercise. They liked the idea of combining gaming and activity. I4 said that \emph{"they [elderly] might think it would be fun to do this and be active at the same time"}. I6 explained that she felt different from when she was watching the other informants play, to when she was playing herself. \emph{"It gave more to participate that I had thought. Because, when I sat and watched it felt so unreal to have someone on the screen, but in the activity, when you got in to it, it was not so stupid after all"}. The observations we did during the gaming session supports the feedback the informants gave us. There were a lot of smiling and laughing, and it seemed like they had fun. However, perceived value of entertainment and perceived usefulness do not have to be the same. 

The informants liked some games better than others. The two games from Kinect Sports Season Two, tennis and skiing, were the game that the informants liked the most. They thought the activities were fun and they liked the challenges the game provided. \emph{"I liked it very much. I like that kind of activity"}, I4 said (om hva da?). The informants also liked that the game required something from them. About skiing, I3 said \emph{"This was a bit fun. Yes, it was. Because, here you have to pay attention and spend some effort [...]"}. I3 also liked tennis for the same reasons.  

Your Shape Fitness Evolved 2012, the personal trainer game, and Fruit Ninja were the two games that the informants liked the least. They did not see the need for the personal trainer game, as they rather would do these type of exercises in a training centre or by themselves. I3 said she did not like this game because it did not require anything from her. \emph{"The aerobic game I did not care for. That game anyone could do anywhere"}. However, when I1 chose which game she liked the most, she chose the personal trainer game as her favourite. This was because the other games had too much noise and loud sounds. Other aspects that were reasons for why the the informants did not like a game were because the game did not present what you gained from doing the exercises, and it was hard to get points because of significant delay. This shows the importance of technical correct games, as well as a meaning or a goal, to motivate people to play. 

When it came to Fruit Ninja, the informants did not see the usefulness of this game. They laughed a lot while playing it, but they thought the game itself was stupid. I3 said \emph{No, I thought it was stupid. It does not put any requirements on you. It was just.. [waving with his hands]"}. We did ask the informants if they thought the game was fun, even though it was a quite unrealistic game. I1 answered that \emph{"Yes, it was a bit fun, I think. You see all the fruit that smashes. That was lovely"}. Even though some of the informants thought the game was fun, they did not see the point in playing it. On reason was that they did not see a connection between their movements and the outcomes on the screen. \emph{"And if you hit something or not, that you did not had any sense of. You did not get any feeling of that"}. This was perceived as confusing, and they did not understand how to do things the right way. 

\section{Motivation and Mastery}

\emph{"Motivation is extremely important"}, was I5's opinion about playing games for exercising. Motivation was one of two things the informants mentioned as an important aspect for an exercise game. They told us that for elderly to use a video game for exercise, it has to possess features that will make them wanting to play. The informants stated goals and socialisation as motivational aspects for a game. Goals could be either achieving a high score, or just to move your body to music. \emph{"I think that what is important is one's own movement and motivation"}, I6 said. \emph{"And to move after a rhythm, that is always positive."}. Social aspects of gaming was mentioned as important for the informants. I4 said that she thinks meeting others for exercising is motivating.   Another aspect that was considered as motivating was to get information about why they should play the game. The informants stated that if they could get to know, e.g. the training benefits, from doing the different activities, it would be more fun and motivating doing them. \emph{"That we get information about what we exercise when, and why we should do this. [...] I think that it is very important for people as grown up as us. We have to know why we should do this"}. We can see from the informants' feedback that entertainment alone is not motivation enough. They need to know why they should play the game, and how it can be good for them. 

One interesting observation we did during the gaming session was that the features in the games that were supposed to be motivating, were perceived as the opposite. Cheering, loud music, encouraging comments, fans and high scores were perceived as noisy and annoying, and not motivating at all. \emph{"I became a bit irritated. When it comes a cute voice saying "yeah, that is great!", "hurrey!". I think it is stupid, I have to admit"}. Some informants did not even notice all this "hustle" described. \emph{"I did not notice it at all"}, I3 said. Since these features was either perceived as annoying or not noticed at all, we conclude that we should avoid features like this in our video game concept. We should keep it simple, and focus on a few motivating aspects. 

\emph{"Motivation and mastery. That is very significant"}. Mastery plays a huge part for the players feeling of success and accomplishment, which affects their self esteem. Mastery is also mentioned by the informants as a motivational factor. I6 told us that she had been teaching an activity class and that their main goal and focus was to experience mastery. She continues with \emph{"It is all about the experience of mastery, which is essential. And the older people get, the more important it is"}. I6 emphasised how important the feeling of mastery is for elderly, in particular, because they would not continue doing something they do not master. It is important for us to think about how to achieve experience of mastery for elderly when creating our video game concept. 
  
I6 presented how the feeling of mastering the game will make everyone want to play. In addition to this, the desire to master could lead to wanting to play the game over again. I1 said that \emph{"I think that I want to try it one more time. To master it [...]"}. When the informants first started playing the games, they were quite insecure and did not know how to perform the different activities and exercises. They completed the activities, but without the feeling of mastering it. But the informants did not see this as negative at all. They were all positive about it, and their common opinion were that it is all about training. \emph{"Nothing of it is difficult. It is just a matter of training"}, I3 said. 

While observing the informants interact with the technology, play the games, and walking through some of the menus, we saw that the informants mastered the different challenges better and better. They learned how to connect with the sensor, they quickly understood what was required from them in the various games, and they responded faster to feedback and information. This was both our observations and the informants own perception. When we asked them if they felt it was easier the second time they played, most of the informants answered \emph{"yes"}. The informants stated that the feeling of mastery came fast, which was positive for the gaming experience. Some of the informants stated that they learned how they were suppose to play the game and how they should move to interact with the sensor by looking at the other informants play. The first time information messages were shown on the screen, the majority of the informations did not respond to it. We had to assist them in what they were suppose to do. After playing for a while, they responded to the information messages without guidance, and they did it faster for each time. It was fun to watch how fast the informants learned. Even though they felt that they did not master the different tasks at first, they kept on trying, and after a short time they had learned a lot and enjoyed the feeling of mastery. 

\section{Immersion and Engagement}
The informants showed a lot of engagement and different feelings during the gaming session. They immersed into to game, and gave it all to complete the different challenges. In most of the games the informants used body movements eagerly. The informants also burst out with various comments like \emph{"yes yes yes yes"}, \emph{"noooooo"} and \emph{"ooh, I wanted that pineapple!"} while playing. In addition, the informants laughed a lot during game play. A few of the informants only used monotonous movements, like just waving their arm up and down with not that much engagement.  

Not all the games created feelings of immersion and engagement. The personal trainer game was a game that the informants "just did", without showing any particular emotions. There were not many comments during game play, and there were no laughter or engagement. We only observed concentrated faces. If this was because the informants thought it was difficult or boring, we do not know. However, the informants tried their best, and most of them went through the exercises with controlled and powerful movements, in tempo with good rhythm.       

The informants expressed that they felt, in some of the games, that there were a lot going on at the same time. We asked them if they felt it was easy to focus on the challenges, exercises and activities, although there were a lot of hustle. The informants agreed on that while playing, they were totally focused. I7 said \emph{"When you are in the game, then you do not see the outside world, then you are concentrated"}. The informants stated that they lived into the character of the avatar. They did not feel that the avatar was a person separate from themselves. \emph{"Yes, there I felt like the person [avatar] itself"}, I3 said. 

\section{Functionality and Usability}
\subsection{Introduction and instructions}
It was a common opinion that the different games should spend more time on instructions and explanation on the different exercises. \emph{"It needs to be a softer and more instructive introduction in the game's rules"}, I5 said. Especially in  the personal trainer game, this was the case. I1 compared this to the way they did it in an aerobic class she used to attend in her earlier days. She explained how they practised the exercise slowly before they could do it fast. The personal trainer game started up quickly, with no room for practice or warm up, and they held a high tempo. The informants did not know how to do the required movements, as no information where given. Most of the informants got the movements after a few steps, as they recognised the exercises from aerobic classes in their training centres. After playing, most of them agreed on that the exercises were OK and that it was only a question of learning. 

When the informants played their first game, they seemed confused and unknowing. One informant started pointing on the screen and asked \emph{"How do I point?"}, meaning how she could press the buttons. When playing the personal trainer game I1 asked \emph{"Press the button? Is there any button here?"}. It was clear that it was not obvious what was meant by pressing a non-physical button. None of the informants had ever played games like these, and did not understand how the game could be controlled by their body movements, even though we had explained this. This suggest that a thorough introduction explaining the interaction between the player and the game should be given. 

\subsection{Complicated menus}
There was a common agreement that the games had too complicated menus.  The avatar hand that was navigating on the screen was too sensitive, and it was hard to keep the hand still long enough to actually "press the button". In the sports game buttons were pressed by holding the hand over the button for a certain amount of time. \emph{"This is worse than working with a mouse this [on the computer]"}, I6 said. The time needed to hold over the different buttons seemed to be too long. This especially apply for people with declined strength in their arm(s). In the personal trainer game the avatar hand was, in addition to being too sensitive, unclear and at times almost invisible. It did not really look like a hand. This seemed to cause some problems because not all informants understood that the object on the screen was their hand. Two informants suggested that an arrow-marker like the one used on most computers would be more intuitive and easy to use. 

\emph{"The menu was extremely difficult [...]"}. In the same game, the menu appeared to be very complicated, as a result of being big, complex and sensitive. This made it difficult to know where to go, as well as to push the right buttons. There were a great amount of information on the screen, which made it difficult to choose the right alternative. Three informants was challenged to go through the whole menu, and they, especially, struggled a lot. To navigate between the pages in the menu, there where arrows on the sides that should be pressed. These arrows were hard to see, and they were very sensitive. When telling the informants to scroll through the menu by pressing the right arrow, I1 said \emph{"Which arrow?"}. In addition, this menu had a huge "back"-button that the informants pressed several times without intention. We found it strange that this menu was so complicated, as it actually was aimed for the older user group with their game "sprek alderdom", or "fit old aged" in English.  

Also with Fruit Ninja the informants had problems with the menu. The menu was crowded with elements, which made it hard to hit the correct buttons. In addition, the menu made it difficult for the informants to distinguish between the menu and the actual game. I6 and I7, who were playing together, stood in the background and waved their arms towards the screen while we tried to go through the menu and start up the game. 

\subsection{Information and feedback}
There were various perceptions of the information and feedback given during game play. Some of the informants was not affect by the information given, and at times they did not even recognise its presence. \emph{"I did not react to the text at all. If you had asked me, I would not know it had been text"}. I5, on the other hand, felt that there was too much information at the same time, which made it hard to concentrate. \emph{"No, limit the total amount of information, I think"}.  I2 said she had seen the text but it disappeared too fast, so she did not have the time to read it all. I2 suggested that it should be a way to confirm that you have read the information before it disappears. 

It seemed that it might have been too small text in the menu in the personal trainer game, after observing that at least one informant spent some time reading the text. At one point the informant had to move closer to the TV-screen to read the text. This informant had the same problem when navigating through the menu in the tennis game. Because there was only one informant that seemed to have problems with this, we assume that this informant might have had impaired vision. However, it is important to acknowledge that impaired vision is a common problem for the older population \cite{ijsselsteijn2007digital}, and it should therefore be considered in a game designed for this group. 

Most of the informants agreed that it was too much unnecessary information, however, it seemed that it was desirable with feedback on what they were doing. \emph{"I think it is very nice to see how far I have walked, and how fast I have walked, and how much downhill and uphill and things like that"}, I1 said, referring to a mobile application she was using to give her information about her progress when cross-country skiing. This implies that feedback on their progress is important for some of them.  

The instructions in both the tennis and skiing game, like "raise hand above head to play" or "move closer to the sensor" were well understood by almost all informants. Although we had to assist some by reading the message given on the screen, we believe that these messages should have been clear enough. A possible reason for why the informants did not see the messages might have been because they were not concentrated and focused enough. 

Not all informants understood that there were introduction videos and instructions in the beginning of the game. This became clear to us as most of the informants tried to do what the avatars did in the video, and seemed confused when nothing happened. \emph{"Should I try to hit it [the ball]? What am I suppose to do now?"}, I1 asked during the introduction video. However, it appeared that the second instruction video that was shown between two matches in tennis, was understood as an introduction video, and that they all learned from it. However, when the game was finished, some of the informants did not understand that the game was over. Especially did this apply for the tennis game, where most of the informants ended up just looking at the screen. Apparently, there was not given a clear enough information about the end of the game.  

\subsection{Graphics and sound}

We asked the informants about their opinion of the music in the games, and I1 answered that \emph{"I think everything about that was too much"}. I2 replied with \emph{"I like Mozart better"}. Other types of music that was mentioned as more suitable than ordinary pop music was swing. All the informants agreed that music is important to keep the rhythm when exercising, but that they would prefer the music not being too noisy. It was also a general opinion that the music was inappropriate.  I7 said \emph{"You get sensitive [to sound]. You do not want it. You want it quite. You would want to be active, but without too much background noise"}. I4 added \emph{"[...] I would prefer walking out in the nature. Then you can listen to the birds [...]"}.

It was a general opinion that there were too much elements and information on the screen during game play. I7 stated this by saying \emph{"It is too much elements on the screen. Where should you look?"}. Some of the informants also felt it was too much going on at the same time during game play, and I5 suggested that there could be different levels of things that happens in the game. \emph{"[...] Eventually when you get better and manage to keep track of more things, you can add more things [to the game] that happens. A lot of what happens in these games are not relevant"}. 

On the question about what they thought about the avatars and the picture in general, I5 answered \emph{"I think it was very confusing to see myself. Especially to see both of us [herself and the trainer] [...]"}. In addition, most informants agreed that the avatar of themselves in the personal trainer game made them look fat, which they did not like.

\subsection{The nature of the game}

\emph{"It bothered me that I did not see the correlation or the relationship between my own body movements and what was happening on the screen"}, I2 said about the gaming experienced. All of the informants complained about the delay that was present the games. This applied for all the games but Fruit Ninja, and it was mentioned as a problem several times by the informants. The delay made it hard to move correctly and in time, like in e.g the skiing game to pass through the gates, and to coordinate movements to perform a perfect jump. \emph{"Missed the jump? I jumped so quickly. That is just nonsenses"}. The delay in the games made it difficult to get points and achieve high scores, which again gave the informants the feeling of not mastering the movements. \emph{"I think my movements on the screen look too slow. Are they that slow?"}, I6 asked while playing. 

The technological aspects of the games became a source of confusion for most of the informants. \emph{"There were some small technical details that were problematic. [...] I could not see in the skiing game that anything special happened with me. [...] And in tennis, it was something about the time aspect, with when you tried to hit and when you actually hit [the ball]. I got a feeling that they [the game] helped me a lot in the beginning"}. For instance, this applied the tennis game, as there was a problem when the informants tried to coordinate the hand with the racket and the hand with the ball. The delay made it difficult for the informants to see when they actually hit the ball. Some of the informants expressed that the significant delay interfered with the gaming experience. I5 said that \emph{"I experienced the tennis and skiing game as games with lack of technical perfection. It was not good enough technically to make a real time experience"}. However, I5 said that if players were to get used to the delay, they might have more fun.

Most of the informants had a hard time coordinating their moves in the personal trainer game, as the movements were quite fast. Some of the movement was for many difficult to perform correctly. I6 commented that it was all about training, and that they would master it after getting familiar with the game. I5 answers to this with \emph{"Sure, but it is possible to make it less confusing"}. I1 replied that she rather would exercise without the game. The informants also experienced similar problems with the tennis game. The ball came quickly, and it was difficult to coordinate the hand fast enough. This applied especially when the ball came on the opposite side of the body from where the racket was held. One of the informants ended up trying to hit the ball with her left hand, even though she had the racket in her right hand. 

[KANSKJE HA MED I FEEDBACK??] Fruit Ninja, unlike the other games, did not show a clear avatar, which could have made it difficult for the informants to see a connection between their own movements and the actions on the screen. The avatar was hard to see as it was shown as a diffuse shadow on the wall behind all the fruit. I7 told us that she had trouble understanding how she could see where she hit. When I1 played Fruit Ninja it seemed like she did not understand what actions did cut the fruit in half, she just stood there and waved her arms while she looked quite confused. All of the informants agreed that Fruit Ninja was a bit silly and that it did not require anything from them. \emph{"This is just to wave your arms. There is no system"}. In addition, they did not see the relationship between what they did and what they achieved. \emph{"[...] I did not get any feedback on my movements"}, I2 said. It seemed that all informants found it important to get feedback on their movement, see results, and experience a learning effect.

\emph{"The game that required something from you was the skiing game. The rest of the games did not require you to follow the instructor apparently"}, I3 answered after being asked how they experienced the game. It was apparent that some of the informants did not understand what they were meant to do in the games and that they just "did something". 

Some informants mentioned that the skiing game was very fast. At the same time there was a comment about the importance of having the fast pace, to make the game exciting and fun. In addition, there was one situation in the skiing game that especially seemed to cause confusion and frustration. When playing multi player, there were two matches, where the players in the the second match would have to switch tracks. The informants had troubles understanding which track that was their track. \emph{"Eh.. I do not understand"}, \emph{"are we going in the same lane?"}. I5 was confused while playing, \emph{"First red and then blue and then blue and then.. [...] This does not work"}. The fun and enjoyment we observed the first round of skiing was the second round replaced with frustration. 

\section{Physical Outcome}

The informants has a general opinion that all movements are good for your body, and that these games required you to move in a good way. I3 said \emph{"If it [the game] do good for your body? Yes it does. All kind of movements are of the good"}, while I4 said  \emph{"[...] I felt it was useful, in a health-related way"}. I5 was amazed by how much body she actually moved, \emph{"It was amazing how much body you actually used on so little"}. All the informants could see that there were some positive health effects, however, they did not exactly know what they were exercising. They urged that the game should include information about which body parts you are exercising when you are doing the different exercises. Another general opinion was that they wanted to know if they were doing the exercises right or not when they were playing. [HA MED DETTE siste (fra However) UNDER INFORMASJON OG INSTRUKSJON???]

The informants said that if they were just going to do basic exercises, like in the personal trainer game, they would rather do this without a game. It they should use a game for exercising, it would need to contain a sport, or something else related to real life. \emph{"It would have been better to chop wood"}, I6 said after playing Fruit Ninja. It seemed like the informants wanted games with meaningful content and realistic activities. Game themes mentioned by the informants were swimming, rowing, picking apples, biathlon, interval exercises and a walk in the nature. \emph{"You mentioned apples. It might have worked to get it synchronised so that when you gathered apples and put it down, you could summarise the number. That might have been a competition"}. One informant also suggested doing puzzle games on the map of Europe. In that way you could learn geography at the same time which would make the game more meaningful.

It seemed important for the informants to see a progress in the game, and that they learning something. \emph{"I believe in games that have the characteristic of play and that you can see that you get better"}, I2 said. This will provide the players with something useful, that will make them want to play again.
 
From what we observed and from what was mentioned by the informants, it is clear that it is important to have a structured program where you equally, and symmetrically exercise similar body parts. This came clear from for example tennis and Fruit Ninja, where several of the informants mentioned that their shoulder and arm were in pain after playing. In tennis, the player only uses one arm, and the workload gets asymmetrical. In addition it was mentioned that if the aim of the game was exercising, they need the whole range of exercising, from warm-up to stretching. 
    
\section{Social Interaction}
\subsection{Playing together}
\emph{"I think the slalom was fun. But I am thinking more if I had a grandchild in a suitable age, and I could say "should you and grandma play the ski-game together? Just for fun?""}, I5 said about the social aspects of the different games. She continued with \emph{"I am thinking that in a nursing home we would probably like to play with someone"}. Some of the informants were clear in their opinion about whether they would play alone or with others, and I7 stated that \emph{"[...] I do not want to do this alone at home"}. I4 said she  enjoyed playing alone, \emph{"I think both [playing alone and together] was OK. [...] I believe in  exercising in small groups"}. This statement shows that playing in a group setting might be preferable.  

A general opinion was that this game could suit well in a nursing home setting, in a small group setting, or as an activity to do with grandchildren. It seemed that most informants liked to play together, however some informants mentioned that they were indifferent. The enjoyed playing together in the games that were made for fun and entertainment, like skiing and tennis, while with games meant for exercising, like the personal trainer game, they felt it would be better to do alone. One informant told about her mother in law that got inactive on her last years. She believed that the only way to encourage her to become more active by playing games would be if there were some social aspects within the game, like playing together with a grandchild. 

Everyone agreed that being social and meeting people are important parts in life, but that they would do it on their own time, and not be locked to a specific group or time. I5 felt that meeting together in a group for exercise gives her a sense of pressure that she does not like. \emph{[...] I have had pressure all my life. I do not want it anymore. I do not want to put myself in that situation where people come and ask "are you going to join this?""}. I7 agreed that this type of pressure is not motivating. \emph{"It is like that, that well-being is attractive, pressure is not attractive"}, I6 said, referring to the importance of doing things voluntarily. However, just having the opportunity to go out and meet people, like in an arranged training group, seemed to be important for all of the informants. 

\subsection{Watching others play}
\emph{"You are talented"}, and \emph{"You are starting to get it now"}, are examples of comments the "audience" told their co-informants while they watch the other informants play. They encouraged each other, and provided positive and motivating messages. The informants that were playing seemed to enjoy this feedback. This feedback made the gaming session more social and fun.  

\subsection{How to play together}
We asked the informants if the could imagine themselves playing at home in their own living room, and competing with friends over Internet. To this they were not very positive, and it seemed like they did not understand what we meant. \emph{"This is very distant for me"}, was I6's comment. \emph{"It would probably be more motivating to just ask her if she wanted to come over and play"}. These answers show how being social over Internet is quite unfamiliar to the informants. To make a game like this social, all the informants agreed that meeting in person would be better. On the other hand, the informants using this training mobile application, seemed to like to share exercise information with friends.

\emph{"[...] If there is going to be any point in doing something together, it needs to be that you are enhancing each other. Like that you get a better result if you are cooperating. [...]"}, I5 said. She had a strong meaning about how to play together, and meant that it was more motivating to cooperate rather than compete. The other informants seemed to be indifferent whether they were competing or collaborating.

\section{Experience and Understanding of Technology}
None of the informants had used a technology like Xbox Kinect before, and they all said that it was a completely new experience for them. \emph{"I have practically never seen it before"}. In fact, the informants said that they never had used any kind of video game technology. \emph{"Waste of time"}, I5 said, \emph{"I could never have time for something like that"}.  

The informants generally showed interest in the technology and our project during the workshop. They asked about the EU-project, where the games were made, and how much the Xbox Kinect costs. The informants showed interest in what equipment needed for them to play at home. \emph{"Does it connect to a TV?"}. I7 commented that it is important that the technology is easy to install. The informants were also eager to ask questions about the games, where this was related to information given, their own performance, difficulty levels and confusion. While looking at two informants playing Fruit Ninja I7 had questions as \emph{"Can you steal others fruit when you play?" and "What do you lose points for?"} 

During the group conversation there were questions about our work. They wanted to know how we would proceed from now, and what we imagined as our result. One informant also asked the facilitators that joined us the first day about their role in this project. When we were about to finish the group conversation, I6 asked \emph{"[...] What we are coming with, is it usable? Can you use if for something?"}. We answered that their feedback through the workshop is highly valuable, and that it will be used as a basis for the video game concept in our master thesis. I3 complimented our work with \emph{"it is price worthy what you are working with this, that I have to say"}, which meant a lot for us. It is good to know that the informants, representatives for the group of elderly, feels that what we do is something useful.  

\section{Aspects to Consider About an Exergame}

To summarise our observations, even though the informants experienced some confusion while playing, we saw that they had fun and enjoyed playing these games. However, most of them did not see the games so useful that they would have bought them and used them themselves. They had other ways to exercise that they would rather prefer. The games they liked the most were those involving real-life activities, like tennis and skiing, which combines movement and entertainment. They did not care for the personal trainer game, even though it involved realistic exercises, because it did not give them any value of 
entertainment. Fruit Ninja was perceived as stupid, and mostly, it gave them nothing.
 

\subsection{Different user groups}
One of the informant emphasised the importance of remembering the different target groups, and that it will be impossible to make a concept that suits everyone. She mentioned different groups to keep in mind, "the people who already have decided to keep their body in shape", "people who wants to know if they are doing the exercises right", and "the people who are already inactive". For the latter group she mentioned the importance of getting help, e.g. from grandchildren.  Another group mentioned are the people who can not stand or walk, who might be sitting in a wheelchair.

\subsection{In what setting could this game be suitable}
All of the informants could see a game like this as a fun activity to do with their grandchildren, which makes it clear that the social parts of playing games are important. Other relevant arenas for the game mentioned were small groups and at the nursing home. One informant talked about her mother in law, who got total inactive her last years. She said that the only way she could see her using a game like this, was with a grandchild or great-grandchild. This shows that it might be hard to reach the people that are already in bad physical condition and inactive, because they will already be depended on someone else. This build up under the importance on preventing people from ending up inactive.

Only one of the seven informants could see herself use a game like this. Most of the informants were in such good physical shape that they were still doing other types of activities. However, they did agree that the game could be an alternative in the future, when they might not be able to do the same activities they are doing today. \emph{"[...] If I can't go out skiing anymore, for example, maybe"}. 

\emph{"This can be a way to shorten the long waiting lists in the hospitals, by keeping people healthy, instead of using the hospitals for every little thing"}. Generally, the informants has a positive attitude towards the exergame, and believed that the game could work well as an alternative method for exercising. However, they agreed that this might be more relevant for our generation (20 years +). 

\subsection{Would rather do ordinary exercising}
\emph{"This [the exercises in the personal trainer game] is what we do at SATS [a Norwegian training centre] too. It is much more fun, I think, to be in a room with others, and to get the same instructions. However, of course, if you live far away and can't get there, this [the game] could be a replacement. It does not give the same, but could be a replacement"}. I2 said, \emph{"I do not think I will buy it [the game]. I would rather go to Elexia [a physical health center in Norway]"}. It became clear that as long as they had good health and were in good physical shape, they would rather keep on doing their regular activities and that they would never replace these activities with a game. 




\cleardoublepage
\chapter{Concept for an Exergame for Elderly}
\cleardoublepage
\chapter{Findings from Workshop 2}
\label{chap:findW2}

According to the fourth activity in the cycle of user centered design (depicted in Figure \ref{userdesign}), we wanted to conduct a second workshop to evaluate the designed exergame against the requirements. A description of the execution of this workshop can be found in Section \ref{sec:ws2}. 

In this chapter we will present the findings from workshop 2. The presented findings are based upon feedback from the informants, and our perception of their reactions to the presentation. The five informants are referred to by using I1 to I5, and are randomised so there is no relation to the references in workshop 1. Due to requirements to anonymity we have decided to not distinguish between male and female. Therefore, we refer to all the informants as females. Quotes are translated from Norwegian to English to  preserve the meaning. 

\section{General Perception Regarding the Game's Story and Elements}

\subsection{Nature Trail}

When we presented the "Nature Trail" game it took some time before the informants had any comments. It appeared that it was not clear to them from the pictures how this game really would work. \emph{"It is impossible for me to say anything about what I think about this game. This is because I do not have any sense of how it works"}, I4 said. She continued by asking, \emph{"This girl on the screen, if I do certain movements, will she do the same?"}. After we confirmed this, it seemed that the informants understood more about how the game would work, and more comments appeared. I3 said, \emph{"I think this was a good idea because the environment you are in is familiar to me"}. The other informants agreed. Several times during the discussion, I1 complemented our concept. \emph{This was very nice"}, she said, while I4 followed with \emph{"fun development"}.  

About having quizzes in the "Nature Trail" game, I1 was very sceptical. She was afraid that it would draw attention away from the physical tasks. I3 agreed and said \emph{"Answering the questions in the quiz will become some kind of test on how good you are, and that is not how I have understood the point of these games"}. I1 meant strongly that we should separate the quizzes from the rest of the game. She said \emph{"I think this is very nice. [...] However, you should think very thoroughly through the cognitive aspects of linking the questions with the physical tasks. You need to have a purpose about it, a goal on what you want to achieve"}. 

It was not clear for all the informants when the quizzes would appear in the nature trail. They all seemed to believe that they would appear at the same time as they were doing something else, like while balancing over a log. To be able to focus on both the physical challenges and the quiz, I4 suggested that the questions should appear somewhere where it was natural to take a break, and that the player should be able to sit down while answering them. I4 also proposed that the player could get to see the questions before the game started, and then answer them after a while. Her experience was that if she was thinking about something, the answer usually came up eventually. 

From the discussion we understood that there was confusion about the link between the cognitive and physical challenges. This was not only about the goal of having a quiz, but also the quiz's purpose in the game. \emph{"The points shown in the top-bar, do you get them just by answering the questions? It does not have anything to do with the physical skills?"}, I2 asked. Several of the informants also wondered about how the question allocation would work. \emph{"When I'm doing the game time number three, will it still be the same assignments that appears?"}, I3 asked. I4 asked if the player could choose category for the questions.  

The informants were mostly concerned about doing two things at the same time, like answering questions at the same time as doing an exercise. Another challenge they commented on was the one where they were supposed to balance over a log. \emph{"This will exercise your balance, right, so it is easy to fall yourself, if you get dizzy"}, I1 said, referring to the possibility of falling not only in the game, but also in real life. There were also some technical questions about this challenge. \emph{"If I am balancing on the log, and I fall down, would I feel that? If I did not manage to keep the balance, and fell in the water?"}, I4 asked. 

Most of the elements in the game environment were familiar to the informants. However, the hearts were not that obvious. \emph{"I did not quite understand. The heart floating there on the screen, is the idea that you should take it?"}, I4 asked. I2 followed with \emph{"If you get the heart, does it disappear?"}. Further the informants wondered where the number of gathered hearts was shown on the screen. We explained that to limit the information on the screen the hearts will only fill the "health bar", as an indication of good health. They understood, and agreed that it would be hard to count the number of hearts on the screen, and that the "health bar" was a good idea. Not all of the informants were positive about the hearts. \emph{"But hearts.. It is a little feminine"}, said I1, and suggested that maybe hunting butterflies would be better, especially for men because it could trigger the hunting instinct. I3 joked \emph{"A bottle of beer?"}, while I4 added \emph{"Let's see.. A feather? Flying by?"}. 

At first the informants did not understand how the different difficulty levels would work. I1 said that she did not want to get forced into something she did not want to do, and that she wanted to be able to choose which difficulty level to play in. \emph{"I am thinking that I do not want to get forced into something that is hard, that I do not master. Because then I get mad"}, she said. We explained how the game will remember the players progress and adjust difficulty level accordingly, and that there also will be initial difficulty levels that the player can choose between. She agreed that this was a nice way to do it, as long as she had the possibility to choose herself. I4 also agreed on the way we had organised the different levels. \emph{"I think it is an advantage that everyone starts at the easy level, and the more confident you get, the harder it gets. I think that is a good way to be controlled"}.

\subsection{Picking Apples}

The informants all seemed to like the "Picking Apples" game. \emph{"I think this was nice"} I3 said. However, there were some concerns related to how the apples would appear on the tree, and how they could plan the apple picking. I4 said \emph{"I feel that two apples is fast to gather. If the tree was full of apples I would be more eager to play. [...] It might be weird if they should pop up all the time"}. We explained that the apples appear randomly to cover both the cognitive and physical exercise. They all agreed that this was a nice solution. 


\section{Information, and the Menu}

More information was stated as important in workshop 1, and we had therefore tried to include this in the exergame. The informants seemed satisfied with the way we had presented this, and they also seemed to recognise the included instructions inspired by the sports game, like "raise hand above head to play". We asked if they noticed the arrow in the upper left corner, and I5 immediately said \emph{"back"}. Everyone agreed that it was clear that the arrow indicated a back-button. About the combination of colours and how the pictures looked in general, I1 said \emph{"beautiful!"}. I3 mentioned that it would be better with more shadow around the buttons, to indicate that it actually is a button. On the question on whether there were too many steps in the menu before the game started, the only comment was from I1, who said \emph{"When you have used the menu, then you want to have shortcuts"}. 

When we presented the possibility to choose game play based on muscle groups, there arose many questions on what we meant about the term "muscle group". It became clear that we had been inconsistent with this categorisation. I5 said \emph{"When you use the term muscle group, then I think that endurance do not fit under this term"}.   

All of the informants had problems understanding the different difficulty levels in the games. \emph{"My immediate reaction when I saw these three different forests with different difficulty levels, was that it took me some time to understand that the different environments represent different difficulty levels"}, I3 said. The other informants agreed. Further I3 suggested \emph{"Easy forest, medium forest, hard forest, or what you would call it"}. About the different difficulty levels, some of the informants wondered if the same challenges would appear in all levels, just more frequently. This was the way we have thought the game to be, in addition to adding new obstacles in the higher levels.

\section{Music and Atmosphere}

The informants were curious about the music and sounds in the games. I4 asked \emph{"I am wondering about the atmosphere and environment. When I am balancing on the log, will I hear the sound of water?"}. We explained that we wanted to include sounds that are natural to the environment. I2 said \emph{"Not noisy, like last time"}, referring to the music in the games played in workshop 1. We told them that we want to use peaceful music, like classical music. I1 suggested \emph{"You should think about rhythmic music. Maybe it is just as easy to pick apples to a rhythm, instead of picking as many as possible? [...] Then, when the apples ripen, it will be according to the rhythm"}. 


\section{Other Aspects Concerning the Games}

Some of the informants were concerned about getting tired of the game after playing it several times. One informant asked if it would be possible to exchange the game if it got boring. We presented our idea for a video game series, where each game could be downloaded from the Internet for a small amount of money, in this case 99 NOK (about 18 USD). They agreed that this was an affordable price. The informants showed interest and curiosity about how they could get the game. I1 asked \emph{"Do you think you could get it on prescription?"}, and I4 continued by asking if it would be possible to rent such a game from the library. When discussing the possibility to exchange games if they got bored, I3 expressed that she did not think this would be a problem. \emph{"I would think that when I have completed the game I have exercised, and that would be satisfactory for me. And then I could easily do it over again"}. After I3 had put it that way, the other informants seemed to agree.

One of the informants remembered the technical aspects related to the delay from the last workshop. She said \emph{"this is a technical question which is about the capacity in the computing system. It is not straightforward to solve technically. [...] Do you have any thoughts about this?"}. We informed them that we will recommend the delay issue for further work.

All of the informants were eager to hear more about the development of the exergame, and they wanted to get information about when the game would appear on the market. I4 said \emph{"It would have been fun to know when it comes. I have faith in this project"}. \emph{"Include us in the customer list!"}, I1 laughed.

\section{Summary}

We will summarise important findings from workshop 2 below:

\begin{itemize}
\item Informants liked the familiar environment, both in the "Nature Trail" game and in the "Picking Apples" game.
\item Informants were sceptical to including quizzes in the "Nature Trail" game, and suggested to separate the quizzes from the physical tasks.
\item Informants stated that the balance task could be challenging.
\item Technical details should be included, i.e. what happens if the avatar falls of the log.
\item The purpose of the hearts was not obvious. In addition, the hearts were considered as feminine.
\item Informants suggested that clearer instructions on the different difficulty levels should be included.
\item Informants were curious about music and sounds. 
\item Informants proposed to use music in rhythm with the rate of apples appearing on the tree. 
\item Informants suggested more shadow around buttons to highlight them even more.
\item Informants expressed lack of consistency related to what goes under the term "muscle group".

\end{itemize}




\cleardoublepage
\chapter{Discussion}
\label{chap:discussion}

In this chapter we will provide a discussion of the findings done in the two workshops we have conducted, and on the concept we have created. We will link our findings and results with theory and previous studies, as well as provide our own opinions. The concept is developed based on findings from workshop 1, findings in related research, and official guidelines. Some of the choices made in the concept come clear from this, however, some choices were made by us with no foundation in theory. This will be discussed in Section \ref{sec:discconcept}. In the discussion of the findings from workshop 1, found in Section \ref{sec:discfindings1}, we conclude in each subsection with what was taken into consideration in the game concept. In this way, the reasons for the choices made in the concept will come clear from the here. There were some topics we learned during the conduction of the research, that are out of the scope of this thesis, but that there were spend some time on, nevertheless. An example of this is the delay problem which was discussed several times with the informants. Because of the great importance of this issue, we will provide a brief discussion on this topic. However, studying this problem in depth, and finding a solution for it, will be left for future work. At last in this chapter, we discus the quality of this study.

\section{Discussion of Findings from Workshop 1}
\label{sec:discfindings1}

In Section 6.1. we discussed the importance of usability which is about how easy a system is to use, learn and understand. Workshop 1 was performed to see how a set of relevant users interacted with existing commercial exergames, and to identify which aspects of these games that do or do not work for this user group. In Section \ref{sec:usability} we present three relevant elements that can say something about a system's usability: \emph{effectiveness}, \emph{efficiency}  and \emph{satisfaction}. These were aspects among others that we tried to measure in this workshop. We also tried to answer the exergame's \emph{context of use}. We wanted to discover the needs of the intended users, the need for functionality, and the environment for the game. Where and in what circumstances the game can be used, were discussed in our previous project assignment \cite{project}. This is not our main focus in this thesis, but will briefly be discussed in Section \ref{subsec:whatwhere} 

In this section, we will discuss the findings from workshop 1 and relate these findings to the literature provided in precious chapters. From this we will emphasise what we have considered in the concept. In workshop 1, and in our literature study we have tried to answer these two research questions: 

\emph{RQ1: Are existing commercial Xbox Kinect games suitable for exercising purpose for the senior user group?}

\emph{RQ2: What are the design challenges when developing video games aimed for exercising for the senior user group?}

First, we will provide a general discussion, and then we will summarize by more precisely answer the two questions at the end of this section. 

\subsection{Control of Character, Clear Goals and Mastery}
The GameFlow model \cite{sweetser} discussed in Section \ref{sec:heur} describes eight core elements that should be present to experience enjoyment in games. One of these is \emph{control}. The informants expressed that they did not feel that they had control over their character. This was due to the significant delay that was present in the games. In addition it came clear from the observation that it was not always easy to understand when the game started and ended, as well as what was an instruction video and what was the actual game. The element \emph{clear goals} did also seem to lack in the commercial games, and it was expressed by one of the informants that there was a need for more instructions on what was actually expected from them and the rules of the game. In Section \ref{sec:sergames} we discussed how video games can function as a pedagogical tool and it was shown that important factors to focus on includes motivation, effectiveness and intuitiveness. Another aspect discussed in the same chapter is behaviourism, which states that if someone is rewarded for something he or she is likely to repeat the action that triggered the reward \cite{understandingvg}. These aspects were also mentioned by the informants, who stated that the system needs to be easy to understand, and that the feeling of mastery and that you learn something are important.  One informant mentioned that if they did not manage to do something, they would stop doing it. In our game concept we will focus on emphasising the goals in the game, and the player will be rewarded with points. In addition we provide different levels. This will be done in two ways: Different initial difficulty levels the player can choose between, and different difficulty levels within the initial levels, where the next level depends on the previous.  

In Section \ref{sec:motivators} we discussed self-efficacy as an important determinant of exercise behaviour. Elements that are relevant to sustain this exercise behaviour are the feeling of pleasure and satisfaction, and self-regulatory skills. This was also discussed in workshop 1 as important aspects, and the informants mentioned goal setting, the possibility for socialising, and the possibility to self decide what to do, as important. The feeling of mastery was seen as significant. They were clear that if they did not get the feeling of mastery, they would not play these games. This relates to the elements discussed in the GameFlow model \cite{sweetser}, where some of the criteria for player enjoyment in games are to include challenges that match the player's skill level, to have different levels of challenges, and clear goals. Clear goals, with appropriate challenges to reach goals, are aspects highly considered in the new game concept. The delay problematic will be discussed in more detail in Section \ref{sec:delay}.
 
\subsection{Immersion and Concentration}
As presented in the GameFlow model \cite{sweetser} immersion and concentration are important aspects of the gaming experience. It was hard to evaluate from observing and interviewing informants if they were immersed into the game and concentrated on the tasks. However, comments like \emph{"I felt like the person [avatar] itself"}, and cheerful comments given by the informants while they played, like \emph{"I wanted that pineapple"}, suggested that the informants immersed into the game. If we are to evaluate, we would say that all the games required concentration to perform the tasks right. However, it did not always seem like the informants were that concentrated. One example of this is that we sometimes had to assist the informants in reading messages appearing on the screen, like "raise hand above head to play", while they other times read the same type of messages themselves. We believe that the text should have been clear enough, and that the reason for them not reading the text, was because they were not concentrated enough on the game. 

Concentration and immersion are considered in the exergame concept. For the player to see obstacles on the trail, as well as see when apples get ripe, they need to concentrate. In addition, the activities are provided in a real-life environment well known for elderly, and are based on activities suggested as interesting for them, which should make it likely to immerse into the game.

\subsection{The Possibility to Customise}
As mentioned, in our previous project \cite{project} we evaluated the game to fit as a tool that physiotherapists can use as an alternative exercise method for their patients. The value proposition of this game was described as: \emph{"A tool with the ability to customize an exercise program, and to offer an alternative, fun and motivating training method, while at the same time ease the workload of the physiotherapist"} \cite{project}. In this thesis we have focused on one part, of this description: \emph{an alternative, fun and motivating training method}. For this game to meet these three criteria, the end-users had to be included in the development process. However, as learned from g.1, as well as from the informants, there is a need to customise these kind of games and acknowledge elderly's various limitations and disabilities. One example given by the informants was that the skiing game could have too fast pace for some people within this user group, and that it could cause problems for people with decline in their balance function. The possibility to customise the games can for example be a feature in the physiotherapists interface, where they can put together different exercises within the game story, that fits their patient. Another example is for the user herself to have an interface where she can put together her own program.  Requirements 1.25, 1.26, and 1.27 in Table \ref{tab:func2} in Section \ref{sec:req} are presented to meet this possibility. We have limited our thesis to not design a user-interface for this and this is therefore left for future studies. 

\subsection{Social Aspects}
The Game Flow model says that games should support social interaction to meet player enjoyment. Also in Section \ref{sec:exergames} the importance of social interaction in exergames are discussed, and  in \cite{statistics2012} it is stated that as much as 62 percent of all gamers say they play with others. Guideline g.8 discussed in Section \ref{sec:summaryguidelines} also stress the importance of the social factors of a game. Offering social interaction, can especially be important for elderly who experience loneliness in their everyday life, due to inactivity \cite{exergamesforelderly}. It was interesting to learn that social interaction also was seen as important by the informants. The majority of the informants would rather play together than alone. However, none of the informants could see themselves playing together with others over the Internet. This relates to the findings done in \cite{Gajadhar} where it was shown that elderly enjoyed playing together in the same room more than playing online. We believe that one of the reasons for the informants stating this was because they did not understand the concept of "playing over the Internet", as a result of their inexperience with this technology, which is also stated as characteristic c.7 in Section \ref{subsec:characteristics}. This might indicate that the market is too immature for this. 

One of the informants meant it was more motivating to cooperate than compete. This was also found in \cite{Gajadhar} where they in line with the result from their study and from previous studies conclude that the focus should be on cooperative play rather than competition for this group of people. This was supported by one of the informants. However, the others seemed indifferent. In our game concept we will provide both the possibility to compete and collaborate, because of two reasons: First, several of the informants wanted the possibility to play with grandchildren. The findings they did in \cite{Gajadhar} about elderly preferring collaborating and helping each other, rather than competing, were in contrast to what they had found about young people in previous studies. Second, the majority of the informants were indifferent. 
 

\subsection{Appropriate and Simple Feedback and Information}
In Section \ref{sec:summaryguidelines} we listed typical characteristics of elderly based on findings from the literature. In our research we had a group of informants, who were relatively physically and mentally fit. From the list provided in Section \ref{sec:summaryguidelines}, we only experienced three out of the nine characteristics. We experienced that one of the informants had problems reading the text in some of the menus. Because there was only one informant that seemed to have problems with this, we assume that this informant might have had impaired vision. However, it is important to acknowledge that impaired vision is a common problem for the older population as discussed in Sections \ref{sec:summaryguidelines} and  \ref{sec:designelderly}, and it should therefore be considered in a game designed for this group. The group of informants had interest in technology and used different types of it, like computer, tablets, mobile phones, e-mail, e-banking, TV, etc. However, none of the informants had any experience with video games. In the beginning of the workshop we experienced that some of the informants were insecure, and had problems understanding what they were suppose to do. It took some time for most of them to understand that they had to use their body to play. Therefore, we see a need for clearer instructions both before and under game play. This includes an introduction to how the system works, like how to interact with the sensor. The last of the characteristics listed in Section \ref{sec:summaryguidelines}  we experienced, was that some of the informants expressed that it was hard to do more than one thing at the same time. Therefore, the information given, and the tasks to be done, should be limited, and adjustable. The possibility to add more functionality after the existing functionalities are managed was suggested by one informant. This is also listed as one of the guidelines in Section \ref{sec:summaryguidelines}, and suits well with the requirements of simplicity discussed in Section \ref{sec:simplicity}. 

The last of the GameFlow elements the informants were not satisfied with was the \emph{feedback.} Number 3 and 4 of the eight golden \cite{mmi} rules presented in Section \ref{subsec:golden} discuss the importance of getting informative feedback at appropriate time, and guideline g.6 presented in Section \ref{sec:summaryguidelines} suggest that feedback should be given in a motivating form. The informants desired more feedback on their actions, and they especially wanted to know whether they did the exercises right or wrong. They did not feel that they got this feedback in the games played in the workshop, which made them both confused and frustrated at times. Some of the games that were played had a lot of different features that were suppose to be motivating. However, by some of the informants this was rather seen as annoying.  In addition the amount of the information given, both text and audio, was experienced as too much. At the same time, some of the informants stated that they did not recognise these type of messages at all. In Section \ref{sec:simplicity} we discussed minimalistic design which is about bringing the most important elements into focus, without elements that will distract the user. Microsoft presents it as "Simple Can Be Powerful", which means that simplistic design not necessarily needs to mean lack of functionality. From this we conclude that we should avoid too much features in our video game concept. We should keep it simple, and focus on a few motivating aspects. One of the informants desired more time to read information and instructions, and suggested that there should be a way to tell the system when you are finished reading. This is supported by guideline o.22 in Section \ref{sec:designelderly}, and will be taken into consideration in our exergame concept. 

\subsection{Cultural and Lifestyle Diversity}
Guideline g.7 presented in Section \ref{sec:summaryguidelines}, and the second of the eight golden rules presented in Section \ref{subsec:golden}, are about matching cultural and lifestyle diversity in the games. This was expressed by the informants as important, and they suggested different type of themes for a possible game concept, like dance, swimming, apple picking etc.. If they were to play a game like this they stated that it would need to include sports or activities related to real life. They also mentioned the importance of appropriate music. They did not like the music in the commercial games because they were not the type of music they used to listen to. Using appropriate music was discussed in Section \ref{sec:motivators} as a motivator for exercising, and \cite{schutzer} sees music also as a way to divert from pain coming from the exercises. The majority of the informants agreed that music was important, especially to keep the rhythm. However, they stated that they wanted it quiet, and they wanted music more related to their generation. This is an important requirement we will set for the exergame, however, it is outside our profession to come up with specifics about what music to include. 

The informants' opinions did not differ much according to gender on what they liked and did not like in the games we tested. However, as discussed in Section \ref{sec:motivators}, different people's needs and expectations, as well as aspects such as gender and ethnicity should be considered. We will meet this by creating a concept with a series of games that covers a variety of interests, as well as getting to choose a male or female character. Race might also be considered, but it is difficult to cover all races. Chao et al. \cite{chao} discuss that to meet diversity requirements, it is important to be in contact with the relevant people. Even though we clearly have not covered the total group of elderly, we have included a small group to understand some needs and expectations.

\subsection{Summary}
We will now provide a summary of the discussion to in a more precise way answer research questions 1 and 2. 

In our previous project \cite{project} we discussed that the existing commercial games are not aimed for the elderly user group, as they are too rapid and complicated. In the several related research presented in Section \ref{sec:relatedresearch} this was seen, and the different characteristics of elderly was discussed, and different sets of guidelines were suggested. Based on the literature studied, we have learned that the commercial exergames are not suitable for the senior user group. However, we wanted to explore this in more detail ourself. To see with different types of games what are good aspects and bad aspects with the existing games. Together with the previous research we will answer this question:

\emph{RQ1: Are existing commercial Xbox Kinect games suitable for exercising purpose for the senior user group?}

As mentioned initially, effectiveness, efficiency  and satisfaction are relevant measurements when evaluating the usability of systems. From findings done in workshop 1 we can conclude the following about the commercial games tested: 
\begin{itemize}
\renewcommand{\labelitemi}{$\bullet$}
\item The existing commercial games tested on a group of elderly do not meet the requirements of effectiveness. The games do not spend enough time on instructions and information, and do not give sufficient feedback on what the players are doing is right. The menus are too complicated and it is too many elements showing at the same time. The buttons presented in the games are too sensitive, and it is not intuitive how to press the buttons. 
\item The existing games meet only to a degree the requirement of efficiency. In FruitNinja it was clear that the informants did not understand what was required from them, and they just waved their hands uncontrollably. The majority of the the informants understood what was expected from them in the tennis game, and played through this game without problems. The skiing game was also well understood, until the second match where the two players that played together switched tracks. All of the informants had problems relating to their player after switching tracks. None of the menus met the requirements of efficiency. We had to assist the informants through the menus, and because of too much information and too sensitive buttons, the informants spend unnecessary time on getting through the menus.
\item Two of the four games played in workshop 1 meet the requirement of satisfaction. All of the informants liked playing the tennis and skiing game and they had fun while playing. FruitNinja, they did not like that much, which relates to their wish for playing games with a meaningful content that they could relate to everyday life, which is not the case for FruitNinja. The informants were not satisfied with the personal trainer game either, because this focused on "just" regular training, which they would rather do without the game. This strengthens up under Michael Zyda's statement on that the main focus in serious games should be on fun and entertainment \cite{zyda2005visual}. 
\end{itemize}

It is important to remember that this was the first time the informants played games like these, and initially they did only play the games once \footnote{Three of the informants got to play the skiing game a second time because they specifically asked for it. We did not observe anything new in this second session, and have because of this, and because the rest of the informants did not try a second time, not included this in our analysis}, which did not enable us to say anything about the informants learning curve. However, we did see that they understood more what was expected from them, and that they got more confident after a while. If we had went on with several rounds with the same games, we might have seen improvements. This was also mentioned by the informants. 

From research and workshop 1 we have also looked for answer to this research question RQ2. This is to make it easier to know what to focus on when developing a exergame for the senior user group.

\emph{RQ2: What are the design challenges when developing video games aimed for exercising for the senior user group?}

The game should meet the requirements from the GameFlow model \cite{sweetser} that we discussed to be important in Section \ref{sec:heur}. We found this model to serve as good guidelines, as many of them was mentioned as important aspects by the informants, and also because there is a clear relation between these guidelines and the guidelines we can draw out from the literature, presented in the previous chapters. Specifically the informants desired a game with a story that appeals and relates to real life, as well as appropriate music to their age. It was urged for more instructions on how to interact with the game, as well as what was expected from the players, the goals, and the rewards. The menus in the commercial games were seen as challenging because of the amount of information and the sensitive buttons. In a game for this user group this needs to be improved. The majority of the informants would only play these type of games together with others. Guideline g.8 presents social aspects as important, as well as it is a recommended guideline in the Game Flow model \cite{sweetser}. Therefore, it is important to include this in our game concept. The delay in the games was seen as a big problem, and should be acknowledged.  This is a technical issue that we are not in the position to evaluate the reason for. We believe that if this game is to be used for exercising, both at home and in clinical settings, technical precision have to be present. 

\section{Discussion of the Concept}
\label{sec:discconcept}
The idea for our concept is not random. It is well thought through, founded on background theory and our own findings. However, some of the choices made do differ from the findings. For example o.18 presented in Section \ref{sec:designelderly} is about the importance of using a one-coloured background, as well as not having text on pictures. We chose to have transparent text-boxes in front of the pictures when giving information and instructions. We chose this to not interrupt with the picture of the game environment. Having the box filled with colors hides too much of the picture, and it does not look very nice. In this case, we chose design over usability. However, o.18 also says that the text should be put where there are light areas in the picture, which wee meet by putting black text over a white transparent box.    

A limitation of the presentation of our exergame concept is that we do only....SE HER

\section{Discussion of Findings from Workshop 2}
\label{sec:discfindings2}
In Section \ref{sec:userinvolvement} we presented how important user involvement is during system development, as the end-users are the ones who will use the final product, and are the only ones who know what they want. After making a concept based on specified requirements, we wanted to invite elderly to a second workshop to get feedback on our ideas and design proposals. In workshop 2, we presented our prototypes for an exergame concept for a group of elderly, and we opened up for questions and discussions. Figure \ref{userdesign}, from ISO 13407, is a cycle of four steps for user centered design, where workshop 2 is about the fourth step, evaluating system design up against usability. As presented in Section \ref{sec:userinvolvement}, ISO 9241-210, states that feedback from users will help designers not only evaluate design, but also improve them according to the users needs. Involving the end-user in the design process will increase the possibility of making a user-friendly interface for the intended user group, which is stated in Section \ref{sec:usability} as crucial for making a successful system. 

In this section feedback from workshop 2 will be discussed, and we will present important aspects to be included in the future work of designing an exergame for elderly. These aspects will only be discussed, we will not make any changes to our current exergame concept, as this is out of the scope of this thesis. Our findings from workshop 2 will be related to theory provided in previous chapters.

\subsection{Few Comments and Moderate Response}
Generally there were few comments, but several questions, during the workshop. When we presented the various prototypes, the informants responded with silent nodding or just looking mutely at the screen. There could be several reasons for this. One reason could be because of the informants lack of experience with this type of technology. This can make it difficult to comment on a video game concept, as they do not know what to look for and respond to. One of the informants stated that it was impossible for her to say something about our exergame concept, as she did not know anything about how the game would work. Another reason could be related to how we presented our prototypes. Our exergame concept is based on highly interactive video game technology, and as presented in \ref{sec:prototypes}, this is very difficult to make good prototypes for. We did not make the prototypes in a way where it was possible for the informants to interact with them, we focused on making prototypes for presenting scenarios. We presented the prototypes with the "Wizard of Oz", meaning that one of us controlled the interactive system, while the other one took the role as the user, while the informants where observing. This was to try to give the informants a feeling of interaction. However, the lack of interaction between the informants and the prototype could have been a reason for the informants poor understanding of how the game would work. 

A final reason could be that we did not present the information good enough. A question from one informant supports this assumption. She asked if the avatar in the prototype would respond to her movements, even though she had played games like this before. After we explained the relationship between the avatar and the player, it seemed that the informants understood quickly. This increased understanding lead to more questions and feedback. Also, it became clear that it was not clear how the quizzes would appear. I4 thought the quizzes had to get answered at the same time as doing other tasks. She suggested that the quizzes should be separated from the other tasks. This was the way we initially had thought it to be. This misunderstanding was a result of us not presenting this clear enough.

\subsection{Difficulty Levels}

From the challenge element in the GameFlow model we know that games should offer difficulty levels that matches the player's skills. This, combined with the concentration element, have lead to the idea of adjusting frequency between the appearance of obstacles and number of simultaneous tasks according to the various difficulty levels. The difficulty of required exercises and movement should in line with guideline g.11 and g.12 also be controlled by the difficulty level, as this is an important feature for elderly that might suffer from various physical challenges. One informant stated that balancing over a log looked challenging, as she though she would feel dizzy, and that she was afraid of falling in real-life. This challenge was also reviewed by a physiotherapist . She stated that it was a difficult exercise to perform, and therefore should be included in higher levels in the game. This support the importance of providing the possibility for players to choose difficulty levels themselves, like stated in guideline g.13. However, the GameFlow model emphasises that the game also should support adjustment of difficult level after the player's skills and progress. By offering both these opportunities in this exergame, we meet the feedback from the informants, as they want to be able to master the various tasks, and not be forced into something they would not master. They both wanted to choose themselves, and to let the game follow their increased skills, so they could feel that they learned something and got better. 

There were some concerns and uncertainties about integrating quizzes in the trail. The GameFlow model with its concentration element, emphasises the importance of not distracting players from tasks they want or need to concentrate on. In addition, players should not be bothered with tasks they see as unimportant. The informants expressed concerns about having quizzes together with physical tasks, as they felt that might draw attention away from one of them. They did not like the idea of doing two thing as the same time. \emph{"I would like to focus on the task I am supposed to do"}, one of the informants said. It is important to take the element of concentration into consideration when developing an exergame for elderly. Characteristic c.9 says that elderly has difficulties doing more than one thing at the same time. We suggest that one solution for future work could be to choose, when starting up the game, whether you want to include the cognitive challenges or not. 

One informant suggested that the quizzes should provide different categories, to meet different interests and knowledge areas. On the work on the concept, we have not focused on what kind of questions or puzzles will appear. We just provided an example question. Further work should include finding an appropriate way to engage cognitive skills and to find suitable questions and tasks for this. 

\subsection{Graphics, Sounds and Interface}

Feedback shows that the informants liked the exergame concept we presented for them. The GameFlow model states that immersion is important for the player to effortless be involved in the game \cite{sweetser}. The informants expressed that the concept with the nature trail was a good idea because they were familiar with the environment. This supports findings from previous studies, like in \cite{gerling2}, where the participants appreciated the use of a real-life environment.  

Guideline g.4 and o.8 emphasise that an interface should be simple and not to complex, and that important elements should be brought into focus. The informants seemed to like and understand the menu interface we presented for them. Their general perception of the interface was that is was \emph{"beautiful"}. The back-button and the purpose of it was observed and understood immediately. A comment was to highlight the buttons even more, making them stand out more with a hint of shadow around them. SilverPromenade, presented in Section \ref{sec:relatedresearch}, had success with an intuitive and user-friendly interface, which quickly and easily guided the users to game play. 

As mentioned in \ref{sec:menu}, we have chosen a longer menu to include the possibility to choose actions, and at the same time avoid too much information at each step. The informants felt the menu was OK, but they wanted to use shortcuts when they had got familiar with the game. Inclusion of shortcuts for the experienced user is mentioned as an important feature in the second of the Eight Golden Rules, presented in \ref{subsec:golden}, and should therefore be included in future work for this exergame.

Generally, there were a lot of questions related to various elements in the prototypes. An example was a question about what the hearts meant, and what would happen if the player touched them. That the hearts were observed quickly is positive, as these are important elements of the game. This means that they stand out, which is stated as important. However, from the last of the Eight Golden Rules, we know that intuitive interfaces also are stated as important. What is not so positive with all the questions, is that it shows that our interfaces are not that intuitive as we had imagined them to be. This has to be taken into consideration for future work. 

Two elements in the menu were a source to a lot of discussion due to lack of intuitiveness and consistency. One of these was where the player was to choose environment and difficulty level. This especially yielded the understanding of the increased difficulty level between the three environments. It came clear to us that it was not very intuitive that the three different forest environments we presented were meant as three different difficulty levels. Future work would be to focus on including textual information to make this part more intuitive. A suggestion would also be to have instructions on what different difficulty level implies and means. The other menu step was where the player had to chose to play according to a specific muscle group. The informants did not see consistency between the term "muscle group" and what we presented in the menu. I1 said \emph{"The term muscle group, it does not fit with what you show here [...]"}. This is not positive for our interface, as the first of the Eight Golden states that designers has to strive for consistency to achieve good system design. We acknowledge that we did not think through this step thoroughly. However, what we presented was just an example, and we concluded that it is future work for professionals, like physiotherapists, to find appropriate muscle groups to include in the exergame.

As a motivator for exercising discussed in Section \ref{sec:motivators} and \ref{sec:georep}, we know that sound and music are important for the players enjoyment and gaming experience. We did not present any music, or soundtrack, for the elderly, as this, as mentioned in Section \ref{sec:outinthenature}, is out of our competence area. However, we told the informants that we wanted to use calm and peaceful music, to meet their feedback from workshop 1, where they expressed that it was too much noise. We also presented the informants for ambient effects, like birdsong and sound of running water, as this are sounds related to the environment of the game, and sound effects like a "ping" when collecting a heart. They seemed OK with this music and sound, but they wanted music with more rhythm to easily immerse into the game and exercise. Future work would therefore be to find appropriate music for the nature trail and the single games, based on the exercise and pace in each game. 

The informants had questions about what would happen if they were to fall down from the log and into the river. Technical issues similar to this question could include the possibility to walk outside the path and into the forest. These kind of limitations are aspect we have not considered in our concept. However, it is something that needs to be looked further into and integrated into the system requirements. We had not included this in the system requirements because we simply did not think about these kind of limitations in the environment of the game. Because we are familiar with computer and video games it is obvious for us that there are limitations to avoid the player from "walking our of the game world". However, this is not necessarily obvious for the user group. In addition, requirements on this are important for the programmers of this game. 

\subsection{Summary}
We will now provide a summary of the discussion of findings from workshop 2, where we will draw important aspects to consider in the future work of the developing this exergame for elderly. We will also try to answer research question 4.

\textbf{Future Work for an Exergame for Elderly}
\begin{itemize}
\renewcommand{\labelitemi}{$\bullet$}
\item Future work should include working on how to present the idea of difficulty levels for the users. There were a lot of questions and confusion related to what each difficulty level implied, and also how the difficulty levels were chosen. When we explained how it would work, the informants liked the idea, but this did not came clear from our prototyped concept. Future work should focus on including instructions and information about how the difficulty level is chosen and controlled. In addition, textual information should be included in the menu step where the player is to choose environment, to indicate that each environment holds initial different difficulty levels.  
\item Our prototypes showed obstacles, hearts and quiz icons all in the same picture. We presented it this way to show how elements will appear along the way in the nature trail, but the informants did not see the depth in the picture, and believed that they were suppose to do everything at the same time. The elements shown all together also made the prototype appear as distracting, as the informants did not know what to focus on. Future work should include studying the presentation of elements. A suggestion could be to not show the elements from a long distance, but let them appear as the player comes closer to them. 
\item In future work, the exergame should hold the possibility to separate the quiz from the nature trail. This should be done to let the player more easily focus on one task at the time. It might be more natural to include the quiz when the player has gotten more familiar with the game.
\item Future work should focus on how the quizzes should work, and how and when they will appear. How we presented the prototypes, made the informants believe that they should answer quizzes at the same time as doing something else. We clearly did not present this element good enough. This shows the need for a thorough introduction to how the quizzes will work. Future work could also include looking into presenting the quiz icons in a different way, to not take focus away from the other current tasks. 
\item The buttons in the menu should be highlighted to let them stand more out. This can be done by adding some shadow around the buttons.
\item It should be the future work for professionals to find appropriate music to the game. This yield ambient effects that fit the environment, sound effects to give users feedback on actions, like when collecting a heart, and music suitable for the intensity in the game.  
\item It is important to include the purpose of all important elements in the instruction video. There were a lot of questions about the hearts and their purpose, and we presented earlier that this could be a result of lack of intuitiveness. To our defence, the idea was that the purpose of the hearts should be presented in an instruction, but this was not included in our presentation. We observed how the lack of instruction created confusion, and we therefore see the importance of having these instructions.  
\item It should be the future work for professionals, like physiotherapists, to find out which muscle groups to present as choices when the player wants to play according to a specific muscle group. Our lack of knowledge in this area shone through when we presented this part, as the informants quickly commented the lack of consistency between the term "muscle group" and what we had presented as muscle groups to choose from. 
\end{itemize}

\section{Challenges Related to User Involvement}

It is important that we acknowledge the difficulties about engaging a user group into a setting they are completely unfamiliar with. It is hard to evaluate some users needs, when they do not even know about these needs themselves. We have learned from our two workshops and our previous project \cite{project} that today's elderly are not necessarily the right user group for a game like this, but that the next generation of elderly might be more suitable, because they are more familiar with the use of technology in their everyday life. However, it is hard to test a tool that is meant as an alternative form for exercising on people who are physically in good shape, and who keep one doing regular exercising. Therefore, a group of today's elderly was included.

Most elderly are inexperienced with video game technology, and it could be difficult to include them in a development process for an exergame, which is completely new for them, as well as something they have not acknowledged the need for. We found it challenging to include such an inexperienced user group in an early development phase. This it not just based on the development being in an early stage, but more the fact that elderly are very inexperienced and unfamiliar with the video game technology Kinect provides. The elderly did not know what to look for or respond to when prototypes were shown to them, as they had problems relating to what we presented. We believe that this could be a reason for not getting that much feedback on our prototypes. However, it could also be a result of how we presented our prototypes. It is difficult to make interactive prototypes for such a complex and interactive system. We know from Section \ref{sec:prototypes}, that how users interact with the prototype is crucial for understanding the system. It was not possible for the informants to interact with our prototypes, which might have lead to lack of understanding. It was not easy for the inexperienced informants to come with feedback on such an interactive system, when they were presented for still pictures, even though they had played Kinect games in workshop 1. We believe that a more interactive prototype would have lead to more understanding and that it would have triggered more feedback. Interactive prototypes would make it possible for the users to try to play, which would make it easier for them to understand how the game would work. This should be provided in a next phase of the testing and evaluation of this game concept. However, this is out of scope of this thesis.  
 
 
\section{The Delay Problem}
\label{sec:delay}
Both in workshop 1 and workshop 2, there were a lot of time spent discussing and commenting technical aspects related to the delay between the player and the avatar on the screen. This problem was a source to much confusion and frustration, and it did not give the informants a real-time experience. The informants also stated that the delay ruined the overall gaming experience. It is obviously crucial to take this into consideration when developing an exergame, as the delay partly destroyed enjoyment and immersion when playing. The control element in the GameFlow model \cite{sweetser} is about the player being in control of their actions, characters and movements. Feedback from the two workshops were that the informants did not felt this control, due to the delay. \emph{"There was no relation between our own movements and actions on the screen"}. For an exergame to be accepted and used in physiotherapy clinics or training groups, there has to be done something with this delay. 

When searching Google for "Kinect" together with "delay" or "lag", we got a lot of results, both articles, and forum questions and discussion. This can indicate that we are not the only ones who have experienced problems with Kinect's real-time presentation. We wanted to play some games over again to see if the delay was still there, or if it was just a one-time coincident. We started by playing the "Your Shape Fitness Evolved 2012", where we chose to play the same game as we presented for the informants in workshop 1, the Humana game "Aging with Grace". This is shown in Figure \ref{fig:remakeDelay}. Observing this figure, we clearly see that there is a significant delay between the players movements and the avatar on the screen. The avatar to the left is the trainer, while the other avatar portrays the player. What we can see here is that the player is following the trainer's movement, having the arms down, while the avatar portraying the player are behind, having the arms straight out to the sides.

\begin{figure} [H]
\centering
\includegraphics[scale=0.6]{kineDelay.jpg}
\caption[Kinect sensor delay]{This figure shows that there is a clear delay between the players movements and the avatar on the screen.}
\label{fig:remakeDelay}
\end{figure} 

We also tried "Fruit Ninja" over again. [HER SÅ VI IKKE LIKE TYDELIG DELAY].

We noticed that a clear difference between the "Your Shape Fitness Evolved 2012" and "Fruit Ninja" is the amount of information being processed. In the personal trainer game, the player's movements are tracked and compared to the required movements, and based on that, precision and various variables, like calories burned, are calculated. These calculations are not present in "Fruit Ninja". As these calculations are both process and time consuming, we believe that it is a source to the experienced delay. Our assumptions are supported by various articles, like \cite{kinectLag}, where it is stated that fitness games that uses skeletal systems requires more time to calculate than other games, and that these games normally experience delay. 

From a review done on some of forums found when searching Google, it seems like the degree of delay varies within different games. It is clear that people have different opinions about Kinect and the delay problem. Some people do not notice the delay at all, while others feel that Kinect is not nearly as good as its reputation, due to the delay \cite{kinectLagForum1} \cite{kinectLagForum2}. \cite{kinectLag} may support the assumption about delay depending on the game played. Here, Andrew Oliver from Blitz Games, states that software could be the source to the delay, and that the delay can be eliminated if developers choose the right software to build their game upon. Shotton et al. have written a paper where they present a method for giving quick and accurate prediction of 3D positions for Kinect. By learning the Kinect software how to predict human movement and behaviour patterns, process speed can be increased while delay is increased \cite{artikkelKinectLag} \cite{artikkelKinectLagIntro}. This paper gives a detailed presentation on how this could be done, and we refer the interested reader to \cite{artikkelKinectLag}.       

\section{Miscellaneous Aspects to Consider About the Exergame Concept}
\label{sec:misc}

\subsection{Discussion on where and in what circumstances the game can be used}
DENNE ER IKKE FERDIG, MEN SKAL SNAKKES LITT MER OM.
\label{subsec:whatwhere}

In our previous project \cite{project} we evaluated the game to have the most potential in a clinical setting, offered as a training method at physiotherapy clinics. This was based on two main things. First, "Samhandlingsreformen" encourage the use of welfare technology where possible in health care. Second, the majority of the senior user group, have little or no experience with video games, which will make it hard to reach out to these customers. In this thesis we have not focused on where the game should be implemented. However, we did ask the informants where they could see the game be used and two main settings were mentioned: in a group setting, for example in nursing homes, and in a setting with grandchildren. We decided to not study any further where the game can be implemented. This was both because we experienced that the majority of the informants had a hard time picturing themselves own and use a game like this, and also because from what we learned from the interviews conducted in our previous project, where physiotherapists could not say anything about whether they would use this game or not before they could test the actual game. For more discussion around where the game could fit, we will direct the reader to \cite{project}, and in particular Section 8.2 and Chapter 9 in that report. 

In Section \ref{sec:barriers} we discussed some challenges when it comes to motivating elderly to exercise. Some of these challenges were also mentioned by the informants. One informant saw it as a barrier to exercise if she lived far away from the training centres, and could see that the game could have a potential as a replacement for going to the gym. Another informant did not want to get controlled by time and appointments, which suggest that an exergame could be an alternative way to exercise on their own time. 
- Kommentar til avsnittet over: Fokusere det litt bort fra "oppsummering" av findings, og mer mot hva vi tenker. Som - i fremtiden, gjøre det mer tilgjengelig. Et slikt spill - få det hjem, trene på egen tid og premisser osvosv. 


- Nevne dette som skjer i Trondheim kommune, det at hjemmehjelpere tar med spill hjem til eldre. 

\subsection{Non-functional requirements not included in the game concept}

Dette er inkludert i diskusjon og ikke i konsept, fordi vi ikke har nok kunnskap til å være nok spesifikke på disse kravene. 

\begin{table} [H]
\label{tab:nfunc2}
\centering
\begin{tabular}{|l|l|}
\hline
3.1 & The system shall be able to run on both PC and Xbox. \\ \hline
3.2 & The system shall be easy to set up (physically).\\ \hline
3.3 & The system shall include Kinect functionality, like pausing \\ & a game by holding one arm out from the body. \\ \hline
3.4 & The system shall load within few seconds.\\ \hline
3.5 & The system shall be small in size and do not require too \\&  much space.\\ \hline
3.6 & The system shall not require too much capacity. It shall \\ & be able to run on a regular PC. \\ \hline
3.7 & The system shall not require too much power. \\ \hline
3.8 & The system shall avoid delay between the player's \\ & movement and action on the screen.\\ \hline
3.9 & The system shall ensure secure storage and sharing of \\ & profiles. \\ \hline
\end{tabular}
\caption[Miscellaneous non-functional requirements]{Miscellaneous non-functional requirements}
\end{table} 

\section{Quality of the Gathered Information}
\label{sec:discQuality}

Section \ref{sec:qualityresearch} states the importance of evaluating quality of the research we have done. This can be done by looking into the three criteria, \emph{reliability}, \emph{validity} and \emph{generalizability}. Based on this, we will discuss each criteria up against our work in this thesis. In addition, we will discuss some pitfalls that we might have experienced during our qualitative research.  

\subsection{Reliability}
It has been important in our thesis to distinguish between our own findings, and findings from theory and previous studies conducted by others. Our findings are presented in chapters separate from theory and literature, see Chapter \ref{chap:findW1} and \ref{chap:findW2}. In these chapters, neither our own opinions or theory are included. These chapters only emphasise feedback and opinions from the informants, where feedback are written as summary of discussion, or as direct citation. Findings from the two workshops are related to relevant theory in our concept chapter, Chapter \ref{chap:concept}, and in our discussion in this chapter.

How we have chosen informants for our qualitative research might have affected the quality of our study. Our informants are all members of "Seniornett", and are as mentioned, very committed to learning technology. All of them already use a wide range of technology devices, as mobile devices, tablets, and computers. We see them as more experienced with technology than most elderly in the same age group. Most of the informants have a high education, and they are relatively active in their everyday life. Five out of seven informants stated that they are active in the means of exercise, while the two others mentioned that they are active due to everyday tasks. We will state these informants as fit older people. The informants average age in workshop 1 were 70.6 years (with a standard deviation of 7,9 years), and 74,5 years (with a standard deviation of 7,5 years) in workshop 2. We feel that these informants are representative for our qualitative research based on their age span, however, they do not meet all the characteristics that are common for the older group. These informants are a group of fit elderly, and we want the exergame to also be accessible for both frail and disabled older people. The exergame is especially aimed at elderly and frail other people that are inactive. This group of people might have experienced reduction in functionality, which make it difficult to perform regular exercise and everyday tasks, or they might just be lazy. They are in the need of a tool that can help and motivate them to exercise. However, the exergame is also meant for fit elderly that wants to stay in shape, get feedback on exercises, and have an alternative to regular exercise. 

Our relationship with the informants is an important aspects when discussion reliability. With the exception of one informant, we had never met the informants before. The one informant we had been in touch with was the manager of "Seniornett", when he helped us to set up meetings and getting in contact with seniors. In addition, non of the informants knew each other. We met the informants at the first workshop, where we introduced ourselves. The fact that we were unknown for the informants could have resulted in the informants not daring to say what the actually meant and felt. It might also have been the case that this sat a kind of barrier, making it intimidation talking to us. However, when first meeting the informants, we welcomed them and talked about familiar everyday topics, just making conversation. We felt this helped establish a sort of comfort. Our observation during the workshop, showed no sort of discomfort. They talked, laughed and joked a lot, both with us and each other, and we felt that the atmosphere was good. 

Our role in this workshop might have affected the reliability of our findings. What is preferred is that our participation would have been neutral, but that was not the case. We participated in the workshop by informing about the technology, showing how the different games would work, we guided the informants when needed, and we partly participated in the discussion. This might have influenced the feedback from the informants. In addition, in workshop 2, we presented a design and concept that we had made. We wished for a valuable brainstorming around the concept, and urged the informants to be critical and come with both positive and negative comments. The informants were polite, and where mostly positive what we presented. I1 said \emph{"No, it is nothing negative"} about our concept. The fact that we presented our own concept, may have lead to the informants being afraid of hurting our feelings by saying what they actually meant. 

Our choice of games that we presented for the informants was not a coincidence. Each game were chosen for a reason, which can be read in Section \ref{sec:chosengames}, and we wanted to trig different reactions. This could have affected the outcome of our qualitative research. Choosing different games might have changed the findings from our research.  

\subsection{Validity}

In our discussion we have related findings from the qualitative research with theory and previous studies within the same subject, and what we see is that our results support what is found in this related theory and findings. We did not experience much deviation in our findings from findings in previous studies. The only aspect we can comment on, is clear interest, and questions from the informants, related to various technical aspects. The findings we got from the two conducted workshops covers a wide range of topics related to our master thesis, and they form a thorough basis for answering our research questions. We will therefore state that our findings are valid and of relevance to the purpose of the study. 

We had some difficulties placing our qualitative research method within a specific research method. Overall, our research included use of different research methods, which made it difficult to put one descriptive "title" on which research method we have used. In addition, we have used observation as a part of our overall research, which is about observing participants in a natural environment. However, this was not the case in our research, as playing Xbox Kinect games are not familiar to elderly. We tried to use an environment that was as realistic as possible, as "Gulhuset" is a place where elderly use to meet, and we imagine that to be a place where a future exergame could be used. We have, in Chapter \ref{chap:metode}, presented the research methods we have used. We have in the same chapter discussed and explained the reason for why we have chosen the various methods. We ended up describing our overall qualitative research as \emph{experimental simulation}, which is about observing people in a fixed, but as natural and familiar setting as possible. We feel that our discussion is thorough, and that it strengthen the validity of the information gathered. 
    
\subsection{Generalizability}    
Our focus in this thesis has been to develop an exergame concept for elderly. We have studied how elderly interact with commercial video games, with a goal of explore what aspects that are important to consider when developing a game for this user group. In addition, we have studied their attitudes towards exercising, and technology in general. This, together with theory and related literature, are used to specify system requirements for a video game for elderly, where our exergame concept is based on these requirements. We feel that these system requirements are so general, and are based on such thorough research, that it could be used as guidelines for others who want to develop video games for elderly. This also includes use for other cases than the one we have studied. About generalizability, Section \ref{sec:qualityresearch} mentions three types, naturalistic, moderate, and conceptual. Based on this discussion, we see conceptual generalizability as relevant for our thesis.    


\subsection{Other Quality Aspects to Consider}

In our qualitative interviews we have used focus groups as setting. Section \ref{sec:qualitativeInterviews} states that a preferred size for a focus group is 6-12 participants, but that mini-focus groups with 3-4 participants are accepted if the participants are experts on the discussed topic. Totally, we had seven informants for workshop 1, but these were divided into two days, with three and four informants each day. Theoretically, this means that we have conducted mini-focus groups. The problem with that was that the informants were far from being experts on the topic, as they in fact never before had seen the technology we presented for them. However, what was a bit different with this first workshop was that the we, in beforehand of the focus group interview, had introduced the informants for the Xbox Kinect technology. We asked about their perceived gaming experience in the interview, and felt that we received good, detailed and descriptive answers. 

The second workshop we involved five informants, where one informant that did not participate in workshop 1. That means that she attended workshop 2 with no previous experience with the Xbox Kinect technology, except from what we presented during our first meeting with "Seniornett". This made it difficult for her to relate to and understand what we presented during workshop 2, and she was therefore not able to provide us with much feedback. This is negative for the quality of the information gathered at workshop 2, as the number of involved informants already was less than what is preferred. 

In \ref{sec:otherQualityAspects} we present \emph{elite bias}, which is a pitfall that might have affected our results. Including only one group of people in qualitative interviews can give incomplete and under-representative data, and as a result, it could be hard to understand the broader situation. Because of the overall time limit for this master thesis, and other practical reasons, we only included this one group of people, even though they have the same interests for learning technology. Our data gathering would probably been different if we had gathered a group of "random" people, because they would have different backgrounds and interests.

Another pitfall, presented in \ref{sec:otherQualityAspects}, that we might have experienced  is the \emph{Hawthorn effect}. This is about observation and the risk of having people behaving differently because they know they are being observed. This may have been an issue in our qualitative research, since the Hawthorn effect may have an even bigger impact with the use of video recording. However, we tried to make a comfortable setting, and to "hide" the video recording equipments as much as possible. The informants seemed calm and unaffected by the video recording, and we therefore conclude that the Hawthorn effect did not affect the quality of our work. 

In the beginning of workshop 1, we had a presentation where we told the informants about the goal for our master thesis, the agenda for the workshop, and what we wanted to figure out and explore during the workshop. We also informed them about the technology used and the games they were about to play. About providing the informants with that much information, Section \ref{sec:ethicalchallenges} presents that participants with insight into what the researcher is looking for, might behave according to that information. This might have been the case in our workshop, as we described quite detailed what we were looking for. This could have set some boundaries related to creative thinking and what they felt was appropriate feedback. However, we felt that a thorough introduction was necessary when including a group of people with such an inexperience with video game technology. 

We will short list some additional quality aspects: 
\begin{itemize}
\renewcommand{\labelitemi}{$\bullet$}
\item User involvement is about involving users from the start to the end of the system development process. We only included the informants two times in the development process, to discover their needs and opinions, and to evaluate our design. Basically, due to the theory of user centered design the informants should have been more involved. However, this was not done due to time constraints in this master thesis, and due to the informants inexperience with technology. 
\item Workshop 1 was held over two days, but it was not executed exactly equally the two days. One day the informants got a little more instruction and guidance than the next day, some informants got to play longer than others, and not all the same questions were asked during the discussion. All this can have affected the findings from this workshop. 
\item The informants assumptions of what the workshop would involve could have affected the outcome of the workshop. 
\item Since we performed focus group interviews, the informants were influenced by each other statements and opinions. They discussed an unfamiliar topic, and it seemed easy to just "mean the same" as the informant stating something. As we had two groups of informants, the informants in within each groups formed similar opinions, that not necessary was the same in the two groups.   
\end{itemize}

\subsection{Our Conclusion}

\section{General stuff}
- Diskusjon av gjennomføring av workshop
- Guidelines/tips til nestemann


\cleardoublepage
\chapter{Conclusion}

During our study we found commercial exergames to be not as inappropriate for elderly as previous research first indicated. There were functionality and elements that did not suited the characteristics of elderly, but mostly, we evaluate the commercial exergames to meet some needs for effectiveness, efficiency and satisfaction.  What has to be taken into consideration when developing an exergame for this user group are real-life experiences, more instructions, motivational factors, clear goals, simple menus, technical precision, and social aspects. This thesis presents an exergame concept, that are based on a real-life theme with familiar activities and challenges, that has included most the considerations mentioned above. We believe it is a good foundation for a future exergame for elderly. In addition, our thesis provides general guidelines and requirements that can help developers create user-friendly video games that meet the needs and characteristics of elderly.  

\section{Future Work}

Future work for an exergame for elderly can head in different directions. 

One direction will be to continue working on the 
- Ta tak i future work presentert i diskusjonen. Fungerer for vårt konsept. Ta dette med i ny interasjon. Lage og bruke mer interaktive prototyper for å øke forståelse. 
- Teste på en mer diverse gruppe. Også se på de som er frail. 
- Se på fysioterapeutenes side
- Validere og  utvide våre funn og guidelines.  
- Jobbe med å finne ut hvordan det kan brukes, og hvor det kan brukes.
- Finne ut hvordan man kan eliminere delay.

\cleardoublepage
\bibliography{bibl}
\bibliographystyle{unsrt}
\addcontentsline{toc}{chapter}{Bibliography}
\pagenumbering{gobble}
\pagestyle{plain}
\cleardoublepage
\appendix 
%\appendixpage*
\appendix

\section*{Appendix A - Presentasjon/infomøte 25.02.2013 kl. 14.15}
\label{A}

Ca 12 personer kommer

Dette er noe mindre enn vanlig, og A (en tidligere kollega av Lill fra NTNU) tror det kan skyldes temaet "spill" som kanskje ikke føles like "matnyttig" som endel av de andre foredragene.

En kort intro av ansvarlig for seminaret: Han presenterer kort litt problemer relatert til det å bli eldre og han spør blant annet hvor mange som har gått på ski i år (2-3 hender i været).

Scenen overlates til oss ca 14.20.

Etter presentasjonen av SuperMario (muntlig, ved bilder og ved to filmklipp) kom det noen spørsmål:

Kvinne: "var det dere viste oss nå ferdig innspilt? Hvor kommer du (spilleren) inn i bildet? Hvordan styrer man dukken?" \\
Det forklares at man bruker kontroll (som løftes opp igjen)
Kontrolleren ble også vist ved starten av omtalen om videospill. \\
Mann: "Er spillet for barn opptil 3 år?" (ironisk!).

En mann forteller at kona bruker Nintendo Wii og trener balanse. Hun er 72 år. Hun pleier komme ganske svett og sliten opp fra kjelleren. Han forteller at dette koster ca. 3000 kr.

Etter at Lill (professor) kort har sagt litt om samhandlingsreformen og velferdsteknologi opplyses det av seminaransvarlig om at de skal ha besøk av Klara Borgen fra kommunen om få uker. Hun skal snakke om velferdsteknologi.

Lill kommenterer at spillprosjektet har kontakt med kommunen og at vi vil snakke med kommunen om evtentuelt å arrangere slike spillegrupper 
som vist på en av slidene (fellesdansing).

Så demonstrerer vi to spill med Xbox Kinect: \\
Tennis (en spiller)\\
Slalom (to spillere som kunkurrerer)

Lill åpner for spørsmål/kommentarer: \\
Kvinne spør om oppsettet med teknologien. Hun uttrykker at dette virker vanskelig og lurer på om man trenger prosjektorboks?\\
Vi forklarer at man må ha kameradelen (Kinect-sensoren) og TV eller PC, men ikke prosjektoren.
Vi kommenterer også at oppsettet ikke er helt intuitivt idag, det er mye som må settes opp riktig. Vi forteller at dette skal tas hensyn til når vi jobber frem et konsept til spill for eldre.

En mann spør hvor mye et slikt spill koster. \\
Vi svarer at det koster ca. 2000 kr for Xboxen og Kinect-sensoren. Da følger det også med to spill. Vi forteller også at det deretter kan bli kjøpt og lastet ned ganske billige spill.

Det kommer en kommentar fra en kvinne om svimmelhet. Hun syntes det virket som at slalomspillet kunne gi svimmelhet. Hun syntes det var mye detaljer og ting som skjer på skjermen. \\
Vi forteller at det faktisk har vært kommentert før for akkurat det spillet.

Mann: Nevner prosjektet "Generasjon 100" på NTNU. Dette prosjektet inkluderer medisinske tester og noen som trenger ganske hardt. \\
Vi sier at vi selvsagt skal se om dette er relevant for oss. \\ 
Det nevnes også at Generasjon 100 skal undersøke om man får flere friske hjerneceller når man trener mer.

Angående balanse, så nevnes det at noen er medlemmer i GAK 
(Gløs. Akad. Klubb), og at et av medlemmene der hadde falt og brukket
begge beina i ene leggen forrige uke

Kvinne kommenterer type musikk og type aktivitet på spillene.
Hun vil ikke ha "barnslige spill" (Med dette tror vi hun mente at uttrykket til de forskjellige spillene vi viste frem blir litt barnslige. Dette inkluderer tema og musikk). Hun synes ikke de spillene vi viste appelerte til henne. Hun foreslår at det burde heller vært mer "gammeldags" musikk (feks. 60-70-tallsmusikk) og god rytme. Hun nevner at hun kunne tenke seg et spill med dans, og spesifikt så nevnes Rørospols.

Flere er enige i en påstand om at det er mere sosialt å møtes fysisk 
i gruppe, fremfor å spille alene hjemme. Det ble også uttrykt usikkerhet om hvorvidt det ville være plass til å spille et slikt spill i en liten leilighet. 

Diskusjonen er over og vi åpner for påmelding til workshop og deler ut 2 ark til hver deltaker, der vi også oppfordrer dem til å snakke med en nabo.

Vi småprater litt med noen deltakere etter at møtet er hevet ca. kl 15.12

Fra diskusjon etter møtet: 

Kvinne forteller at hun trener på "senior puls" på Sats med musikk som passer for eldre.

En dame sa hun trodde den største utfordringen ved et slikt spill ville være farten. Hun mente det burde være mulig å velge hastighet selv, og at man burde ha mulighet til å øke denne ettersom man ble mer kjent med spillet.

Athar (Post Doc fra NTNU som var med for å observere presentasjonen) forteller at han observerte ansiktene når vi presenterte 
tennis og ski (slalom). Deltakerne "lyste opp" når ski ble nevnt.

\newpage
\section*{Appendix B - Exercises from "{Ø}velsbanken"}
\label{app:exercises}

The exercises presented here are modified from \cite{eldretrening}. We have got permission to use the exercises, but we could not use their pictures. Therefore, we made our own. We will present a range of exercises that, together with feedback from workshop 1, have been used as a basis for our video game concept.

\begin{figure} [ht!]
\centering
\includegraphics[scale=0.8]{WeightShift.jpg}
\label{weightshift}
\end{figure} 

\begin{figure} [ht!]
\centering
\includegraphics[scale=0.8]{Swimming.jpg}
\label{swimming}
\end{figure}

\begin{figure} [ht!]
\centering
\includegraphics[scale=0.8]{StrechFlank.jpg}
\label{stretchflank}
\end{figure} 

\begin{figure} [ht!]
\centering
\includegraphics[scale=0.8]{LiftingYourLegs.jpg}
\label{liftlegs}
\end{figure} 


\begin{figure} [ht!]
\centering
\includegraphics[scale=0.8]{Squats.jpg}
\label{squats}
\end{figure}  

\begin{figure} [ht!]
\centering
\includegraphics[scale=0.8]{GoSideways.jpg}
\label{gosideways}
\end{figure} 

\newpage
\section*{Appendix C - Review of the original Norwegian menu}
\label{app:menureview}

\begin{figure} [H]
\centering
\includegraphics[scale=0.45]{menuStep1.jpg}
\caption[Menu review -  part one]{This figure shows the menu step by step, from the beginning to playing a single game, here picking apples. The selection of single games is a result of the chosen muscle group.}
\label{menu1}
\end{figure}

\begin{figure} [H]
\centering
\includegraphics[scale=0.45]{menuStep2.jpg}
\caption[Menu review - part two]{This figure shows the menu step by step, from the beginning to playing a single game, here picking apples. Single player game and difficulty level easy are chosen. When ready to start the text "raise hand above head to play" is shown.}
\label{menu2}
\end{figure}  
\addcontentsline{toc}{chapter}{Appendix}
\cleardoublepage
\end{document}

