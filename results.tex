\chapter{Results from Workshop 1}

\section{Functionality and Usability}
\subsection{Introduction and instructions}
\emph{"It needs to be a softer and more instructive introduction in the game’s rules"}, was I5's opinion about the information given about the game. It was a common opinion that the different games should spend more time on instructions and explanation on the different exercises, especially in a game like the personal trainer game. I1 compared this to the way they did it in an aerobic course she used to attend in her earlier days. She explained how they practised the exercise slowly before they could do it fast. Everyone agreed that the exercises were OK and that it was only a question of learning in the personal trainer game. 

As a part of improving the introduction, a thorough explanation of how the player interacts with the game should be given. When the informants played their first game, they seemed confused and unknowing. One informants started pointing on the screen and asked \emph{"How do I point?"}, meaning how he could press the buttons. When playing the personal trainer game I1 asked \emph{"Press the button? Is there any button here?"}. It was clear that it was not obvious what was meant by pressing a non-physical button. Another informant was confused when first trying to interact with the game and waved in front of the screen and asked \emph{"Is that mine?"}.  None of the informants had ever played games like these, and did not understand how the game could be controlled by their body movements, even though we had explained this. Therefore, we had to explain and show them how they could navigate their hand on the screen. After explaining this once for each informants, we did not need to explain this any further. This suggest that a thorough introduction video explaining the interaction between the player and the game is important, especially for this user group. 

\subsection{Complicated menus}
There was a common agreement that the games had too complicated menus.  The avatar of the hand that was navigating on the screen was too sensitive, and it was hard to keep the hand still long enough to actually "press the button". In the sports game, where the participants played tennis and skiing, buttons were pressed by holding the hand on the button for a certain amount of time. \emph{"This is worse than working with a mouse this [on the computer]"}, said I6 when trying to navigate in the menu on the sports game. The time needed to hold on the different buttons seemed to be too long. This especially applies for people with declined strength in their arm(s). One of the informants had some trouble keeping the hand long enough out to the side to pick hands for the racket in the tennis game and needed several trials, as well as help from us.

In addition to being sensitive, the hand in the personal training game was also very unclear, and did not really look like a hand. This seemed to cause some problems because not all informants understood that the object on the screen was their hand. Two informants suggested that an arrow-marker like the one used on most computers would be more intuitive and easy to use. I1 added \emph{"You could just walked up and just touched the screen, hehe. I just wanted to go over there and touch it"}, referring to today's technology where almost every device has a touch screen. 

In "Your Shape Fitness Evolved 2012" the menu appeared to be very complicated. There were too much information on the screen, and even though we instructed them on what to choose, the three first informants had a lot of problems choosing the right alternative. To get to the right alternative, they had to scroll to the next page by pressing and arrow on the left or right side of the screen. These arrows were very hard to see and very sensitive. When telling the participants to scroll through the menu by pressing the right arrow, I1 said \emph{"Which arrow?"}.  Because of the huge amount of information on every menu site, the informants kept pressing the wrong button several times. In addition, this menu had a huge "back"-button, and the informants kept pressing this button without intention. All of the informants participating in Workshop 1 had problems with this menu, and it was clear that this was a bad experience.  Therefore, we decided to start further in the menu in Workshop 2, to not spend too much time on this, and also to avoid the bad experience for the informants. Therefore, the last group managed the few steps in the menu they were allocated very well. We found it strange that this menu was so complicated, as it actually was aimed also for the older user group with their game "sprek alderdom", or "fit old aged" in English. 

Even though the different menus were hard to understand, most of the informants agreed that this was a matter of practising, and that it would be easier if they knew what to do. 

\subsection{Information and feedback}
The information and feedback given during the play, did not seem to affect the informants. Some of them did not even recognize it was there. \emph{"I do not think I read that much text"}, said I1. \emph{"I did not react to the text at all. If you had asked me, I would not know it had been text
"}, said I3, while I4 said \emph{"It [the text] did not need to be there [in the game]. It could as well be gone"}. I2 on the other hand said he had seen the text but it disappeared too fast, so he did not have the time to read it all. I2 suggested that it should be a way to confirm that you have read the information before it disappears. However, all of the participants said they had seen the instructions and the information in the menu. 

Most of the informants did not always understand when the game was over. Especially did this apply for the tennis-game, where most of the informants were just standing looking into the screen after the game was over.  The instructions in both the tennis- and ski game, like "take you hand over your head when you are ready to play", was well understood by almost all participants. Although, we had to assist some by reading the message, we believe that these messages should have been clear enough, and the informants did not see the message because they were not concentrated on their task. It was not obvious for all participants that they had the racket in one hand, and the tennis ball in the other, so we had to explain this a couple of times. 

It appeared that it might have been too small text in the menu in "Your Shape Fitness Evolved 2012", after observing that at least one informant spend some time reading the text, and at one point had to move closer to the TV-screen to read the text. The same informant had the some problem when navigating through the menu in the tennis-game. Because it was only one informant that appeared to have problem with this, we assume that this informant had impaired vision. However, it is important to acknowledge that this is a common problem for the older population (kilde kanskje?), and should therefore be considered in a game designed for them. 

Two participants mentioned that it is hard to do several times at the same time, and that it therefore is hard to concentrate about both the oral information and the textual information. It also appeared that some of the informants had not seen all the messages shown on the screen during the play, and P5 thought it was too much: \emph{"No, limit the total amount of information, I think"}. 


Even though a most of the informants agreed that it was too much unnecessary information, it seemed that it was desirable with feedback on what they were doing. \emph{"I think it is very nice to see how far I have walked, and how fast I have walked, and how much downhill and uphill and things like that"}, said I1, when she was telling about an mobile app she was using to give her information about her progress when cross-country skiing. One of the other informant knew what app she was talking about, and seemed to agree. This implies that feedback on their progress is important for some of them.  \emph{"I am sitting here thinking about the program "Puls" on TV. There there are many exercises I wish I knew how to be done and how to be done right"}, said I3. Some of the other informants agreed. 

(The messages given by the commentator seem to entertain the informants watching the game being played more than the player him/herself. )

Not all informants understood that there were introduction videos and instructions in the beginning of the game. This became clear to us as most of the informants tried to do what the avatars did in the video, and seemed confused when nothing happened. \emph{"Should I try to hit it [the ball]? What am I suppose to do now?"}, asked I1 during the intro video. However,it appeared that the second instruction video that was shown between two matches in tennis, was understood as an introduction video, and that they all learned from it.  

\subsection{The nature of the game}
\emph{"It bothered me that I did not see the correlation or the relationship between my own body movements and what was happening on the screen"}, answered I2 on our question about how they experienced the game. I1 agreed: \emph{"There are some delays, yes"}. All of the participants complained about the delay in the different games. This especially applied for the tennis-, personal trainer- and ski game, and was mentioned as a problem several times by the informants. Also I5 complained about the technological aspects of the games: \emph{"There was some small technical details that were problematic. [...] I could not see in the ski game that anything special happened with me. [...] And in tennis, it was something about the time aspect, with when you tried to hit and when you actually hit [the ball]. I got a feeling that they [the game] helped me a lot in the beginning"}, and added \emph{"I experienced the tennis- and ski games as games with lack of technical perfection. It was not good enough technically to make a real time experience."}  I5 mentions several times that when you get use to the delay, it might be more fun, as I5 is actually accepting that there is a delay. 

Most of the informants had a hard time coordinating their moves in the personal trainer game, this applied especially for the men. There were fast movements and a clear delay. \emph{"[...] I think I could do this [the exercises] better without that [the game]"}, said I1 after playing the personal trainer game. \emph{"I think my movements on the screen look too slow. Are they that slow?"}, asked I6 when playing. There is a clear delay between the players actual movement and what is happening on the screen.  \emph{"I think it was confusing to see yourself, an especially both [himself and the personal trainer]. They were not synchronized"}. On the question on whether this destroyed the playing experience, I5 answered \emph{"I think so"}. 

\emph{"The game that required something from you was the ski game. The rest of the games did not require you to follow the instructor apparently"}, answered I3 after being asked how they experienced the game. It was apparent that some of the participants did not understand what they were meant to do in the games and that they just "did something". All of the participants agreed that FruitNinja was a bit silly and that it did not require anything except from waving hands and that they did not see the relationship between what they did and what they achieved. \emph{"[...] I did not get any feedback on my movements"}, said I2.  It seemed that all informants found it important to get feedback, see results, and experience a learning effect. 

Some informants mentioned that the ski game was very fast. At the same time there was a comment about the importance of having the fast pace, to make the game exciting and fun. I1 thought it would be too fast for older adults that lived in a nursing home, referring to her own mother. There were one problem that seemed to appear for most informants when playing two player on the ski game. There were two matches and in the second match, the player switched track. This appeared to be very hard to understand, and they kept asking several times which track was their. \emph{"I did not understand anything"}, said I2 after playing the ski game. 

\subsection{Sound and picture}
\emph{"It was much hubbub about it, right. "Are you leaving now?"(imitating the voice of the lady in the personal trainer game). I got the feeling of when you are listening to the GPS in the car "take a u-turn", hehe"}, said I1 about what she felt about everything happening around the game.

One informant meant strongly that it was too much fuzz around the game, and she was uncertain how motivating this was for people playing, and knew for sure it did not motivate her. However, these aspects were mostly irritating when watching other informants play, and not so apparent when playing themselves. 

On the question on what they thought about the music I1 answered: \emph{"I think everything about that was too much"}, while I2 answered \emph{"I like Mozart better"}. Other types of music that was mentioned as more suitable than ordinary pop music was swing. All informants agreed that music is important to keep the rhythm when exercising, but they would prefer the music not being too noisy. It was a general opinion that the music was inappropriate.  I5 said \emph{"Yes, the sound was a bit irritating. Tiring"}, while I7 says \emph{"You get sensitive [to sound]. You do not want it. You want it quite. You would want to be active, but without too much background noice"}. I4 added \emph{"[...] I would prefer walking out in the nature. Then you can listen to the birds [...]"}.

\emph{"I think it was very confusing to see myself. Especially to see both of us [himself and the trainer] [...]"}, said I5 on the question about what they thought about the avatars and the picture in general. Except from that, there were not many comments about the avatars. However, most informants agreed that the avatar of themselves in the personal trainer game mad them look fat, which they did not like. 

It seemed like most informants saw the objects relevant to the game, and that the only visual problem was the text that was not seen by all.

\emph{"It is too much elements on the screen. Where should you look?"}, said I7 when watching I5 play. It was a general opinion that there were too much information in the menu in the personal trainer game. 

Some of the informants also felt it was too much happening in the games at the same time and I5 suggested that there could be different levels of things that happens in the game. \emph{"[...] Eventually when you get better and manage to keep track of more things, you can add more things [to the game] that happens. A lot of what happens in these games are not relevant"}.

\section{Perceived Usefulness and Value of Entertainment}
\emph{"Mostly, it was quite amusing"}. This was the general feedback from the informants after the gaming session. When we asked the informants right after they were finished playing, if they enjoyed it, the response came quickly and the answer was “yes!”. One of the informants stated that he could see that it could be fun and useful for someone, but it was not for him. He would not have bought it for himself. The informants were divided in their opinions about if they would buy a game like this or not. Some of them would rather workout for themselves or go to training centres, while others see this gaming as more amusing than using a treadmill or an exercise bike. I2 stated that “No, I do not think I am going to buy it. Then I rather go to Elixia [training centre]”, while I5 stated that “I think this seemed very fun, so I want to buy one like this. I have one of those exercise bikes in my basement, but it is so boring that I can not bear it”.  Some of the informants stated that one of the reasons for why it was fun playing was because it was a completely new experience. I6 said that “No I have been involved in something I have never been a part of before. [...] It is a new world that has opened up, that is for sure”. Other reasons for why they enjoyed playing was that they could imagine playing with their grandchildren, and that it was a fun way to exercise. They liked the idea of combining gaming and activity. I4 said that “they [elderly] might think it would be fun to do this and be active at the same time”. I6 explained that she felt different from when she was watching the other informants play, to when she was playing herself. “It game more to participate that I had thought. Because, when I sat and watched it felt so unreal to have someone on the screen, but in the activity, when you go in to it, it was not so stupid after all”. The observations we did during the gaming session supports the feedback the informants have given us. There were a lot of smiling and laughing, and it seemed like they had fun. However, perceived value of entertainment and perceived usefulness does not has to be the same. 

There were some of the games that the informants liked better than others. Kinect Sports Season Two with its tennis and skiing was the game that the informants liked the most. They thought it was fun activities and they liked the challenges the game provided. “I liked it very much. I like that kind of activity”, I4 said. The informants also liked that the game required something from them. I3 said about skiing that “This was a bit fun. Yes, it was. Because, here you have to pay attention and spend some effort [...]”. I3 also liked tennis for the same reasons.  

Your Shape Fitness Evolved 2012, the “personal trainer” game, and Fruit Ninja were the two games that the informants liked the least. They did not see the need for the personal trainer game, as they rather would do these type of exercises in a training centre or by themselves. I3 said he did not like this game because it did not required anything from him. “The aerobic game I did not care for. That game anyone could do anywhere”. However, when I1 should choose which game she liked the most, she chose the personal trainer game as her favourite. This was because the other games had too much noises and loud sounds. When it comes to Fruit Ninja, the informants did not see the usefulness of this game. They laughed a lot while playing it, but they thought the game itself was stupid. I3 said “No, I thought it was stupid. It does not put any requirements on you. It was just.. [showing, waving with his hands”. We did ask the informants if the thought the game was fun, even though it was a quite unrealistic game. I1 answered that “Yes, it was a bit fun, I think. You see all the fruit that smashes. That was lovely”. I1 laughs. I6 stated that she did not see the fun in all the smashed fruit. “[...] I think that it was just chaos. I think. It becomes much mess on the wall, but it was just chaos”. I7 is more of the same perception as I1, and answers I6 with “yes, it was chaos, but at the same time fun!”. Another reason for why the informants did not care for Fruit Ninja was that they did not see a connection between their movements and the outcomes on the screen. “And if you hit something or not, that you did not had any sense of. You did not get any feeling of that”. This was perceived as confusing, and they did not understand how to do things the right way. 

To summarise our observations, we saw that the informants had fun and enjoyed playing these games. However, most of them did not see the games so useful that they would have bought it and used it themselves. They have other ways to exercise that they would rather prefer. The games they liked the most were those involving real-life activities, that combined movement and entertainment. The did not care for the personal trainer game, even though it involved realistic exercises, because it did not give them any value of entertainment. Fruit Ninja was perceived as stupid, and mostly, it gave them nothing.    

HER HADDE DU SKREVET NOE MER; SKAL DET VÆRE MED?

\section{Physical Outcome}
\subsection{All kind of movements are good, but what are we exercising?}
 “If it [the game] do good for your body? Yes it does. All kind of movements are of the good” -I3

“I liked it very much. I like this kind of activity. I felt it was useful, healthwise(helsemessig)” -I4 

“It was amazing how much body you actually used on so little” -I5

Everyone of the informants were positive to the games, and could see that there were some kind of positive health effects, but they did not know exactly what they were exercising. They urged that the game should include information about which body parts you are exercising when you are doing the different exercises. Another general opinion was that they wanted to know if they were doing the exercises right or not when they were playing. 

From what we observed and from what was mentioned by the informants, it is clear that it is important to have a structured program where you equally, and symmetrically exercise similar body parts. This came for example clear from tennis and FruitNinja. Several of the informants mentioned that their shoulder and arm was in pain after playing tennis and FruitNinja. In tennis, the player is only playing with one hand, and the workload gets unsymmetrically. In FruitNinja there is no structure, and all of the informants just waves uncontrollably with their arms, which might have been too strong movements.

\subsection{Did not need the game to do the exercises}
“The tennis game require something. You have to pay attention. However, I did not like the aerobic game. Everyone can do this game anywhere”, said I3 about how he liked the games. 

It was a general opinion that the personal trainer game (aerobic) was confusing and not good enough. Several of the informants said they only looked at the personal trainer, and not on their own avatar due to the significant delay. Some mentioned that they would rather switch the TV off and do the exercises without the game. It was not very clear from this game how what you gained from doing the exercises, and it was hard to get points because of the delay. This shows the importance of technical correct games, as well as a meaning or a goal, to motivate people to play. 

“You get hot” -I7
 (vet ikke helt om denne skal være med ennå)

\section{Immersion and Engagement}

\section{Motivation and Mastery}
\subsection{Motivation}
“Motivation is extremely important” - I5. Motivation is one of two things the informants mentions as an important aspect for a game like this. They tell us that for elderly to use a video game for exercise, it has to possess features that will make them wanting to play. The informants stated goals and socialisation as motivational aspects for a game. Goals were mentioned to be to achieve a high score, or it could just be to move your body to music. “I think that what is important is one’s own movement and motivation”, I6 said. “And to move after a rhythm, that is always positive. If you do not get it all perfect, it does not have to be that important”. Social aspects of gaming is mentioned as important for the informants. I4 said that she thing meeting others for exercising is motivating. I5 feels that meeting together in a group for exercise gives him a sense of pressure that he does not like. “Yes.. I do not want to.. I mean, I have had pressure all my life. I do not want it anymore. I do not want to put me in that situation where people come and ask “are you going to join this?”. I7 agrees that this type of pressure is not motivating. I6 said that “It is like that, that well-being is attractive, pressure is not attractive anymore”.  Another aspect that were considered as motivating was to get information about why they should play the game. The informants stated that if they could get to know, e.g. the training benefits, from doing the different activities, it would be more fun and motivating doing them. “For me it would have been important to know that; okay, now we do the slalom, but what we actually do is training balance, and that we train strength in thighs and stuff like that. For me that would be important for motivation”, I6 said. “That we get information about what we exercise when, and why we should do this. [...] I think that is very important for people as grown up as us. We have to know why we should do this”. We can see from the feedback from the informants, that entertainment alone is not motivation enough. They need to know why they should play the game, and how it can be good for them. 

One interesting observation we did during the gaming session was that the features in the games that were supposed to be motivating, were perceived as the opposite. Cheering, loud music, encouraging comments, fans and high scores were perceived as noisy and annoying, and not motivation at all. Some of the informants thought it was too much “going on”, and it became too much nagging. The comments that was meant to be encouraging became irritating and stupid instead of motivating as it was intended. “I became a bit irritated. When it comes a cute voice saying “yeah, thats great!”, “hurrey!”. I think it is stupid, I have to admit”.  I1 explained how she perceived the encouragement from the game by saying “I thought it was too much hustle. [..] Especially in the tennis game, you shall have the fans and you shall have the “hurray”, and you shall have comments and stuff like that. I do not know how motivation it is, but for me it was not motivating, and I thought it was annoying. I do not think I would bare playing that game because I do not bare the hustle, in a way”. Some informants did not even notice all this “hustle”, that I1 described. “I did not notice it at all”, I3 said, and I2 agrees with him and said that “I did not think it was a problem”. Since these features was either perceived as annoying or not noticed at all, we conclude that we should avoid features like this in our video game concept. We should keep it simple, and focus on a few motivating aspects. 

\subsection{Mastery}
“Motivation and mastery. That is very significant” - I5. The other aspect the informants pointed out as an important part of a video game for elderly is mastery (“mestring”). Mastery plays a huge part for the players feeling of success and accomplishment, which affects their self esteem. I1 commented this after playing tennis. “I think I do not would do this.. hehe, if I should be completely honest. [...] Maybe if I had felt I had mastered it better”. Mastery is also mentioned by the informants as a motivational factor. I6 tells us that she has been teaching an activity class and that their main goal and focus was to experience mastery. She continues with “It is all about the experience of mastery, which is essential. And the older people get, the more important it is”. I6 emphasises how important the feeling of mastery is for elderly. It is not fun doing something one is not able to do, and elderly, especially, do not continue doing something they do not master. It is important for us to think about how to achieve experience of mastery for elderly when creating our video game concept. 

Some of the informants said that they felt that the game helped them to perform the various activities. “I got the feeling that in the beginning, they [the skiing game] helped me alot”, I5 said. This was perceived as positive, as it led to a feeling of mastery. I6 described it as “When we hit the ball here now [tennis], we mastered even though we not really did. Then we had a huge experience of mastery! And that gives, that is positive”. She said that the feeling of mastery the game gave will make everyone wanting to play.  

Mastery, or the desire to master, could lead to wanting to play the game over again. I1 said that “I think that I want to try it one more time. To master it [...]”. When the informants first started playing the games, they were quite insecure and did not know how to perform the different activities and exercises. They completed the activities, but without the feeling of mastering it. “And the feeling of mastery. This we did not master”. But the informants did not see this as negative at all. They were all positive about it, and their common opinion were that it is all about training. “Nothing of it is difficult. It is just a matter of training”, I3 said. 

While observing the informants interact with the technology, play the games, and walking through some of the menus, we saw that the informants mastered the different challenges better and better. They learned how to connect with the sensor, they quickly understood what was required from them in the various games, and they responded faster to feedback and information. This was both our observations and the informants own perception. When we asked them if they felt it was easier the second time they played, most of the informants answered “yes”. We observed that it did not take long before the informants mastered exercises and understood how they were suppose to move. They just needed some time to observe and try it out. The informants stated that the feeling of mastery came fast, which was positive for the gaming experience. Some of the informants stated that they learned how they were suppose to play the game and how they should move to interact with the sensor by looking at the other informants play. Some of the informants did not fully understand e.g. Fruit Ninja before they had sat down and got the chance to see the next pair play. The first times information message was shown on the screen, the informations did not respond to it. We had to assist them in what they were suppose to do. After playing for a while, they responded to the information messages without guidance, and they did it faster for each time. This could be messages like “raise your hand above your head”, “move closer to the screen” etc. It was fun to watch how fast the informants learned. Even though they felt that they did not master the different tasks at first, they kept on trying, and after short time they had learned a lot and enjoyed the feeling of mastery. 

\section{Social Aspects}
\subsection{Playing together}
“I think the slalom was fun. But I am thinking more if I had a grandchild in a suitable age, and I could say “should you and grandma play the ski-game together? Just for fun?””. -I1

“I am thinking that in a nursing home we would probably like to play with someone” -I1

“[...] I do not want to do this alone at home” -I7

“I think both [playing alone and together] was ok. This was all new for me. It was a fun and nice experience in many ways. [...] I believe in  exercising in small groups” -I4

A general opinion was that this game could suit well in a nursing home setting, in a small group setting, or as an activity to do with grandchildren. It seemed that most informants liked to play together, however some informants mentioned that they were indifferent. A common opinion was that it would be fun to play together in the games that were made for fun and entertainment, like skiing and tennis, while games meant for exercising, like the personal trainer game, would be better to do alone. One informant told about her mother in law that got inactive on her last years. She believed that the only way to encourage her to become more active by playing games would be if there were some social aspects within the game. She was sure that her mother in law never would have switched on the TV and started playing this game herself, but by for example the help from grandchildren this could be more likely.

Everyone agreed that being social and meeting people is important in life, but that they would do it on their own time, and not be locked to a specific group or time. However, just having the opportunity to go out and meet people for example in an arranged training group would be nice. 

“It is like this: welfare is appealing, pressure is not appealing anymore”, said I6 referring to the importance of doing things voluntarily. 

\subsection{Watching others play}
“You are talented” -I1
“You are starting to get it now” -I1
“You have already played this game” -I5

These are examples of comments given by the “audience” while watching their co-informants play. They encouraged each other when playing and gave positive and motivating messages. The informants that were playing, seemed to enjoy the feedback they got from their co-informants. This seemed to make the play session more social and fun.  

\subsection{How to play together}
“[...] If there is going to be any point in doing something together, it needs to be that you are enhancing each other. Like that you get a better result if you are cooperating. [...]” -I5

One informant had a strong meaning about how to play together, and meant that it was more motivating to cooperate rather than compete. The other informants seemed to be indifferent whether they were competing or cooperating.

On the question on if they could see themselves play the game in their living room, and compete with friends over the internet, they were not very positive.

“This is very distant for me”, was I6s comment.

“It would probably be more motivating to just ask him if he wanted to come over and play”, said I1

These answers show how being social over the internet is quite unfamiliar to the informants. On the other hand, one informants told about an app she is using to get information about her progress when she is out cross-country skiing. This information could be shared with friends, and she did seem to like that. However, to make a game like this social, they all agreed that meeting in person would be better.

\section{Experience and Understanding of Technology}

\section{Understanding and Confusion During the Gaming Session}

\section{Aspects to Consider About an Exergame}