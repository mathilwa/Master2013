\chapter{Usability}
\subsection{Usability}
“Usability is the ease of use and learnability of a human-made object. The object of use can be a software application, website, book, tool, machine, process, or anything a human interacts with.” [fra Wikipedia]. Usability has played a huge part in the evolvement of bringing digital systems into people’s homes and everyday life. Making good, intuitive, easy to understand systems is essential for a system to be successful, accepted and used.\\ \\
In the beginning of the development of computers and digital systems only users with special knowledge could use these systems. They consisted of complex and not understandable applications meant to be used by professionals and not for the normal user, but in addition, they were not meant to. First, when computers and digital systems were developed with the intention of being used by the normal user, developers had to think about usability. The developers had to put the user in the center of the computer system, and not only focus on functionality and system features.\\ \\
The great shift in technology the world has experienced has been possible because of designers and researchers with focus on human needs. The term human-computer interaction was created, which is about including psychology in developing human-centric design. This is not an easy task, and it includes people from a lot many different sciences. Ben Shneiderman list “psychologist, instructional and graphic designers, technical writers, experts in human factors or ergonomics, information architects, and adventuresome anthropoplogists and sociologist” as some of the people to be included in the process of saying something about human-computer interaction. \\ \\  
To make a system design experience success one has to pay attention to various aspects of the user. The best way to do this is to involve users in the process of developing the system. When creating something completely new, it is all about understanding what the users want and need, but often the users themselves do not know what they want. It will not be possible to walk up to a potential customer and ask what he or she wants and needs. This does not mean that a developer should be creative, come up with a great idea and bring it into life without ever talking to the user group. The development of a new system should be done as a cyclic process, where development and user involvement goes hand in hand.   \\ \\
“Make it or break it” is a slogan that connects well to usability. A system can possess the best functionality there is, but if the users do not understand how to use it, the system will fail. Apple made a huge breakthrough when they launched they touch phone, iPhone. One may associate Apple with inventing the touch phone, but the truth is that it has been invented long before Apple launched their iPhone. Apple’s success is fully based on good, intuitive and user-friendly design and interfaces. Apple had their user’s needs as main focus. \\ \\       
Usability is a wide and quite abstract term, and it is not easy to understand or to perform right. There is no definitive solution on how to make a good and user friendly system, and it is hard, and it requires time to find out what the user needs. However, usability is extremely important for the success and adoption of a new system!