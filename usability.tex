\chapter{How to Design a System}
To design and develop a system, there are several aspects to take into consideration. What is the system going to be? What functionality shall the system hold, and what should the system look like? Who are the users, and what are their needs? What software should the system be built upon? There are many questions to be asked and figured out before starting with the development. In this chapter we will present theory about some aspects of system development. We will in Section \ref{sec:fourpillarsofdesign} look into general guidelines for how to design a system. An important part of system design is to set up system requirements, and this will be presented in Section \ref{sec:systemreq}. Since the focus of this master thesis is to design an exergame for elderly, we will also discuss various aspects of usability, both in general and specific for elderly. This will be done in Section \ref{sec:usability} and \ref{sec:designguide}. At the end of this chapter we will present a list of heuristics implemented in a model of "flow". This model will serve good to evaluate a system's usability and for developing a system in an early phase.

\section{The Four Pillars of Design}
\label{sec:fourpillarsofdesign}
It is the work of an interactive system designer to combine the sense of what attracts the user with system functionality. To help these designers develop successful systems, a theory called the four pillars of design has been developed. The four pillars of design does not guarantees brilliant systems, but it could be helpful along the way in a development process. The four pillars of design consist of \emph{user-interface requirements}, \emph{guidelines documents and processes}, \emph{user-interface software tools} and \emph{expert reviews and usability testing} \cite{mmi}.    

\subsubsection{User-interface requirements (Ethnographic observations)}   
A major key to success when developing a system is connected to specifying the user requirements, and how well these requirements are defined and understood. The way to specify requirements differs from system to system, but what the final result always should include is the same, the system's \emph{context of use}: who should use the system, where should it be used and what should it be used for. 

\subsubsection{Guidelines documents and process (Theories and models)}
It is important for the interactive system designer to generate a document that obtains a set of guidelines which specifies how the design should be. This could be design guidelines for the whole system, or it can be design guidelines for parts of the system, like functional design, implementation design, and interface design. Companies like e.g. Apple uses guidelines documents to specify design principles developers should follow. This is to create consistency in design across systems and products. Design may differ as different systems have different needs, but there are still some elements that should be considered in the guidelines document. It is important that the guidelines are flexible, so that they can adapt to changes in needs and experiences \cite{mmi}. Examples of what guidelines should describe are:

\begin{itemize}
\renewcommand{\labelitemi}{$\bullet$}
\item Words, icons, and graphics.
\item Screen-layout issues.
\item Input and output devices.
\item Action sequences.
\item Training.
\end{itemize}

\subsubsection{User-interface software tools (Algorithms and prototypes)}
In the early stages of development, it is difficult for users to picture what the final result will look like. One way to address this problem is to let the users get a realistic impression of the final result, by presenting different types of mock-ups and prototypes \cite{mmi}. What prototypes are, and how and when they should be used will be presented in Section \ref{sec:prototypes}. 

When deciding on which development environment to use, there are numbers of good products to choose from. Most of them are easy to use, and offers good features. The important part is for the developers to choose the development environment that is most suitable for the product they are going to make, due to performance, cost, and how easy it is to use and learn \cite{mmi}.
	
\subsubsection{Expert reviews and usability testing (Controlled experiments)}
To be able to launch a successful system, it is important with testing along the way in the development process. System testing could involve both experts and the intended users \cite{mmi}. 

\section{System Requirements}
\label{sec:systemreq}
To be able to make a system, designers must start by finding out what the system actually is going to be, what it should do and what functionality that should be included. Finding out and documenting all this is called a requirement analysis. The requirements in this analysis should focus on the role and the purpose of the system, viewed from the environment. They should include what is essential to the system, and avoid details that are unnecessary. \emph{How} the system is going to be realised, is not common to include in these requirements. From a requirements analysis there is natural to produce what is called a requirement specification \cite{braude2000software}. Requirements specification is defined in \cite{systemutviklingDel1} as \emph{"A specification that sets forth the requirements for a system [...]. Typically included are: functional requirements, performance requirements, interface requirements, design requirements and development standards"}. The different requirements are often separated into two categorisations, functional and non-functional requirements. 

\subsection{Functional Requirements}
Functional requirements are about system behaviour, and the provided functionality seen by the user \cite{systemutviklingDel1}. In addition it says something about input/output devices, range of users etc. \cite{mmi}. \cite{systemutviklingDel1} mentions examples of functional requirements:
\begin{itemize}
\item \emph{Requirements to ideal (functional) services and dialogues, possibly on several layers of abstraction.}
\item \emph{Requirements to ideal (conceptual) knowledge about the environment.}
\item \emph{Requirements to size: numbers of users, terminals etc.}
\end{itemize}     

\subsubsection{Functional Design}
Functional requirements will be used as a primary input to functional design. Functional design should say something about the complete functionality of the system that can be seen by the users, and is defined as \emph{"what the system shall do in a way that can be compared to the functional requirements. [...] it provides a basis for selecting the implementation. It is therefore idealised with respect to the concrete system and will hold for a range of technical solutions"}. Functional design is independent from technology and will not say anything about how the system is going to be implemented \cite{systemutviklingDel1}. 

\subsection{Non-Functional Requirements}
Non-functional requirements include performance requirements, interface requirements, reliability and availability requirements, error handling, and constraints. Performance requirements address speed, capacity and memory usage. Reliability and availability requirements specify reliability in quantitative terms, and the amount of time the system should be available to the users. Error handling is about how the system should respond to errors, and interface design says something about how the system will interact with the users. Constraints restrict how the system should be designed and implemented. This is done by describing accuracy, tool and language to be used, design constraints, which standards that should be used, and the hardware platform the system should be built upon \cite{braude2000software}. To summarise, non functional requirements are about ensuring quality of a system, and it says something about constraints on hardware, software, and the implementation of the system in general \cite{mmi}. These requirements are hidden from the users point of view. Examples of non-functional requirements are presented in \cite{systemutviklingDel1} as:
\begin{itemize}
\item \emph{Requirements to (concrete) physical interfaces.}
\item \emph{Requirements to physical conditions like temperature, humidity, power, consumption etc.}
\item \emph{Requirements to processing capacity: response times, traffic load etc.}
\item \emph{Requirements to exception handling.} \\ \\ 
\end{itemize}   

\subsubsection{Implementation Design}
Non-functional requirements form a basis for implementation design. While functional design is about what the system shall do, implementation design describes how the system should be realised. Implementation design connects the technical solution with the functional design, which makes the "manual" for how the final system will be implemented \cite{systemutviklingDel1}.  

An important part of developing a system is to decide and specify functional and non-functional requirements. It is not always easy to distinguish between the different requirements, in terms of which category they belong under. It is important to notice that categorisation of requirements is not the essential part. What is the main goal is to define and express the requirements as clear, simple and understandable as possible \cite{systemutviklingDel1}.  

\section{The Importance of Usability}
\label{sec:usability}
When developing a system it is important to keep in mind usability. Usability says something about how easy it is to use, learn and understand a human-made system. Examples of systems can be a machines, software applications, websites, tools, or anything else that involves human interaction with an object. Usability is often used in association with technology development, in terms of making digital systems understandable and intuitive for the users through user-friendly interfaces. Usability has played a huge part in the evolvement of bringing digital systems into people's homes and everyday life. The first computers and digital systems that were developed consisted of complex and not understandable applications that only professionals with special knowledge could use. There was little focus on simple and accessible systems, and complex interfaces were actually appreciated and gave the system credibility. First, when computers and digital systems were developed with the intention of being used by the normal user, developers had to think about usability. The developers had to put the user in the center of the computer system, and not only focus on functionality and system features \cite{mmi}.

We have experienced a great shift in technology from the first computer was invented and until today. Technology has been more mobile due to laptops, smart phones and other portable devices, and it is also used more often because of instant messaging, e-business and social networks \cite{mmi}. Users are no longer just "computer professionals", but normal people in all age groups, with different skills and interests, that are both experienced and inexperienced with technology. This has been possible because of designers and researchers with focus on human needs. The term human-computer interaction was created, which is about including psychology in developing human-centric design. This is not an easy task, and it includes people from  many different sciences. Ben Shneiderman lists "psychologist, instructional and graphic designers, technical writers, experts in human factors or ergonomics, information architects, and adventuresome anthropologists and sociologist" as people to be included in the process of saying something about usability and human-computer interaction \cite{mmi}.  

Usability is a wide and quite abstract term, and it is not easy to understand, to measure, or to practise right. ISO 9241-11 states usability as \emph{"The extent to which a product can be used by specified users to achieve specified goals with effectiveness, efficiency and satisfaction in a specified context of use."} \cite{usabilitydef}. From this definition we can see that there are three elements that could help us say something about a system's usability; \emph{effectiveness}, \emph{efficiency} and \emph{satisfaction}. \emph{Effectiveness} measures to which degree the systems covers all necessary functionality, and how easy they are to use and understand. \emph{Efficiency} is about how well different tasks are performed. This requires measurement on how much time that is used to accomplish a task. The last element is \emph{satisfaction}, which is about the overall user experience. This could be measures through interviews, studies, questionnaires etc. The degree of satisfaction is important for the system to be accepted \cite{mmi}. 

\subsection{Context of Use}
\emph{Context of use} is an important concept within the definition of usability, and is defined in ISO 9241-11 as \emph{"users, tasks and equipment (hardware, software and materials), and the physical and social environment in which a product is used"} \cite{maguire2001context}. The degree of usability and quality in user experience for a system is dependent on how well it is related to its specified context of use \cite{bevan1995human}. A system will be used by a specific population for specific reasons within a specific environment. It is therefore crucial that the system fits the needs of its intended users, tasks, equipment, and environment, as depicted in Figure \ref{contextofuse}. Analysing a system's context of use will help developers to specify who the users are, what are their characteristics, which functionality do they want, and where and in which circumstances do they want to use it. This understanding about the system can be used all through the development process, from system specification to the test phase \cite{maguire2001context}.

\begin{figure} [ht!]
\centering
\includegraphics[scale=0.5]{contextOfUse.jpg}
\caption[Context of use]{This figure shows how ISO 9241-11 presents a product's context of use. Modified from \cite{bevan1995human}.}
\label{contextofuse}
\end{figure} 

\subsection{Simplicity}
\label{sec:simplicity}
Making good, intuitive, easy to understand systems is essential for a system to be successful, accepted and used. "Make it or break it" is a slogan that connects well with success and acceptance. A system can possess the best functionality there is, but if the users do not understand how to use it, the system will fail. An example of this is Apple's huge breakthrough when they launched their iPhone in 2007 \cite{iphone2007}. One might associate the invention of the touch phone with Apple's iPhone, but the truth is that touch phones was invented long before the iPhone. The first touch screen was published as early as in 1968, where it was used for air traffic control, and the first smart phone with touch screen technology was released early in the 1990s \cite{touchphone}. Apple was an eager participant in the development of touch screen devices. Already in 1983 Apple had a prototype of a touch screen phone \cite{applefirst1983}. Apple's success with their iPhone is based on focus on user's needs throughout the development process, which has resulted in good, intuitive and user-friendly design and interfaces. 

"KISS" and "Less is more" are other terms related to usability. "KISS" is an acronym that stands for "Keep it simple, stupid". This principle was formulated by the American aircraft designer Kelly Johnson in the middle of the 1900s, and it states that simple systems work better than complex ones. KISS is not related to stupidity, but rather to intelligent systems that due to their simplistic design may be perceived as stupid. The KISS principle has been adopted into software engineering, and subjects as design and usability. It states that simplicity should be the main focus in design, and that every element that leads to unnecessary complexity should be avoided \cite{kiss1} \cite{kiss2}. Ludwig Mies van der Rohe was a German architect that used the term "Less is more" to describe his extreme simplistic and minimalistic design style, and his use of that term became a guiding principle in modern design. "Less is more" has also been widely used as a slogan in association with usability. \cite{rohe}. Minimalistic design can be described as \emph{"design at its most basic, stripped of superfluous elements, colors, shapes and textures"}. With minimalistic design, the most important elements are brought into focus. In this way the user will not be distracted from, or miss out on, the content that is important \cite{lessismore}. Also big companies, like Microsoft, focus on simplicity in their design. Microsoft has launched an article called "The Importance of Simplicity" in their developer network. This is about how to design user-friendly systems while still keeping good functionality. Microsoft presents a topic called "Simple Can Be Powerful". This means that simplistic design not necessarily implies lack of functionality. Simplistic design will provide ease of use for first timers. The idea is to present a design that is intuitive, understandable and easy to learn, with a possibility for the experienced user to choose to add more functionality. A possible solution could be to include customisation so the users can set up their own workspace, and include more features if wanted \cite{msdnsimple}.            
    
\section{Design Guidelines}
\label{sec:designguide}
In order for a system to become successful it has to be easy to interact with, and it has to offer functionality that are attractive to the user. There have been developed several guidelines to help designers make successful, user-friendly systems. In this section we will present a list of eight principles with focus on interface design, called "The Eight Golden Rules". We will also look into designing interfaces with focus on the senior user group. 

\subsection{The Eight Golden Rules}
\label{subsec:golden}
The "Eight Golden Rules", presented in \cite{mmi}, are a set of guidelines that have been developed over three decades with research and experiences. It does not exist a solution for how to make good and user-friendly interface design, but these "Eight Golden Rules" can serve as a starting point and a helpful design guide if they are used correctly. When using the "Eight Golden Rules", it is important that designers refines and implements the principles into the environment they are working in. 

We will now present the "Eight Golden Rules" \cite{mmi}:

\begin{enumerate} 
\item \emph{Strive for consistency:} Consistency in interfaces requires identical terminology for actions and layout. This is important for users not to wonder whether words, icons or situations means the same. 
\item \emph{Cater to universal usability:} Designers have to see the need for making a design that fits the diversity of users. There could be differences in age and technology experience, that requires transformation of content. Beginners would need guiding and explanations, while experts should have features for short cuts. This could improve quality of the system experience. 
\item \emph{Offer informative feedback:} The users should always receive feedback on their actions. Appropriate system feedback should be chosen in accordance to the importance of the actions performed. Process bars, sound as a response for clicking a button, or visual presentation for showing object in actions, is possible ways to give users feedback on actions.  
\item \emph{Design dialogues to yield closure:} It is important to create distinct work steps in dialogues. This means organising similar actions into separate groups with a clear start, middle, and an ending. To provide users with a feeling of accomplishment, feedback should be provided when a particular sequence is finished.     
\item \emph{Prevent errors:} The best solution to this problem would be not to experience any errors at all. Designers should prevent users from doing serious errors by e.g. not allowing inappropriate digits in a field or "hiding" buttons that could cause errors. However, when errors do occur users should be provided with informative instructions for how to recover from the problem.   
\item \emph{Permit easy reversal of actions:} Users should always be provided with the possibility to regret a performed action. This will make the system easy and comforting to use, as users know that every action can be undone. 
\item \emph{Support internal locus of control:} User should feel that they are controlling the interface, and not the other way around. This might be especially important for experienced users. Surprising changes in design and actions, in addition to boring, time-consuming situations, will not be well received. 
\item \emph{Reduce short-term memory load:} Designers should reduce the need for memorising information and how actions should be performed. The focus should be on designing an interface with visible information and intuitive actions.
\end{enumerate}

These presented guidelines are far from being the only guides for how to design a user interface. There have been done a huge amount of research in the area of usabilit. Jacob Nielsen is one of the participants \cite{nielsen2005ten}. He is a Ph.D from Denmark, and an expert in human-computer interaction. He has established a movement for how to easy improve user interfaces, invented several methods for how to achieve good usability, and he has also published a great amount of articles and books with usability as main subject \cite{JNielsen}. As a part of his work Nielsen has created a list of ten usability heuristics, which can be used as general principles when designing a user interface\cite{nielsen2005ten}. We will discuss heuristics in more detail in Section \ref{sec:heur}. Now, we will present difficulties and possible solutions related to developing interfaces for elderly users.

\section{Heuristics}
\label{sec:heur}
Heuristics are designed guidelines made to assess how good software design is, and it has become a widely used method for usability evaluation in software development. As learned from this chapter so far, it is important to develop software interfaces that are easy to understand, learn and conduct. Heuristic evaluation method allows for insight into users' point of view, even before there is an actual system, and is actually best suited in an early phase, before spending a lot of money on expensive prototypes \cite{desurvire}. As mentioned in \ref{subsec:golden}, Jacob Nielsen developed a set of heuristics which can be used as guidelines when developing user interfaces. These can be found in \cite{nielsen2005ten}. However, these heuristics are made for software development in general. We sought to find heuristics that were more applicable for game development. 

We found that a lot of research have been done on heuristics for games and different sets have been suggested. Some worth mentioning are Desurvire et al. \cite{desurvire}, Malone \cite{malone}, Shelley \cite{shelley}, and Federoff \cite{federoff}. Many of these overlap, and in some way, they all tell the same. 

Sweetser and Wyeth discovered that many of the heuristics proposed in the literature, did not evaluate the enjoyment in games. They argue that the many valid sets of heuristics presented in the literature should be integrated in a model where also player enjoyment can be assessed. How much someone enjoys something can be described by the concept of flow. The concept of flow was first proposed by  Mihaly Csikszentmihalyi, when he many years ago started  to study how people could be so immersed and engaged in something they did not get money for. He wanted to find out why they did these things. He found that the reason was the enjoyment they felt when doing it. He called this state "flow" because "many of the respondents described the feeling when things were going well as an almost automatic, effortless, yet highly focused state of consciousness" \cite{flow}.  Sweetser and Wyeth integrated the already existing heuristics into the model of "flow", and called this new model "GameFlow".  They argued that the nature of flow fits well as a way to structure the different heuristics found in the literature, into a model of player enjoyment. The "GameFlow" model has eight core elements which are related to Csikszentmihalyi's defined elements. The core elements are: \emph{concentration, challenge, skills, control, clear goals, feedback, immersion} and \emph{social interaction}, see Figure \ref{fig:gameflow1} and \ref{fig:gameflow2} \cite{sweetser}. 


\begin{figure} [ht!]
\centering
\includegraphics[scale=0.9]{gameflow1}
\caption[GameFlow criteria for player enjoyment in games, part 1]{GameFlow criteria for player enjoyment in games, part 1 \cite{sweetser}}
\label{fig:gameflow1}
\end{figure}  

\begin{figure} [ht!]
\centering
\includegraphics[scale=0.7]{gameflow2}
\caption[GameFlow criteria for player enjoyment in games, part 2]{GameFlow criteria for player enjoyment in games, part 2 \cite{sweetser}}
\label{fig:gameflow2}
\end{figure}  

We find most of the guidelines listed in the model also relevant for an exercise game for elderly. We will now emphasise the guidelines that based on literature discussed in previous chapters, are found to be most relevant. 
To do the exercises right and properly the person doing it should be \emph{concentrated} on what she is doing. Therefore, we evaluate guidelines 3 and 6 to be important. Not all exercises are fun to do, however, most exercises are good for your body. 4 states that the player should not have to do tasks that do not feel important. Therefore, it is crucial to give the player information about how the tasks in the game relates to exercise so that the tasks seem meaningful. \emph{Challenge} is also important in games for elderly. This is a user group with diversity, where many suffer from different types of physical and mental decline. It is not fun to play something that is too easy, and it is definitely not fun to play something that the player is not able to do. Therefore, number 7 is in particular important.  In addition, 8, 9 and 10 are important, because as with all other user groups, elderly will experience learning effects, and will get better. The possibility to add more functionality after initial tasks are managed, and offering different difficulty levels were suggested in Chapter \ref{subsec:guidelines}. In Chapter \ref{sec:motivators} self-efficacy is discussed as an important determinant of exercise behaviour and self-esteem is shown to be an element that makes people continue doing physical activities. Self-esteem will likely be reached when challenges are overcome.  The element \emph{player skills} is related to challenge. Because of memory related problems seen in many elderly as discussed in Chapter \ref{subsec:guidelines} we found especially 14 to be relevant. Players should not be bothered with long textual information beforehand, but should rather learn through tutorials as they are playing. 15 is important for the same reasons as why 8,9 and 10 are, and should be considered. In this case, the Kinect sensor have the property of being able to detect the people playing (see Appendix \ref{app:kinectsensortech} for a brief description of the Kinect technology), and can in that way detect if the movements are been done right or not. This makes it possible to individually increase the players skills at an appropriate pace for this player. As suggested as guideline G.4 in Section \ref{subsec:guidelines} 17 should be met. This is in particular important for this user group, because of their inexperience with technology, as well as the common problem of decline in visual acuity \cite{Billis}, \cite{gregor}. The feeling of \emph{control} is an element in both the game flow model and in the eight golden rules presented in \ref{subsec:golden}. Control is especially applicable for exercise games. This is because for the player to do the exercises right a realistic picture of the player and her movements should by shown on the screen.  18 should therefore be considered. 19 and 20 are crucial for the player to be able to interact with menus and to start and stop the game. Because of many elderly's lack of technology experience, this should be presented in an easy and understandable way. If the player does not understand how to interact with the game, she is likely to not use the game. As for every other technology system, errors should not affect the user. This makes also 21 relevant. The game should meet the requirement of \emph{clear goals} and especially 25. It is especially important to present intermediate goals, because of memory related problems which are quite common in elderly people, as discussed in Chapter \ref{subsec:characteristics}. Goal setting is also discussed as important to promote adherence to physical activity in Chapter \ref{sec:motivators}. Appropriate \emph{feedback} should, as suggested in \ref{subsec:guidelines}, be given as the player achieve goals, but at an appropriate time without disturbing the player. Therefore, 27 and 28 are relevant. \emph{Immersion} is about feeling involved in the game and is a determinant of player enjoyment. In Chapter \ref{sec:barriers} it is discussed that many look at exercising as time-consuming, and many see exercising as boring (Finne en kilde på at trening er kjedelig). Therefore, an altered sense of time will be positive for the exergaming experience, as described in guideline 31. At last, social interaction is, as mentioned several times in the previous chapters, a very important aspect to get elderly to exercise. 35 and 36 are not necessarily relevant in this case, because as discussed in Chapter \ref{chap:exforseniors} the elderly user group does not like to play over the internet. However, 34 should be considered, with emphasis on cooperation, as discussed as the preferable way of multi-play by \cite{Gajadhar}.

The GameFlow model will be used in our further work, both when evaluating elderly's enjoyment of existing commercial games, and when developing a early-phase game concept. To come up with a game appealing to elderly, this user group has to be involved in the development phase. We will now move on to the next chapter, where we will discuss the methods used to involve the user in the development process.
