\chapter{Characteristics of Video Games}
Game designer Sid Meier came up with one of the most famous definitions of game: “A game is a series of interesting choices” \cite{understandingvg}. This of course, does not apply for every video game. In \cite{understandingvg} the authors present the MDA-model, which was developed by Robin Hunicke, Marc LeBlanc and Robert Zubeck, after several workshops conducted at the “Game Developers Conference” in California between 2001 and 2004. This model divides games into three elements: mechanics, dynamics, and aesthetics, and is a very useful tool for designers to understanding games. Mechanics are not something we can see or hear, but rather the rules and basic code of the game. It is for example the algorithms that lies in the ground for creating the reaction pattern of a computer-controlled character. Dynamics is based on the mechanics and describes what events do and can occur during the gameplay, seen from the players point of view. Aesthetics say something about the emotions triggered when interacting with the game. A list of elements  that attract us to games is provided in the book (directly drawn from the book): \\
- Sensation (game as sense-pleasure)\\
- Fantasy (game as make-believe)\\
- Narrative (game as drama)\\
- Challenge (game as obstacle course)\\
- Fellowship (game as social framework)\\
- Discovery (game as uncharted territory)\\
- Expression (game as self-discovery)\\
- Submission (game as pastime).\\
One or more of these can be included in a game, but not all of them \cite{understandingvg}.

(I denne boken presenterer de fire sjangere av spill: action games, adventure games, strategy games and process-oriented games. Jeg klarer ikke helt se hvor exergames faller under, men om det måtte være noen blir det den siste. skriver ikke noe om det ennå, ettersom det ikke virket veldig relevant. )

\section{Aesthetics}
The aesthetics describes everything that can be experienced by the player. These are the elements that actually makes the game. The aesthetics can be divided in three, namely rules, geography and representation, and number of players. The rules say something about what the players can and cannot do, as well as what actions will make the score increase or decrease. The geography and representation element says something about how the video game is represented through graphics and sound. Here there are a lot of different design possibilities. The number of players is important, because there are huge differences between a single-player game and a multi-player game. This is an important choice because it affects a number of game elements. In a single-player game artificial intelligence (AI) is important, because the player has to play against the computer and not real humans. The computer can never be as intelligent as the human brain, and generally employ a very limited set of strategies. Therefore, it can be easy for an experienced player to win over the opponent. When designing multi-player games, artificial intelligence is not needed, as the player will play against other human players. These games faces other issues instead. For example it has be possible to distinguish between characters. This can be done by giving them different unique features, but at the same time not make any character superior. It is also important to facilitate social interaction between the players, like for example cooperation or competition \cite{understandingvg}. Often in multi-player games, it is possible to chat with the opponents or team members.

We will now look a little closer into the three categories rules, geography and representation, and number of players.

\subsection{Rules}
In \cite{understandingvg}, they distinguish between two types of rules; interplay rules and evaluation rules.The former are the physical laws. It determines what properties the different elements in the game should have, as well as the actions that can be done, as well as what will happen corresponding to the action action. An example of this is “What will happen when the player presses button A? Jump”. The latter defines what will happen when an action is made. For example if an action will be rewarded or punished \cite{understandingvg}. In \cite{understandingvg} gameplay is defined as “the game dynamics emerging from the interplay between rules and game geography”. The dynamics can be of different types. They can be entertaining or they can be predictable, or they can not be any of those. 

\subsection{Geography and representation}
Geography and representation is about how the game is represented through graphics and sound \cite{understandingvg}. Graphics and sounds are used to set the games environment and enhance the players enjoyment of the game, but has no effect on how the game is played.Graphics is needed in a videogame to in a best possible way imitate reality and enhance the player experience. This means that instead of telling out what is happening, it will be shown visually. Sound is important in video games for the same reasons as why graphics is important. Background music can set the atmosphere of the game, as well as give an indication on when actions will change and if the atmosphere is changing (for example from happy to sad). Sound can also be used as feedback to the player \cite{umlapproach}. There are different types of sound included in a  video game. \cite{understandingvg} distinguish between four categories:
vocalization, which is the game’s characters’ voices,
sound effects, which are sounds made by the different objects in the game, ambient effects, which is non-specific sounds that makes the atmosphere in the game, and music, which is the game’s soundtrack, but also a part of setting the atmosphere. The latter is a very important part of the game. Other elements that can affect sound effects are the environment, spatiality and physics. What kind of environment the player is in can affect the  sounds. An example of this, is what kind of floor the characters are walking on or weather conditions. The sound will also be affected by where in space it will come from. For example, a sound from far away sounds different from a sound close up. In addition, sound will be affected by movements. Imagine the sound of the sirens on an fast passing ambulance.

There are many things to consider when it comes to how the game should be represented. Ta med tabell side 107.  It has to be decided if the game will be in 2D or 3D and in first-person or third-person. Two-dimensional graphics are represented by only two coordinates and does not have any depth. This makes a very unrealistic scene. Three-dimensional graphics, on the other hand, has depth, and is therefore much more realistic. Every game will either have a first-person perspective, a third-person perspective, or a mix of both. In a first-person perspective, the game is played through the characters eyes, while in a third-person perspective the player can see the character he controls through the game. Games will also either have an isometric perspective or a top-down perspective. Isometric perspective is when three-dimensional objects are presented in a two-dimensional form, like in a architectural drawing, while a top-down perspective is, as the name suggest, when the scene is shown from above. The choice of perspective is important because it decides how the player will perceive the game world. It is also important to choose the right perspective with the right dimension. A third-person game can be played in a two-dimensional and a three-dimensional space, while a first-person game should only be played in a three-dimensional space \cite{understandingvg}. Another aspect experiences by the player is time. It is proposed to distinguish between play time and event time, where the former is the time a player uses playing the game, and the latter is the time in the game world. If play time and event time are the same, we can say that the game is played in real time. In many games it is possible to save the game, and return to the same state at a later time \cite{understandingvg}.

There are many different ways to present a game graphically, and Aki Järvinen defined three different styles common for video games. \\ 
Photorealism: “a style of painting that tried to completely mimic photographs”. There are two subcategories of photorealism: \\
Televisualism tries to give the same visually look as shown on the television. Typical for this, can be for example football games. Illusionism is the other category. Here it is used photorealistic graphics in the service of non-realistic content (siste setning skrevet av. skjønte ikke helt). \\
Caricaturism: Here the use of so called caricatures are used. Caricatures are drawings that presents an object by exaggerating the prominent features of the object, and often gives the feeling of a cartoon. \\
Abstractionism: In this category there are no real people of real-life objects involved. Instead  the game has a rather abstract form. An example of this is the well known game Tetris. A problem with these kind of games is that they often face a hard time on the market. This is because humans mostly get attracted by the story, and it can therefore be hard to create attention just around the mechanics of the game. Therefore, it is within this category very important with the story \cite{understandingvg}. 

\section{Story}
One important element of a videogame is the story, or the narrative. The story is included in the game to make the involvement and enjoyment better for the player. The story is just a part of the game, and is not the game \cite{umlapproach}.  Different events make up the narrative and will usually contain settings and characters \cite{understandingvg}. Unlike literature and film, which centers on the story, the game centers on play. Therefore, the narrative should be looked at in a player-centric context (Towards a Game Theory of Game, Pearce). The game designer is kind of creating the story together with the audience. This is because the player often make their own story throughout the game \cite{umlapproach}. This means that the game designer creates the background storyline, while the player creates the experimental story while interacting with the game or others. 

In \cite{understandingvg} they introduce interesting elements of narrative in video game, and discuss three different categories: “The fictional world: settings and actors”, “Mechanics: organizing narrative action”, and “Reception: the player’s experience of story”.

The fictional world: settings and actors can also be thought of as who and what. The game space, which is the game settings, is on of the most important parts of a game. The game space is a reproduction of some of the features of the real world and contains specific rules to make it possible to play the game. There are different characters in a game, and they are very important for the story. This includes not only the characters that the game is about, but also characters that make things happen and that interacts with the main characters. Therefore, all the different characters need to be considered. \cite{understandingvg} propose four different categories of characters: \\
-Stage Characters: These characters can not be interacted with, and serve more as just a part of a scenario. \\
-Functional Characters: These are also just a part of a scenario, but in addition, they have a general function, which makes it possible for the player to interact with them in some way. \\
-Cast Characters: These have specific functions in the game that has something to do with the story. \\
-Player Characters: These are the characters that the player is in control of. 

The different characters can be constructed in different ways. They can be constructed through description, which means through the way we can see them on the screen. This can be done in a symbolic, naturalistic or a “real-life” way. See table blabla for examples.

her skal det være en tabell

The characters can also be described through their actions, through their relationship to space, through other character’s views, or just through a meaningful name. The player character is the most important type, and was by Toby Gard divided in three different categories, in accordance to how easy the player will identify with them, avatars, actors and roleplaying: 

“Avatars are a non-intrusive representation of ourselves, actors are always part of a story (or have a story, albeit minimal sometimes), and roleplaying characters have very different abilities that we can raise according to our performance”. \cite{understandingvg} 

With the use of avatars the game is in first-person and can therefore not be seen, while actors usually will be seen in third-person. In addition, actors will usually have a personality and they are well integrated in the story, while an avatar usually do not have a personality at all. Roleplayers are quite different. Here the player create their own characters. The player can choose their name, and their abilities, as well as if they want to play in first-person or third-person.  

Mechanics: organizing narrative action can also be thought of as how the action of the story is organized. There is a basic concept for how to organize the story in a game, which is called “branching”. This means to have multiple paths in the story. 
Ha med figur 8.5 på side 181 i boka: a model of classical linear fiction. 
The figure shows the standard progression of a story of linear fiction. In traditional stories, it goes through a resolution. This can not be applied in a game, because it would not let the player do anything. Therefore, another model is applied for a game:
Ha med figur 8.6 på side 182: a model of interactive fiction
This model has no continuous curve. In this model it is more about finishing each chapter by solving puzzles, and relies on the emotional satisfaction the players get from the victory of solving a puzzle. Even though an action in a game can lead to different endings, the player is actually solving a story. These games are called progression games because the player has to finish different actions, before proceeding. The different chapters are cummulative, meaning that each chapter are building on each other. Very common in these kind of games it is a climax or a resolution at the end of each chapter (in many games you have to fight “the boss”). 

The other structure of narratives, is emergence games. This structure depends on a more active artificial intelligence where all the objects in the game has behaviours (Example directly drawn from the book: In a progression game, the dragon will always attack the player when she steps into the cave, but in an emergence game, this might depend on how the player behaves towards the dragon, which is a more “active” object with a few possible different responses. Some game designers (Smith and Juul) means that an emergent structure is preferable in games. This is because this structure gives the player more freedom. To be able to easily explore the smaller events in a game, game designers are making so called “quests”, which is the submission the players have to perform in the game. When the different quests are defined, it can be put together and be told as a story. In \cite{understandingvg} that say that “ideally, quests are the glue where world, rules and themes come together in a meaningful way”. Quests can be seen differently from the designer perspective and the player perspective: The designers look at quests as “a set of parameters in the game world (making use of the game’s rules and gameplay) that creates a challenge for the player.”, while the players look at the quests as “a set of specific instructions for action”. 

Reception is how the players experience a story. In \cite{understandingvg} they argue that: ”a reception-theory based analysis can explain the way that narrative and gameplay together determine the player experience in games that make use of stories”. We will not describe this theory any further here.. blabla.. 

\section{Gameplay}
NB: dette delkapittelet skal merges litt med tidligere delkapitteler. Er fra en annen kilde, så bruker en del andre begreper.

In a video game, everything really depends on the gameplay. Gameplay defines how the game is played and is the most important element of a game. In short, gameplay can be describes as the interaction between the player and the game. Siang. et al. describes two different kinds of interactions: player-object and object-object. When creating a game, a set of rules about what kind of interactions can be made have to be set. These rules are set by determining what kind of interactions are allowed between the two components: players and game objects. The paper describes the player-token interactions with this model:
figure “the player-token interaction”
which is a description of gameplay, set aside the players experience. 
Different from objects is that tokens are conceptual (hva betyr det?), and may not have a one-to-one mapping with the programming language (skrevet rett av, vet ikke hvor viktig det er).
In a video game tokens is a thing that can react with something in a game. It can be defined three categories of different tokens: static, dynamic and behavioural. Static tokens has the same visual look throughout the whole game. Dynamic tokens, also called entities, can react to certain events in a game. Behavioural tokens are the parameters how they will react to the player and other tokens and are not physically represented in the game. Example is a game level manager which tells the player about high scores etc. The tokens can have attributes, which is representing the tokens state. Example of an attribute is the number of life. The tokens can also have behaviour, which say something about how the token can react in certain events \cite{umlapproach}.

\section{User Interface}
The interface is there to enable interaction between the player and the token. In their paper Siang et al. present a figure describing the gameplay:

figure

The interface is where and with what the player interacts with the game and includes the input that is sent through the input devices, like a game console, and the output the user receives on the screen and the speaker \cite{umlapproach}.
